\section{Numerično integriranje}

\subsection{Kvadraturne formule}

\begin{definicija}
\emph{Kvadraturna formula}\index{Kvadraturna formula} je enačba
\[
\int_a^b f(x)\,dx = \sum_{i=0}^n \alpha_i f(x_i) + R(f),
\]
Pri čemer velja
\[
\alpha_i = \int_a^b \prod_{j \ne i} \frac{x - x_j}{x_i - x_j}\,dx.
\]
Tu so $\alpha_i$ uteži, $x_i$ vozli in $R(f)$ napaka.
\end{definicija}

\begin{opomba}
Kvadratna formula je vedno točna za polinome stopnje največ $n$. Če
izberemo primerne vozle, lahko dosežemo, da je formula točna za
polinome stopnje največ $2n+1$.\footnote{Gaussova kvadraturna
formula, glej poglavje~\ref{subs:gauss}.}
\end{opomba}

\begin{definicija}
\emph{Newton-Cotesove formule}\index{Newton-Cotesove formule} so
kvadratne formule, pri katerih za vozle vzamemo ekvidistantne točke
\[
x_i = a + ih,
\]
kjer je $h = \frac{b-a}{n}$. Formula je \emph{zaprta}, če sta
krajišči intervala vozla, sicer je \emph{odprta}.
\end{definicija}

\begin{opomba}
Naj bo $f \in \mathcal{C}^2([a,b])$. Za $n=1$ dobimo
\[
f(x) = f(a) \cdot \frac{x-b}{a-b} + f(b) \cdot \frac{x-a}{b-a} +
\frac{f''(\xi_x)}{2} (x-a)(x-b),
\]
od koder z izrekom o vmesni vrednosti izpeljemo
\[
\int_a^b f(x)\,dx =
\frac{h}{2} \cdot \br{f(a) + f(b)} - \frac{h^3}{12} f''(\xi).
\]
\end{opomba}

\begin{opomba}
Če je $n$ sod, ne Newton-Cotesova formula točna za polinome stopnje
največ $n+1$.
\end{opomba}

\begin{opomba}
Iz Peanovega izreka lahko izpeljemo, da za
$f \in \mathcal{C}^r([a,b])$ velja
\[
R(f) = c h^m f^{(r)}(\xi),
\]
kjer je $r$ najnižja stopnja polinoma, za katerega je $R \ne 0$.
\end{opomba}

\begin{opomba}
Naj bo $f \in \mathcal{C}^3([a,b])$. Za $n=2$ dobimo\footnote{Tudi
Simpsonovo pravilo.}
\[
\int_a^b f(x)\,dx =
\frac{h}{3} \cdot \br{f(x_0) + 4 f(x_1) + f(x_2)} +
\int_a^b \frac{f'''(\xi_x)}{6} \cdot (x-x_0)(x-x_1)(x-x_2)\,dx.
\]
Z uporabo prejšnje opombe lahko izpeljemo še
\[
R(f) = -\frac{h^4}{90} \cdot f^{(4)}(\xi).
\]
\end{opomba}

\begin{opomba}
Napaka metode kvadraturnih formul je enaka $R(f)$. Če je
$\widetilde{f(x_i)}$ naš približek za $f(x_i)$ in velja
$\abs{\widetilde{f(x_i)} - f(x_i)} < \varepsilon$, lahko
neodstranljivo napako ocenimo kot
\[
\abs{D_n} =
\abs{\sum_{i=0}^n \alpha_i \cdot \br{\widetilde{f(x_i)} - f(x_i)}}
<
\varepsilon \cdot \sum_{i=0}^n \abs{\alpha_i}.
\]
Če imajo vse $\alpha_i$ enak predznak, sledi
$\abs{D_n} < \varepsilon \cdot (b-a)$.
\end{opomba}

\begin{definicija}
\emph{Sestavljeno pravilo} za integriranje je enačba, ki jo dobimo
tako, da integral odsekoma aproksimiramo s kvadraturno formulo.
\end{definicija}

\begin{opomba}
Za odsekoma linearno aproksimiacijo dobimo sestavljeno trapezno
pravilo
\[
\int_a^b f(x)\,dx =
\frac{h}{2} \cdot \br{f(x_0) + 2\sum_{i=1}^{n-1} f(x_i) + f(x_n)} -
\sum_{i=0}^{n-1} \frac{h^3}{12} f''(\xi_i),
\]
kar lahko z izrekom o povprečni vrednosti poenostavimo do
\[
\frac{h}{2} \cdot \br{f(x_0) + 2\sum_{i=1}^{n-1} f(x_i) + f(x_n)} -
(b-a) \cdot \frac{h^2}{12} f''(\xi).
\]
\end{opomba}

\begin{opomba}
Za odsekoma kvadratne polinome pri sodih $n$ dobimo sestavljeno
Simpsonovo pravilo
\[
\int_a^b f(x)\,dx =
\frac{h}{3} \cdot \br{f(x_0) +
4 \sum_{\substack{0 < i < n \\2 \nmid i}} f(x_i) +
2 \sum_{\substack{0 < i < n \\2 \mid i}} f(x_i) + f(x_n)} -
(b-a) \cdot \frac{h^4}{180} f^{(4)}(\xi).
\]
\end{opomba}

\newpage

\subsection{Gaussove kvadraturne formule}
\label{subs:gauss}

\begin{trditev}
Naj bo $\rho$ nenegativna funkcija. Tedaj je
\[
\skl{f, g} = \int_a^b f(x) \oline{g(x)} \rho(x)\,dx
\]
skalarni produkt.
\end{trditev}

\obvs

\begin{definicija}
Naj bodo $p_n(x) = x^n$ polinomi. Polinomom, ki jih dobimo z
Gram-Schmidtovo ortogonalizacijo na $p_i$ z zgornjim skalarnim
produktom pravimo
\emph{ortogonalni polinomi}\index{Ortogonalni polinomi} za interval
$[a, b]$ in utež $\rho$.
\end{definicija}

\begin{lema}
Vse ničle ortogonalnih polinomov so enostavne, realne in ležijo
znotraj intervala $(a,b)$.
\end{lema}

\begin{proof}
Naj bo $p$ ortogonalen polinom in
\[
q(x) = \prod_{i=1}^k (x - x_i)^{\alpha_i},
\]
kjer so $x_i$ vse ničle polinoma $p$, ki ležijo na $(a,b)$, in
\[
\alpha_i =
\begin{cases}
1, & 2 \nmid \ord_{x_i}(p), \\
0, & 2 \mid \ord_{x_i}(p).
\end{cases}
\]
Očitno je $\skl{p, q} \ne 0$, zato je $\deg p = \deg q$. Sledi,
da vse ničle polinoma $p$ ležijo na intervalu $(a, b)$ in so
enostavne.
\end{proof}

\begin{definicija}
\emph{Gaussova kvadraturna formula}\index{Kvadraturna formula!Gaussova}
je kvadraturna formula, pri kateri za vozle vzamemo ničle
ortogonalnega polinoma $p_{n+1}$ (z utežjo $1$).
\end{definicija}

\begin{trditev}
Naj bo $f$ polinom stopnje največ $2n+1$. Tedaj je Gaussova
kvadraturna formula točna za $f$.
\end{trditev}

\begin{proof}
Naj bo $q$ ortogonalen polinom stopnje $n+1$. Vozli so njegove
ničle. Za polinoma $g$ in $h$ stopnje največ $n$ lahko zapišemo
$f = gq + h$. Tako sledi
\[
\int_a^b f(x) \rho(x)\,dx =
\int_a^b g(x) q(x)\,dx + \int_a^b h(x)\,dx =
0 + \sum_{i=0}^n \alpha_i h(x_i) =
\sum_{i=0}^n \alpha_i f(x_i). \qedhere
\]
\end{proof}

\begin{lema}
Uteži Gaussovih kvadraturnih formul so pozitivne.
\end{lema}

\begin{proof}
Naj bo $q$ ortogonalen polinom. Polinom
$p_k(x) = \frac{q(x)^2}{(x - x_k)^2}$ je stopnje $2n$, zato je
\[
0 < \int_a^b p_k(x) \rho(x)\,dx =
\sum_{i=0}^n \alpha_i p_k(x_i) =
\alpha_k p_k(x_k). \qedhere
\]
\end{proof}
