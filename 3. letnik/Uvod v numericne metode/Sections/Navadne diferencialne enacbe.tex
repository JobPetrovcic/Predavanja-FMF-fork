\section{Navadne diferencialne enačbe}

\subsection{Eulerjeva metoda}

\begin{definicija}
Naj bo $D = [a, b] \times \Omega \subseteq \R^2$. Funkcija
$f \colon D \to \R$ je
\emph{Lipschitzova}\index{Funkcija!Lipschitzova} glede na $y$ s
konstanto $L$, če za vsaka $(x, y_1)$ in $(x, y_2)$ v $D$ velja
\[
\abs{f(x, y_1) - f(x, y_2)} \leq L \cdot \abs{y_1 - y_2}.
\]
\end{definicija}

\begin{opomba}
Če je $f$ Lipschitzova glede na $y$, ima navadna diferencialna
enačba $y' = f(x,y)$ z začetnim pogojem $y(x_0) = y_0$ enolično
lokalno rešitev ne glede na $x_0$ in $y_0$.
\end{opomba}

\begin{opomba}
Občutljivost problema določimo z zvezo
\[
\abs{\tilde{y}(x) - y(x)} \leq
e^{L(x - x_0)} \abs{\tilde{y}_0 - y_0} +
\frac{e^{L(x - x_0)}}{L} \cdot \norm{\tilde{f} - f}_\infty.
\]
\end{opomba}

\begin{definicija}
\emph{Eksplicitna Eulerjeva metoda}\index{Numerična metoda!Eulerjeva}
je numerična metoda, s katero izračunamo približek vrednosti $y$ v
točki $x_n = x_0 + n h$. Približke vrednosti v točkah $x_k$ dobimo
tako, da se premaknemo v smeri tangente.
\end{definicija}

\begin{opomba}
Algoritem za Eulerjevo metodo je naslednji:

\begin{algorithmic}[1]
\State $y = y_0$
\For{$i = 0$ to $n-1$}
  \State $y \gets y + h f(x + ih, y)$
\EndFor
\end{algorithmic}
\end{opomba}

\begin{opomba}
Poznamo tudi \emph{implicitno Eulerjevo metodo}, pri kateri
$y_{n+1}$ izračunamo iz zveze
\[
y_{n+1} = y_n + h f(x + (n+1)h, y_{n+1}).
\]
\end{opomba}

\begin{definicija}
\emph{Taylorjeva metoda}\index{Taylorjeva metoda} je numerična
metoda, podobna Eulerjevi. Pri tem $y(x+h)$ izračunamo kot
\[
y(x+h) = \sum_{n=0}^N \frac{h^n}{n!} \cdot y^{(n)}(x).
\]
\end{definicija}

\begin{definicija}
Pravimo, da ima metoda lokalno napako reda $k$, če se pri točni
vrednosti $y(x_n)$ izračunani približek $y_{n+1}$ ujema z razvojem
$y(x_n+h)$ v Taylorjevo vrsto okoli $x_n$ do vključno člena s
$h^k$.
\end{definicija}

\begin{opomba}
Ker velja $y'' = f_x + f \cdot f_y$, lahko odvode $y$ izrazimo z
odvodi funkcije~$f$.
\end{opomba}

\newpage

\subsection{Metode tipa Runge-Kutta}

\begin{definicija}
\emph{Metoda Runge-Kutta}\index{Numerična metoda!Runge-Kutta} je
numerična metoda, s katero izračunamo približek vrednosti $y$ v
točki $x_n = x_0 + nh$. Približke vrednosti $y$ v točkah $x_k$
dobimo tako, da izračunamo koeficiente
\[
k_i = h f \br{x_n + \alpha_i h, y_n + \sum_{j=1}^m \beta_{i,j} k_j}
\]
in rekurzivno izračunamo
\[
y_{n+1} = y_n + \sum_{i=1}^m \gamma_i k_i.
\]
\end{definicija}

\begin{opomba}
Metoda je eksplicitna, če velja $\beta_{i,j} = 0$ za vse
$j \geq i$.
\end{opomba}

\begin{opomba}
Da ima metoda lokalno napako reda $1$, mora veljati
\[
\alpha_i = \sum_{j=1}^i \beta_{i,j}
\quad \text{in} \quad
\sum_{i=1}^m \gamma_i = 1.
\]
\end{opomba}


