\section{Nelinearne enačbe}

\subsection{Definicije in bisekcija}

\begin{definicija}
\emph{Nelinearna enačba}\index{Nelinearna enačba} je enačba oblike
$f(x) = 0$, kjer je $f \colon \R \to \R$ (gladka) funkcija.
\end{definicija}

\begin{definicija}
Naj bo $f \colon \R \to \R$ funkcija z ničlo $\alpha$.
Predpostavimo, da je $f$ gladka\footnote{Dovoljkrat zvezno
odvedljiva.} v okolici $\alpha$. Pravimo, da je $\alpha$ $m$-kratna
ničla, če je
\[
m = \min \setb{n \in \N}{f^{(n)}(\alpha) \ne 0}.
\]
\end{definicija}

\begin{opomba}
Če je $\alpha$ $m$-kratna ničla funkcije $f$, za $x$ v okolici
$\alpha$ velja
\[
\abs{f(x)} =
\frac{\abs{f^{(m)}(\xi)}}{m!} \cdot \abs{x - \alpha}^m,
\]
zato velja
\[
\abs{x - \alpha} =
\sqrt[m]{\frac{m! \abs{f(x)}}{\abs{f^{(m)}(\xi)}}} =
O \br{\abs{f(x)}^{\frac{1}{m}}}.
\]
Natančnost računanja ničel torej pada z redom ničle.
\end{opomba}

\begin{izrek}
Če je $f \colon [a, b] \to \R$ zvezna funkcija in je
$f(a) f(b) < 0$, obstaja tak $c \in (a, b)$, da je $f(c) = 0$.
\end{izrek}

\begin{proof}
Glej dokaz izreka~3.5.3 v skripti predmeta Analiza~1 v 1.~letniku.
\end{proof}

\begin{opomba}
Ničlo poiščemo z metodo
\emph{bisekcije}\index{Numerična metoda!Bisekcija}. Algoritem za
bisekcijo je naslednji:
\begin{algorithmic}[1]
\State $e = b - a$
\While{$e > \varepsilon$}
  \State $e \gets \frac{e}{2}$
  \State $c \gets a + e$
  \If{$\sgn(f(c)) = \sgn(f(a))$}
  	\State $a \gets c$
  \Else
  	\State $b \gets c$
  \EndIf
\EndWhile
\end{algorithmic}
Do napake $\varepsilon$ pridemo v
\[
k = \ceil{\log_2 \br{\frac{\varepsilon}{b-a}}}
\]
korakih.
\end{opomba}

\newpage

\subsection{Navadna iteracija}

\begin{izrek}
Naj bo $g \colon I = [\alpha - \delta, \alpha + \delta] \to \R$
skrčitev s faktorjem $m < 1$ in fiksno točko $\alpha$. Tedaj za
vsak $x_0 \in I$ zaporedje, podano z rekurzivno zvezo
$x_n = g(x_{n-1})$, konvergira k $\alpha$. Pri tem velja
\[
\abs{x_r - \alpha} \leq m^r \abs{x_0 - \alpha}
\quad \text{in} \quad
\abs{x_{r+1} - \alpha} \leq
\frac{m}{1-m} \cdot \abs{x_{r+1} - x_r}.
\]
\end{izrek}

\begin{proof}
Za dokaz konvergence in prve neenakosti glej dokaz izreka~7.4.2 v
skripti predmeta Analiza~1 v 1.~letniku. Velja
\begin{align*}
\abs{x_{r+1} - \alpha} &=
\abs{\sum_{n=r+1}^\infty x_n - x_{n+1}}
\\
&\leq
\sum_{n=r+1}^\infty \abs{x_n - x_{n+1}}
\\
&\leq
\sum_{n=1}^\infty m^n \abs{x_r - x_{r-1}}
\\
&=
\frac{m}{1-m} \cdot \abs{x_r - x_{r-1}}. \qedhere
\end{align*}
\end{proof}

\begin{posledica}
Naj bo $g \colon I \to \R$ zvezno odvedljiva funkcija s fiksno
točko $\alpha \in \Int I$. Če je $\abs{g'(\alpha)} < 1$, obstaja
tak $\delta > 0$, da za vsak
$x_0 \in [\alpha - \delta, \alpha + \delta]$ zaporedje, podano z
rekurzivno zvezo $x_n = g(x_{n-1})$, konvergira k $\alpha$.
\end{posledica}

\obvs

\begin{definicija}[Navadna iteracija]
\emph{Navadna iteracija}\index{Numerična metoda!Navadna iteracija}
je numerična metoda, ki rešitev enačbe $g(x) = x$ aproksimira z
uporabo začetnega približka $x_0$ po predpisu
$\widetilde{x} = g^n(x_0)$ za dovolj velik $n$.
\end{definicija}

\begin{definicija}
Naj bo $g \colon I \to \R$ funkcija s fiksno točko
$\alpha \in \Int I$, v kateri je zvezno odvedljiva. Pravimo, da je
točka $\alpha$ \emph{privlačna}\index{Privlačna, odbojna točka}, če
velja $\abs{g'(\alpha)} < 1$, in \emph{odbojna}, če velja
$\abs{g'(\alpha)} > 1$.
\end{definicija}

\begin{definicija}
Naj zaporedje $(x_n)_{n=1}^\infty$ konvergira proti $\alpha$.
Pravimo, da je $p$ \emph{red konvergence}\index{Red konvergence},
če obstaja taka konstanta $C > 0$, da je
\[
\lim_{n \to \infty}
\frac{\abs{x_{n+1} - \alpha}}{\abs{x_r - \alpha}^p} = C.
\]
\end{definicija}

\begin{opomba}
Pravimo, da je konvergenca linearna, če je $p = 1$, kvadratična, če
je $p = 2$, superlinearna, če je $1 < p < 2$ in podobno za večje
$p$. Če je red koncergence enak $p$, število točnih decimalnih mest
z iteracijo narašča asimptotsko kot $A^p$.
\end{opomba}

\begin{izrek}
Naj bo $g \colon I \to \R$ gladka na $I$ in $\alpha \in \Int I$
njena fiksna točka. Naj bo
\[
p = \min \setb{n \in \N}{g^{(n)}(\alpha) \ne 0}.
\]
Tedaj zaporedje, podano z rekurzivnim predpisom $x_n = g(x_{n-1})$
in začetnim členom v okolici $\alpha$, konvergira z redom
$p$.\footnote{Pri $p=1$ zahtevamo še $\abs{g'(\alpha)} < 1$.}
\end{izrek}

\begin{proof}
Po Taylorjevem izreku velja
\[
x_{n+1} - \alpha = \frac{g^{(p)}(\xi)}{p!} (x_n - \alpha)^p
\]
za nek $\xi$ med $\alpha$ in $x_n$. Sledi, da je
\[
\lim_{n \to \infty}
\frac{\abs{x_{n+1} - \alpha}}{\abs{x_n - \alpha}^p} =
\lim_{n \to \infty}
\frac{\abs{\frac{g^{(p)}(\xi)}{p!} (x_n - \alpha)^p}}{
\abs{x_n - \alpha}^p} =
\frac{\abs{g^{(p)}(\alpha)}}{p!}. \qedhere
\]
\end{proof}

\begin{definicija}
\emph{Tangentna metoda}\index{Numerična metoda!Tangentna} je
navadna iteracija za funkcijo
\[
g(x) = x - \frac{f(x)}{f'(x)}.
\]
\end{definicija}

\begin{opomba}
Če je $\alpha$ enostavna ničla za $f$, sledi
\[
g'(\alpha) = \frac{f(\alpha) f''(\alpha)}{f'(\alpha)^2} = 0,
\]
zato je red konvergence vsaj kvadratičen.\footnote{Če velja še
$f''(\alpha) = 0$, je red celo vsaj kubičen.} Če je $\alpha$ ničla
reda $m \geq 2$, pa se izkaže, da velja
\[
g'(\alpha) = 1 - \frac{1}{m},
\]
zato je konvergenca linearna.
\end{opomba}

\begin{izrek}
Naj bo $f \colon [a, \infty) \to \R$ dvakrat zvezno odvedljiva,
naraščajoča in konveksna funkcija z ničlo $\alpha \geq a$. Tedaj
je $\alpha$ edina ničla $f$ in za vsak $x_0 \geq a$ tangentna
metoda konvergira k $\alpha$.
\end{izrek}

\begin{proof}
Naj bo $e_n = x_n - \alpha$ za vsak $n \in \N$. Ker velja
\[
0 = f(\alpha) = f(x_n) + f'(x_n) (\alpha - x_n) +
\frac{f''(\xi_n)}{2} (\alpha - x_n)^2,
\]
sledi
\[
0 = \frac{f(x_n)}{f'(x_n)} + \alpha - x_n +
\frac{f''(\xi_n)}{2f'(x_n} (\alpha - x_n)^2,
\]
od koder lahko izrazimo
\[
e_{n+1} =
\alpha - x_n + \frac{f(x_n)}{f'(x_n)} =
\frac{f''(\xi_n)}{2 f'(x_n)} \cdot e_n^2.
\]
Tako sledi
\[
x_{n+1} - \alpha = \frac{f''(\xi_n)}{2 f'(x_n)} \cdot e_n^2 \geq 0,
\]
zato je $x_k \geq \alpha$ za vse $k \in \N$. Za $n \geq 1$ je tako
\[
x_{n+1} = x_n - \frac{f(x_n)}{f'(x_n)} \leq x_n,
\]
zato je zaporedje padajoče in posledično konvergentno, saj je
omejeno.
\end{proof}

\newpage

\subsection{Sekantne metode}

\begin{definicija}
\emph{Sekantna metoda}\index{Numerična metoda!Sekantna} je
numerična metoda, ki približek ničle funkcije $f$ aproksimira z
uporabo začetnih približkov $x_0$ in $x_1$ ter rekurzivne zveze
\[
x_{n+1} = x_n - \frac{f(x_n) (x_n - x_{n-1})}{f(x_n) - f(x_{n-1})}.
\]
\end{definicija}

\begin{opomba}
Izkaže se, da velja $e_{n+1} \approx c e_n e_{n-1}$, od koder
sledi, da je red konvergence enak $p = \frac{1 + \sqrt{5}}{2}$,
zato je konvergenca superlinearna.
\end{opomba}

\begin{definicija}
\emph{Mullerjeva metoda}\index{Numerična metoda!Mullerjeva} je
numerična metoda, pri kateri naslednji člen zaporedja izračunamo
kot ničlo kvadratnega polinoma, ki interpolira vrednosti v
prejšnjih treh točkah.
\end{definicija}

\begin{opomba}
Za red konvergence v tem primeru velja $p^3 = p^2 + p + 1$, oziroma
$p \approx 1{,}84$, zato je konvergenca superlinearna.
\end{opomba}

\begin{opomba}
Za razliko od prejšnjih metod lahko s to metodo za realne funkcije
najdemo kompleksne ničle tudi pri realnih začetnih približkih.
\end{opomba}

\begin{definicija}
\emph{Inverzna interpolacija}\index{Numerična metoda!Inverzna interpolacija}
je numerična metoda, pri kateri poiščemo kvadratni polinom $p$, ki
interpolira točke $(f(x_n), x_n)$, $(f(x_{n-1}), x_{n-1})$ in
$(f(x_{n-2}), x_{n-2})$, in naslednji člen izračunamo po predpisu
$x_{n+1} = p(0)$.
\end{definicija}

\begin{opomba}
Kot pri Mullerjevi metodi je tudi tu red konvergence enak
$p \approx 1{,}84$.
\end{opomba}

\begin{opomba}
Zgoraj naštete metode lahko med seboj seveda tudi kombiniramo.
\end{opomba}
