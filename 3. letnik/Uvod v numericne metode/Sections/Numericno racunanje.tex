\section{Računalniški zapis realnih števil}

\subsection{Predstavljiva števila}

\begin{definicija}
\emph{Numerična metoda}\index{Numerična metoda} je postopek, ki s
končnim številom elementarnih operacij izračuna približek iskane
numerične vrednosti.
\end{definicija}

\begin{definicija}
\emph{Množica predstavljivih števil}\index{Predstavljiva števila}
je množica
\[
P(b, t, L, U) =
\setb{\pm \sum_{i=1}^t c_i b^{e-i}}{
\forall i \colon 0 \leq c_i < b \land L \leq e \leq U}.
\]
Pravimo, da je število
\emph{normalizirano}\index{Število!Normalizirano}, če v tem zapisu
velja $c_1 \ne 0$, sicer pravimo, da je \emph{subnormalizirano}.
\end{definicija}

\begin{definicija}
Ob zgornjih oznakah številu $u = \frac{1}{2} b^{1-t}$ pravimo
\emph{osnovna zaokrožitvena napaka}.
\end{definicija}

\begin{izrek}
Če za realno število $x$ velja
\[
\min P(b, t, L, U) \leq \abs{x} \leq \max P(b, t, L, U),
\]
velja
\[
\min_{y \in P(b, t, L, U)} \frac{\abs{y-x}}{\abs{x}} \leq u.
\]
\end{izrek}

\begin{definicija}
Število $y \in P(b, t, L, U)$, za katerega je zgornja vrednost
najmanjša, označimo s $\fl(x)$.
\end{definicija}

\begin{opomba}
Za računske operacije po standardu IEEE za poljubno operacijo
$* \in \set{+, -, \cdot, \div}$ velja\footnote{Izjema so overflowi
in underflowi.}
\[
\fl(x*y) = (x*y) \cdot(1 + \delta)
\]
za nek $\delta \in [-u,u]$. Podobno velja za korenjenje.
\end{opomba}

\newpage

\subsection{Numerične napake}

\begin{definicija}
Naj bo $f \colon \R \to \R$ funkcija in $x \in \R$ število.

\begin{enumerate}[i)]
\item \emph{Neodstranljiva napaka}\index{Numerična napaka} pri
računanju vrednosti $f(x)$ je razlika $D_n = f(x) - f(\fl(x))$.

\item \emph{Napaka metode} je razlika
$D_m = f(\fl(x)) - \widetilde{f}(\fl(x))$, kjer je $\widetilde{f}$
numerična metoda za funkcijo $f$.

\item \emph{Zaokrožitvena napaka} je razlika
$D_z = \widetilde{f}(\fl(x)) - g(\fl{x})$, kjer je $g$ funkcija, ki
jo dobimo tako, da vsako operacijo v metodi $\widetilde{f}$
komponiramo s $\fl$.
\end{enumerate}
Skupno napako izračunamo kot $D = D_n + D_m + D_z$.
\end{definicija}

\begin{definicija}
Naj bo $f \colon \R^n \to \R^m$ preslikava.
\emph{Stopnja občutljivosti}\index{Preslikava!Stopnja občutljivosti}
preslikave $f$ v točki $a$ je limita\footnote{Ob predpostavki, da
obstaja.}
\[
\lim_{x \to 0} \frac{\norm{f(a+x) - f(a)}}{\norm{x}}.
\]
\end{definicija}

\begin{opomba}
Za odvedljive realne funkcije realne spremenljivke je stopnja
občutljivosti kar $\abs{f'(a)}$.
\end{opomba}

\begin{definicija}
Naj bo $f \colon \R^n \to \R^m$ preslikava in
$\widetilde{f} \colon \R^n \to \R^m$ numerična metoda za~$f$.

\begin{enumerate}[i)]
\item Metoda $\widetilde{f}$ je
\emph{direktno stabilna}\index{Stabilnost}, če je napaka
$\frac{\norm{\widetilde{f}(x) - f(x)}}{\norm{f(x)}}$
">majhna"<.\footnote{Omejena z dovolj majhnim številom.}
\item Metoda $\widetilde{f}$ je \emph{obratno stabilna}, če je
napaka
\[
\min_{f(y) = \widetilde{f}(x)} \frac{\norm{x-y}}{\norm{x}}
\]
">majhna"<.
\end{enumerate}
\end{definicija}

\begin{opomba}
Direktna napaka je načeloma manjša od obratne napake pomnožene z
občutljivostjo metode.
\end{opomba}

\begin{trditev}
Pri računanju produkta $p = \prod_{i=0}^n x_i$ za relativno
zaokrožitveno napako $\gamma$ velja $\abs{\gamma} \leq nu + o(u)$.
\end{trditev}

\begin{proof}
Produkt izračunamo z naslednjim algoritmom:
\begin{algorithmic}[1]
\State $p = x_0$
\For{$i=1,\dots,n$}
  \State $p \gets p \cdot x_i$
\EndFor
\end{algorithmic}
Za rezultat $\widetilde{p}$, ki ga na koncu dobimo, tako zaradi
zaokroževanja dobimo
\[
\widetilde{p} = p \cdot \prod_{i=1}^n (1+\delta_i),
\]
kjer je $\abs{\delta_i} \leq u$ za vsak $i$. Tako z Bernoullijevo
neenakostjo ocenimo
\[
1 - nu \leq (1-u)^n \leq 1 + \gamma \leq (1+u)^n = 1 + nu + o(u).
\qedhere
\]
\end{proof}

\begin{opomba}
Računanje produkta je direktno stabilno. Ni težko preveriti, da je
tudi obratno stabilno.
\end{opomba}

\begin{trditev}
Računanje skalarnega produkta je obratno, ne pa direktno stabilno.
\end{trditev}

\begin{proof}
Skalarni produkt vektorjev $x$ in $y$ izračunamo z naslednjim
algoritmom:
\begin{algorithmic}[1]
\State $s = 0$
\For{$i=1,\dots,n$}
  \State $s \gets s + x_i \cdot y_i$
\EndFor
\end{algorithmic}
Dejanski numerični rezultat je zaradi zaokroževanja enak
\[
\widetilde{s} = \sum_{i=1}^n x_i y_i (1 + \alpha_i) \cdot
\prod_{j=i}^n (1 + \beta_j),
\]
pri čemer je $\beta_1 = 0$. Označimo
\[
1 + \gamma_i = (1 + \alpha_i) \cdot \prod_{j=i}^n (1 + \beta_j).
\]
Tedaj je $\widetilde{s} = \skl{x, \widetilde{y}}$, kjer vzamemo
$\widetilde{y}_i = y_i \cdot (1 + \gamma_i)$. Metoda je torej
obratno stabilna, saj je $\abs{\gamma_i} \leq nu + o(u)$. Velja pa
\[
\frac{\abs{\widetilde{s} - s}}{\abs{s}} \leq
\frac{\skl{\abs{x}, \abs{y}}}{\abs{\skl{x,y}}} \cdot (nu + o(u)),
\]
kar ni omejeno navzgor. Če sta vektorja $x$ in $y$ skoraj
pravokotna, je lahko napaka zelo velika. Metoda ni direktno
stabilna.
\end{proof}
