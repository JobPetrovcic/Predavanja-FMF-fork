\section{Nelinearni sistemi}

\subsection{Navadna iteracija}

\begin{izrek}
\index{Numerična metoda!Navadna iteracija}
Naj bo $G \colon \R^n \to \R^n$ zvezno odvedljiva na
$\Omega \subseteq \R^n$. Če je $G(\Omega) \subseteq \Omega$ in
obstaja tak $m < 1$, da za vse $x \in \Omega$
velja\footnote{Tu $\rho(A)$ označuje \emph{spektralni radij}, za
matrike je to absolutna vrednost največje lastne vrednosti.}
\[
\rho(JG(x)) \leq m,
\]
ima $G$ v $\Omega$ natanko eno fiksno točko $\alpha$ in zaporedje
$x_n = G^n(x_0)$ konvergira k $\alpha$ za vsak
$x_0 \in \Omega$.\footnote{Tega dokaza ni v profesorjevih zapiskih,
za teoretični izpit ga ni potrebno znati.}
\end{izrek}

\begin{proof}
Naj bo $\varphi(t) = \norm{F(a + t(b-a)) - F(a)}$ za preslikavo
$F \colon \Omega \to \R^n$. Tedaj je
\begin{align*}
\varphi'(t)
&=
\frac{1}{\varphi(t)} \cdot
\sum_{i=1}^n \br{F_i(s) - F_i(a)}
\sum_{j=1}^n \br{b_j - a_j} \parc{F_i}{x_j}(s)
\\
&=
\frac{1}{\varphi(t)} \cdot \skl{F(s) - F(a), JF(s) \cdot (b-a)},
\end{align*}
kjer je $s = a + t(b-a)$. Po Lagrangeevem izreku tako obstaja tak
$c \in (0, 1)$, da je
\[
\varphi(1) - \varphi(0) = \varphi'(c).
\]
Za $p = a + c(b-a)$ tako dobimo
\[
\norm{F(b) - F(a)} =
\frac{1}{\norm{F(p) - F(a)}} \cdot
\skl{F(p) - F(a), JF(p) \cdot (b-a)} \leq
\norm{JF(p) \cdot (b-a)}.
\]
Posebej, za $F = G^n$ dobimo
\[
\norm{G^n(b) - G^n(a)} \leq
\norm{JG(p_n)^n \cdot (b-a)} \leq
\norm{JG(p_n)^n} \cdot \norm{b-a}.
\]
Naj bo $m < m_1 < m_2 < 1$. Za vsak $x$ obstaja tak $c(x)$, da za
vse $n \in \N$ velja\footnote{Glej dokaz leme 6.1.17 iz skripte
predmeta Uvod v funkcionalno analizo magistrskega študija.}
\[
\norm{JG(x)^n} < c(x) \cdot m_1^n.
\]
Sedaj naj bo
\[
A = \setb{a + t (b-a)}{t \in [0,1]}
\]
in
\[
U_c = \setb{x \in A}{\exists p < m_2 \colon \forall n \in \N \colon
\norm{JG(x)^n} < cp^n}.
\]
To je očitno pokritje množice $A$. Naj bo $x \in U_c$ in
$x_1$ v taki okolici $x$, da za matriko $M = JG(x_1) - JG(x)$ velja
$\norm{M} < m_2 - p$. Tedaj dobimo
\[
\norm{JG(x_1)^n} \leq
\sum_{i=0}^n \binom{n}{i} \norm{JG(x)^{n-i}} \norm{M^i} <
c \cdot \br{\norm{M} + m_2}^n.
\]
Sledi, da je $x_1 \in U_c$, zato je to pokritje odprto. Zaradi
kompaktnosti množice $A$ obstaja končno podpokritje. Ekvivalentno
obstaja tak $c$, da je
\[
\norm{JG(x)^n} < c m_2^n
\]
za vsak $x \in A$ in $n \in \N$. Za vse $p > q$ tako dobimo
\[
\norm{G^p(a) - G^q(a)} \leq
\sum_{k=q}^{p-1} \norm{G^{k+1}(a) - G^k(a)} <
c \cdot \norm{G(a) - a} \cdot
\frac{m_2^q \cdot (1 - m_2^{p-q})}{1-m_2}.
\]
Od tod ni težko videti, da je zaporedje $G^n(a)$ Cauchyjevo in zato
konvergentno. Limita je zaradi zveze
\[
\norm{G^n(b) - G^n(a)} < c m_2^n \norm{b - a}
\]
neodvisna od izbire začetne točke, za limito $\alpha$ zaporedja pa
velja še
\[
G(\alpha) = G \br{\lim_{n \to \infty} G^n(a)} =
\lim_{n \to \infty} G^{n+1}(a) = \alpha. \qedhere
\]
\end{proof}

\begin{opomba}
Če je $\alpha = G(\alpha)$ in $\rho(JG(\alpha)) < 1$, je $\alpha$
privlačna fiksna točka. Zadošča že, da velja
$\norm{JG(\alpha)} < 1$ za neko matrično normo.
\end{opomba}

\begin{definicija}
\emph{Seidelova iteracija}\index{Numerična metoda!Seidelova iteracija}
je numerična metoda, ki rešitev enačbe $G(x) = x$ aproksimira z
začetnim približkom $x^{(0)}$ in rekurzivnim predpisom
\[
x_i^{(r+1)} =
G_i(x_1^{(r+1)}, \dots, x_{i-1}^{(r+1)}, x_i^{(r)}, \dots).
\]
\end{definicija}

\begin{definicija}
\emph{Newtonova metoda}\index{Numerična metoda!Newtonova} je
navadna iteracija za preslikavo
\[
G(x) = x - JF(x)^{-1} \cdot F(x).
\]
\end{definicija}

\begin{opomba}
Metoda zagotovo konvergira, če je začetni približek dovolj dober.
Za korak iteracije moramo izračunati $\Delta x$ kot rešitev enačbe
$JF(x) \cdot \Delta x = -F(x)$, za kar potrebujemo $O(n^3)$
operacij.
\end{opomba}

\begin{lema}
Matrika $A$, za katero je $Ax = y \ne 0$ in ima najmanjšo normo, je
oblike
\[
A = \frac{y x^\top}{\norm{x}^2}.
\]
\end{lema}

\begin{proof}
Očitno velja $Ax = y$ in $\norm{A} \geq \frac{\norm{y}}{\norm{x}}$.
Ker je
\[
\norm{yx^\top z} =
\norm{y} \cdot \abs{x^\top z} \leq
\norm{y} \cdot \norm{x} \cdot \norm{z}
\]
z enakostjo, ko je $z = x$, sledi
\[
\norm{\frac{y x^\top}{\norm{x}^2}} = \frac{\norm{y}}{\norm{x}}.
\qedhere
\]
\end{proof}

\begin{definicija}
\emph{Broydenova metoda}\index{Numerična metoda!Broydenova} je
metoda, pri kateri $\Delta x$ izračunamo po predpisu
\[
B_r \Delta x = -F(x).
\]
Pri tem za $B_r$ vzamemo matriko, za katero je
\[
B_{r+1} \cdot \br{x^{(r+1)} - x^{(r)}} =
F \br{x^{(r+1)}} - F \br{x^{(r)})}
\]
in je najbližje matriki $B_r$.
\end{definicija}

\begin{opomba}
Velja
\[
B_{r+1} = B_r + \frac{F \br{x^{(r+1)}} \br{\Delta x^{(r)}}^\top}
{\norm{\Delta x^{(r)}}^2}.
\]
\end{opomba}

\begin{opomba}
Za reševanja sistema z $B_{r+1}$ si pomagamo s Sherman-Morrisonovo
formulo
\[
\br{A + uv^\top}^{-1} =
A^{-1} + \frac{A^{-1} u v^\top A^{-1}}{1 + \skl{v, A^{-1} u}}.
\]
\end{opomba}
