\section{Polinomska interpolacija}

\subsection{Lagrangeeva interpolacija}

\begin{izrek}
Za paroma različne točke $x_0, \dots, x_n$ in vrednosti
$y_0, \dots, y_n$ obstaja natanko en polinom $p$, za katerega je
$\deg p \leq n$ in $p(x_i) = y_i$ za vse $i$.
\end{izrek}

\begin{proof}
Če sta $p$ in $q$ dva taka polinoma, ima njuna razlika stopnjo
največ $n$, ima pa $n+1$ ničel. Sledi, da imamo kvečjemu en tak
polinom. Ni pa težko videti, da
\[
p(x) =
\sum_{i=0}^n y_i \cdot \prod_{j \ne i} \frac{x - x_j}{x_i - x_j}
\]
zadošča danim pogojem.
\end{proof}

\begin{izrek}
Naj bo $f \in \mathcal{C}^{n+1}([a,b])$ funkcija in
$x_0, \dots, x_n \in [a,b]$ paroma različne točke. Naj bo $p$
interpolacijski polinom za $f$ in točke $x_i$. Tedaj za vsak
$x \in [a,b]$ obstaja tak $\xi$, da je
\[
\min \br{\set{x} \cup \setb{x_i}{0 \leq i \leq n}} <
\xi <
\max \br{\set{x} \cup \setb{x_i}{0 \leq i \leq n}}
\]
in je
\[
f(x) - p(x) =
\frac{f^{(n+1)}(\xi)}{(n+1)!} \prod_{i=0}^n (x - x_i).
\]
\end{izrek}

\begin{proof}
Brez škode za splošnost naj bo $x \ne x_i$ za vsak $i$. Definirajmo
\[
w(t) = \prod_{i=0}^n (t - x_i)
\]
in
\[
g(t) = f(t) - p(t) - \frac{f(x) - p(x)}{w(x)} w(t).
\]
Tako velja $g(x) = 0$ in $g \in \mathcal{C}^{n+1}([a,b])$. Po
Rollovem izreku ima $g^{(n+1)}$ vsaj eno ničlo $\xi$ na želenem
intervalu. Tako dobimo
\[
0 = f^{(n+1)}(\xi) - \frac{f(x) - p(x)}{w(x)} (n+1)!. \qedhere
\]
\end{proof}

\newpage

\subsection{Deljene diference}

\begin{definicija}
Za paroma različne točke $x_0, \dots, x_k$ in funkcijo $f$ je
\emph{deljena diferenca}\index{Deljena diferenca}
$[x_0, \dots, x_k] f$ koeficient pred $x^k$ pripadajočega
interpolacijskega polinoma. Če se katera izmed točk ponovi, tam
interpoliramo tudi vrednosti odvodov.
\end{definicija}

\begin{izrek}
Za paroma različne točke $x_0, \dots, x_n$ velja
\[
p(x) = \sum_{i=0}^n
\br{[x_0, \dots, x_i]f \cdot \prod_{j=0}^{i-1} (x - x_j)}.
\]
\end{izrek}

\begin{proof}
Trditev dokažemo z indukcijo, baza je trivialna. Velja
\[
p_{n+1}(x) = p_n(x) + c \prod_{i=0}^n (x - x_i),
\]
kjer je
\[
c =
\frac{f^{(k)}(x_{n+1}) - p_n^{(k)}(x_{n+1})}
{\br{\displaystyle \prod_{i=0}^n (x_{n+1} - x_i)}^{(k)}}.
\]
Očitno je $c$ vodilni koeficient $p_{n+1}$.
\end{proof}

\begin{izrek}
Veljajo naslednje trditve:

\begin{enumerate}[i)]
\item Deljena diferenca $[x_0, \dots, x_n] f$ je simetrična
funkcija.
\item Deljena diferenca je linearen funkcional.
\item Če je $x_0 \ne x_n$, velja
\[
[x_0, \dots, x_n] f =
\frac{[x_1, \dots, x_n] f - [x_0, \dots, x_{n-1}] f}{x_n - x_0}
\]
\end{enumerate}
\end{izrek}

\begin{proof}
Simetričnost in linearnost sta očitni. Če je $q$ interpolacijski
polinom za točke $x_0, \dots, x_{n-1}$ in $r$ interpolacijski
polinom za $x_1, \dots, x_n$, velja
\[
p(x) =
\frac{x - x_n}{x_0 - x_n} q(x) + \frac{x - x_0}{x_n - x_0} r(x).
\]
S primerjavo koeficientov dobimo rekurzivno formulo.
\end{proof}

\begin{opomba}
Če je $x_i = x_0$ za vsak $i$, deljeno diferenco namesto z
rekurzivno formulo izračunamo kot
\[
[x,\dots,x] f = \frac{f^{(n)}(x)}{n!}.
\]
\end{opomba}

\begin{izrek}
Za vsako funkcijo $f \in \mathcal{C}^k([a,b])$ velja
\[
[x_0, \dots, x_k] f =
\int_0^1 \int_0^{t_1} \dots \int_0^{t_{k-1}}
f^{(k)} \br{x_0 + \sum_{i=1}^k t_i (x_i - x_{i-1})}
\,dt_k \,dt_{k-1}\dots\,dt_1
\]
\end{izrek}

\begin{proof}
Izrek očitno velja za $k = 0$. Najprej predpostavimo, da je
$x_k \ne x_{k-1}$. Tedaj sledi
\[
\int_0^{t_{k-1}} f^{(k)}
\br{X + t_k (x_k - x_{k-1})}\,dt_k
\\
=
\frac{f^{(k-1)} \br{X + t_{k-1} (x_k - x_{k-1})} -
f^{(k-1)} \br{X}}{x_k - x_{k-1}},
\]
kjer je
\[
X = x_0 + \sum_{i=1}^{k-1} t_i (x_i - x_{i-1}).
\]
Po indukcijski predpostavki je ta izraz enak
\[
\frac{[x_0, \dots, x_{k-2}, x_k] f -
[x_0, \dots, x_{k-1}] f}{x_k - x_{k-1}} =
[x_0, \dots, x_k] f.
\]
Če je $x_i = x_0$ za vsak $i$, pa dobimo
\[
\int_0^1 \int_0^{t_1} \dots \int_0^{t_{k-1}}
f^{(k)} \br{x_0} \,dt_k \,dt_{k-1}\dots\,dt_1 =
\frac{f^{(k)}(x_0)}{k!}. \qedhere
\]
\end{proof}

\begin{posledica}
Za vsako funkcijo $f \in \mathcal{C}^k([a,b])$ in točke
$x_0, \dots, x_n$ velja
\[
[x_0, \dots, x_n] f = \frac{f^{(k)}(\xi)}{k!}
\]
za $\xi \in \br{\min \setb{x_i}{0 \leq i \leq n},
\max \setb{x_i}{0 \leq i \leq n}}$.
\end{posledica}

\begin{proof}
Volumen množice, po kateri integriramo, je enak $\frac{1}{k!}$.
Izrek tako sledi iz izreka o vmesni vrednosti in zveznosti funkcije
$f^{(k)}$.
\end{proof}

\begin{izrek}
Za funkcijo $f$ in interpolacijski polinom $p$ na točkah
$x_0, \dots, x_n$ velja
\[
f(x) = p(x) + [x_0, \dots, x_n, x] f \cdot \prod_{i=0}^n (x - x_i).
\]
\end{izrek}

\begin{proof}
Desna stran enačbe je interpolacijski polinom za $f$ na točkah
$x_0, \dots, x_n, x$.
\end{proof}

\begin{posledica}
Za $f \in \mathcal{C}^{n+1}([a,b])$ naj bo $p$ njen interpolacijski
polinom za točke $x_0, \dots, x_n$.\footnote{Ne nujno različne.}
Tedaj za vsak $x \in [a,b]$ obstaja tak $\xi$, da velja
\[
\min \br{\set{x} \cup \setb{x_i}{0 \leq i \leq n}} \leq
\xi \leq
\max \br{\set{x} \cup \setb{x_i}{0 \leq i \leq n}}
\]
in
\[
f(x) - p(x) =
\frac{f^{(n+1)}(\xi)}{(n+1)!} \prod_{i=0}^n (x - x_i).
\]
\end{posledica}

\obvs

\begin{opomba}
To je pravzaprav posplošitev Taylorjevega izreka.
\end{opomba}

\begin{opomba}
Boljšo aproksimacijo funkcije lahko poskusimo dobiti tako, da
povečamo število interpolacijskih točk. Včasih je pri tem namesto
ekvidistantnih smiselno vzeti Čebiševe točke
\[
x_i = \lambda \cos \br{\frac{j \pi}{n}} + \mu.
\]
\end{opomba}

\begin{opomba}
Kljub večjemu številu točk polinomi pogosto ne konvergirajo proti
$f$. V takem primeru je bolj smiselno $f$ aproksimirati odsekoma.
Na odsekih lahko vzamemo linearne funkcije, lahko pa vzamemo
polinome tretje stopnje in dosežemo, da je dobljena funkcija zvezno
odvedljiva.
\end{opomba}
