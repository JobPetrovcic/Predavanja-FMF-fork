\section{Slučajne spremenljivke}

\subsection{Diskretne slučajne spremenljivke}

\datum{2022-10-19}

\begin{definicija}
\emph{Slučajna spremenljivka}\index{Slučajna spremenljivka} $X$ je
taka funkcija $X \colon \Omega \to \R$, da je $X^{-1}((a,b])$
dogodek v $\mathcal{F}$ za vsak interval $(a,b]$.
\end{definicija}

\begin{opomba}
Iz definicij sledi, da so dogodki tudi praslike unij intervalov.
Takim množicam pravimo \emph{Borelove množice}.
\end{opomba}

\begin{definicija}
Slučajna spremenljivka $X$ je
\emph{diskretna}\index{Slučajna spremenljivka!Diskretna}, če je
zaloga vrednosti diskretna množica.
\end{definicija}

\begin{opomba}
Za diskretne slučajne spremenljivke označimo
\[
P(X=k) = P(X^{-1}(\set{k})).
\]
\end{opomba}

\begin{definicija}
\emph{Porazdelitev slučajne spremenljivke}\index{Slučajna spremenljivka!Porazdelitev}
$X$ je dana z naborom verjetnosti $P(X^{-1}((a,b]))$ za vse $a<b$.
\end{definicija}

\begin{definicija}
Slučajna spremenljivka $X$ ima
\emph{hiper-geometrijsko porazdelitev}\index{Porazdelitev!Hiper-geometrijska}
s parametri $B$, $R$ in $n$, če velja
\[
P(X=k) = \frac{\binom{B}{k} \cdot \binom{R}{n-k}}{\binom{B+R}{n}}.
\]
\end{definicija}

\begin{opomba}
Verjetnosti $P(X=k)$ je največja pri
\[
k = \floor{\frac{(B+1)(n+1)}{N+2}}.
\]
Če je zgornji ulomek celo število, je maksimum dosežen tudi pri
$k-1$.
\end{opomba}

\datum{2022-10-25}

\begin{definicija}
Slučajna spremenljivka $X$ ima
\emph{binomsko porazdelitev}\index{Porazdelitev!Binomska} s
parametroma $n$ in $p$, če velja
\[
P(X=k) = \binom{n}{k} \cdot p^k \cdot (1-p)^{n-k}.
\]
\end{definicija}

\begin{opomba}
Verjetnosti $P(X=k)$ je največja pri
\[
k = \floor{(n+1)p}.
\]
Če je $(n+1)p$ celo število, je maksimum dosežen tudi pri $k-1$.
\end{opomba}

\begin{definicija}
Slučajna spremenljivka $X$ ima
\emph{geometrijsko porazdelitev}\index{Porazdelitev!Geometrijska} s
parametrom $p$, če velja
\[
P(X=k) = p \cdot (1-p)^{k-1}.
\]
\end{definicija}

\begin{definicija}
Slučajna spremenljivka $X$ ima
\emph{negativno binomsko porazdelitev}\index{Porazdelitev!Negativna binomska}
s parametroma $m$ in $p$, če velja
\[
P(X=k) = \binom{k-1}{m-1} p^m (1-p)^{k-m}.
\]
\end{definicija}
