\section{Moduli}

\epigraph{"Moj tinitus samo slabši postaja."<}{-- prof.~dr.~Primož
Moravec}

\subsection{Definicija}

\begin{definicija}
Naj bo $M$ neprazna množica in $R$ kolobar. Množica $M$ z
operacijama $+ \colon M \times M \to M$ in
$\cdot \colon R \times M \to M$ je \emph{$R$-modul}\index{Modul},
če velja naslednje:

\begin{enumerate}[i)]
\item $(M, +)$ je abelova grupa.
\item Za vse $r \in R$ in $m_1, m_2 \in M$ velja
$r (m_1 + m_2) = r m_1 + r m_2$.
\item Za vse $r_1, r_2 \in R$ in $m \in M$ velja
$(r_1 + r_2) m = r_1 m + r_2 m$.
\item Za vse $r_1, r_2 \in R$ in $m \in M$ velja
$r_1 (r_2 m) = (r_1 r_2) m$.
\item Za vse $m \in M$ je $1 \cdot m = m$.
\end{enumerate}
\end{definicija}

\begin{definicija}
Naj bo $M$ $R$-modul. Neprazna množica $N \subseteq M$ je
\emph{podmodul}\index{Modul!Podmodul} v $M$, če je $R$-modul z
induciranimi preslikavami.
\end{definicija}

\datum{2023-3-17}

\begin{definicija}
Naj bo $M$ $R$-modul in $X \subseteq M$. Najmanjši podmodul v $M$,
ki vsebuje $X$, označimo z $\skl{X}$.
\end{definicija}

\begin{trditev}
Velja
\[
\skl{X} = \setb{\sum_{i=1}^n r_i x_i}
{n \in \N \land r_i \in R \land x_i \in X}.
\]
\end{trditev}

\obvs

\begin{definicija}
Podmodul $N$ modula $M$ je
\emph{končnogeneriran}\index{Modul!Končnogeneriran}, če obstaja
končna množica $X \subseteq M$, za katero je $\skl{X} = N$.
\end{definicija}

\begin{definicija}
Podmodul $N$ modula $M$ je \emph{cikličen}\index{Modul!Cikličen},
če za nek $x \in M$ velja $N = \skl{\set{x}} = \skl{x}$.
\end{definicija}

\newpage

\subsection{Homomorfizmi modulov}

\begin{definicija}
Naj bosta $M$ in $M'$ $R$-modula. Preslikava
$\varphi \colon M \to M'$ je
\emph{homomorfizem modulov}\index{Modul!Homomorfizem}, če za vse
$m_1, m_2 \in M$ in $r \in R$ velja
\[
\varphi(m_1 + m_2) = \varphi(m_1) + \varphi(m_2)
\quad \text{in} \quad
\varphi(r m_1) = r \varphi(m_1).
\]
\end{definicija}

\begin{definicija}
Naj bo $\varphi \colon M \to M'$ homomorfizem modulov. Označimo
\[
\ker \varphi = \setb{x \in M}{\varphi(x) = 0}
\]
in
\[
\im \varphi = \setb{\varphi(x)}{x \in M}.
\]
\end{definicija}

\begin{definicija}
Naj bo $\varphi \colon M \to M'$ homomorfizem modulov. Pravimo, da
je $\varphi$

\begin{enumerate}[i)]
\item \emph{endomorfizem}, če je $M = M'$,
\item \emph{monomorfizem}, če je injektiven,
\item \emph{epimorfizem}, če je surjektiven,
\item \emph{izomorfizem}, če je bijektiven.
\end{enumerate}

Če obstaja izomorfizem $\varphi \colon M \to M'$, pravimo, da sta
modula $M$ in $M'$ izomorfna, oziroma $M \cong M'$.
\end{definicija}

\begin{definicija}
Naj bo $M$ $R$-modul in $N$ njegov podmodul.
\emph{Kvocientni modul}\index{Modul!Kvocientni} je definiran kot
\[
\kvoc{M}{N} = \setb{m+N}{m \in M}
\]
z naravno definiranim seštevanjem in množenjem.
\end{definicija}

\begin{opomba}
Kvocientni modul je spet $R$-modul.
\end{opomba}

\begin{definicija}
Preslikavi $\pi \colon M \to \kvoc{M}{N}$, podani s predpisom
$\pi(m) = m + N$, pravimo
\emph{kanonični epimorfizem}\index{Modul!Kanonični epimorfizem}.
\end{definicija}

\begin{opomba}
Kanonični epimorfizem je seveda epimorfizem.
\end{opomba}

\newpage

\subsection{Izreki o izomorfizmih}

\begin{trditev}
Naj bo $\varphi \colon M \to N$ homomorfizem $R$-modulov in
$L \leq M$. Če je $L \leq \ker \varphi$, $\varphi$ inducira
homomorfizem $\tilde{\varphi} \colon \kvoc{M}{L} \to N$ s predpisom
$\tilde{\varphi}(m + L) = \varphi(m)$. Pri tem velja
$\im \tilde{\varphi} = \im \varphi$ in
$\ker \tilde{\varphi} = \kvoc{\ker \varphi}{L}$.
\end{trditev}

\begin{proof}
Če je $m_1 + L = m_2 + L$, je očitno $\varphi(m_1) = \varphi(m_2)$,
zato je preslikava dobro definirana. Ni težko videti, da je
preslikava homomorfizem z zgoraj naštetima jedrom in sliko.
\end{proof}

\begin{izrek}[O izomorfizmu]\index{Izrek!O izomorfizmu}
Naj bo $\varphi \colon M \to N$ homomorfizem $R$-modulov. Tedaj je
$\kvoc{M}{\ker \varphi} \cong \im \varphi$, $\varphi$ pa ima
epi-mono razcep
\[
\begin{tikzcd}[column sep=large, row sep=large]
M \arrow[r, "\varphi"] \arrow[d, "\pi"'] &
N \\
\kvoc{M}{\ker \varphi} \arrow[r, "\cong", "\varphi"'] &
\im \varphi \arrow[u, hook, "i"']
\end{tikzcd}
\]
\end{izrek}

\begin{proof}
V prejšnji trditvi vzamemo $L = \ker \varphi$.
\end{proof}

\begin{izrek}[Noether]\index{Izrek!Noether}
Naj bodo $M$, $N$ in $L$ $R$-moduli.

\begin{enumerate}[i)]
\item Če je $M \leq N \leq L$, velja
$\kvoc{N}{M} \leq \kvoc{L}{M}$ in
\[
\kvoc{\kvoc{L}{M}}{\kvoc{N}{M}} \cong \kvoc{L}{N}.
\]
\item Če je $M, N \leq L$, velja
\[
\kvoc{M}{M \cap N} \cong \kvoc{M+N}{N}
\]
\item Če je $M \leq L$, so podmoduli v $L$, ki vsebujejo $M$, v
bijektivni korespondenci s podmoduli v $\kvoc{L}{M}$.
\end{enumerate}
\end{izrek}

\begin{proof}
Preslikavi
$\varphi \colon \kvoc{\kvoc{L}{M}}{\kvoc{N}{M}} \to \kvoc{L}{N}$ in
$\psi \colon \kvoc{M}{M \cap N} \to \kvoc{M+N}{N}$ s predpisoma
\[
\varphi \br{(x + M) + \kvoc{N}{M}} = x + N
\]
in
\[
\psi(x + (M \cap N)) = x + N
\]
sta izomorfizma.
\end{proof}

\newpage

\subsection{Direktna vsota modulov}

\begin{definicija}
Naj bodo $M_1, \dots, M_s$ $R$-moduli. Množici
\[
\prod_{i=1}^s M_i
\]
s seštevanjem in množenjem po komponentah pravimo
\emph{direktna vsota}\index{Modul!Direktna vsota} modulov in jo
označimo z
\[
\bigoplus_{i=1}^s M_i.
\]
\end{definicija}

\begin{definicija}
$R$-modul $M$ je \emph{notranja direktna vsota} podmodulov
$N_1, \dots, N_2$, če je abelova grupa $(M, +)$ notranja direktna
vsota grup $(N_i, +)$.
\end{definicija}

\begin{trditev}
Naj bo $M$ $R$-modul in $N_1, \dots, N_s$ njegovi podmoduli. Modul
$M$ je direktna vsota $N_1, \dots, N_s$ natanko tedaj, ko se da
vsak $m \in M$ na enoličen način zapisati kot
\[
m = \sum_{i=1}^s n_i,
\]
pri čemer je $n_i \in N_i$ za vsak $i$.
\end{trditev}

\obvs

\begin{trditev}
Če je $M$ notranja direktna vsota podmodulov $N_1, \dots, N_s$, je
izomorfen njihovi direktni vsoti. Zunanja direktna vsota modulov
$N_1, \dots, N_s$ je izomorfna notranji direktni vsoti podmodulov
\[
\prod_{i=1}^{k-1} \set{0} \times
N_k \times \prod_{i=k+1}^s \set{0} \leq \prod_{i=1}^s N_i.
\]
\end{trditev}

\obvs

\begin{opomba}
Lahko definiramo tudi direktno vsoto neskončno mnogo modulov, a pri
tem zahtevamo, da je kvečjemu končno komponent neničelnih.
\end{opomba}

\newpage

\subsection{Prosti moduli}

\begin{definicija}
Naj bo $M$ $R$-modul. Množici $X \subseteq M$ pravimo
\emph{baza}\index{Modul!Baza} za $M$, če velja

\begin{enumerate}[i)]
\item $\skl{X} = M$ in
\item za vse $n \in \N$, $\lambda_i \in R$ in $x_i \in X$ iz
enakosti
\[
\sum_{i=1}^n \lambda_i x_i = 0
\]
sledi $\lambda_i = 0$ za vse $i$.
\end{enumerate}
\end{definicija}

\begin{definicija}
$R$-modul $M$ je \emph{prost}\index{Modul!Prost}, če ima bazo.
\end{definicija}

\begin{izrek}
Za $R$-modul $M$ so ekvivalentne naslednje trditve:

\begin{enumerate}[i)]
\item $M$ je prost.
\item $M$ je izomorfen direktni vsoti kopij $R$-modula $R$.
\item Velja
\[
M = \bigoplus_{\lambda \in \Lambda} M_\lambda
\]
za $M_\lambda \subseteq M$ in $M_\lambda \cong R$.
\end{enumerate}
\end{izrek}

\begin{proof}
Drugi točki sta očitno ekvivalentni. Naj bo $M$ prost $R$-modul.
Tedaj velja
\[
M = \bigoplus_{x \in X} \skl{x},
\]
kjer je $X$ baza $M$, očitno pa je $R \cong \skl{x}$, saj je
$r \mapsto r \cdot x$ izomorfizem.

Denimo sedaj, da obstaja izomorfizem
\[
f \colon \bigoplus_{\lambda \in \Lambda} R \to M.
\]
Tedaj je
\[
X = \setb{f(e_\lambda)}{\lambda \in \Lambda}
\]
očitno baza za $M$.
\end{proof}

\begin{posledica}
Vsak $R$-modul je kvocient prostega $R$-modula.
\end{posledica}

\begin{proof}
Naj bo $M$ $R$-modul. Tedaj je preslikava
\[
f \colon \bigoplus_{m \in M} R \to M
\]
s predpisom $f(e_m) = m$ epimorfizem $R$-modulov.
\end{proof}

\begin{opomba}
Če je vsak $R$-modul prost, je $R$ obseg.
\end{opomba}

\begin{opomba}
Podmodul prostega $R$-modula ni nujno prost.
\end{opomba}

\begin{trditev}
Naj bo $D$ obseg in $M$ $D$-modul. Tedaj veljajo naslednje trditve:

\begin{enumerate}[i)]
\item $M$ je prost.
\item Iz vsakega ogrodja $M$ lahko izberemo bazo.
\item Vsako linearno neodvisno množico lahko razširimo do baze.
\item Poljubni bazi imata enako kardinalnost $\dim_D M$.
\item Za vsak $N \leq M$ velja
$\dim_D M = \dim_D N + \dim_D \kvoc{M}{N}$.
\item Za vsak homomorfizem $\varphi \colon M \to N$ je
$\dim_D M = \dim_D \ker \varphi + \dim_D \im \varphi$.
\end{enumerate}
\end{trditev}

\begin{proof}
Enak kot za vektorske prostore.
\end{proof}

\begin{trditev}[Univerzalna lastnost prostih modulov]
\index{Modul!Prost!Univerzalna lastnost}
$R$-modul $M$ je prost natanko tedaj, ko obstaja neprazna množica
$X$ in preslikava $\iota \colon X \to M$, za katera za vsak
$R$-modul $N$ in preslikavo $f \colon X \to N$ obstaja natanko en
homomorfizem $g \colon M \to N$, za katerega je
$f = g \circ \iota$.
\[
\begin{tikzcd}[column sep=large, row sep=large]
X \arrow[r, "f"] \arrow[d, "\iota"'] & N
\\
M \arrow[ur, dashrightarrow, "g"']
\end{tikzcd}
\]
\end{trditev}

\begin{proof}
Naj bo $X$ baza za $M$ in $\iota \colon X \hookrightarrow M$
vložitev. Naj bo $N$ poljuben $R$-modul in $f \colon X \to N$
prelikava. Zdaj za vse $x_i \in X$ in $r_i \in R$ definiramo
\[
g \br{\sum_{i=1}^n r_i x_i} = \sum_{i=1}^n r_i f(x_i).
\]
Očitno je to homomorfizem. Ni težko videti, da je edini, ki zadošča
$f = g \circ \iota$.

Sedaj predpostavimo, da za $M$ velja univerzalna lastnost. Naj bo
\[
N = \bigoplus_{x \in X} R
\]
in $f \colon X \to N$ preslikava, ki deluje po predpisu
$f(x) = e_x$. Sledi, da obstaja enolična preslikava
$g \colon M \to N$, za katero je $f = g \circ \iota$. Ker je $N$
prost, po univerzalni lastnosti obstaja enolična preslikava
$h \colon N \to M$, za katero je $\iota = h \circ f$. Sedaj
opazimo, da je
\[
(g \circ h) \circ f = g \circ \iota = f = \id \circ f
\]
in
\[
(h \circ g) \circ \iota = h \circ f = \iota = \id \circ \iota.
\]
Iz enoličnosti tako sledi $g \circ h = \id$ in $h \circ g = \id$,
zato je $M \cong N$.
\end{proof}

\begin{definicija}
Kolobar $R$ ima
\emph{lastnost enoličnega ranga}\index{Kolobar!Enoličen rang}, če
ima za vsak prost $R$-modul vsaka njegova baza enako moč. Moč baze
označimo z $\rang M$.
\end{definicija}

\begin{definicija}
Naj bo $M$ $R$-modul in $I \edn R$. Tedaj definiramo
\[
I \cdot M = \setb{\sum_{i=1}^n u_i m_i}
{n \in \N \land u_i \in I \land m_i \in M}.
\]
\end{definicija}

\begin{opomba}
Velja $IM \leq M$.
\end{opomba}

\begin{lema}
Če je $M$ prost $R$-modul z bazo $X$, je $\kvoc{M}{IM}$ prost
$\kvoc{R}{I}$-modul z bazo $X + IM$.
\end{lema}

\begin{proof}
Očitno je $\skl{X + IM} = \kvoc{M}{IM}$. Denimo, da je
\[
\sum_{i=1}^n (r_i + I)(x_i + IM) = 0.
\]
Ekvivalentno, velja
\[
\sum	_{i=1}^n r_i x_i \in IM,
\]
zato je
\[
\sum_{i=1}^n r_i x_i = \sum_{i=1}^m u_i \tilde{x}_i,
\]
pri čemer so $u_i \in I$. Ker je $X$ baza $M$, sledi $r_i \in I$ za
vse $i$.
\end{proof}

\begin{trditev}
Naj bo $X$ baza $R$-modula $M$. Če je $I$ pravi ideal v $R$, je
$\abs{X + IM} = \abs{X}$.
\end{trditev}

\begin{proof}
Denimo, da je $x + IM = y + IM$ za $x, y \in X$. Tedaj je
\[
1 \cdot x - 1 \cdot y = \sum_{i=1}^n u_i x_i,
\]
zato je $x = y$.
\end{proof}

\begin{izrek}
Veljata naslednji trditvi:

\begin{enumerate}[i)]
\item Če ima kakšen netrivialen kvocient kolobarja $R$ lastnost
enoličnega ranga, jo ima tudi $R$.
\item Vsi komutativni kolobarji imajo lastnost enoličnega ranga.
\end{enumerate}
\end{izrek}

\begin{proof}
\phantom{a}
\begin{enumerate}[i)]
\item Naj bo $I \edn R$ pravi ideal v $R$, za katerega ima
$\kvoc{R}{I}$ lastnost enoličnega ranga. Tedaj za prost $M$-modul z
bazama $X$ in $Y$ velja
\[
\abs{X} = \abs{X + IM} = \abs{Y + IM} = \abs{Y}.
\]
\item Naj bo $R$ komutativen kolobar. Predpostavimo, da $R$ ni
polje, saj ta imajo lastnost enoličnega ranga. Naj bo $I \edn R$
pravi ideal. Po Zornovi lemi obstaja maksimalni ideal $J \edn R$,
ki vsebuje $I$. Tedaj je $\kvoc{R}{J}$ polje, saj nima pravih
netrivialnih idealov. Po prvi točki ima $R$ lastnost enoličnega
ranga. \qedhere
\end{enumerate}
\end{proof}

\newpage

\subsection{Projektivni moduli}

\datum{2023-3-24}

\begin{definicija}
Naj bo $R$ kolobar. $R$-modul $P$ je
\emph{projektiven}\index{Modul!Projektiven}, če za vsak
homomorfizem $R$-modulov $f \colon P \to N$ in epimorfizem
$g \colon M \to N$ obstaja homomorfizem $h \colon P \to M$, za
katerega je $g \circ h = f$.
\[
\begin{tikzcd}[column sep=large, row sep=large]
& P \arrow[dl, dashrightarrow, "h"] \arrow[d, "f"] \\
M \arrow[r, two heads, "g"'] & M'
\end{tikzcd}
\]
\end{definicija}

\begin{trditev}
Vsak prost $R$-modul $F$ je projektiven.
\end{trditev}

\begin{proof}
Naj bo $f \colon F \to M'$ homomorfizem in $g \colon M \to M'$
epimorfizem. Za vsak $x \in X$ velja $f \circ \iota(x) \in M'$,
zato obstaja tak $m_x \in M$, da je $f \circ \iota(x) = g(m_x)$.
Sedaj definiramo preslikavo $\tau \colon X \to M$ s predpisom
$\tau(x) = m_x$. Po univerzalni lastnosti prostih modulov obstaja
homomorfizem $h \colon F \to M$, za katerega je
$h \circ \iota = \tau$. Ker velja
\[
(g \circ h) \circ \iota = g \circ \tau = f \circ \iota,
\]
po univerzalni lastnosti sledi $g \circ h = f$.
\end{proof}

\begin{izrek}
Za $R$-modul $P$ so ekvivalentne naslednje trditve:

\begin{enumerate}[i)]
\item $P$ je projektiven.
\item Za vsak epimorfizem $\varphi \colon M' \to P$ je
$M' \cong P \oplus \ker \varphi$.
\item Obstaja $R$-modul $M$, za katerega je $P \oplus M$ prost
$R$-modul.
\end{enumerate}
\end{izrek}

\begin{proof}
Denimo, da je $P$ projektiven in $\varphi \colon M' \to P$ poljuben
epimorfizem. Zaradi projektivnosti $P$ obstaja tak homomorfizem
$\psi \colon P \to M'$, da je $\varphi \circ \psi = \id$. Tako
sledi, da je $\psi$ injektiven in zato $\im \psi \cong P$. Ni težko
preveriti, da je $\im \psi \cap \ker \varphi = \set{0}$. Ker za
vsak $m \in M'$ velja $m - \psi \circ \varphi(m) \in \ker \varphi$
in lahko zapišemo
\[
m = (m - \psi \circ \varphi(m)) + \psi \circ \varphi(m),
\]
je $M' = \im \psi \oplus \ker \varphi$.

Sedaj predpostavimo, da velja druga točka. Ker je $P$ kvocient
nekega prostega $R$-modula $F$, velja
\[
F \cong P \oplus \ker \pi.
\]

Denimo sedaj, da je $P \oplus N$ prost $R$-modul in dokažimo, da je
$P$ projektiven. Naj bo $f \colon P \to M'$ homomorfizem,
$g \colon M \to M'$ pa epimorfizem. Definirajmo homomorfizem
$\tilde{f} \colon P \oplus N \to M'$ kot $\tilde{f} = f \circ \pi$,
kjer je $\pi \colon P \oplus N \to P$ projekcija. Ker je
$P \oplus N$ projektiven, obstaja tak homomorfizem
$\tilde{h} \colon P \oplus N \to M$, da je
$g \circ \tilde{h} = \tilde{f}$. Sedaj lahko za $h \colon P \to M$
izberemo kar $h(p) = \tilde{h}(p, 0)$, saj velja
\[
g \circ h(p) = g \circ \tilde{h}(p,0) = \tilde{f}(p,0) = f(p).
\qedhere
\]
\end{proof}

\begin{opomba}
Vsak projektiven modul nad lokalnim\footnote{Za vsak $x \in R$ je
$x$ ali $1-x$ obrnljiv.} kolobarjem je prost.
\end{opomba}

\newpage

\subsection{Tenzorski produkt modulov}

\datum{2023-3-31}

\begin{definicija}
Naj bo $R$ komutativen kolobar z enoto, $M$ in $N$ pa $R$-modula.
Naj bo $P$ prost $R$-modul, generiran z množico $M \times N$. Naj
bo $Y$ podmodul v $P$, generiran z naslednjimi množicami:
\begin{gather*} 
\setb{(m + m', n) - (m, n) - (m', n)}{m, m' \in M \land n \in N},
\\
\setb{(m, n + n') - (m, n) - (m, n')}{m \in M \land n, n' \in N},
\\
\setb{(rm, n) - r (m, n)}{r \in R \land m \in M \land n \in N},
\\
\setb{(m, rn) - r (m, n)}{r \in R \land m \in M \land n \in N}.
\end{gather*}
Modulu
\[
M \otimes_R N = \kvoc{P}{Y}
\]
pravimo \emph{tenzorski produkt}\index{Modul!Tenzorski produkt}
modulov $M$ in $N$.
\end{definicija}

\begin{opomba}
Sliko para $(m, n) \in P$ v $M \otimes_R N$ označimo z
$m \otimes n$. Takim tenzorjem pravimo \emph{enostavni tenozorji}.
\end{opomba}

\begin{opomba}
Preslikava $\tau \colon M \times N \to M \otimes_R N$ s predpisom
$\tau(m,n) = m \otimes n$ je bilinearna.
\end{opomba}

\begin{izrek}[Univerzalna lastnost tenzorskih produktov]
Naj bodo $M$, $N$ in $T$ $R$-moduli.

\begin{enumerate}[i)]
\item Vsaka $R$-bilinearna preslikava
$\varphi \colon M \times N \to T$ inducira enolično določen
homomorfizem $R$-modulov
$\tilde{\varphi} \colon M \otimes_R N \to T$, za katero je
$\tilde{\varphi} \circ \tau = \varphi$.
\[
\begin{tikzcd}[column sep=large, row sep=large]
M \times N \arrow[r, "\varphi"] \arrow[d, "\tau"'] & T \\
M \otimes_R N \arrow[ur, dashrightarrow, "\tilde{\varphi}"']
\end{tikzcd}
\]
\item Naj bo $\psi \colon M \times N \to T$ bilinearna preslikava.
Če za vsako bilinearno preslikavo $\varphi \colon M \times N \to L$
obstaja natanko en homomorfizem $R$-modulov
$\tilde{\varphi} \colon T \to L$, za katerega je
$\tilde{\varphi} \circ \psi = \varphi$, je $T \cong M \otimes_R N$.
\[
\begin{tikzcd}[column sep=large, row sep=large]
M \times N \arrow[r, "\varphi"] \arrow[d, "\psi"'] & L \\
T \arrow[ur, dashrightarrow, "\tilde{\varphi}"']
\end{tikzcd}
\]
\end{enumerate}
\end{izrek}

\begin{proof}
\phantom{a}
\begin{enumerate}[i)]
\item Ker je $P$ prost $R$-modul nad $M \times N$, po univerzalni
lastnosti obstaja enoličen homomorfizem
$\hat{\varphi} \colon P \to T$, za katerega na $M \times N$ velja
$\hat{\varphi}(m,n) = \varphi(m,n)$. Ni težko videti, da je
$Y \subseteq \ker \hat{\varphi}$, zato $\hat{\varphi}$ inducira
homomorfizem $\tilde{\varphi} \colon M \otimes_R N \to T$.
\item Obstaja enoličen homomorfizem
$\tilde{\psi} \colon M \otimes_R N \to T$, za katerega je
$\tilde{\psi} \circ \tau = \psi$. Po predpostavki izreka obstaja
enoličen homomorfizem $\tilde{\tau} \colon T \to M \otimes_R N$, za
katerega je $\tilde{\tau} \circ \psi = \tau$. Opazimo, da velja
\[
\tilde{\psi} \circ \tilde{\tau} \circ \psi =
\tilde{\psi} \circ \tau = \id \circ \psi
\]
in
\[
\tilde{\tau} \circ \tilde{\psi} \circ \tau =
\tilde{\tau} \circ \psi = \id \circ \tau.
\]
Po enoličnosti sledi, da je $\tilde{\psi} = \tilde{\tau}^{-1}$.
\qedhere
\end{enumerate}
\end{proof}

\begin{opomba}
Če je $M = \skl{X}$ in $N = \skl{Y}$, je
\[
M \otimes_R N = \skl{\setb{x \otimes y}{x \in X \land y \in Y}}.
\]
\end{opomba}

\begin{opomba}
Naj bosta $f \colon M_1 \to M_2$ in $g \colon N_1 \to N_2$
homomorfizma $R$-modulov. Tedaj $f$ in $g$ inducirata homomorfizem
$f \otimes g \colon M_1 \otimes_R N_1 \to M_2 \otimes_R N_2$ s
predpisom
\[
(f \otimes g)(m \otimes n) = f(m) \otimes g(n).
\]
\end{opomba}

\begin{trditev}
Naj bo $M$ $R$-modul. Potem je
\[
M \otimes_R R \cong M \cong R \otimes_R M.
\]
\end{trditev}

\begin{proof}
Preslikava $\varphi \colon R \times M \to M$ s predpisom
$\varphi(r, m) = rm$ je bilinearna, zato inducira homomorfizem
$\tilde{\varphi} \colon M \otimes_R R \to M$. Ni težko videti, da
je $\tilde{\varphi}^{-1}(m) = m \otimes 1$.
\end{proof}

\begin{trditev}
Naj bo $M$ prost $R$-modul z bazo $\setb{m_i}{i \in I}$, $N$ pa
prost $R$-modul z bazo $\setb{n_j}{j \in J}$. Potem je
$M \otimes_R N$ prost $R$-modul z bazo
\[
\setb{m_i \otimes n_j}{i \in I \land j \in J}.
\]
\end{trditev}

\begin{proof}
Množica očitno generira $M \otimes_R N$, zato je dovolj preveriti
linearno neodvisnost. Predpostavimo torej, da je
\[
\sum r_{i,j} (m_i \otimes n_j) = 0.
\]
Za vsak $k \in J$ definiramo homomorfizem $f_k \colon N \to R$,
ki na bazi deluje po predpisu $f_k(n_j) = \delta_{k,j}$. Če
zgornjo enačbo preslikamo z $(\id \otimes f_k)$, dobimo
\[
0 = \sum r_{i,j} (m_i \otimes \delta_{k,j}),
\]
oziroma
\[
0 = \sum r_{i,k} m_i.
\]
Sledi, da so vsi koeficienti enaki $0$.
\end{proof}

\begin{trditev}
Naj bodo $A$, $B$ in $C$ $R$-moduli. Tedaj velja

\begin{enumerate}[i)]
\item $A \otimes_R B \cong B \otimes_R A$,
\item $A \otimes_R (B \oplus C) \cong
A \otimes_R B \oplus A \otimes C$,
\item $(A \otimes_R B) \otimes_R C \cong
A \otimes_R (B \otimes_R C)$.
\end{enumerate}
\end{trditev}

\pagebreak[2]
\begin{proof}
\phantom{a}
\begin{enumerate}[i)]
\item Preslikava $(a, b) \mapsto b \otimes a$ je bilinearna, zato
inducira homomorfizem
$\varphi \colon A \otimes_R B \to B \otimes_R A$, ki slika po
predpisu $\varphi(a \otimes b) = b \otimes a$. Simetrično dobimo še
njegov inverz.
\item Naj bo $\varphi \colon A \times (B \oplus C) \to
A \otimes_R B \oplus A \otimes C$ preslikava, ki deluje po predpisu
\[
\varphi(a, (b, c)) = (a \otimes b, a \otimes c).
\]
Očitno je bilinearna, zato inducira homomorfizem
$\tilde{\varphi} \colon A \otimes_R (B \oplus C) \to
A \otimes_R B \oplus A \otimes C$.

Definirajmo preslikavi
$\psi_1 \colon A \times B \to A \otimes_R (B \oplus C)$ in
$\psi_2 \colon A \times B \to A \otimes_R (B \oplus C)$ s
predpisoma
\[
\psi_1(a, b) = a \otimes (b, 0)
\quad \text{in} \quad
\psi_2(a, c) = a \otimes(0, c).
\]
Očitno sta bilinearni, zato inducirata homomorfizma na tenzorskem
produktu, to pa sta ravno komponenti inverza homomorfizma
$\tilde{\varphi}$.
\item Naj bo
$\varphi_a \colon B \times C \to (A \otimes_R B) \otimes_R C$
preslikava s predpisom $\varphi_a(b,c) = (a \otimes b) \otimes c$.
Očitno je bilinearna, zato inducira homomorfizem
$\tilde{\varphi}_a$ $R$-modulov $B \otimes_R C$ in
$(A \otimes_R B) \otimes_R C$ s predpisom
\[
\tilde{\varphi}_a(b \otimes c) = (a \otimes b) \otimes c.
\]
Preslikava $\varphi \colon A \times (B \otimes_R C) \to
(A \otimes_R B) \otimes_R C$ s predpisom
$\varphi(a, \omega) = \tilde{\varphi}_a(\omega)$ je bilinearna,
zato inducira homomorfizem $\tilde{\varphi}$ ustreznih tenzorskih
produktov. Simetrično lahko skonstruiramo tudi inverz
$\tilde{\varphi}$. \qedhere
\end{enumerate}
\end{proof}

\datum{2023-4-7}

\begin{trditev}
Naj bo $R$ komutativen kolobar, $M$ in $N$ pa prosta $R$-modula z
bazama $\setb{m_i}{i \in I}$ in $\setb{n_j}{j \in J}$. Tedaj je
$M \otimes_R N$ prost $R$-modul z bazo
\[
\setb{m_i \otimes n_j}{i \in I, j \in J}.
\]
\end{trditev}

\begin{proof}
Množica očitno generira $M \otimes_R N$, zato je dovolj preveriti
linearno neodvisnost. Predpostavimo torej, da je
\[
\sum r_{i,j} (m_i \otimes n_j) = 0.
\]
Za vsak $k \in J$ definiramo homomorfizem $f_k \colon N \to R$,
ki na bazi deluje po predpisu $f_k(n_j) = \delta_{k,j}$. Če
zgornjo enačbo preslikamo z $(\id \otimes f_k)$, dobimo
\[
0 = \sum r_{i,j} (m_i \otimes \delta_{k,j}),
\]
oziroma
\[
0 = \sum r_{i,k} m_i.
\]
Sledi, da so vsi koeficienti enaki $0$.
\end{proof}

\begin{opomba}
Za nekomutativne kolobarje definiramo $M \otimes_R N$ na naslednji
način:

Naj bo $M$ desni, $N$ pa levi $R$-modul. Naj bo $F$ prosta abelova
grupa nad množico $\setb{(m,n)}{m \in M, n \in N}$. Tedaj je
\[
M \otimes_R N = \kvoc{F}{T},
\]
kjer je $T$ podgrupa v $F$, generirana z
\[
(m_1 + m_2, n) - (m_1, n) - (m_2, n),
\quad
(m, n_1 + n_2) - (m, n_1) - (m, n_2)
\quad \text{in} \quad
(mr, n) - (m, rn).
\]
\end{opomba}

\newpage

\subsection{Skrčitve in razširitve skalarjev}

\begin{trditev}
Naj bo $f \colon R \to S$ homomorfizem kolobarjev.

\begin{enumerate}[i)]
\item Naj bo $M$ $S$-modul. Tedaj je $M$ tudi $R$-modul z
operacijo
\[
r \cdot m = f(r) \cdot m.
\]
\item Naj bo $R$ komutativen kolobar, $M$ pa $R$-modul. Tedaj je
$S \otimes_R M$ $S$-modul.
\end{enumerate}
\end{trditev}

\obvs

\begin{trditev}
Naj bo $f \colon R \to S$ homomorfizem kolobarjev, kjer je $R$
komutativen. Naj bo $M$ $R$-modul, $N$ pa $S$-modul. Tedaj sta
\[
\Hom_R(M,N) \cong \Hom_S(S \otimes_R M, N)
\]
izomorfni abelovi grupi.
\end{trditev}

\begin{proof}
Naj bo $\Phi \colon \Hom_R(M,N) \to \Hom_S(S \otimes_R M, N)$
preslikava, ki deluje po predpisu
\[
\varphi \mapsto \br{(s \otimes m) \mapsto s \varphi(m)},
\]
preslikava $\Psi \colon \Hom_S(S \otimes_R M, N) \to \Hom_R(M,N)$
pa po predpisu
\[
\psi \mapsto \br{m \mapsto \psi(1 \otimes m)}.
\]
Ni težko preveriti, da sta to inverzni preslikavi.
\end{proof}

\begin{izrek}
Naj bo $R$ komutativen kolobar, $M$, $N$ in $P$ pa $R$-moduli.
Tedaj je
\[
\Hom_R(M \otimes_R N, P) \cong \Hom_R(M, \Hom_R(N,P)).
\]
\end{izrek}

\begin{proof}
Preslikava
\[
\varphi \mapsto
\br{m \mapsto \br{n \mapsto \varphi(m \otimes n)}}
\]
je izomorfizem, saj je
\[
\varphi \mapsto (m \otimes n \mapsto \varphi(m)(n))
\]
njen inverz.
\end{proof}

\newpage

\subsection{Eksaktna zaporedja modulov}

\begin{definicija}
\emph{Zaporedje modulov}\index{Modul!Zaporedje} je zaporedje
modulov $(M_n)_n$ in preslikav $(f_n)_n$, kjer je
$f_n \colon M_n \to M_{n+1}$.
\end{definicija}

\begin{definicija}
Zaporedje modulov je
\emph{eksaktno}\index{Modul!Zaporedje!Eksaktno} v $M_n$, če je
$\im f_{n-1} = \ker f_n$. Zaporedje je eksaktno, če je eksaktno v
vsakem modulu.
\end{definicija}

\begin{definicija}
Zaporedje modulov je
\emph{verižni kompleks}\index{Modul!Verižni kompleks}, če za vsako
naravno število $n$ velja $\im f_n \subseteq \ker f_{n+1}$.
\end{definicija}

\begin{definicija}
Eksaktnim zaporedjem $R$-modulov oblike
\[
\begin{tikzcd}
0 \arrow[r] & M \arrow["f",r] & N \arrow["g", r] & P \arrow[r] & 0
\end{tikzcd}
\]
pravimo \emph{kratka eksaktna zaporedja}.
\end{definicija}

\begin{trditev}
Za kratko eksaktno zaporedje
\[
\begin{tikzcd}
0 \arrow[r] & A \arrow[r] & B \arrow[r] & C \arrow[r] & 0
\end{tikzcd}
\]
so ekvivalentne naslednje trditve:

\begin{enumerate}[i)]
\item Obstaja homomorfizem $p \colon B \to A$, za katerega je
$p \circ f = \id_A$.
\item Obstaja homomorfizem $i \colon C \to B$, za katerega je
$g \circ i = \id_C$.
\item Obstajata homomorfizma $p \colon B \to A$ in
$i \colon C \to B$, za katera je $p \circ f = \id_A$,
$g \circ i = \id_C$ in $f \circ p + i \circ g = \id_B$.
\end{enumerate}
\end{trditev}

\begin{proof}
Predpostavimo, da obstaja homomorfizem $p \colon B \to A$, za
katerega je $p \circ f = \id_A$. Ker je
$\im f \cap \ker p = \set{0}$ in za vsak $x \in B$ velja
\[
x = (x - p \circ f(x)) + p \circ f(x),
\]
je $B = \im f \oplus \ker p$. Sledi, da je
$\eval{g}{\ker p}{} \colon \ker p \to C$ izomorfizem. Preslikavo
$i$ tako dobimo kot inverz tega izomorfizma.

Predpostavimo, da obstaja homomorfizem $i \colon C \to B$, za
katerega je $g \circ i = \id_C$. Podobno kot zgoraj opazimo, da
velja $B = \ker g \oplus \im s$. Sedaj lahko definiramo
$p \colon B \to A$ tako, da za vsak $b \in B$ zapišemo
$b = b_1 + b_2$, kjer je $b_1 \in \ker g = \im f$ in
$b_2 \in \im s$. Sedaj preprosto vzamemo $p(b) = f^{-1}(b_1)$.

Prvi dve točki sta tako ekvivalentni. Pokazati moramo še, da
implicirata tretjo točko, oziroma $i \circ g + f \circ p = \id_B$.
Za $b \in B$ zapišimo $b = b_1 + b_2$, kjer je $b_1 \in \ker g$
in $b_2 \in \im i$. Tedaj je
\[
f \circ p(b) + i \circ g(b) =
f(f^{-1}(b_1)) + i \circ g(i(i^{-1}(b_2)) =
b_1 + b_2 =
b. \qedhere
\]
\end{proof}

\begin{opomba}
Če veljajo ti pogoji, sledi $B \cong A \oplus C$. Pravimo, da je
zaporedje \emph{razcepno eksaktno}. Velja tudi obratno -- če za
$f \colon A \to B$ vzamemo inkluzijo, za $g \colon B \to C$ pa
projekcijo, tvorijo moduli $A$, $B$ in $C$ razcepno eksaktno
zaporedje.
\end{opomba}

\begin{trditev}
Če je v kratkem eksaktnem zaporedju $R$-modulov
\[
\begin{tikzcd}
0 \arrow[r] & A \arrow[r, "f"] & B \arrow[r, "g"] & C \arrow[r] & 0
\end{tikzcd}
\]
$C$ prost, je zaporedje razcepno.
\end{trditev}

\begin{proof}
Naj bo $\setb{c_\lambda}{\lambda \in \Lambda}$ baza za $C$. Po
aksiomu izbire lahko za vsak $c_\lambda$ izberemo
$b_\lambda \in g^{-1}(c_\lambda)$ in definiramo
\[
i(c_\lambda) = b_\lambda. \qedhere
\]
\end{proof}

\begin{trditev}
Če je v kratkem eksaktnem zaporedju $R$-modulov
\[
\begin{tikzcd}
0 \arrow[r] & A \arrow[r, "f"] & B \arrow[r, "g"] & C \arrow[r] & 0
\end{tikzcd}
\]
$C$ projektiven, je zaporedje razcepno.
\end{trditev}

\begin{proof}
Ker je $g$ epimorfizem, je
\[
B \cong C \oplus \ker g = C \oplus \im f \cong C \oplus A. \qedhere
\]
\end{proof}

\begin{izrek}[Kratka lema o petih]\index{Izrek!Kratka lema o petih}
Denimo, da sta v diagramu $R$-modulov
\[
\begin{tikzcd}[row sep=large]
0 \arrow[r] &
A \arrow[r, "f"] \arrow[d, "\alpha"] &
B \arrow[r, "g"] \arrow[d, "\beta"] &
C \arrow[r] \arrow[d, "\gamma"] & 0
\\
0 \arrow[r] &
A' \arrow[r, "f'"] &
B' \arrow[r, "g'"] &
C' \arrow[r] & 0
\end{tikzcd}
\]
vrstici eksaktni in diagram komutira. Tedaj veljajo naslednje
trditve:

\begin{enumerate}[i)]
\item Če sta $\alpha$ in $\gamma$ monomorfizma, je tudi $\gamma$
monomorfizem.
\item Če sta $\alpha$ in $\gamma$ epimorfizma, je tudi $\gamma$
epimorfizem.
\item Če sta $\alpha$ in $\gamma$ izomorfizma, je tudi $\gamma$
izomorfizem.
\end{enumerate}
\end{izrek}

\begin{proof}
\phantom{a}
\begin{enumerate}[i)]
\item Naj bo $b \in B$ ničla $\beta$. Tedaj je
\[
0 = g' \circ \beta (b) = \gamma \circ g(b).
\]
Ker je $\gamma$ monomorfizem, sledi $g(b) = 0$. Sledi, da je
$b \in \ker g = \im f$, zato lahko pišemo $b = f(a)$. Opazimo še,
da je
\[
0 = \beta \circ f(a) = f' \circ \alpha(a).
\]
Ker sta tako $f'$ kot $\alpha$ monomorfizma, je tak tudi njun
kompozitum, zato je $a = 0$ in posledično $b = 0$.

\item Naj bo $b \in B'$. Ker sta $g$ in $\gamma$ epimorfizma,
obstaja tak $b_2 \in B$, da je $\gamma \circ g(b_2) = g'(b)$.
Opazimo, da je $g'(b - \beta(b_2)) = 0$, zato je
$b - \beta(b_2) \in \ker g' = \im f$. Ker je $\alpha$ epimorfizem,
obstaja tak $a \in A$, da je $b - \beta(b_2) = \alpha \circ f'(a)$.
Naj bo $b_1 = f(a)$. Dobimo
\[
\beta(b_1 + b_2) =
\beta(f(a)) + \beta(b_2) =
\alpha(f'(a)) + \beta(b_2) = b.
\]
\item Sledi iz prvih dveh točk. \qedhere
\end{enumerate}
\end{proof}
