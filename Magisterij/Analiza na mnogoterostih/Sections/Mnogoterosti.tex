\section{Mnogoterosti}

\subsection{Topološke mnogoterosti}

\datum{2022-10-5}

\begin{definicija}
Naj bo $n \in \N_0$. Topološki prostor $X$ je
\emph{$n$-dimenzionalna topološka mnogoterost}\index{Mnogoterost},
če velja:

\begin{enumerate}[i)]
\item za vsako točko $p \in X$ obstaja njena okolica $U$ in
homeomorfizem $\phi \colon U \to V$, kjer je $V \subseteq \R^n$
odprta,
\item prostor $X$ je Hausdorffov in
\item prostor $X$ je 2-števen.
\end{enumerate}

Vsak par $(U, \phi)$ iz prve točke imenujemo
\emph{lokalna karta}\index{Karta}, inverz $\phi^{-1}$ pa
\emph{lokalna parametrizacija}\index{Mnogoterost!Lokalna parametrizacija}
množice $U \subseteq X$.
\end{definicija}

\begin{definicija}
\emph{Topološki atlas}\index{Mnogoterost!Atlas} je kolekcija
$\mathcal{U} = \setb{(U_i, \phi_i)}{i \in I}$ lokalnih kart, za
katero je
\[
X = \bigcup_{i \in I} U_i.
\]
\end{definicija}

\begin{opomba}
Po definiciji mnogoterosti lahko vselej najdemo števni atlas. Če je
$X$ kompakten, dobimo atlas s končno kartami.
\end{opomba}

\datum{2022-10-7}

\begin{opomba}
Zgornja formulacija ustreza mnogoterostim brez roba. Če v zgornji
definiciji zamenjamo $\R^n$ s
\[
\HH^n = \setb{(x_1,x_2,\dots,x_n) \in \R^n}{x_n \geq 0},
\]
dobimo splošno definicijo. Definiramo rob $\partial X$ mnogoterosti
kot točke, ki se s homeomorfizmi preslikajo v $\partial \HH^n$, in
notranjost $\itr{X}$ kot točke, ki se preslikajo v
$\HH^n \setminus \partial \HH^n$.
\end{opomba}

\begin{izrek}[Brouwer]\index{Izrek!Brouwer}
Naj bo $V \subseteq \R^n$ odprta množica in $\phi \colon V \to V'$
homeomorfizem, kjer je $V' \subseteq \R^n$. Tedaj je $V'$ odprta v
$\R^n$.
\end{izrek}

\begin{proof}
Glej izrek~2.3.2 v zapiskih predmeta Uvod v geometrijsko topologijo
v 2.~letniku.
\end{proof}

\begin{opomba}
Ta izrek zagotavlja, da sta rob in notranjost mnogoterosti dobro
definirana.
\end{opomba}

\begin{trditev}
Naj bo $X$ $n$-dimenzionalna mnogoterost z nepraznim robom. Tedaj
je $\partial X$ zaprta množica, ki je $(n-1)$-dimenzionalna
mnogoterost brez roba.
\end{trditev}

\begin{proof}
Glej izrek~3.2.2 v zapiskih predmeta Uvod v geometrijsko topologijo
v 2.~letniku.
\end{proof}

\begin{primer}
Naj bo $f \colon \R^n \to \R$ zvezno odvedljiva funkcija in
\[
X = \setb{x \in \R^n}{f(x) \leq 0}.
\]
Če za vsak $x$, za katerega je $f(x) = 0$, velja $df_x \ne 0$, je
\[
\partial X = \setb{x \in \R^n}{f(x) = 0}.
\]
Dokaz je preprost -- uporabimo izrek o implicitni funkciji.
\end{primer}

\newpage

\subsection{Gladke mnogoterosti}

\begin{definicija}
Naj bo $\Omega \subseteq \R^n$ in $f \colon \Omega \to \R^m$.
Pravimo, da je za $r \in \N_0$ funkcija $f$ razreda
$\mathcal{C}^r$, če njeni parcialni odvodi do reda $r$ obstajajo in
so zvezne funkcije na $\Omega$. Funkcija $f$ je razreda
$\mathcal{C}^\infty$, če obstajajo njeni parcialni odvodi
poljubnega reda.

S $\mathcal{C}^\omega(\Omega)$ označujemo realno analitične
funkcije na $\Omega$.

Za $\Omega \subseteq \C^n$ je funkcija $f \colon \Omega \to \C^m$
razreda $\mathcal{O}(\Omega)$, če so njene komponente holomorfne na
$\Omega$.
\end{definicija}

\begin{definicija}
Bijektivna preslikava $f \colon \Omega \to \Omega'$, kjer sta
$\Omega, \Omega' \subseteq \R^n$, je $\mathcal{C}^r$
\emph{difeomorfizem}\index{Preslikava!Difeomorfizem}, če sta
$f$ in $f^{-1}$ razreda $\mathcal{C}^r$.
\end{definicija}

\begin{definicija}
Naj bo $X$ topološka mnogoterost dimenzije $n$ in
$\mathcal{U} = \setb{(U_i, \phi_i)}{i \in I}$ njen atlas. Za vsaka
$i, j \in I$ označimo $U_{i,j} = U_i \cap U_j$. Za vsak par
$i, j \in I$, za katerega je $U_{i,j} \ne \emptyset$, definiramo
\emph{prehodno preslikavo}\index{Preslikava!Prehodna}
$\phi_{i,j} \colon \phi_j(U_{i,j}) \to \phi_i(U_{i,j})$ kot
$\phi_{i,j} = \phi_i \circ \phi_j^{-1}$.

Pravimo, da je $\mathcal{U}$ razreda $\mathcal{C}^r$, če so vse
njegove prehodne preslikave $\mathcal{C}^r$ difeomorfizmi.
\end{definicija}

\begin{primer}
Naj bosta $\phi$ in $\psi$ stereografski projekciji glede na
severni in južni pol sfere $S^2$. Tedaj je prehodna preslikava
enaka
\[
\phi \circ \psi^{-1}(\zeta) = \frac{1}{\zeta}
\]
za $\zeta \in \C^*$. Sledi, da je to kompleksen atlas na $S^2$.
\end{primer}

\begin{definicija}
Dva $\mathcal{C}^r$ atlasa
$\mathcal{U} = \setb{(U_i, \phi_i)}{i \in I}$ in
$\mathcal{V} = \setb{(V_i, \psi_i)}{i \in I}$ sta
\emph{$C^r$ kompatibilna}\index{Mnogoterost!Atlas!Cr kompatibilnost@$\mathcal{C}^r$ kompatibilnost},
če je tudi $\mathcal{U} \cup \mathcal{V}$ spet $\mathcal{C}^r$
atlas.
\end{definicija}

\begin{opomba}
$\mathcal{C}^r$ kompatibilnost je ekvivalenčna relacija.
\end{opomba}

\begin{opomba}
Struktura $\mathcal{C}^r$ mnogoterosti na topološki mnogoterosti
$X$ je določena z izbiro ekvivalenčnega razreda $\mathcal{C}^r$
atlasov na $X$.
\end{opomba}

\begin{definicija}
\emph{Maksimalen}\index{Mnogoterost!Atlas!Maksimalen}
$\mathcal{C}^r$ atlas na $X$ je unija vseh ekvivalenčnih
$\mathcal{C}^r$ atlasov.
\end{definicija}

\begin{definicija}
Topološka mnogoterost s $\mathcal{C}^r$ atlasom je

\begin{enumerate}[i)]
\item \emph{gladka}, če je $r = \infty$,
\item \emph{realno analitična}, če je $r = \omega$.
\end{enumerate}

Če je atlas razreda $\mathcal{O}$, je mnogoterost
\emph{kompleksna}\index{Mnogoterost!Gladka}.
\end{definicija}

\begin{definicija}
Zvezna preslikava $f \colon X \to Y$ med dvema $\mathcal{C}^r$
mnogoterostima je razreda $\mathcal{C}^r$, če je za vsak par kart
$(U, \phi)$ na $X$ in $(V, \psi)$ na $Y$ preslikava
$\widetilde{f} = \psi \circ f \circ \phi^{-1}$ razreda
$\mathcal{C}^r$.
\[
\begin{tikzcd}[column sep=large, row sep=large]
U \arrow[r, "f"] \arrow[d, "\phi"'] & V \arrow[d, "\psi"'] \\
U' \subseteq \R^n \arrow[r, dashrightarrow, "\widetilde{f}"'] &
V' \subseteq \R^n
\end{tikzcd}
\]
\end{definicija}

\datum{2022-10-12}

\begin{trditev}
Preslikava $f \colon X \to Y$ je $\mathcal{C}^r$ v neki okolici
točke $p \in X$ natanko tedaj, ko je za nek par kart
$p \in U \subseteq X$ in $f(p) \in V \subseteq Y$ s homeomorfizmoma
$\phi \colon U \to U'$ in $\psi \colon V \to V'$, preslikava
$\widetilde{f} = \psi \circ f \circ \phi^{-1}$ razreda
$\mathcal{C}^r$ v okolici točke $\phi(p)$.\footnote{Lastnost
$\mathcal{C}^r$ je neodvisna od izbire parov kart v danem atlasu.}
\end{trditev}

\begin{proof}
Naj bosta $(U', \phi')$ in $(V', \psi')$ neki drugi karti na $X$ in
$Y$ v okolici $p$ in $f(p)$ zaporedoma. Velja
\[
\psi' \circ f \circ (\phi')^{-1} =
(\psi' \circ \psi^{-1}) \circ (\psi \circ
f \circ \phi^{-1}) \circ (\phi \circ (\phi')^{-1}).
\]
Opazimo, da sta prvi in zadnji oklepaj $\mathcal{C}^r$
difeomorfizma, saj sta prehodni preslikavi, drugi oklepaj pa je kar
$\widetilde{f}$.
\end{proof}

\begin{definicija}
\emph{Rang}\index{Preslikava!Rang} preslikave
$f \colon \R^n \to \R^m$ v točki $p$ je rang diferenciala
preslikava v tej točki, oziroma
\[
\rang_p f = \rang df(p).
\]
\end{definicija}

\begin{definicija}
Naj bo $f \colon X \to Y$ preslikava razreda $\mathcal{C^1}$ in
$p \in X$. \emph{Rang} preslikave $f$ v točki $p$ je definiran kot
\[
\rang_p f = \rang_{\phi(p)} \psi \circ f \circ \phi^{-1},
\]
kjer sta $(U, \phi)$ in $(V, \psi)$ poljubni karti na $X$ in $Y$,
za kateri je $p \in U$ in $f(p) \in V$.
\end{definicija}

\begin{opomba}
Definicija je neodvisna od izbire kart -- velja namreč
\[
\psi' \circ f \circ (\phi')^{-1} =
(\psi' \circ \psi^{-1}) \circ (\psi \circ
f \circ \phi^{-1}) \circ (\phi \circ (\phi')^{-1}).
\]
Ker sta prvi in zadnji oklepaj difeomorfizma, sta ranga preslikav
\[
\psi \circ f \circ \phi^{-1}
\quad \text{in} \quad
\psi' \circ f \circ (\phi')^{-1}
\]
enaka.
\end{opomba}

\begin{definicija}
Pravimo, da je $f \colon \R^n \to \R^m$ ($f \colon X \to Y$)
\emph{imerzija}\index{Preslikava!Imerzija, submerzija} v točki $p$,
če je $n \leq m$ in je $\rang_p f = n$. Če je $n \geq m$ in je
$\rang_p f = m$, preslikavi pravimo \emph{submerzija}. Če je
$n = m = \rang_p f$, preslikavi pravimo
\emph{lokalni difeomorfizem}\index{Preslikava!Difeomorfizem}.
\end{definicija}

\begin{opomba}
Če je $f$ imerzija, obstaja tak par kart $(U, \phi)$ in $(V, \psi)$
v okolici $p$ in $f(p)$, da je
\[
\widetilde{f}(x_1, \dots, x_m) = (x_1, \dots, x_n, 0, \dots, 0).
\]
Če je $f$ submerzija, obstaja tak par kart $(U, \phi)$ in
$(V, \psi)$ v okolici $p$ in $f(p)$, da je
\[
\widetilde{f}(x_1, \dots, x_m) = (x_1, \dots, x_m).
\]
\end{opomba}

\datum{2022-10-14}

\begin{definicija}
Naj bosta $X$ in $Y$ mnogoterosti razreda $\mathcal{C}^r$.
Preslikava $f \colon X \to Y$ je $\mathcal{C}^r$ difeomorfizem, če
je homeomorfizem in sta $f$ in $f^{-1}$ razreda $\mathcal{C}^r$.
\end{definicija}

\begin{opomba}
Preslikava $f$ je difeomorfizem natanko tedaj, ko je $f$
homeomorfizem razreda $\mathcal{C}^r$ in je maksimalnega ranga v
vsaki točki.
\end{opomba}

\begin{trditev}
Naj bo $r \geq 1$, $X$ mnogoterost razreda $\mathcal{C}^r$ z
atlasom $\mathcal{U} = \setb{(U_i, \phi_i)}{i \in I}$ in
$f \colon X \to X$ homeomorfizem. Tedaj je
\[
\mathcal{V} =
\setb{(f(U_i), \phi_i \circ (\eval{f}{U_i}{})^{-1}}{i \in I}
\]
$\mathcal{C}^r$ atlas.
\end{trditev}

\begin{proof}
Velja
\[
\psi_i \circ \psi_j^{-1} = \phi_i \circ
((\eval{f}{V_i}{})^{-1} \circ \eval{f}{U_j}{}) \circ \phi_j^{-1} =
\phi_i \circ \phi_j^{-1}.
\qedhere
\]
\[
\begin{tikzcd}[column sep=large, row sep=large]
U_i \arrow[r, "\phi_i"] \arrow[d, "\eval{f}{U_i}{}"'] &
\phi_i(U_i) \arrow[d, "\id"]
\\
V_i \arrow[r, rightarrow, "\psi_i"'] &
\phi_i(U_i)
\end{tikzcd}
\]
\end{proof}

\begin{trditev}
Naj bo $r \in \N \cup \set{\infty, \omega, \mathcal{O}}$. Če sta
$f \colon X \to Y$ in $g \colon Y \to Z$ $\mathcal{C}^r$
preslikavi, je tudi $g \circ f$ $\mathcal{C}^r$ preslikava.
\end{trditev}

\begin{proof}
Naj bo $p \in X$ poljubna, $U$ pa njena okolica.
\[
\begin{tikzcd}[column sep=large, row sep=large]
U \arrow[r, "f"] \arrow[d, "\phi"'] &
V \arrow[r, "g"] \arrow[d, "\psi"'] &
W \arrow[d, "\theta"]
\\
U' \arrow[r, "\widetilde{f}"] &
V' \arrow[r, "\widetilde{g}"] &
W'
\end{tikzcd}
\]
Iz predpostavk sledi, da sta $\widetilde{f}$ in $\widetilde{g}$
razreda $\mathcal{C}^r$. Sledi, da je tak tudi njun kompozitum. Ker
je
\[
\widetilde{(g \circ f)} = \widetilde{g} \circ \widetilde{f},
\]
je $\eval{g \circ f}{U}{}$ razreda $\mathcal{C}^r$.
\end{proof}

\begin{posledica}
Mnogoterosti razreda $\mathcal{C}^r$ kot objekti in $\mathcal{C}^r$
preslikave med njimi kot morfizmi tvorijo kategorijo
$\mathcal{C}^r$. Izomorfizmi v tej kategoriji so difeomorfizmi.
\end{posledica}

\begin{opomba}
Vse zgornje definicije lahko posplošimo na mnogoterosti z robom.
\end{opomba}

\begin{opomba}
Naj bo $X$ mnogoterost dimenzije $n$ z robom. Tedaj je
\[
\setb{(U_i \cap \partial X, \eval{\phi_i}{U_i \cap X}{})}{i \in I}
\]
$\mathcal{C}^r$ atlas, zato je $\partial X$ $\mathcal{C}^r$
mnogoterost dimenzije $n-1$.
\end{opomba}

\begin{definicija}
Mnogoterost $X$ je \emph{kompleksna mnogoterost s $\mathcal{C}^r$
robom $\partial X$}, če obstaja tak atlas
\[
\mathcal{U} = \setb{(U_i, \phi_i)}{i \in I}
\]
na $X$, da so prehodne preslikave $\mathcal{C}^r$ difeomorfizmi, ki
so holomorfni na $\phi_j(U_{i,j} \setminus X)$.
\end{definicija}

\begin{opomba}
Pri $\dim_\C X = 1$ lahko zaradi Riemannovega upodobitvenega izreka
zahtevamo $\phi_i(U_i \cap \partial X) \subseteq \set{\Im z = 0}$.
\end{opomba}

\begin{trditev}
Če je $f$ meromorfna funkcija na domeni $D \subseteq \C$ s poli v
točkah množice $A$, je preslikava $\widetilde{f} \colon D \to S$,
podana s predpisom
\[
\widetilde{f}(z) = \begin{cases}
f(z), & z \not \in A,
\\
\infty, & z \in A.
\end{cases}
\]
\end{trditev}

\begin{opomba}
Pojem meromorfne funkcije lahko definiramo na poljubni Riemannovi
ploskvi oziroma kompleksni mnogoterosti.
\end{opomba}

\newpage

\subsection{Primeri in konstrukcije mnogoterosti}

\begin{definicija}
Naj bo $\K$ obseg, $V$ pa vektorski prostor nad $\K$.
\emph{Projektivni prostor}\index{Projektivni prostor} prostora $V$
je množica enodimenzionalnih vektorskih podprostorov v $V$.
Označimo ga z $\P(V)$. Za $V = \K^{n+1}$ označimo
\[
\P(V) = \proj{\K}{n}.
\]
\end{definicija}

\datum{2022-10-19}

\begin{opomba}
Velja $\proj{\C}{n} = \C^n \cup \proj{\C}{n-1}$. Prostor
$\proj{\C}{n}$ je kompaktifikacija prostora $\C^n$ s hiperravnino v
neskončnosti.
\end{opomba}

\begin{trditev}
Vsak projektiven prostor $\proj{\K}{n}$ je kompakten.
\end{trditev}

\begin{proof}
Prostor je zvezna slika sfere, ki je kompaktna.
\end{proof}

\begin{opomba}
Vsako vlakno\footnote{Ekvivalenčni razred projekcije.} v
$\proj{\C}{n}$ je krožnica.
\end{opomba}

\begin{trditev}
Prostor $\proj{\C}{1}$ je Riemannova sfera.
\end{trditev}

\begin{proof}
Ker je $\proj{\C}{0}$ ena točka, lahko zapišemo
$\proj{\C}{1} = \C \cup \set{\infty}$. Ker sta karti oblike
\[
[z_0 : z_1] \mapsto \frac{z_0}{z_1} = \zeta
\quad \text{in} \quad
[z_0 : z_1] \mapsto \frac{z_1}{z_0} = \zeta^{-1},
\]
dobimo enako prehodno preslikavo kot na Riemannovi sferi.
\end{proof}

\begin{definicija}
\emph{Projektivno zaprtje}\index{Projektivni prostor!Projektivno zaprtje}
zaprte množice $E \subseteq \C^n$ je množica
$\overline{E} \subseteq \proj{C}{n}$.
\end{definicija}

\begin{opomba}
Če je $E$ omejena, je $\overline{E} = E$.
\end{opomba}

\datum{2022-10-20}

\begin{definicija}
Naj bo $\Sigma \cong \C^{k+1}$ $(k+1)$-razsežen vektorski prostor v
$\C^{n+1}$. Prostoru $\pi(\Sigma \setminus \set{0})$ pravimo
\emph{projektivni podprostor}\index{Projektivni prostor!Podprostor}
v $\proj{\C}{n}$.\footnote{$\pi$ označuje kvocientno projekcijo.}
\end{definicija}

\begin{opomba}
Projektivni podprostor dimenzije $k$ je množica točk, ki rešijo
$(n-k)$ neodvisnih linearnih enačb.
\end{opomba}

\begin{trditev}
Naj bosta $\Sigma$ in $\Sigma'$ vzporedni hiperravnini v $\C^n$.
Tedaj je $\oline{\Sigma} \cap H = \oline{\Sigma'} \cap H$, kjer $H$
označuje hiperravnino v neskončnosti.
\end{trditev}

\obvs

\begin{definicija}
Če je $P(z_0, \dots, z_n)$ homogen polinom stopnje $d \in \N$,
množici
\[
\setb{[z_0 : \cdots : z_n]}{P(z_0, \dots, z_n) = 0}
\]
pravimo
\emph{Algebraična hiperploskev}\index{Mnogoterost!Algebraična hiperploskev}
stopnje $d$.
\end{definicija}

\begin{opomba}
Algebraična hiperploskev v projektivnem prostoru je projektivno
zaprtje afine algebraične množice.
\end{opomba}

\begin{definicija}
Za mnogoterost $X$ in preslikavo $F \colon X \to \C^{n+1}$ lahko
definiramo preslikavo
$f \colon X \setminus \setb{x \in X}{F(x) = 0} \to \proj{\C}{n}$
kot $f = \pi \circ F$.
\end{definicija}

\begin{opomba}
Preslikava $f$ je gladka, če so kvocienti $\frac{f_i}{f_j}$ gladki
na $f_j \ne 0$.
\end{opomba}

\begin{primer}
Naj bo $X \subseteq \C$ območje v $\C$, $f_0, \dots, f_n$ pa
holomorfne funkcije, pri čemer vsaj ena ni konstantno enaka $0$.
Naj bo $a \in X$ skupna ničla teh funkcij, $k$ pa njena minimalna
stopnja. Tedaj lahko zapišemo
\[
f(\zeta) =
[(\zeta-a)^k g_0(\zeta) : \dots : (\zeta-a)^k g_n(\zeta)] =
[g_0(\zeta) : \dots : g_n(\zeta)].
\]
Tako lahko vedno definiramo preslikavo
$f \colon X \to \proj{\C}{n}$.
\end{primer}

\begin{definicija}
Naj bo $P \colon \C_*^{n+1} \to \C_*^{N+1}$ preslikava, katere
komponente so polinomi stopnje $d$. Tedaj lahko definiramo
preslikavo
$f \colon \proj{\C}{n} \setminus
\setb{z \in \proj{\C}{n}}{P(z) \ne 0} \to \proj{\C}{N}$
s predpisom
\[
f([z_0 : \dots : z_n]) = [P_0(z) : \dots : P_n(z)].
\]
\end{definicija}

\begin{primer}
Preslikave $P \in \GL_{n+1}(\C)$ porodijo holomorfne avtomorfizme
$\proj{\C}{n} \to \proj{\C}{n}$. Dobljena grupa je izomorfna
\[
\operatorname{PGL}_{n}(\C) \cong \kvoc{\GL_{n+1}(\C)}{\C^*}.
\]
\end{primer}

\datum{2022-10-21}

\begin{definicija}
Naj bo $1 \leq k < n$. Z $V_k(\K^n)$ označimo vse $n \times k$
matrike maksimalnega ranga. Na $V_k(\K^n)$ uvedemo ekvivalenčno
relacijo
\[
v \sim w \iff \Lin v = \Lin w.
\]
\emph{Grassmanova mnogoterost}\index{Mnogoterost!Grassmanova} je
množica
\[
G_k(\K^n) = \kvoc{V_k(\K^n)}{\sim}.
\]
\end{definicija}

\begin{opomba}
Grupa $\GL_k(\K)$ deluje na $V_k(\K^n)$ z množenjem na desno.
\end{opomba}

\begin{opomba}
Na Grassmanovi mnogoterosti definiramo karto tako, da jo množimo z
inverzom nesingularnega $k \times k$ minorja. Tako dobimo $I_k$ in
še $k(n-k)$ koordinat, ki nam dajo točko v $\K^{k(n-k)}$.
\end{opomba}

\begin{opomba}
Mnogoterost $G_k(\R^n)$ je realno analitična, $G_k(\C)$ pa
kompleksna mnogoterost.
\end{opomba}

\begin{definicija}
Naj bosta $X$ in $Y$ $\mathcal{C}^r$ mnogoterosti z atlasoma
$\mathcal{U} = \setb{(U_i, \phi_i)}{i \in I}$ in
$\mathcal{U} = \setb{(V_j, \psi)}{j \in J}$.
\emph{Kartezični produkt}\index{Mnogoterost!Kartezični produkt} je
mnogoterost $X \times Y$ z atlasom
\[
\setb{(U_i \times V_j, \phi_i \times \psi_j)}{i \in I, j \in J}.
\]
\end{definicija}

\begin{opomba}
Če sta $X$ in $Y$ $\mathcal{C}^r$ mnogoterosti, je $X \times Y$
spet $\mathcal{C}^r$ mnogoterost.
\end{opomba}

\newpage

\subsection{Orientabilne in orientirane mnogoterosti}

\begin{definicija}
Naj bo $U \subseteq \R^n$ množica in $f \colon U \to f(U)$
difeomorfizem. Pravimo, da $f$ \emph{ohranja orientacijo}, če za
vse $x \in U$ velja $\det(Jf(x)) > 0$.
\end{definicija}

\begin{definicija}
Naj bo $X$ $\mathcal{C}^r$ mnogoterost. Atlas mnogoterosti $X$ je
\emph{orientiran}, če vse prehodne preslikave ohranjajo
orientacijo. V tem primeru pravimo, da je $X$
\emph{orientirana}\index{Mnogoterost!Orientirana}. Mnogoterost $X$
je \emph{orientabilna}, če ima kak orientiran atlas, in
\emph{neorientabilna}, če ni orientabilna.
\end{definicija}

\begin{opomba}
Če je $X$ povezana in orientabilna, ima natanko dve orientaciji.
\end{opomba}

\begin{definicija}
Naj bo $X$ $\mathcal{C}^r$ mnogoterost za
$r \in \N_0 \cup \set{\infty}$, $\psi \colon X \to \R$ pa
$\mathcal{C}^r$ funkcija. \emph{Nosilec}\index{Preslikava!Nosilec}
funkcije $\psi$ je množica
\[
\supp \psi = \oline{\setb{x \in X}{\psi(x) \ne 0}}.
\]
\end{definicija}

\begin{opomba}
Če je $X$ kompaktna, je $\supp X$ kompakten.
\end{opomba}

\begin{definicija}
Particija enote na $\mathcal{C}^r$ mnogoterosti $X$ je družina
$\setb{\psi_i}{i \in \N}$ $\mathcal{C}^r$ funkcij s kompaktnimi
nosilci, za katero velja

\begin{enumerate}[i)]
\item za vsak kompakt $k \subseteq X$ je
$\supp \psi_i \cap K \ne \emptyset$ za kvečjemu končno indeksov in
\item na $X$ velja
\[
\sum_{i=1}^\infty \psi_i = 1.
\]
\end{enumerate}
\end{definicija}

\begin{opomba}
Zgornja vrsta je dobro definirana, saj je lokalno končna.
\end{opomba}

\begin{izrek}
Naj bo $X$ $\mathcal{C}^r$ mnogoterost za
$r \in \N \cup \set\infty$,
$\mathcal{U} = \setb{U_\alpha}{\alpha \in A}$ pa neko odprto
pokritje $X$. Potem obstaja $\mathcal{C}^r$ particija enote
$\setb{\psi_i}{i \in \N}$ na $X$, pri čemer za vsak $i \in I$
obstaja nek $\alpha \in A$, za katerega je
\[
\supp \psi_i \subseteq U_\alpha.
\]
\end{izrek}

\begin{opomba}
Če je $\mathcal{U}$ lokalno končno, je brez škode za splošnost
$A = \N$. Tedaj obstaja taka particija enote, da je v zgornjem
izreku $i = \alpha$.
\end{opomba}

\begin{posledica}
Če sta $E$ in $F$ disjunktni zaprti podmnožici v $\mathcal{C}^r$
mnogoterosti $X$, obstaja taka $\mathcal{C}^r$ funkcija
$\chi \colon X \to [0,1]$, da je $\eval{\chi}{E}{} = 1$ in
$\eval{\chi}{F}{} = 0$.
\end{posledica}

\begin{proof}
Množici $U = X \setminus E$ in $V = X \setminus F$ tvorita odprto
pokritje in lahko preprosto izberemo
\[
\chi = \sum_{\supp \psi_i \in V} \psi_i. \qedhere
\]
\end{proof}

\newpage

\subsection{Podmnogoterosti}

\datum{2022-10-26}

\begin{definicija}
Podmnožica $N$ $\mathcal{C}^r$ mnogoterosti $X$ dimenzije $m$ je
\emph{$\mathcal{C}^r$ podmnogoterost}\index{Mnogoterost!Podmnogoterost}
dimenzije $n$, če za vse $p \in N$ obstaja karta $(U, \phi)$ na
$X$, za katero velja $p \in U$ in
\[
\phi(U \cap N) = \phi(U) \cap (\R^n \times \set{0}^{m-n}).
\]
Številu $\dim X - \dim N = m - n$ pravimo
\emph{kodimenzija}\index{Mnogoterost!Podmnogoterost!Kodimenzija}
podmnogoterosti $N$ v $X$.
\end{definicija}

\begin{opomba}
Podobno definiramo kompleksne podmnogoterosti.
\end{opomba}

\begin{opomba}
Karta $(U, \phi)$ je iz maksimalnega atlas na $C$, ki določa dano
$\mathcal{C}^r$ strukturo na $X$.
\end{opomba}

\begin{opomba}
Brez izgube splošnosti lahko karto izberemo tako, da velja
\[
\pi(\phi(U)) = \phi(U) \cap (\R^n \times \set{0}^{m-n}),
\]
kjer je $\pi$ projekcija na prvih $n$ koordinat. Taki karti pravimo
\emph{odlikovana karta}\index{Karta!Odlikovana} glede na $N$.
\end{opomba}

\begin{opomba}
Če v definiciji zamenjamo $\R^n$ s $\HH^n$, dobimo podmnogoterosti
z robom.
\end{opomba}

\begin{trditev}
Če je $N$ $\mathcal{C}^r$ podmnogoterost v $X$, obstaja na $N$
natanko določena struktura $\mathcal{C}^r$ mnogoterosti, za katero
je inkluzija $\iota \colon N \hookrightarrow X$ $\mathcal{C}^r$
preslikava.
\end{trditev}

\begin{proof}
Za vsako karto $(U, \phi)$ na $X$ v danem maksimalnem
$\mathcal{C}^r$ atlasu, izbrano glede na $N$, je zožitev
\[
\eval{\pi \circ \phi}{N \cap U}{} \colon N \times U \to
\pi(\phi(U) \cap \R^n \times \set{0}^{m-n})
\]
karta na $N$. Ker so prehodne preslikave na $N$ zožitve prehodnih
preslikav na $X$, so gladke. Če je $(U, \phi)$ izbrana karta na
$X$, potem je glede na to in prirejeno karto inkluzija
$\iota \colon N \to X$ podana z inkluzijo
\[
\phi(U) \cap (\R^n \times \set{0}^{m-n}) \hookrightarrow \R^m.
\qedhere
\]
\end{proof}

\datum{2022-10-28}

\begin{trditev}
Naj bo $N$ podmnogoterost v $X$. Tedaj obstaja odprta množica
$\Omega \subseteq X$, v kateri je $N$ zaprta podmnožica.
\end{trditev}

\begin{proof}
Vzamemo unijo vseh kart iz definicije podmnogoterosti.
\end{proof}

\begin{trditev}
Naj bo $N \subseteq X$ $\mathcal{C}^r$ podmnogoterost. Naj bo $Z$
$\mathcal{C}^r$ mnogoterost, $f \colon Z \to N$ pa zvezna
preslikava. Tedaj je $f$ gladka kot preslikava v $N$ natanko tedaj,
ko je gladka kot preslikava v $X$.

Če je $f \colon X \to Z$ gladka preslikava, je taka tudi
$\eval{f}{N}{}$.
\end{trditev}

\begin{proof}
Oboje sledi iz definicije.
\end{proof}

\begin{izrek}
Če je $N$ zaprta $\mathcal{C}^r$ podmnogoterost v $X$ in je
$f \colon N \to Z$ $\mathcal{C}^r$ preslikava, obstaja odprta
okolica $\Omega \subseteq X$ podmnogoterosti $N$ in $\mathcal{C}^r$
preslikava $F \colon \Omega \to Z$, za katero je $\eval{F}{N}{}=f$.
\end{izrek}

\begin{proof}
l8r nerdz
\end{proof}

\begin{trditev}
Če je $N \subseteq X$ zaprta $\mathcal{C}^r$ podmnogoterost za
$r \in \N_0 \cup \set\infty$, za vsako $\mathcal{C}^r$ funkcijo
$f \colon N \to \R$ obstaja taka $\mathcal{C}^r$ funkcija
$F \colon X \to \R$, da je $\eval{F}{N}{} = f$.
\end{trditev}

\begin{proof}
Naj bo $\mathcal{U} = \setb{(U_i, \phi_i)}{i \in I}$ lokalno končno
pokritje $N$ z odlikovanimi kartami. Tedaj za vsak $i \in I$
obstaja $\mathcal{C}^r$ preslikava
$\pi_i \colon U_i \to U_i \cap N$, za katero je
$\eval{\pi_i}{U_i \cap N}{} = \id$. Res, vzamemo lahko kar
$\pi_i = \phi_i^{-1} \circ \pi \circ \phi_i$, kjer je $\pi$
projekcija na prvih $n$ koordinat. Sedaj lahko na vsaki odlikovani
karti definiramo $f_i \colon U_i \to \R$ s predpisom
$f_i = f \circ \pi_i$. Sedaj lahko s particijo enote na $X$,
podrejeno pokritju $\mathcal{U} \cup \set{X \setminus N}$,
definiramo
\[
F = \sum_{i \in I} \chi_i \cdot f_i
\]
in jo na $X$ razširimo z $0$.
\end{proof}

\begin{opomba}
Analogen dokaz deluje, če v trditvi $\R$ zamenjamo s poljubnim
kontraktibilnim prostorom.
\end{opomba}

\begin{izrek}[Homotopski princip]
Naj bo $N \subseteq X$ zaprta $\mathcal{C}^r$ podmnogoterost za
$r \in \N_0 \cup \set\infty$ in $f \colon N \to Z$ $\mathcal{C}^r$
preslikava. Če obstaja zvezna razširitev $F_0 \colon X \to Z$
preslikave $f$, obstaja tudi $\mathcal{C}^r$ razširitev
$F_1 \colon X \to Z$ preslikave $f$.
\end{izrek}

\begin{opomba}
Homotopski princip pravi, da rešitev nekega analitičnega problema
obstaja, če ni topoloških obstrukcij.
\end{opomba}

\begin{opomba}
Obstaja celo razširitev, ki je homotopna $F_0$.
\end{opomba}

\begin{izrek}
Naj bo $N \subseteq X$ zaprta podmnožica $\mathcal{C}^r$
mnogoterosti $X$, pri čemer je $r \ne 0$. Denimo, da za vsak
$p \in N$ obstaja taka odprta množica $U \subseteq X$ točke $p$ in
$\mathcal{C}^r$ funkcije $g_1, \dots, g_d \colon U \to \R$, da ima
preslikava $g = (g_1, \dots, g_d) \colon U \to \R^d$ maksimalen
rang v točki $p$ in je $\setb{x \in U}{g(x) = 0} = U \cap N$. Tedaj
je $N$ $\mathcal{C}^r$ podmnogoterost kodimenzije $d$.
\end{izrek}

\begin{proof}
Lahko vzamemo dovolj majhno okolico $U$, da imamo $\mathcal{C}^r$
karto $\phi \colon U \to U'$ za odprto množico $U' \subseteq \R^m$.
Brez škode za splošnost naj bo $\phi(p) = 0$. Velja, da je
$\widetilde{g} = g \circ \phi^{-1}$ $\mathcal{C}^r$ preslikava, ki
ima v $0$ maksimalen rang. Sledi, da so diferenciali
$d \widetilde{g}_i(0)$ linearno neodvisni. Obstajajo funkcije
$h_1, \dots, h_n \colon U' \to \R$, za katere so diferenciali
$dh_i(0)$ in $d \widetilde{g}_i(0)$ linearno neodvisni (vzamemo
lahko kar linearne preslikave). Po izreku o inverzni preslikavi je
preslikava $f = (h, \widetilde{g})$ lokalni difeomorfizem v točki
$0$. Če zožimo okolico $U$, s tem $f \colon U' \to f(U')$ postane
difeomorfizem. S tem dobimo karto $f \circ \phi$ na $U$. Nivojnice
$g = c$ se s to karto preslikajo v afine ravnine
$\R^n \times \set{c}$. Velja torej
\[
(f \circ \phi)(N \cap U) = f(U') \cap (\R^n \times \set{c}).
\qedhere
\]
\end{proof}

\begin{posledica}
Naj bo $g \colon X \to \R^d$ $\mathcal{C}^r$ preslikava, kjer je
$r \ne 0$. Če je $c \in \R^d$ v $g(x)$
\emph{regularna vrednost},\footnote{Točka $c$ je regularna
vrednost, če je $g$ v vsaki točki $x \in f^{-1}(c)$ maksimalnega
ranga.} je nivojnica $g^{-1}(c)$ zaprta $\mathcal{C}^r$
podmnogoterost kodimenzije $d$.
\end{posledica}

\begin{posledica}
Če je $g \colon X \to \R^d$ $\mathcal{C}^r$ submerzija, je vsaka
neprazna nivojnica $\mathcal{C}^r$ zaprta podmnogoterost.
\end{posledica}

\begin{opomba}
S tem dobimo razslojitev\footnote{Tudi \emph{foliacija}.} $X$ na
paroma disjunktne podmnogoterosti kodimenzije $d$.
\end{opomba}

\begin{opomba}
Podobno velja tudi za nivojnice $\mathcal{C}^r$ preslikav
$g \colon X \to Z$, kjer je $Z$ mnogoterost dimenzije $d$.
\end{opomba}

\datum{2022-11-2}

\begin{izrek}
Naj bo $f \colon X \to Y$ injektivna $\mathcal{C}^r$ imerzija.
Potem je $f(X)$ $\mathcal{C}^r$ podmnogoterost v $Y$ natanko tedaj,
ko je vložitev.
\end{izrek}

\begin{proof}
Recimo, da je $f \colon X \to f(X)$ homeomorfizem. Naj bo
$p \in X$. Obstaja okolica $U \subseteq X$ točke $p$, za katero je
$\eval{f}{U}{}$ vložitev, na okolici $f(U)$ pa imamo karto
$(V, \psi)$, ki jo izravna. Tedaj je $f(U)$ odprta v $f(X)$, zato
jo lahko zapišemo kot $V' \cap f(X)$, kjer je $V' \subseteq Y$
odprta. Za $W = V \cap V'$ je $W \cap f(X) = f(U)$, zato je
$(W, \eval{\psi}{W}{})$ karta na $Y$, ki zadošča definiciji
podmnogoterosti.

Recimo, da je $f(X)$ podmnogoterost v $Y$ in $p \in X$. Tedaj
obstaja karta $(V, \psi)$ na $Y$, za katero je $f(p) \in V$ in
\[
\psi(f(X) \cap V) = \psi(V) \cap \br{\R^n \times \set{0}^{m-n}}.
\]
Naj bo $U \subseteq X$ taka okolica točke $p$, da je
$f(U) \subseteq f(X) \cap V$. Sledi, da je $\psi(f(U))$
difeomorfizem slike $U$ v $\R^n$, zato je odprta. Sedaj naj bo
$W \subseteq \psi(V)$ odprta množica, za katero je
\[
W \cap \br{\R^n \times \set{0}^{m-n}} = \psi(f(U)).
\]
Velja, da je $\psi^{-1}(W)$ odprta v $Y$, zato je
\[
\psi^{-1}(W) \cap f(X) = f(U)
\]
odprta v $f(X)$. Ker je odprtost dovolj preveriti na bazi, je $f$
odprta in zato vložitev.
\end{proof}

\begin{definicija}
Naj bosta $X$ in $Y$ lokalno kompaktna Hausdorffova prostora.
Zvezna preslikava $f \colon X \to Y$ je
\emph{prava}\index{Preslikava!Prava}, če je za vsako kompaktno
množico $K \subseteq Y$ tudi $f^{-1}(K)$ kompaktna.
\end{definicija}

\begin{trditev}
Če je $f \colon X \to Y$ prava preslikava lokalno kompaktnih
Hausdorffovih prostorov, je $f$ zaprta preslikava.
\end{trditev}

\begin{proof}
Naj bo $E \subseteq X$ zaprta množica. Naj bo $y_0$ stekališče
množice $f(E)$. Obstaja torej zaporedje $(x_n)_{n=1}^\infty$ v $E$,
za katero je
\[
\lim_{n \to \infty} f(x_n) = y_0.
\]
Naj bo $K \subseteq Y$ kompaktna okolica točke $y_0$. Sledi, da je
$f^{-1}(K)$ kompaktna. Za vse $n \geq N$ velja $f(x_n) \in K$, zato
je zaporedje $(a_n)_{n=N}^\infty$ vsebovano v $E \cap f^{-1}(K)$.
Sledi, da obstaja konvergentno podzaporedje z limito $x_0 \in E$ in
je $y_0 = f(x_0)$.
\end{proof}

\begin{posledica}
Naj bosta $X$ in $Y$ mnogoterosti. Če je $f \colon X \to Y$
injektivna prava preslikava, je $f(X)$ zaprta podmnogoterost v $Y$
in je $f$ vložitev.
\end{posledica}

\begin{proof}
Po zgornji trditvi je $f$ zaprta.
\end{proof}
