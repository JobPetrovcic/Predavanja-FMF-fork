\section{Linearni funkcionali, operatorji, dualni prostor}

\epigraph{">Ta dokaz je spet priložnost da pokažemo svoje likovne
sposobnosti."<}{-- prof.~dr.~Igor Klep}

\subsection{Osnovni pojmi}

\begin{definicija}
Naj bosta $E$ in $F$ normirana prostora.
Operator\footnote{Linearna preslikava.} $T \colon E \to F$ je
\emph{omejen},\index{Operator!Omejen} če obstaja tak $c > 0$, da za
vse $x \in E$ velja
\[
\norm{Tx} \leq c \cdot \norm{x}.
\]
\end{definicija}

\begin{izrek}
Naj bosta $E$ in $F$ normirana prostora, $T \colon E \to F$ pa
operator. Naslednje trditve so ekvivalentne:

\begin{enumerate}[i)]
\item Operator $T$ je zvezen na $E$.
\item Operator $T$ je zvezen v $x_0 \in E$.
\item Operator $T$ je omejen.
\end{enumerate}
\end{izrek}

\begin{proof}
Denimo, da je $T$ zvezen v $x_0$. Sledi, da obstaja $\delta > 0$,
za katero velja
\[
\norm{x - x_0} \leq \delta \implies \norm{Tx - Tx_0} \leq 1.
\]
Vzemimo poljuben $y \in E$, za katerega je $\norm{y} \leq \delta$.
Ker je
\[
\norm{y} = \norm{(y+x_0) - x_0} \leq \delta,
\]
sledi
\[
\norm{T(y + x_0) - Tx_0} \leq 1,
\]
zato je $\norm{Ty} \leq 1$. Za $y \in E \setminus 0$ tako sledi
\[
\norm{T\left(\delta \cdot \frac{y}{\norm{y}}\right)} \leq 1.
\]
Za $c$ iz definicije lahko tako vzamemo kar $\frac{1}{\delta}$.

Naj bo $T$ omejen operator, $x_0 \in E$ poljuben in
$\varepsilon > 0$. Za vse $x$, za katere velja
\[
\norm{x - x_0} < \frac{\varepsilon}{c},
\]
tako velja
\[
\norm{Tx - Tx_0} \leq c \cdot \norm{x - x_0} < \varepsilon.
\]
Operator $T$ je torej zvezen.
\end{proof}

\begin{definicija}
Naj bosta $E$ in $F$ normirana prostora. Označimo
\[
\mathcal{B}(E,F) =
\setb{T \colon E \to F}{\text{$T$ je omejen operator}}.
\]
Na prostoru vpeljemo še normo
\[
\norm{T} = \sup_{\norm{x}_E \leq 1} \norm{Tx}_F.
\]
\end{definicija}

\begin{trditev}
Naj bodo $E$, $F$ in $G$ normirani prostori.

\begin{enumerate}[i)]
\item Prostor $(\mathcal{B}(E,F), \norm{.})$ je normiran prostor.
Za $T \in \mathcal{B}(E,F)$ velja
\[
\norm{T} =
\sup_{\norm{x}=1} \norm{Tx} =
\sup_{x \ne 0} \frac{\norm{Tx}}{\norm{x}} =
\inf \setb{c > 0}
{\forall x \in E \colon \norm{Tx} \leq c \norm{x}}.
\]
\item Naj bosta $T \in \mathcal{B}(E,F)$ in
$S \in \mathcal{B}(F,G)$ operatorja, Tedaj je
$S \circ T \in \mathcal{B}(E,G)$ in velja
\[
\norm{S \circ T} \leq \norm{S} \cdot \norm{T}.
\]
\end{enumerate}
\end{trditev}

\datum{2022-10-13}

\begin{proof}
Če je $\norm{T} = 0$, za vse $\norm{x} \leq 1$ velja
$\norm{Tx} = 0$. Ker je $T$ operator, sledi $T = 0$. Velja še
\[
\norm{\lambda T} = \sup_{\norm{x}=1} \norm{\lambda Tx} =
\abs{\lambda} \cdot \norm{T}
\]
in za vsak $x$, za katerega je $\norm{x}=1$ še
\[
\norm{(T_1 + T_2)x} \leq \norm{T_1x} + \norm{T_2x},
\]
od koder dobimo še trikotniško neenakost.

Za dokaz prve enakosti iz trditve je dovolj opaziti
\[
\norm{T} \geq \sup_{\norm{x}=1} \norm{Tx} =
\sup_{x \ne 0} \frac{\norm{Tx}}{\norm{x}} \geq
\sup_{\substack{\norm x \leq 1 \\ x \ne 0}} \norm{Tx} = \norm{T}.
\]
Če velja $\norm{Tx} \leq c \cdot \norm{x}$ za vse $x$, je
$\norm{T} \leq c$. Hkrati pa velja
\[
\norm{Tx} =
\frac{\norm{Tx}}{\norm{x}} \cdot \norm{x} \leq
\norm{T} \cdot \norm{x},
\]
zato je $\norm{T}$ res iskani infimum.

Ker velja
\[
\norm{S(T(x))} \leq \norm{S} \cdot \norm{Tx} \leq
\norm{S} \cdot \norm{T} \cdot \norm{x},
\]
sledi $\norm{S \circ T} \leq \norm{S} \cdot \norm{T}$.
\end{proof}

\begin{definicija}
Naj bo $E$ normiran prostor. Linearnim preslikavam $E \to \K$
pravimo \emph{linearni funkcionali}\index{Operator!Funkcional}.
Prostoru linearnih funkcionalov pravimo
\emph{dualni prostor}\index{Vektorski prostor!Dualni} in ga
označimo z
\[
E^* = \mathcal{B}(E,\K).
\]
\end{definicija}

\begin{izrek}
Naj bo $E$ normiran, $F$ pa Banachov prostor. Potem je
$\mathcal{B}(E,F)$ Banachov prostor.
\end{izrek}

\begin{proof}
Naj bo $(T_n)_{n=1}^\infty$ Cauchyjevo zaporedje v
$\mathcal{B}(E,F)$. Za vse $x \in E$ je zaporedje
$(T_n x)_{n=1}^\infty$ Cauchyjevo, saj je
\[
\norm{T_n x - T_m x} \leq \norm{T_n - T_m} \cdot \norm{x}.
\]
Sledi, da ima limito, zato lahko definiramo
\[
Tx = \lim_{n \to \infty} T_n x.
\]
Ni težko videti, da je $T$ linearna preslikava. Za dovolj velik $N$
velja
\[
\norm{T_n - T_N x} \leq \norm{x}
\]
za vse $n > N$, zato v limiti dobimo
\[
\norm{Tx} \leq (1 + \norm{T_n}) \cdot \norm{x},
\]
zato je $T$ omejen. Za vsak $\varepsilon > 0$ obstaja $N$, za
katerega velja
\[
\norm{Tx - T_n x} \leq \varepsilon \cdot \norm{x},
\]
za vse $n > N$.Velja torej $\norm{T - T_m} \leq \varepsilon$, zato
je $T$ limita zaporedja.
\end{proof}

\begin{posledica}
Dualni prostor normiranega prostora je Banachov.
\end{posledica}

\begin{posledica}
Naj bo $E$ normiran prostor in $L \leq E$. Naj bo $F$ Banachov in
$T \in \mathcal{B}(L,F)$. Tedaj obstaja natanko en
$S \in \mathcal{B}(\oline{L},F)$, za katerega je
$\eval{S}{L}{} = T$. Pri tem je $\norm{S} = \norm{T}$.
\end{posledica}

\begin{proof}
Naj bo $x \in \oline{L}$. Potem obstaja zaporedje
$(y_n)_{n=1}^\infty$ v $L$, ki konvergira k $x$. Zaporedje
$(Ty_n)_n$ je tako Cauchyjevo v $F$, njegova limita pa je neodvisna
od izbira zaporedja $y_i$. Sledi, da lahko definiramo
$Sx = \lim_{n \to \infty} T y_n$. Ni težko preveriti, da je $S$
linearen, očitno pa je $\eval{S}{L}{} = T$, saj je $T$ omejen.

Očitno velja neenakost $\norm{S} \geq \norm{T}$. Naj bo
$x \in \oline{L}$ in $(x_n)_n$ zaporedje v $L$, ki konvergira k
$x$. Sledi, da je
\[
\norm{Sx} =
\norm{\lim_{n \to \infty} Tx_n} =
\lim_{n \to \infty} \norm{Tx_n} \leq
\norm{T} \cdot \norm{\lim_{n \to \infty} x_n} =
\norm{T} \cdot \norm{x},
\]
zato je $\norm{S} \leq \norm{T}$.

Enoličnost razširitve sledi iz zveznosti in gostosti.
\end{proof}

\begin{trditev}
Naj bosta $E$ in $F$ normirana prostora, $T \colon E \to F$ pa
linearna preslikava. Naslednji trditvi sta ekvivalentni:

\begin{enumerate}[i)]
\item Obstaja omejen inverzni operator $T^{-1} \colon TE \to E$.
\item Obstaja tak $c > 0$, da za vsak $x \in E$ velja
$c \norm{x} \leq \norm{Tx}$.
\end{enumerate}
\end{trditev}

\begin{proof}
Denimo, da ima $T$ omejen inverz z $\norm{T^{-1}} = \frac{1}{c}$.
Za poljuben $y = Tx$ tako velja
\[
\frac{1}{c} \cdot \norm{y} \geq \norm{T^{-1}y} = \norm{x},
\]
oziroma $\norm{Tx} \geq c \cdot \norm{x}$.

Denimo sedaj, da velja druga trditev. Očitno je $T$ injektivna,
zato $T^{-1} \colon TE \to E$ obstaja in je linearen. Naj bo
$y = Tx$ poljuben. Tedaj je
\[
\norm{y} \geq c \norm{x} = c \norm{T^{-1}y},
\]
oziroma $\norm{T^{-1}} \leq \frac{1}{c}$.
\end{proof}

\begin{izrek}
Naj bo $1 \leq p < \infty$ in $\frac{1}{p} + \frac{1}{q} = 1$. Za
$y \in \ell^q$ naj bo $f_y \colon \ell^p \to \K$ preslikava s
predpisom
\[
f_y(x) = \sum_{n=1}^\infty x_n y_n.
\]
Tedaj je $f_y \in \br{\ell^p}^*$ in je $y \mapsto f_y$ izometrični
izomorfizem.
\end{izrek}

\begin{proof}
Opazimo, da po Hölderjevi neenakosti velja
\[
f_y(x) \leq \norm{x}_p \cdot \norm{y}_q.
\]
Sledi, da je $f_y$ dobro definirana preslikava. Ker je ta očitno
linearna, je omejen operator, za katerega je
$\norm{f_y} \leq \norm{y}_q$. Pokažimo, da velja enakost.

Najprej obravnavajmo primer $p=1$. Za poljuben $\varepsilon > 0$
obstaja tak $n$, da je
$\abs{y_n} \geq \norm{y}_\infty - \varepsilon$. Sedaj lahko
preprosto vzamemo $x = e_n$ in dobimo
$\norm{f_y} \geq \norm{y}_\infty - \varepsilon$, oziroma
$\norm{f_y} = \norm{y}_\infty$.

Naj bo sedaj $p > 1$. Vzemimo
\[
x_n = \begin{cases}
0, & y_n = 0, \\
\frac{\abs{y_n}^q}{y_n}, & y_n \ne 0.
\end{cases}
\]
Velja $x \in \ell^p$, saj je
\[
\sum_{n=1}^\infty \abs{x_n}^p =
\sum_{n=1}^\infty \abs{y_n}^q <
\infty.
\]
Sledi, da je
\[
f_y(x) = \sum_{n=1}^\infty x_n y_n = \norm{y}_q^q,
\]
oziroma
\[
\frac{\abs{f_y(x)}}{\norm{x}_p} =
\frac{\norm{y}_q^q}{\norm{y}_q^{\frac{1}{p}}} =
\norm{y}_q,
\]
oziroma $\norm{f_y} = \norm{y}_q$.

Preslikava $y \mapsto f_y$ je tako linearna izometrija. Pokažimo še
surjektivnost. Naj bo $f \in \br{\ell^p}^*$ in definirajmo
$y_n = f(e_n)$. Dokažimo, da je $y \in \ell^q$.

Znova najprej obravnavajmo primer $p=1$. Sledi, da je
\[
\abs{y_n} = \abs{f(e_n)} \leq \norm{f},
\]
zato je $y \in \ell^\infty$.

Naj bo sedaj $p > 1$. Sledi, da je
\[
\sum_{n=1}^m \abs{y_n}^q =
\sum_{\substack{n=1 \\ y_n \ne 0}}^m
\frac{\abs{y_n}^q}{y_n} f(e_n) =
f \br{\sum_{\substack{n=1 \\ y_n \ne 0}}^m
\frac{\abs{y_n}^q}{y_n} e_n} \leq
\norm{f} \cdot \norm{\sum_{\substack{n=1 \\ y_n \ne 0}}^m
\frac{\abs{y_n}^q}{y_n} e_n}_p,
\]
velja pa
\[
\norm{\sum_{\substack{n=1 \\ y_n \ne 0}}^m
\frac{\abs{y_n}^q}{y_n} e_n}_p =
\sqrt[p]{\sum_{\substack{n=1 \\ y_n \ne 0}}^m \abs{y_n}^q}.
\]
Dobimo, da je
\[
\sqrt[q]{\sum_{n=1}^m \abs{y_n}^q} \leq \norm{f},
\]
kar nam v limiti da $\norm{y}_q \leq \norm{f}$.

Očitno se $f$ in $f_y$ ujemata na $\Lin \setb{e_n}{n \in \N}$, ki
je gosta v $\ell^p$, zato je $f = f_y$.
\end{proof}

\begin{posledica}
Velja $\br{\ell^p}^* \cong \ell^q$.
\end{posledica}

\datum{2022-10-19}

\begin{definicija}
Naj bosta $E$ in $F$ normirana prostora in $T \in \mathcal{B}(E,F)$
omejen operator. Operatorju $T^* \colon F^* \to E^*$ s predpisom
$\varphi \mapsto \varphi \circ T$ pravimo
\emph{dualni operator}\index{Operator!Dualni}.
\end{definicija}

\begin{lema}
Dualni operator je omejen operator.
\end{lema}

\begin{proof}
Očitno je $T^*$ linearen. Velja
\[
\abs{(T^* \varphi)(x)} = \abs{\varphi(T(x)} \leq
\norm{\varphi} \cdot \norm{T} \cdot \norm{x},
\]
zato je $\norm{T^* \varphi} \leq \norm{T} \cdot \norm{\varphi}$.
\end{proof}

\newpage

\subsection{Hahn-Banachov izrek}

\begin{definicija}
Naj bo $X$ vektorski prostor nad $\R$. Funkcija $p \colon X \to \R$
je
\emph{sublinearen funkcional}\index{Operator!Funkcional!Sublinearen},
če za vse $x, y \in X$ in $\alpha \geq 0$ velja
\[
p(x+y) \leq p(x) + p(y)
\quad \text{in} \quad
p(\alpha x) = \alpha \cdot p(x).
\]
\end{definicija}

\begin{opomba}
Vsaka polnorma je sublinearen funkcional.
\end{opomba}

\begin{opomba}
Namesto prvega pogoja bi lahko vzeli tudi, da je $p$ konveksen.
\end{opomba}

\begin{izrek}[Hahn-Banach]\index{Izrek!Hahn-Banach!Razširitveni}
\label{iz:hb:1}
Naj bo $p \colon X \to \R$ sublinearen funkcional in $Y \leq X$.
Naj bo $f \colon Y \to \R$ linearen funkcional, za katerega je
$f(y) \leq p(y)$ za vse $y \in Y$. Tedaj obstaja linearen
funkcional $F \colon X \to \R$, za katerega je $\eval{F}{Y}{} = f$
in za vse $x \in X$ velja $F(x) \leq p(x)$.
\end{izrek}

\begin{proof}
Najprej obravnavajmo primer, ko je $\dim \kvoc{X}{Y} = 1$. Tedaj
lahko zapišemo $X = Y \oplus \R x_0$ za nek
$x_0 \in X \setminus Y$. Za vse $y_1, y_2 \in Y$ velja
\[
f(y_1) + f(y_2) = p(y_1 + y_2) \leq p(y_1 + x_0) + p(y_2 - x_0).
\]
Velja torej
\[
f(y_2) - p(y_2 - x_0) \leq p(y_1 + x_0) - f(y_1),
\]
zato obstaja tak $\alpha_0$, da je
\[
\sup_{y_2 \in Y} f(y_2) - p(y_2 - x_0) \leq
\alpha_0 \leq
\inf_{y_1 \in Y} p(y_1 + x_0) - f(y_1).
\]
Sedaj lahko preprosto izberemo $F(y + tx_0) = f(y) + t\alpha_0$.
Operator $F$ je očitno linearen. Velja pa
\[
F(y + tx_0) = f(y) + t\alpha_0 \leq
f(y) + t \cdot p\left(\frac{y}{t} + x_0\right) -
t \cdot f\left(\frac{y}{t}\right) = p(y + t x_0).
\]
Naj bo sedaj $Y \leq X$ poljuben podprostor in
\[
A = \setb{(Y_i, f_i)}{Y \leq Y_i \leq X,
f_i \in \mathcal{B}(Y_i, \R) \land \eval{f_i}{Y}{} = f \land
\forall y \in Y \colon f_i(y) \leq p(y)}.
\]
Očitno je $\mathcal{A} \ne 0$, množico pa lahko delno uredimo z
relacijo
\[
(Y_1, f_1) \leq (Y_2, f_2) \iff
Y_1 \leq Y_2 \land \eval{f_2}{Y_1}{} = f_1.
\]
Vsaka veriga $\mathcal{C} = \setb{(Y_j, f_j)}{j \in \Lambda}$ v
$\mathcal{A}$ ima zgornjo mejo. Res, naj bo
\[
Z = \bigcup_{j \in \Lambda} Y_j
\]
in $g \colon Z \to \R$ funkcional, podan s predpisom
$z \mapsto f_j(z)$, pri čemer je $z \in Y_j$. Očitno je
$(Z, g) \in \mathcal{A}$. Po Zornovi lemi torej obstaja maksimalni
element $(M, F) \in \mathcal{A}$. Po prvem delu dokaza je $M = X$,
zato je $F$ iskana razširitev.
\end{proof}

\begin{lema}\label{lm:hb:1}
Naj bo $X$ vektorski prostor nad $\C$.

\begin{enumerate}[i)]
\item Če je $f \colon X \to \R$ $\R$-linearen funkcional, potem je
s predpisom
\[
\widetilde{f}(x) = f(x) - if(ix)
\]
definiran $\C$-linearen funkcional $\widetilde{f} \colon X \to \C$,
za katerega je $\Re \widetilde{f} = f$.
\item Če je $g \colon X \to \C$ $\C$-linearen funkcional in
$f = \Re g$, je $\widetilde{f} = g$.
\item Če je $p$ polnorma na $X$, velja
\[
\forall x \in X \colon \abs{f(x)} \leq p(x) \iff
\forall x \in X \colon \abs{\widetilde{f}(x)} \leq p(x).
\]
\item Če je $X$ normiran, je $\norm{f} = \norm{\widetilde{f}}$.
\end{enumerate}
\end{lema}

\begin{proof}
Dokažimo vsako točko posebej:

\begin{enumerate}[i)]
\item Očitno je $\widetilde{f}$ $\R$-linearen, saj je tak tudi $f$.
Velja pa
\[
\widetilde{f}(ix) = f(ix) - if(-x) =
i \cdot (f(x)-if(ix)) = i \widetilde{f}(x).
\]
\item Naj bo $g(x) = f(x) + i \cdot h(x)$ za nek $\R$-linearen
funkcional $h \colon X \to \R$. Velja torej
\[
f(ix) + ih(ix) = g(ix) = if(x) - h(x),
\]
zato je $h(x) = -f(ix)$ in $g(x) = \widetilde{f}(x)$.
\item Denimo najprej, da je $\abs{\widetilde{f}(x)} \leq p(x)$.
Velja
\[
\abs{f(x)} = \abs{\Re \widetilde{f}(x)} \leq
\abs{\widetilde{f}(x)} \leq p(x).
\]
Če pa je $\abs{f(x)} \leq p(x)$, pa za nek $\lambda \in \C$,
$\abs{\lambda} = 1$, dobimo
\[
\abs{\widetilde{f}(x)} = \widetilde{f}(\lambda x) =
\Re \widetilde{f}(\lambda x) = f(\lambda x) \leq p(x).
\]
\item Po prejšnji točki za $c \geq 0$ velja
\[
\forall x \in X \colon \abs{f(x)} \leq c \norm{x} \iff
\forall x \in X \colon \abs{\widetilde{f}(x)} \leq c \norm{x}.
\qedhere
\]
\end{enumerate}
\end{proof}

\begin{izrek}[Hahn-Banach]\index{Izrek!Hahn-Banach!Razširitveni}
\label{iz:hb:2}
Naj bo $X$ vektorski prostor nad $\K$ in $Y \leq X$, $p$ pa
polnorma na $X$. Če je $f \colon Y \to \K$ linearen funkcional, za
katerega za vse $y \in Y$ velja $\abs{f(y)} \leq p(y)$, obstaja
linearen funkcional $F \colon X \to \K$, za katerega je
$\eval{F}{Y}{} = f$ in za vse $x \in X$ velja
$\abs{F(x)} \leq p(x)$.
\end{izrek}

\begin{proof}
Če je $\K = \R$, lahko uporabimo izrek~\ref{iz:hb:1}. Za dobljen
funkcional velja $F(x) \leq p(x)$, a je tudi
\[
-F(x) = F(-x) \leq p(-x) = p(x),
\]
zato je $\abs{F(x)} \leq p(x)$.

Naj bo sedaj $\K = \C$ in $f_1 = \Re f$. Po lemi~\ref{lm:hb:1} in
prejšnji točki tega dokaza obstaja $\R$-linearna razširitev
$F_1 \colon X \to \R$ funkcionala $f_1$. Sedaj preprosto vzamemo
$F = \widetilde{F}_1$.
\end{proof}

\begin{izrek}[Hahn-Banach]\index{Izrek!Hahn-Banach!Razširitveni}
Naj bo $X$ normiran prostor nad $\K$, $Y \leq X$ in
$f \colon Y \to \K$ omejen linearen funkcional. Tedaj obstaja
omejen linearen funkcional $F \colon X \to \K$, za katerega je
$\eval{F}{Y}{} = f$ in je $\norm{F} = \norm{f}$.
\end{izrek}

\begin{proof}
Naj bo $p(x) = \norm{f} \cdot \norm{x}$. Velja
$\abs{f(y)} \leq \norm{f} \cdot \norm{x} = p(x)$, zato po
izreku~\ref{iz:hb:2} obstaja razširitev $F \colon X \to \K$, za
katero je
\[
\abs{F(x)} \leq p(x) = \norm{f} \cdot \norm{x}.
\]
Sledi $\norm{F} \leq \norm{f}$, ker pa je $F$ razširitev $f$, velja
tudi $\norm{F} \geq \norm{f}$.
\end{proof}

\begin{posledica}\label{ps:hb:1}
Naj bo $X$ normiran prostor in $x \in X$. Tedaj je
\[
\norm{x} = \sup \setb{\abs{f(x)}}{f \in X^* \land \norm{f} \leq 1}.
\]
Ta supremum je tudi dosežen.
\end{posledica}

\begin{proof}
Naj bo $Y = \K x \leq X$. Na $Y$ definiramo linearni funkcional
$q \colon Y \to \K$ s predpisom $q(\lambda x) = \lambda \norm{x}$.
Po Hahn-Banachovem izreku lahko $q$ razširimo do funkcionala
$f \in X^*$ z normo $\norm{f} = \norm{q} = 1$. Za $f$ torej velja
$f(x) = \norm{x}$. Velja pa tudi
\[
\abs{f(x)} \leq \norm{f} \cdot \norm{x} \leq \norm{x}. \qedhere
\]
\end{proof}

\begin{posledica}
Dualni prostor $X^*$ loči točke normiranega prostora $X$.
\end{posledica}

\begin{proof}
Po prejšnji posledici obstaja $f \in X^*$, za katerega je
$f(x-y) = \norm{x-y} \ne 0$.
\end{proof}

\begin{posledica}
Naj bo $Y$ zaprt podprostor normiranega prostora $X$ in
$x_0 \in X \setminus Y$ z $d = d(x_0, Y) > 0$. Potem obstaja
linearen funkcional $f \in X^*$, za katerega je $f(x_0) = 1$,
$\eval{f}{Y}{} = 0$ in $\norm{f} = \frac{1}{d}$.
\end{posledica}

\begin{proof}
V kvocientnem prostoru $\kvoc{X}{Y}$ velja $\norm{x_0 + Y} = d$. Po
Hahn-Banachovem izreku obstaja $g \in \left(\kvoc{X}{Y}\right)^*$,
za katerega je $\norm{g} = 1$ in $g(x_0 + Y) = d$. Naj bo
$\pi \colon X \to \kvoc{X}{Y}$ kvocientna projekcija in
$f = \frac{1}{d} g \circ \pi$. Ni težko videti, da je $f$ linearen
funkcional, za katerega je $f(x_0) = 1$ in $\eval{f}{Y}{} = 0$.
Ni težko videti, da je
\[
\norm{f} \leq
\frac{1}{d} \norm{g} \cdot \norm{\pi} \leq \frac{1}{d}.
\]
Ker je $\norm{g} = 1$, obstaja zaporedje $(x_n)_{n=1}^\infty$ v
$X$, za katerega je $\norm{x + Y} < 1$
\[
\lim_{n \to \infty} g(x_n + Y) = 1.
\]
Naj bo $y_n \in Y$ tak, da je $\norm{x_n + y_n} < 1$. Sledi, da je
\[
\abs{f(x_n + y_n)} = \frac{1}{d} \abs{g(x_n + y_n + Y)} =
\frac{1}{d} \abs{g(x_n + Y)},
\]
kar limitira proti $\frac{1}{d}$, zato je tudi
$\norm{f} \geq \frac{1}{d}$.
\end{proof}

\begin{izrek}
Naj bo $Y$ podprostor normiranega prostora $X$. Potem je
\[
\oline{Y} =
\bigcap_{\substack{f \in X^* \\ Y \subseteq \ker f}} \ker f.
\]
\end{izrek}

\begin{proof}
Ker je $\ker f \leq X$ zaprt, je zgornji presek zaprt in je
$\oline{Y}$ vsebovan v njem. Naj bo sedaj $x_0 \not \in \oline{Y}$.
Sledi, da je $d(x_0, Y) > 0$, zato po zgornji posledici obstaja
funkcional $f \in X^*$, za katerega je $Y \in \ker f$ in
$f(x_0) = 1$, zato $x_0$ ni element zgornjega preseka.
\end{proof}

\newpage

\subsection{Geometrijske posledice Hahn-Banachovega izreka}

\datum{2022-10-20}

\begin{definicija}
Naj bo $X$ vektorski prostor nad $\R$ in $A \subseteq X$. Točka
$x_0 \in A$ je \emph{notranja točka}\index{Notranjost}, če za vsak
$y \in X$ obstaja tak $\varepsilon > 0$, da je
$x_0 + t \cdot y \in A$ za vse $t \in (-\varepsilon, \varepsilon)$.
\end{definicija}

\begin{definicija}
Naj bo $K$ konveksna podmnožica realnega vektorskega prostora $X$,
pri čemer je $0 \in K$ notranja točka.
\emph{Funkcional Minkowskega}\index{Funkcional!Minkowskega} $p_K$
za $K$ je funkcija
\[
p_K(x) = \inf \setb{a > 0}{\frac{x}{a} \in K}.
\]
\end{definicija}

\begin{zgled}
Če je $K$ enotska krogla v normiranem prostoru $X$, je
\[
p_K(x) = \norm{x}.
\]
\end{zgled}

\begin{trditev}
Funkcional Minkowskega je sublinearen funkcional.
\end{trditev}

\begin{proof}
Homogenost je očitna. Naj bosta $x, y \in X$, $a, b > 0$ pa taki
števili, da velja $\frac{x}{a}, \frac{y}{b} \in K$. Tedaj je
\[
\frac{x+y}{a+b} =
\frac{a}{a+b} \cdot \frac{x}{a} + \frac{b}{a+b} \cdot \frac{y}{b},
\]
kar je po konveksnosti $K$ spet element $K$. Sledi
\[
p_K(x+y) \leq a + b,
\]
zato je funkcional tudi subaditiven.
\end{proof}

\begin{trditev}
Če je $x \in K$, je $p_K(x) \leq 1$. Točka $x \in X$ je notranja za
$K$ natanko tedaj, ko je $p_K(x) < 1$.
\end{trditev}

\begin{proof}
Prvi del trditve je očiten. Če je $x$ notranja točka, obstaja tak
$\varepsilon > 0$, da je $x + \varepsilon x \in K$. Sledi, da je
$(1+\varepsilon) p_K(x) = p_K(x + \varepsilon x) \leq 1$.

Če je $p_K(x) < 1$, obstaja tak $a \in (0, 1)$, da je
$\frac{x}{a} \in K$. Posebej je $x \in K$. Za poljuben $y \in X$
obstaja tak $\varepsilon > 0$, da je
$p_K(x) + \varepsilon p_K(\pm y) < 1$. Sledi, da za
$t \in (-\varepsilon, \varepsilon)$ velja
\[
p_K(x \pm t y) < 1,
\]
zato je $x \pm \varepsilon y \in K$.
\end{proof}

\begin{opomba}
Če so vse točke $K$ notranje, je
\[
K = \setb{x \in X}{p_K(x) < 1}.
\]
\end{opomba}

\begin{posledica}
Naj bo $p$ sublinearen funkcional na realnem vektorskem prostoru
$X$. Tedaj je množica
\[
\setb{x \in X}{p(x) < 1}
\]
konveksna množica, pri čemer so vse njene točke notranje. Tudi
\[
\setb{x \in X}{p(x) \leq 1}
\]
je konveksna množica.
\end{posledica}

\begin{definicija}
Naj bo $f$ neničeln linearen funkcional na realnem vektorskem
prostoru $X$ in $c \in \R$.

\begin{enumerate}[i)]
\item \emph{Hiperravnina}\index{Vektorski prostor!Hiperravnina} je
množica $f^{-1}(c)$.
\item Množica $\setb{x \in X}{f(x) < c}$ je
\emph{odprt polprostor}\index{Vektorski prostor!Polprostor}
\item Množica $\setb{x \in X}{f(x) \leq c}$ je
\emph{zaprt polprostor}.
\end{enumerate}
\end{definicija}

\begin{izrek}[Hahn-Banach]\index{Izrek!Hahn-Banach!Separacijski}
Naj bo $K$ konveksna množica, ki ima samo notranje točke. Potem
lahko vsak $y \not \in K$ ločimo od $K$ s hiperravnino -- obstajata
linearni funkcional $f \colon X \to \R$ in $c \in \R$, pri čemer za
vsak $x \in K$ velja
\[
f(x) < f(y) = c.
\]
\end{izrek}

\begin{proof}
Brez škode za splošnost naj bo $0 \in K$ notranja točka. Tedaj za
vse $x \in K$ velja $p_K(x) < 1$. Sedaj definiramo linearen
funkcional $f \colon \R y \to \R$, ki deluje po predpisu
$\lambda y \to \lambda$. Tedaj velja
$f(\lambda y) \leq p_K(\lambda y)$ -- za negativne $\lambda$ je
neenakost očitna, sicer pa uporabimo homogenost. Sedaj lahko po
Hahn-Banachovem razširitvenem izreku $f$ razširimo do linearnega
funkcionala na $X$, ki je omejen s $p_K$.
\end{proof}

\begin{opomba}
Dovolj je že, da ima $K$ eno notranjo točko, pri čemer zgornja
neenakost ni stroga.
\end{opomba}

\begin{izrek}
Naj bosta $A$ in $B$ disjunktni konveksni podmnožici realnega
vektorskega prostora $X$, pri čemer ima vsaj ena notranjo točko.
Potem ju lahko ločimo s hiperravnino -- obstaja tak neničeln
linearen funkcional $f \colon X \to \R$ in $c \in \R$, da za vse
$a \in A$ in $b \in B$ velja
\[
f(a) \leq c \leq f(b).
\]
\end{izrek}

\begin{proof}
Množica
\[
K = A-B = \setb{a-b}{A \in A, b \in B}
\]
je konveksna, velja pa $0 \not \in K$. Obstaja torej tak neničeln
linearen funkcional $f \colon X \to \R$, da za vsak $x \in K$ velja
\[
f(x) \leq f(0) = 0. \qedhere
\]
\end{proof}

\newpage

\subsection{Separacija v normiranih prostorih}

\begin{lema}
Naj bo $X$ normiran prostor in $K \subseteq X$ odprta konveksna
podmnožica, za katero je $0 \in K$. Potem obstaja tak $M > 0$, da
je
\[
p_K(x) \leq M \cdot \norm{x}.
\]
\end{lema}

\begin{proof}
Naj bo $B = \spr(0, r)$ vsebovana v $K$. Sledi, da je
$p_K \leq p_B = \frac{1}{r} \cdot \norm{x}$.
\end{proof}

\begin{izrek}[Hahn-Banach]\index{Izrek!Hahn-Banach!Separacijski}
Naj bo $X$ normiran prostor in $K \subseteq X$ odprta konveksna
podmnožica. Potem za vsak $y \not \in K$ obstajata taka $f \in X^*$
in $c \in \R$, da za vse $x \in K$ velja
\[
f(x) < f(y) = c.
\]
\end{izrek}

\begin{proof}
Brez škode za splošnost naj bo $0 \in K$. Obstaja torej linearen
funkcional $f \colon X \to \R$, ki strogo loči $y$ od $K$ in za vse
$x \in X$ velja
\[
f(x) \leq p_K(x).
\]
Po zgornji lemi je $f$ omejen.
\end{proof}

\begin{izrek}
Naj bo $X$ normiran prostor, $U, V \subseteq X$ pa disjunktni
konveksni množici. Če je $U$ zaprta in $V$ kompaktna, obstajajo tak
linearen funkcional $f \in X^*$ in $\alpha_1 < \alpha_2$, da za vse
$u \in U$ in $v \in V$ velja
\[
f(u) \leq \alpha_1 < \alpha_2 \leq f(v).
\]
\end{izrek}
