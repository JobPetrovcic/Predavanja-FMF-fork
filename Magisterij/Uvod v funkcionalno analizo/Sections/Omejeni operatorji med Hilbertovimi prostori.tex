\section{Omejeni operatorji med Hilbertovimi prostori}

\epigraph{">Po kratkem razmisleku mislim da že nekaj časa nisi bil
na tabli, tako da lahko nadaljuješ svoj crossfit
trening."<}{-- asist.~Tea Štrekelj}

\subsection{Osnovni pojmi}

\begin{definicija}
Naj bosta $a < b$ realni števili in $k \in \mathcal{C}([a,b]^2)$.
\emph{Integralski operator}\index{Operator!Integralski} je
preslikava $K \colon \mathcal{C}([a,b]) \to \mathcal{C}([a,b])$, ki
deluje po predpisu
\[
(Kf)(x) = \int_a^b k(x,y) f(y)\,dy.
\]
\end{definicija}

\begin{trditev}
Integralski operator je omejen.
\end{trditev}

\begin{proof}
Po Cauchyjevi neenakosti je
\begin{align*}
\abs{(Kf)x} &= \abs{\int_a^b k(x,y) f(y)\,dy}
\\
&\leq
\int_a^b \abs{k(x,y)} \cdot \abs{f(y)}\,dy
\\
&\leq
\norm{k}_\infty \cdot \int_a^b \abs{f(y)}\,dy
\\
&\leq
\norm{k}_\infty \cdot \norm{1}_2 \cdot \norm{f}_2.
\end{align*}
Sledi, da je
\[
\norm{Kf}^2 = \int_a^b \abs{(Kf)(x)}^2\,dx \leq
\norm{k}_\infty^2 \cdot (b-a)^2 \cdot \norm{f}_2,
\]
zato je $\norm{K} \leq \norm{k}_\infty \cdot (b-a)$.
\end{proof}

\newpage

\subsection{Adjungirani operator}

\begin{definicija}
Naj bosta $H$ in $K$ Hilbertova prostora. Preslikava
$u \colon H \times K \to \K$ je
\emph{seskvilinearna forma}\index{Seskvilinearna forma}, če je
linearna v prvi in poševno linearna v drugi komponenti.
\end{definicija}

\begin{definicija}
Seskvilinearna forma je
\emph{omejena}\index{Seskvilinearna forma!Omejena}, če obstaja tak
$M > 0$, da za vse $x \in H$ in $y \in K$ velja
\[
\abs{u(x,y)} \leq M \cdot \norm{x} \cdot \norm{y}.
\]
\end{definicija}

\begin{zgled}
Za $A \in \mathcal{B}(H,K)$ je $u(x,y) = \skl{Ax,y}$ omejena
seskvilinearna forma. Res, po Cauchyjevi neenakosti velja
\[
\abs{u(x,y)} = \abs{\skl{Ax,y}} \leq
\norm{A} \cdot \norm{x} \cdot \norm{y}. \qedhere
\]
\end{zgled}

\datum{2022-11-17}

\begin{izrek}
Naj bo $u \colon H \times K \to \K$ omejena seskvilinearna forma.
Tedaj obstajata enolična $A \in \mathcal{B}(H,K)$ in
$B \in \mathcal{B}(K,H)$, za katera za vsaka $x \in H$ in $y \in K$
velja
\[
u(x,y) = \skl{Ax,y} = \skl{x,By}.
\]
\end{izrek}

\begin{proof}
Naj bo $f_x \colon K \to \K$ preslikava, ki deluje po predpisu
$f_x(y) = \oline{u(x,y)}$. Tedaj je $f_x$ linearen funkcional,
velja pa
\[
\abs{f_x(y)} = \abs{u(x,y)} \leq M \cdot \norm{x} \cdot \norm{y},
\]
zato je funkcional omejen. Po Rieszovem izreku obstaja tak
$z \in K$, da je $f_x(y) \equiv \skl{y,z}$ in velja
$\norm{f_x} = \norm{z}$. S predpisom $Ax = z$ je tako podana
preslikava $A \colon H \to K$.

Preslikava $A$ je očitno linearna, velja pa
\[
\norm{Ax} = \norm{z} = \norm{f_x} \leq M \cdot \norm{x},
\]
zato je $A$ omejen. Dobimo
\[
\skl{y,Ax} = \skl{y,z} = f_x(y) = \oline{u(x,y)}.
\]
Denimo sedaj, da je $u(x,y) = \skl{A_1x,y}$. Sledi, da za vse $x$
in $y$ velja
\[
\skl{Ax,y} = \skl{A_1x,y},
\]
zato je $A_1 = A$.

Z zgornjim dokazom za $v(x,y) = \oline{u(y,x)}$ dobimo še
preslikavo $B$.
\end{proof}

\begin{definicija}
Naj bo $A \in \mathcal{B}(H,K)$. Operatorju
$B \in \mathcal{B}(K,H)$, za katerega za vse $x \in H$ in $y \in K$
velja
\[
\skl{Ax,y} = \skl{x,By},
\]
pravimo \emph{adjungirani operator}\index{Operator!Adjungirani} in
ga označimo z $A^*$.
\end{definicija}

\begin{opomba}
Po zgornjem izreku obstaja natanko en adjungiran operator.
\end{opomba}

\begin{trditev}
Preslikava $U \in \mathcal{B}(H,K)$ je izomorfizem natanko tedaj,
ko je obrnljiv z inverzom $U^*$.
\end{trditev}

\begin{proof}
Če je $U$ obrnljiv, je seveda surjektiven. Če je $U^* = U^{-1}$, pa
velja še
\[
\skl{Ux,Uy} = \skl{x,U^*Uy} = \skl{x,y}.
\]
Če je $U$ izomorfizem, je bijektiven in zato obrnljiv, za vse
$x,y \in H$ pa velja
\[
\skl{x,y} = \skl{Ux,Uy} = \skl{x,U^*Uy},
\]
zato je $U^*U = I$ in $U^* = U^{-1}$.
\end{proof}

\begin{trditev}
Operacija $^*$ je involucija na $\mathcal{B}(H)$ -- za vse
$A, B \in \mathcal{B}(H)$ in $\alpha, \beta \in \K$ velja

\begin{enumerate}[i)]
\item
$(\alpha A + \beta B)^* = \oline{\alpha} A^* + \oline{\beta} B^*$,
\item $(AB)^* = B^* A^*$,
\item $(A^*)^* = A$ in
\item če je $A$ obrnljiv, je obrnljiv tudi $A^*$ in velja
$(A^*)^{-1} = (A^{-1})^*$.
\end{enumerate}
\end{trditev}

\obvs

\begin{zgled}
Naj bo $A \in \mathcal{B}(\ell^2)$ in $(e_n)_n$ kompleten
ortonormiran sistem za $\ell^2$. Če ima $A$ ">neskončno matriko"<
$(a_{i,j})_{i,j \in \N}$, ima $A^*$ neskončno matriko
$(\oline{a}_{j,i})_{i,j \in \N}$.
\end{zgled}

\begin{izrek}\label{iz:ker_im:1}
Naj bo $A \in \mathcal{B}(H,K)$. Tedaj velja
$\ker(A^*) = (\im A)^\bot$.
\end{izrek}

\begin{proof}
Velja
\[
x \in \ker(A^*) \iff A^*x = 0 \iff
\forall y \in H \colon \skl{A^*x,y} = 0 \iff
\forall y \in H \colon \skl{x,y} = 0. \qedhere
\]
\end{proof}

\begin{posledica}
Za $A \in \mathcal{B}(H,K)$ velja
$(\ker A)^\bot = \oline{\im(A^*)}$.
\end{posledica}

\begin{proof}
Velja $\ker A = (\im(A^*))^\bot$.
\end{proof}

\begin{definicija}
Operator $A \in \mathcal{B}(H)$ je

\begin{enumerate}[i)]
\item
\emph{sebiadjungiran}\index{Operator!Sebiadjungiran},\footnote{Tudi
\emph{hermitski}.} če velja $A^* = A$,
\item \emph{normalen}\index{Operator!Normalen}, če velja
$AA^* = A^*A$ in
\item \emph{unitaren}\index{Operator!Unitaren}, če velja
$AA^* = I = A^*A$.
\end{enumerate}
\end{definicija}

\begin{opomba}
Operator $A$ je sebiadjungiran natanko tedaj, ko za vse
$x, y \in H$ velja $\skl{Ax,y} = \skl{x,Ay}$.
\end{opomba}

\begin{opomba}
Tako sebiadjungirani kot unitarni operatorji so normalni.
\end{opomba}

\begin{definicija}
Naj bo $H$ Hilbertov prostor nad $\C$. Za $A \in \mathcal{B}(H)$
definiramo
\[
\Re A = \frac{A + A^*}{2}
\quad \text{in} \quad
\Im A = \frac{A-A^*}{2i}.
\]
\end{definicija}

\begin{opomba}
Realni in imaginarni del operatorja sta sebiadjungirana operatorja.
\end{opomba}

\begin{trditev}
Naj bo $H$ Hilbertov prostor nad $\C$ in $A \in \mathcal{B}(H)$.
Operator $A$ je sebiadjungiran natanko tedaj, ko za vse $x \in H$
velja $\skl{Ax,x} \in \R$.
\end{trditev}

\begin{proof}
Če je $A = A^*$, velja
\[
\skl{Ax,x} = \skl{x,Ax} = \oline{\skl{Ax,x}}.
\]
Naj bosta sedaj $x, y \in H$ in $\alpha \in \C$. Tedaj velja
\[
\skl{A(\alpha x + y), \alpha x + y} =
\abs{\alpha}^2 \skl{Ax,x} + \alpha \skl{Ax,y} +
\oline{\alpha} \skl{Ay, x} + \skl{Ay,y},
\]
zato velja
$\alpha \skl{Ax,y} + \oline{\alpha} \oline{\skl{x,Ay}} \in \R$. Z
$\alpha = 1$ in $\alpha = i$ dobimo $\skl{Ax,y} = \skl{x,Ay}$.
\end{proof}

\begin{definicija}
Za operator $A \in \mathcal{B}(H)$ definiramo
\emph{numerični zaklad}\index{Operator!Numerični zaklad in radij}
\[
W(A) = \setb{\skl{Ax,x}}{x \in H \land \norm{x} = 1}.
\]
\emph{Numerični radij} operatorja $A$ je
\[
w(A) = \sup_{\norm{x} = 1} \abs{\skl{Ax,x}}.
\]
\end{definicija}

\begin{opomba}
Za normirane $x$ velja
\[
\abs{\skl{Ax,x}} \leq \norm{Ax} \cdot \norm{x} \leq \norm{A},
\]
zato je $w(A) \leq \norm{A}$.
\end{opomba}

\begin{trditev}
Naj bo $A \in \mathcal{B}(H)$ sebiadjungiran operator. Tedaj je
$w(A) = \norm{A}$.
\end{trditev}

\begin{proof}
Naj bo $x \in H$ normiran vektor. Tedaj za nek enotski vektor $y$
velja $Ax = \norm{Ax}y$ in je
\[
\norm{Ax} = \skl{Ax,y} = \skl{x,Ay}.
\]
Sledi, da je
\begin{align*}
\skl{A(x+y),x+y} - \skl{A(x-y),x-y} &=
2\skl{Ax,y} + 2\skl{Ay,x}
\\
&=
2 \skl{Ax,y} + 2 \oline{\skl{Ax,y}}
\\
&=
4 \Re \skl{Ax,y}
\\
&= 4 \skl{Ax,y}.
\end{align*}
Tako velja
\begin{align*}
\norm{Ax} &=
\frac{1}{4} \br{\skl{A(x+y),x+y} - \skl{A(x-y),x-y}}
\\
&\leq
\frac{1}{4} \br{\abs{\skl{A(x+y),x+y}} - \abs{\skl{A(x-y),x-y}}}
\\
&\leq
\frac{1}{4} \br{w(A) \norm{x+y}^2 + w(A) \norm{x-y}^2}
\\
&=
\frac{1}{2} w(A) \br{\norm{x}^2 + \norm{y}^2}
\\
&=
w(A). \qedhere
\end{align*}
\end{proof}

\begin{posledica}
Če za sebiadjungiran operator $A \in \mathcal{B}(H)$ velja
$\skl{Ax,x} \equiv 0$, je $A = 0$.
\end{posledica}

\begin{posledica}
Če je $H$ Hilbertov prostor nad $\C$, $A \in \mathcal{B}(H)$ pa
operator, za katerega je $\skl{Ax,x} \equiv 0$, velja $A = 0$.
\end{posledica}

\begin{trditev}
Za operator $A \in \mathcal{B}(H)$ so naslednje trditve
ekvivalentne:

\begin{enumerate}[i)]
\item Operator $A$ je normalen.
\item Za vsak $x \in H$ velja $\norm{Ax} = \norm{A^*x}$.
\item Velja $\Re A \cdot \Im A = \Im A \cdot \Re A$.\footnote{Ta
del trditve je ekvivalenten prejšnjima le v kompleksnih prostorih.}
\end{enumerate}
\end{trditev}

\begin{proof}
Za vsak $x \in H$ velja
\[
\norm{Ax}^2 - \norm{A^*x}^2 =
\skl{Ax,Ax} - \skl{A^*x,A^*x} =
\skl{(A^*A - AA^*)x,x}.
\]
Ker je operator $A^*A - AA^*$ sebiadjungiran, je zgornji izraz enak
$0$ za vse $x \in H$ natanko tedaj, ko je $A^*A - AA^* = 0$.

Naj bo $B = \Re A$ in $C = \Im A$. Ta operatorja sta
sebiadjungirana, velja pa
\begin{align*}
A^*A = (B-iC)(B+iC) &= B^2 + C^2 + i(BC-CB)
\intertext{in}
AA^* = (B+iC)(B-iC) &= B^2 + C^2 - i(BC-CB),
\end{align*}
zato je $A^*A = AA^* \iff BC = CB$.
\end{proof}

\datum{2022-11-23}

\begin{definicija}
Operator $A \in \mathcal{B}(H)$, kjer je $H$ Hilbertov prostor, je
\emph{pozitivno semidefiniten}\index{Operator!Pozitivno semidefiniten},
če velja $A = A^*$ in za vse $x \in H$ velja
\[
\skl{Ax,x} \geq 0.
\]
Označimo $A \geq 0$.
\end{definicija}

\begin{opomba}
Če za vse $x \ne 0$ velja $\skl{Ax,x} > 0$, pravimo, da je $A$
pozitivno definiten z oznako $A > 0$.
\end{opomba}

\begin{definicija}
Naj bo $H$ Hilbertov prostor. Na množici
\[
\mathcal{B}(H)_{\text{sa}} = \setb{A \in \mathcal{B}(H)}{A^* = A}
\]
vpeljemo delno urejenost
\[
A \geq B \iff A - B \geq 0.
\]
\end{definicija}

\begin{opomba}
Če je $\dim H \geq 2$, zgornja urejenost ni linearna.
\end{opomba}

\begin{trditev}
Veljajo naslednje trditve:

\begin{enumerate}[i)]
\item Če je $\alpha \geq 0$ in $A \geq 0$, je tudi
$\alpha A \geq 0$.
\item Če je $A \geq 0$ in $P \in \mathcal{B}(H)$, je tudi
$P^*AP \geq 0$.
\item Za vse $A \in \mathcal{B}(H)_{\text{sa}}$ velja
$-\norm{A} \cdot I \leq A \leq \norm{A} \cdot I$.
\item Za $A \in \mathcal{B}(H)$ velja $AA^* \geq 0$ in
$A^*A \geq 0$.\footnote{Tema operatorjema pravimo \emph{hermitska
kvadrata}.}
\end{enumerate}
\end{trditev}

\begin{proof}
Dokažimo tretjo točko. Za $A \in \mathcal{B}(H)_{\text{sa}}$ velja
\[
\skl{-\norm{A}Ix,x} = -\norm{A} \cdot \norm{x}^2 \leq
\skl{Ax,x} \leq \norm{A} \cdot \norm{x}^2 =
\skl{\norm{A} \cdot I x, x}. \qedhere
\]
\end{proof}

\begin{opomba}
Za vsak operator $B \geq 0$ obstaja tak $A$, da je $B = A^*A$.
\end{opomba}

\newpage

\subsection{Idempotenti in ortogonalni projektorji}

\begin{definicija}
Operator $E \in \mathcal{B}(H)$ je
\emph{idempotent}\index{Operator!Idempotent}, če je $E^2 = E$.
\end{definicija}

\begin{trditev}
Za neničeln idempotent $E \in \mathcal{B}(H)$ so naslednje trditve
ekvivalentne:

\begin{enumerate}[i)]
\item $E$ je ortogonalni projektor.
\item Velja $\norm{E} = 1$.
\item Velja $E^* = E$.
\item Operator $E$ je normalen.
\item Za vsak $x \in H$ velja $\skl{Ex,x} \geq 0$.
\end{enumerate}
\end{trditev}

\begin{proof}
Denimo, da je $E$ ortogonalni projektor. Tedaj je
$\norm{E} \leq 1$, a za vsak $x \in \im E$ velja $Ex=x$, zato je
$\norm{E} = 1$.

Denimo, da je $\norm{E} = 1$. Naj bo $x \in \br{\ker E}^\bot$. Ker
je $x - Ex \in \ker E$, sledi
\[
0 = \skl{x - Ex, x} = \norm{x}^2 - \skl{Ex,x}.
\]
Sledi, da je
\[
\norm{x}^2 = \skl{Ex,x} \leq \norm{Ex} \norm{x} \leq \norm{x}^2,
\]
zato je $\norm{x} = \norm{Ex} = \sqrt{\skl{Ex,x}}$. Sledi
\[
\norm{x-Ex}^2 = \norm{x}^2 + \norm{Ex}^2 - 2 \skl{Ex,x} = 0.
\]
Sledi, da je $\br{\ker E}^\bot \subseteq \im E$.

Za $y \im E$ lahko zapišemo $y = y_1 + y_2$ za enolično določen
$y_1 \in \ker E$ in $y_2 \in \br{\ker E}^\bot$. Sledi, da je
\[
y = Ey = E(y_1 + y_2) = Ey_2 = y_2,
\]
saj je $y_2 \in \im E$. Sledi, da je
$\im E \subseteq \br{\ker E}^\bot$. Prvi dve trditvi sta torej
ekvivalentni.

Naj bo $E$ ortogonalni projektor. Tedaj lahko zapišemo
$H = \ker E \oplus \br{\ker E}^\bot$. Za $x, y \in H$ zapišemo
$x = x_1 + x_2$ in $y = y_1 + y_2$, pri čemer velja
$x_1, y_1 \in \ker E$ in $x_2, y_2 \in \br{\ker E}^\bot$. Velja
torej
\[
\skl{E^*x,y} = \skl{x,Ey} = \skl{x_1 + x_2,y_2} = \skl{x_2, y_2}
\]
in
\[
\skl{Ex,y} = \skl{x_2,y_1 + y_2} = \skl{x_2,y_2},
\]
zato je sebiadjungiran.

Če je $E$ sebiadjungiran, je očitno normalen.

Če je $E$ normalen, za vsak $x \in H$ velja
$\norm{Ex} = \norm{E^*x}$ -- posebej, $\ker E = \ker E^*$. Po
izreku~\ref{iz:ker_im:1} sledi, da je $\ker E = \br{\ker E}^\bot$,
zato je $E$ projektor. Prve štiri trditve so torej ekvivalentne.

Če je $E$ ortogonalen projektor, sledi
\[
\norm{Ex,x} = \norm{E^2x,x} = \norm{Ex,E^*x} = \norm{Ex}^2 \geq 0.
\]
Če velja peta trditev, pa za poljubna $h_1 \in \im E$ in
$h_2 \in \ker E$ velja
\[
0 \leq \skl{E(h_1 + h_2), h_1 + h_2} =
\skl{h_1, h_1} + \skl{h_1, h_2},
\]
kar lahko velja le, če je $h_1 \perp h_2$, zato je
$\im E \perp \ker E$. Sledi, da je $H = \ker E \oplus \im E$, saj
lahko zapišemo $x = (x - Ex) + Ex$. Ker pa je
$\im E \subseteq \br{\ker E}^\bot$ in
$H = \ker E \oplus \br{\ker R}^\bot$, je
$\im E = \br{\ker E}^\bot$.
\end{proof}

\begin{posledica}
Operator $P \in \mathcal{B}(H)$ je projektor natanko tedaj, ko je
$P^2 = P = P^*$.
\end{posledica}

\begin{definicija}
Zaprt podprostor $M \leq H$ je
\emph{invariaten}\index{Vektorski prostor!Invarianten} za
$A \in \mathcal{B}(H)$, če je $A(M) \subseteq M$. Invarianten
prostor $M \leq H$ je
\emph{reducirajoč}\index{Vektorski prostor!Reducirajoč}, če je tudi
$M^\bot$ invarianten za $A$.
\end{definicija}

\begin{opomba}
Naj bo $M \leq H$ zaprt podprostor. Potem lahko vsak
$A \in \mathcal{B}(H)$ zapišemo kot $2 \times 2$ operatorsko
matriko
\[
A = \begin{pmatrix}
B & C \\
D & E
\end{pmatrix},
\]
kjer je $B \in \mathcal{B}(M)$, $C \in \mathcal{B}(M^\bot, M)$,
$D \in \mathcal{B}(M, M^\bot)$ in $E \in \mathcal{B}(M^\bot)$.
Ortogonalni projektor v tej obliki zapišemo kot
\[
P = \begin{pmatrix}
I & 0 \\
0 & 0
\end{pmatrix}.
\]
\end{opomba}

\begin{trditev}
Naj bo $A \in \mathcal{B}(H)$, $M \leq H$ zaprt podprostor in $P$
projektor na $M$. Naslednje trditve so ekvivalentne:

\begin{enumerate}[i)]
\item $M$ je invarianten za $A$.
\item Velja $PAP = AP$.
\item V zapisu iz zgornje opombe je $D = 0$.
\end{enumerate}

Ekvivalentne so tudi naslednje trditve:

\begin{enumerate}[i)]
\item $M$ reducira $A$.
\item Velja $PA = AP$.
\item V zapisu iz prejšnje opombe je $D = 0$ in $C = 0$.
\item $M$ je invarianten za $A$ in $A^*$.
\end{enumerate}
\end{trditev}

\begin{proof}
Naj bo $M$ invarianten za $A$. Za $x \in H$ je $Px \in M$, zato je
$APx \in M$ in je $PAPx = APx$.

Če je $PAP = AP$, sledi
\[
AP =
\begin{pmatrix}
B & C \\
D & E
\end{pmatrix}
\cdot
\begin{pmatrix}
1 & 0 \\
0 & 0
\end{pmatrix}
=
\begin{pmatrix}
B & 0 \\
D & 0
\end{pmatrix},
\]
velja pa
\[
PAP =
\begin{pmatrix}
B & 0 \\
0 & 0
\end{pmatrix},
\]
zato je $D = 0$.

Če je $D = 0$, za $x \in M$ velja
\[
Ax =
\begin{pmatrix}
B & C \\
0 & E
\end{pmatrix}
\cdot
\begin{pmatrix}
x \\
0
\end{pmatrix}
=
\begin{pmatrix}
Bx \\
0
\end{pmatrix}
\in M.
\]
Sledi, da so prve tri trditve ekvivalentne.

Denimo sedaj, da $M$ reducira $A$. Ker je $M$ invarianten za $A$,
sledi $PAP = AP$, ker pa je $I-P$ ortogonalni projektor na
$M^\bot$, dobimo
\[
(I-P)A(I-P) = A(I-P),
\]
od koder sledi $AP = PA$.

Če je $AP = PA$, iz bločnega množenja dobimo $D = 0$ in $C = 0$.

Če velja $D = 0$ in $C = 0$, sledi
\[
A =
\begin{pmatrix}
B & 0 \\
0 & E
\end{pmatrix}
\quad \text{in} \quad
A^* =
\begin{pmatrix}
B^* & 0   \\
0   & C^*
\end{pmatrix},
\]
zato je $M$  invarianten za $A$ in $A^*$.

Denimo, da je $M$ invarianten za $A$ in $A^*$. Za poljuben
$x \in M$ in $y \in M^\bot$ tako velja
\[
\skl{x, Ay} = \skl{A^*x, y} = 0,
\]
zato je $Ay \in M^\bot$ in je $M^\bot$ invarianten za $A$, zato $M$
reducira $A$. Sledi, da so tudi ostale trditve ekvivalentne.
\end{proof}

\begin{posledica}
Če je $A \in \mathcal{B}(H)$ in $M \leq H$ invarianten prostor za
$A$, je $M^\bot$ invarianten za $A^*$.
\end{posledica}

\begin{posledica}
Če je $A \in \mathcal{B}(H)_{\text{sa}}$, je vsak invarianten
podprostor za $A$ tudi reducirajoč.
\end{posledica}

\begin{openprob}[Invariant subspace problem]
\index{Odprt problem!Invariant subspace problem}
Ali ima vsak $A \in \mathcal{B}(H)$ kak netrivialen invarianten
prostor?
\end{openprob}

\begin{opomba}
Če je $\dim H < \infty$, je odgovor da.
\end{opomba}

\begin{opomba}
Če $H$ ni separabilen, je odgovor tudi da -- velja da je
\[
\left[\setb{A^kx}{k \in \N_0}\right] \leq H
\]
netrivialen podprostor.
\end{opomba}

\begin{opomba}
Trditev ne drži v splošnem v Banachovih prostorih.
\end{opomba}

\newpage

\subsection{Kompaktni operatorji}

\datum{2022-11-24}

\begin{definicija}
Naj bosta $X$ in $Y$ normirana prostora. Operator
$T \colon X \to Y$ je \emph{kompakten}\index{Operator!Kompakten},
če je $T(B_X)$ relativno kompaktna množica.\footnote{$B_X$ tu
označuje zaprto enotsko kroglo v $X$.}
\end{definicija}

\begin{opomba}
Ekvivalentno, za vsako omejeno zaporedje $(x_n)_n$ v $X$ ima
zaporedje $(Tx_n)_n$ konvergentno podzaporedje.
\end{opomba}

\begin{opomba}
Vsak kompakten operator je omejen.
\end{opomba}

\begin{definicija}
Množica $A \subseteq X$ je
\emph{povsem omejena}\index{Množica!Povsem omejena}, če za vsak
$\varepsilon > 0$ obstaja končno pokritje množice $A$ z odprtimi
kroglami radija $\varepsilon$.
\end{definicija}

\begin{trditev}
Če je $Y$ Banachov prostor, je operator $T \in \mathcal{B}(X,Y)$
kompakten natanko tedaj, ko je $T(B_X)$ povsem omejena.
\end{trditev}

\begin{proof}
Vsak omejen poln metričen prostor $(X,d)$ je kompakten. Naj bo
$(x_n)_n$ zaporedje. Tedaj za vsak $n \in \N$ obstaja taka končna
podmnožica $D_n \subseteq X$, da je
\[
\bigcup_{a \in D} \spr \br{a,2^{-n}} = X.
\]
Za vsak $n \in \N$ zato obstaja tak $y_n \in D_n$, da je množica
\[
A_n = \setb{n \in A_{n-1}}{x_n \in \spr \br{y_n, 2^{-n}}}
\]
neskončna, kjer je $A_0 = \N$. Za poljubno naraščajoče zaporedje
$(n_k)_k$, pri čemer je $n_k \in A_k$, je tako $\br{x_{n_k}}_k$
Cauchyjevo. Sledi, da ima vsako zaporedje Cauchyjevo podzaporedje.
\end{proof}

\begin{opomba}
Množico kompaktnih operatorjev iz $X$ v $Y$ označimo s
$\mathcal{K}(X,Y)$. Posebej označimo še
$\mathcal{K}(X) = \mathcal{K}(X,X)$.
\end{opomba}

\begin{izrek}
Naj bosta $X$ in $Y$ Banachova prostora. Tedaj je
$\mathcal{K}(X,Y)$ zaprt podprostor v $\mathcal{B}(X,Y)$ in za vse
$T \in \mathcal{K}(X,Y)$, $A \in \mathcal{B}$ in
$B \in \mathcal{B}(Y)$ velja $TA \in \mathcal{K}(X,Y)$ in
$BT \in \mathcal{K}(X,Y)$.
\end{izrek}

\begin{proof}
Naj bosta $S, T \in \mathcal{K}(X,Y)$ in $\lambda, \mu \in \K$,
$(x_n)_n$ pa omejeno zaporedje v $X$. Ker sta $S$ in $T$ kompaktna,
obstaja naraščajoče zaporedje $(n_k)_k$, za katerega sta zaporedji
$\br{Sx_{n_k}}$ in $\br{Tx_{n_k}}$ konvergentni. Sledi, da je
konvergentno tudi zaporedje $\br{(\lambda S + \mu T)\br{x_{n_k}}}$,
zato je $\mathcal{K}$ res podprostor.

Naj bo $(T_n)_n$ zaporedje v $\mathcal{K}(X,Y)$ in
$T \in \mathcal{B}(X,Y)$ limita tega zaporedja. Naj bo
$\varepsilon > 0$. Tedaj obstaja tak $n \in \N$, da je
$\norm{T - T_n} < \frac{\varepsilon}{3}$. Ker je
$T_n \in \mathcal{K}(X,Y)$, obstajajo točke
$x_1, \dots, x_m \in B_X$, za katere je
\[
T_n(B_x) \subseteq
\bigcup_{j=1}^m \spr \br{T_nx_j, \frac{\varepsilon}{3}}.
\]
Za vsak $x \in B_X$ torej obstaja tak $j$, da je
$\norm{T_nx - T_n x_j} < \frac{\varepsilon}{3}$. Sledi, da je
\[
\norm{Tx_j - Tx} =
\norm{Tx_j - T_nx_j + T_nx_j - T_nx + T_nx - T_x} <
\varepsilon
\]
in
\[
T(B_X) \subseteq \bigcup_{j=1}^m \spr(Tx_j, \varepsilon),
\]
zato je $T \in \mathcal{K}(X,Y)$ in je $\mathcal{K}(X,Y)$ zaprt.

Naj bo $(x_n)_n$ omejeno zaporedje. Potem je $(Ax_n)_n$ omejeno,
ker pa je $T$ kompakten, ima $(TAx_n)_n$ konvergentno podzaporedje.
Sledi, da je $TA$ kompakten. Podobno dobimo kompaktnost $BT$.
\end{proof}

\begin{posledica}
Prostor $\mathcal{K}(X)$ je ideal v $\mathcal{B}(X)$.
\end{posledica}

\begin{zgled}
Če ima $T \in \mathcal{B}(X,Y)$ končen rang, je
kompakten.\footnote{Definiramo $\rang T = \dim(\im T)$.} Res, velja
\[
T(B_X) \subseteq \norm{T} \cdot B_{\im T},
\]
ki pa je kompaktna, saj je $\im T$ končnorazsežen.
\end{zgled}

\begin{opomba}
Operatorje s končnim rangom med $X$ in $Y$ označimo z
$\mathcal{F}(X,Y)$. Posebej označimo
$\mathcal{F}(X) = \mathcal{F}(X,X)$.
\end{opomba}

\begin{opomba}
Velja, da je $\mathcal{F}(X,Y)$ podprostor v $\mathcal{K}(X,Y)$ in
$\mathcal{F}(X) \edn \mathcal{B}(X)$.
\end{opomba}

\begin{zgled}
Če je $\dim X < \infty$ ali $\dim Y < \infty$, je
$\mathcal{F}(X,Y) = \mathcal{K}(X,Y) = \mathcal{B}(X,Y)$.
\end{zgled}

\begin{zgled}
Če je $\dim X = \infty$, potem identiteta ni kompakten operator.
Zaporedje ortononrmiranih vektorjev je namreč omejeno, nima pa
konvergentnega podzaporedja.
\end{zgled}

\begin{opomba}
Če je $H$ separabilen Hilbertov prostor, je $\mathcal{K}(H)$
maksimalen ideal v $\mathcal{B}(H)$.
\end{opomba}

\begin{trditev}
Naj bo $X$ normiran, $Y$ pa Banachov prostor. Potem je za vsak
$A \in \mathcal{K}(X,Y)$ tudi razširitev
$\oline{A} \colon \oline{X} \to Y$ kompaktna.
\end{trditev}

\begin{proof}
Naj bo $(x_n)_n$ omejeno zaporedje v $\oline{X}$. Tedaj obstaja
tako omejeno zaporedje $(y_n)_n$ v $X$, da je
\[
\norm{x_n - y_n} < \frac{1}{n}.
\]
Ker je $A$ kompakten, ima zaporedje $\br{Ay_n}_n$ konvergentno
podzaporedje $\br{Ay_{n_j}}_j$ z limito $y$. Sledi, da je
\[
\norm{\oline{A}x_{n_j} - y} \leq
\norm{\oline{A}x_{n_j} - \oline{A}y_{n_j}} +
\norm{\oline{A}y_{n_j} - y} \leq
\norm{\oline{A}} \cdot \frac{1}{n_j} + \norm{Ay_{n_j} - y},
\]
kar konvergira k $0$, zato ima $\br{\oline{A}x_n}_n$ konvergentno
podzaporedje.
\end{proof}

\begin{definicija}
Družina funkcij $F$ je
\emph{enakozvezna}\index{Preslikava!Enakozvezne}, če za vsak
$\varepsilon > 0$ obstaja tak $\delta > 0$, da za vse $f \in F$
in $x,y \in \R$, za katere je $\abs{x-y} < \delta$, velja
\[
\abs{f(x) - f(y)} < \varepsilon.
\]
\end{definicija}

\begin{izrek}[Arzelà-Ascoli]\index{Izrek!Arzelà-Ascoli}
Naj bo $(f_n)_n \subseteq \mathcal{C}([a,b])$ enakozvezno
zaporedje. Če je $(f_n)_n$ omejena množica v
$(\mathcal{C}([a,b]), \norm{.}_\infty)$, je relativno kompaktna.
\end{izrek}

\begin{proof}
Naj bo $(x_n)_n$ množica vseh racionalnih števil v $[a,b]$. Ker je
$(f_n(x_1))_n$ omejena v $\K$, obstaja tako podzaporedje
$(f_n^{(1)})_n$, da zaporedje $(f_n^{(1)}(x_1))_n$ konvergira.
Induktivno tako za vsak $j$ najdemo zaporedje $(f_n^{(j)})_n$, za
katero zaporedja $(f_n^{(j)}(x_i))_n$ konvergirajo za vsak
$i \leq j$. Tedaj zaporedje $\widetilde{f}_n = f_n^{(n)}$
konvergira za vsak $x_j$.

Naj bo $\varepsilon > 0$. Po enakozveznosti obstaja tak
$\delta > 0$, da iz $\abs{x-y} < \delta$ sledi
\[
\abs{f_n(x) - f_n(y)} < \frac{\varepsilon}{3}
\]
za vse $n \in \N$. Ker je $[a,b]$ kompakten, obstaja tak
$p \in \N$, da je
\[
[a,b] \subseteq \bigcup_{j=1}^p \br{x_j - \delta, x_j + \delta}.
\]
Za vse dovolj velike $n$ in $m$ velja
\[
\abs{\widetilde{f}_m(x_j) - \widetilde{f}_n(x_j)} <
\frac{\varepsilon}{3}.
\]
Za poljuben $x \in [a,b]$ tako obstaja tak $j$, da je
\[
\abs{\widetilde{f}_n(x) - \widetilde{f}_m(x)} \leq
\abs{\widetilde{f}_n(x) - \widetilde{f}_m(x_j)} +
\abs{\widetilde{f}_m(x_j) - \widetilde{f}_n(x_j)} +
\abs{\widetilde{f}_n(x_j) - \widetilde{f}_n(x)} <
\varepsilon.
\]
Sledi, da je zaporedje $\br{\widetilde{f}_n}_n$ Cauchyjevo.
\end{proof}

\begin{trditev}
Naj bo $(f_n)_n$ zaporedje zvezno odvedljivih funkcij na $[a,b]$.
Če je
\[
M = \sup_n \norm{f_n'}_\infty < \infty,
\]
je zaporedje enakozvezno.
\end{trditev}

\begin{proof}
Izberemo $\delta = \frac{\varepsilon}{M}$. Tedaj je
\[
\abs{f_n(x) - f_n(y)} = \abs{f_n'(z) \cdot (x - y)} \leq
M \cdot \delta = \varepsilon. \qedhere
\]
\end{proof}

\begin{zgled}
Obrat ne drži -- zaporedje $f_n(x) = \frac{1}{n} \sin(n^2 \pi x)$
na $[0,1]$ je enakozvezno, a odvodi niso omejeni.
\end{zgled}

\begin{trditev}
Integralski operator
\[
(Kf)(x) = \int_a^b k(x,y) f(y)\,dy,
\]
kjer je $k \in \mathcal{C} \br{[a,b]^2}$, je kompakten na
$(\mathcal{C}([a,b]), \norm{.}_2)$ in
$(\mathcal{C}([a,b]), \norm{.}_\infty)$.
\end{trditev}

\begin{proof}
Opazimo, da je $k$ enakomerno zvezen v prvi spremenljivki. Sledi,
da je
\[
\abs{(Kf)(x) - (Kf)(y)} \leq
\int_a^b \abs{k(x,z) - k(y,z)} \cdot \abs{f(z)}\,dz \leq
\varepsilon \cdot \int_a^b \abs{f(z)}\,dz \leq
\varepsilon \cdot \sqrt{b-a} \cdot \norm{f}_2.
\]
Naj bo $(f_n)_n$ omejeno zaporedje v
$(\mathcal{C}([a,b]), \norm{.}_2)$. Po zgornji oceni je zaporedje
$(Kf_n)_n$ enakozvezno. Ker je
\[
\norm{Kf_n}_\infty \leq
\sup_{x \in [a,b]} \int_a^b \abs{k(x,y)} \cdot \abs{f(y)}\,dy \leq
\norm{k}_\infty \cdot \int_a^b \abs{f(y)}\,dy \leq
\norm{k}_\infty \cdot \sqrt{b-a} \cdot \norm{f_n}_2,
\]
je tudi omejeno. Sledi, da je $K$ kompakten.
\end{proof}

\datum{2022-11-30}

\begin{izrek}
Naj bosta $H$ in $K$ Hilbertova prostora. Za operator
$T \in \mathcal{B}(H,K)$ so naslednje trditve
ekvivalentne:\footnote{Prvi dve trditvi sta ekvivalentni tudi v
Banachovih prostorih -- ekvivalenci pravimo \emph{Schauderjev
izrek}.}

\begin{enumerate}[i)]
\item $T$ je kompakten.
\item $T^*$ je kompakten.
\item Obstaja zaporedje operatorjev
$(T_n)_n \subseteq \mathcal{F}(H,K)$, ki konvergirajo k $T$.
\end{enumerate}
\end{izrek}

\begin{proof}
Ker je $\mathcal{F}(H,K) \subseteq \mathcal{K}(H,K)$, kompaktni
operatorji pa so zaprti, je zaprtje $\mathcal{F}(H,K)$ vsebovano v
kompaktnih operatorjih. Sledi, da so vse limite operatorjev končnih
rangov kompaktne.

Denimo, da je $T$ kompakten. Tedaj je množica $\oline{T(B_H)}$
kompaktna. Sledi, da je tudi separabilna, saj jo lahko za vsak $n$
pokrijemo s končno mnogo kroglami z radijem $\frac{1}{n}$. Sledi,
da je
\[
\oline{\im T} \subseteq \Q_{\geq 0} \cdot \oline{T(B_H)}.
\]
Tudi $\oline{\im T}$ je separabilen Hilbertov prostor. Naj bo
$(e_n)_n$ kompleten ortonormiran sistem za $\oline{\im T}$,
$P_n \in \mathcal{B}(K)$ pa projektor na
$\Lin \setb{e_i}{i \leq n}$. Tedaj so $P_n \in \mathcal{F}(K)$,
zato so operatorji $T_n = P_n T \in \mathcal{F}(H,K)$.

Pokažimo, da $T_n$ po točkah konvergirajo k $T$, oziroma, da za
vsak $y \in \oline{\im T}$ velja
\[
\lim_{n \to \infty} \norm{P_n y - y} = 0.
\]
Velja pa
\[
\lim_{n \to \infty} \norm{P_n y - y} =
\lim_{n \to \infty} \norm{\sum_{k=n+1}^\infty \skl{y,e_k}e_k} = 0.
\]
Ker je $T$ kompakten, za vsak $\varepsilon > 0$ obstajajo take
točke $x_1, \dots, x_m \in B_H$, da je
\[
T(B_H) \subseteq
\bigcup_{j=1}^m \spr \br{Tx_j, \frac{\varepsilon}{3}}.
\]
Za poljuben $x \in B_H$ lahko tako izberemo tak $x_j$, da je
\[
\norm{Tx - Tx_j} < \frac{\varepsilon}{3}.
\]
Ker $T_n$ konvergira po točkah, obstaja tak $n$, da za vse $i$
velja
\[
\norm{Tx_i - T_nx_i} < \frac{\varepsilon}{3}.
\]
Sledi, da je
\[
\norm{Tx - T_nx} \leq
\norm{Tx - Tx_j} + \norm{Tx_j - T_nx_j} + \norm{T_nx_j - T_nx} <
\frac{2 \varepsilon}{3} + \norm{P_n} \cdot \norm{Tx_j - Tx} <
\varepsilon.
\]
Pokažimo še ekvivalenco prvih dveh točk -- denimo, da je $T$
kompakten. Velja
\[\norm{T^*x}^2 = \skl{T^*x,T^*x} =
\skl{TT^*x,x} \leq \norm{TT^*x} \cdot \norm{x}.
\]
Ker je $T$ kompakten, je kompakten tudi $TT^*$. Za vsako zaporedje
$(x_n)_n \subseteq B_K$ zaporedje $(TT^*x_n)_n$ tako vsebuje
konvergentno podzaporedje $\br{TT^*x_{n_k}}_k$. Velja pa
\[
\norm{T^*x_{n_k} - T^*x_{n_l}}^2 \leq
\norm{TT^* \br{x_{n_k}-x_{n_l}}} \cdot \norm{x_{n_k}-x_{n_l}} \leq
2 \norm{TT^* \br{x_{n_k}-x_{n_l}}}.
\]
Zaporedje $\br{T^*x_{n_k}}_k$ je zato Cauchyjevo in konvergentno.
\end{proof}

\begin{posledica}
Če je $T \in \mathcal{K}(H,H')$, je $\oline{\im T}$ separabilen
prostor v $H'$. Če je $(e_n)_n$ kompleten ortonormiran sistem za
$\oline{\im T}$, $P_n$ pa ortogonalni projektor na prvih $n$
komponent, zaporedje $(P_nT)_n$ konvergira k $T$.
\end{posledica}

\begin{zgled}[Diagonalni operator]
Naj bo $H$ separabilen Hilbertov prostor z bazo $(e_n)_n$. Naj bo
$\alpha = (\alpha_n)_n \in \ell^\infty$. S predpisom
\[
Ae_n = \alpha_n e_n
\]
je definiran enoličen operator $A \in \mathcal{B}(H)$ in je
\[
\norm{A} = \sup_n \abs{\alpha_n}.
\]
Operator $A$ je kompakten natanko tedaj, ko je $\alpha \in c_0$.
\end{zgled}

\begin{proof}
Naj bo $A_n = A - P_n A$. Potem je
\[
A_n e_k =
\begin{cases}
0,            & k \leq n \\
\alpha_k e_k, & k > n.
\end{cases}
\]
Sledi, da je $A_n$ diagonalen operator in
\[
\norm{A_n} = \sup_{k > n} \abs{\alpha_n}.
\]
Če je $\alpha \in c_0$, norme $\norm{A_n}$ konvergirajo proti $0$,
zato operatorji $P_n A$ konvergirajo k $A$, zato je $A$ kompakten.

Če je $A$ kompakten, velja
\[
\norm{A_n} = \norm{A_n^*} = \norm{A^* - A^* P_n},
\]
zato $\norm{A_n}$ konvergirajo proti $0$ zaradi kompaktnosti $A^*$.
Sledi, da $\alpha_n$ konvergirajo proti $0$.
\end{proof}
