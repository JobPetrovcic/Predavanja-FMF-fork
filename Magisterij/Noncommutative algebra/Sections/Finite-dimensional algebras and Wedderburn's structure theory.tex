\section{Finite-dimensional algebras, Wedderburn's structure theory}

\subsection{Free algebras}

\datum{2023-10-5}

\begin{definicija}
Let $R = K \skl{x, y}$ be a free algebra and
$F = \set{xy - yx - 1}$. The quotient
\[
\mathcal{A}_1(K) = \kvoc{R}{(F)}
\]
is called the \emph{first Weyl algebra}\index{first Weyl algebra}.
\end{definicija}

\begin{opomba}
The first Weyl algebra is generated by elements $\oline{x}$ and
$\oline{y}$ that satisfy
$\oline{x} \cdot \oline{y} - \oline{y} \cdot \oline{x} - 1$.
\end{opomba}

\begin{opomba}
The first Weyl algebra is the algebra of differential operators --
for $D, L \colon K[y] \to K[y]$, defined as $D(p) = \parc{p}{y}$
and $L(p) = yp$, we have $DL - LD = I$.
\end{opomba}

\begin{definicija}
Let $R$ be a ring and $\sigma \in \End(R)$. The
\emph{skew polynomial ring}\index{skew polynomial ring} is the set
\[
R[x, \sigma] =
\setb{\sum_{i=0}^n b_i x^i}{n \in \N \land b_i \in R}
\]
in which for all $b \in R$ the equality in $x b = \sigma(b) x$
holds.
\end{definicija}

\begin{definicija}
Let $R$ be a ring and $\sigma$ a
derivation\footnote{$\sigma(a+b) = \sigma(a) + \sigma(b)$,
$\sigma(ab) = a\sigma(b) + \sigma(a)b$.} on $R$. The
\emph{skew polynomial ring}\index{skew polynomial ring} is the set
\[
R[x, \sigma] =
\setb{\sum_{i=0}^n b_i x^i}{n \in \N \land b_i \in R}
\]
in which for all $b \in R$ the equality in $x b = b x + \sigma(b)$
holds.
\end{definicija}

\newpage

\subsection{Chain conditions}

\begin{definicija}
Let $C$ be a set and $\setb{C_i}{i \in I}$ a set of subsets of $C$.
The set $\setb{C_i}{i \in I}$ satisfies the
\emph{ascending chain condition}\index{ascending chain condition}
if there does not exist an infinite strictly increasing chain
\[
C_{i_1} \subset C_{i_2} \subset C_{i_3} \subset \dots
\]
The
\emph{descending chain condition}\index{descending chain condition}
is defined analogously.
\end{definicija}

\begin{definicija}
Let $R$ be a ring and $M$ an $R$-module.

\begin{enumerate}[i)]
\item $M$ is \emph{noetherian}\index{neotherian module} if the set
of submodules of $M$ satisfies the ascending chain condition.
\item $M$ is \emph{artinian}\index{artinian module} if the set of
submodules of $M$ satisfies the descending chain condition.
\end{enumerate}
\end{definicija}

\begin{trditev}
The following statements are true:

\begin{enumerate}[i)]
\item A module $M$ is noetherian if and only if each submodule of
$M$ is finitely generated.
\item Let $N \leq M$ be a submodule. Then $M$ is noetherian if and
only if both $N$ and $\kvoc{M}{N}$ are noetherian.
\label{fin_wed:prop:noet}
\item Let $N \leq M$ be a submodule. Then $M$ is artinian if and
only if both $N$ and $\kvoc{M}{N}$ are artinian.
\end{enumerate}
\end{trditev}

\begin{proof}
\phantom{a}
\begin{enumerate}[i)]
\item Suppose that each submodule of $M$ is finitely generated and
$M_1 \leq M_2 \leq \dots \leq M$. Define the submodule
\[
N = \bigcup_{j \in \N} M_j.
\]
By assumption, $N$ is finitely generated. But then there exists
some $j \in \N$ such that $M_j$ contains all generators of $N$,
so $M_j = N$. Therefore, the chain cannot be strictly increasing.

Now assume that $M$ is noetherian and let $N \leq M$ be a
submodule. Define
\[
\mathcal{C} = \setb{S \leq N}{\text{$S$ is finitely generated}}.
\]
This set must have some maximal element $N_0 \leq N$. Suppose
$N_0 < N$ and consider some element $b \in N \setminus N_0$. The
module $N + Rb$ is also finitely generated and contained in $N$,
which is a contradiction as $N_0$ was maximal. Therefore we must
have $N = N_0$ and $N$ is finitely generated.

\item Suppose that $M$ is noetherian. Consider the following short
exact sequence:
\[
\begin{tikzcd}
0 \arrow[r] & N \arrow[r, "f"] &
M \arrow[r, "g"] & \kvoc{M}{N} \arrow[r] & 0.
\end{tikzcd}
\]
It is easy to see that $N$ is also noetherian, as the inclusion of
a chain in $N$ is also a chain in $M$. As preimages of submodules
are also submodules, the same conclusion follows for $\kvoc{M}{N}$.

Now suppose that both $N$ and $\kvoc{M}{N}$ are noetherian and
consider a chain $M_1 \leq M_2 \leq \dots \leq M$ of submodules.
As $f^{-1}(M_i)$ and $g(M_i)$ form increasing chains in their
respective modules, it follows that there exists some $n \in \N$
such that both $f^{-1}(M_i)$ and $g(M_i)$ are constant for all
$i \geq n$. Now consider the following diagram:
\[
\begin{tikzcd}
0 \arrow[r] &
f^{-1}(M_n) \arrow[r, "f"] \arrow[d, "\id"] &
M_n \arrow[r, "g"] \arrow[d, hook, "i"] &
g(M_n) \arrow[r] \arrow[d, "\id"] &
0 \\
0 \arrow[r] &
f^{-1}(M_i) \arrow[r, "f"] &
M_i \arrow[r, "g"] &
g(M_i) \arrow[r] &
0.
\end{tikzcd}
\]
By the short five lemma, $i$ is an isomorphism, so $M_n = M_i$.
\item Same as~\ref{fin_wed:prop:noet}. \qedhere
\end{enumerate}
\end{proof}
