\section{Finite-dimensional algebras, Wedderburn's structure theory}

\subsection{Free algebras}

\datum{2023-10-5}

\begin{definicija}
Let $R = K \skl{x, y}$ be a free algebra and
$F = \set{xy - yx - 1}$. The quotient
\[
\mathcal{A}_1(K) = \kvoc{R}{(F)}
\]
is called the \emph{first Weyl algebra}\index{first Weyl algebra}.
\end{definicija}

\begin{opomba}
The first Weyl algebra is generated by elements $\oline{x}$ and
$\oline{y}$ that satisfy
$\oline{x} \cdot \oline{y} - \oline{y} \cdot \oline{x} - 1$.
\end{opomba}

\begin{opomba}
The first Weyl algebra is the algebra of differential operators --
for $D, L \colon K[y] \to K[y]$, defined as $D(p) = \parc{p}{y}$
and $L(p) = yp$, we have $DL - LD = I$.
\end{opomba}

\begin{definicija}
Let $R$ be a ring and $\sigma \in \End(R)$. The
\emph{skew polynomial ring}\index{skew polynomial ring} is the set
\[
R[x, \sigma] =
\setb{\sum_{i=0}^n b_i x^i}{n \in \N \land b_i \in R}
\]
in which for all $b \in R$ the equality in $x b = \sigma(b) x$
holds.
\end{definicija}

\begin{definicija}
Let $R$ be a ring and $\sigma$ a
derivation\footnote{$\sigma(a+b) = \sigma(a) + \sigma(b)$,
$\sigma(ab) = a\sigma(b) + \sigma(a)b$.} on $R$. The
\emph{skew polynomial ring}\index{skew polynomial ring} is the set
\[
R[x, \sigma] =
\setb{\sum_{i=0}^n b_i x^i}{n \in \N \land b_i \in R}
\]
in which for all $b \in R$ the equality in $x b = b x + \sigma(b)$
holds.
\end{definicija}

\newpage

\subsection{Chain conditions}

\begin{definicija}
Let $C$ be a set and $\setb{C_i}{i \in I}$ a set of subsets of $C$.
The set $\setb{C_i}{i \in I}$ satisfies the
\emph{ascending chain condition}\index{ascending chain condition}
if there does not exist an infinite strictly increasing chain
\[
C_{i_1} \subset C_{i_2} \subset C_{i_3} \subset \dots
\]
The
\emph{descending chain condition}\index{descending chain condition}
is defined analogously.
\end{definicija}

\begin{definicija}
Let $R$ be a ring and $M$ an $R$-module.

\begin{enumerate}[i)]
\item $M$ is \emph{noetherian}\index{noetherian!module} if the set
of submodules of $M$ satisfies the ascending chain condition.
\item $M$ is \emph{artinian}\index{artinian!module} if the set of
submodules of $M$ satisfies the descending chain condition.
\end{enumerate}
\end{definicija}

\begin{trditev}
The following statements are true:

\begin{enumerate}[i)]
\item A module $M$ is noetherian if and only if each submodule of
$M$ is finitely generated.
\item Let $N \leq M$ be a submodule. Then $M$ is noetherian if and
only if both $N$ and $\kvoc{M}{N}$ are noetherian.
\label{fin_wed:prop:noet}
\item Let $N \leq M$ be a submodule. Then $M$ is artinian if and
only if both $N$ and $\kvoc{M}{N}$ are artinian.
\end{enumerate}
\end{trditev}

\begin{proof}
\phantom{a}
\begin{enumerate}[i)]
\item Suppose that each submodule of $M$ is finitely generated and
$M_1 \leq M_2 \leq \dots \leq M$. Define the submodule
\[
N = \bigcup_{j \in \N} M_j.
\]
By assumption, $N$ is finitely generated. But then there exists
some $j \in \N$ such that $M_j$ contains all generators of $N$,
so $M_j = N$. Therefore, the chain cannot be strictly increasing.

Now assume that $M$ is noetherian and let $N \leq M$ be a
submodule. Define
\[
\mathcal{C} = \setb{S \leq N}{\text{$S$ is finitely generated}}.
\]
This set must have some maximal element $N_0 \leq N$. Suppose
$N_0 < N$ and consider some element $b \in N \setminus N_0$. The
module $N + Rb$ is also finitely generated and contained in $N$,
which is a contradiction as $N_0$ was maximal. Therefore we must
have $N = N_0$ and $N$ is finitely generated.

\item Suppose that $M$ is noetherian. Consider the following short
exact sequence:
\[
\begin{tikzcd}
0 \arrow[r] & N \arrow[r, "f"] &
M \arrow[r, "g"] & \kvoc{M}{N} \arrow[r] & 0.
\end{tikzcd}
\]
It is easy to see that $N$ is also noetherian, as the inclusion of
a chain in $N$ is also a chain in $M$. As preimages of submodules
are also submodules, the same conclusion follows for $\kvoc{M}{N}$.

Now suppose that both $N$ and $\kvoc{M}{N}$ are noetherian and
consider a chain $M_1 \leq M_2 \leq \dots \leq M$ of submodules.
As $f^{-1}(M_i)$ and $g(M_i)$ form increasing chains in their
respective modules, it follows that there exists some $n \in \N$
such that both $f^{-1}(M_i)$ and $g(M_i)$ are constant for all
$i \geq n$. Now consider the following diagram:
\[
\begin{tikzcd}
0 \arrow[r] &
f^{-1}(M_n) \arrow[r, "f"] \arrow[d, "\id"] &
M_n \arrow[r, "g"] \arrow[d, hook, "i"] &
g(M_n) \arrow[r] \arrow[d, "\id"] &
0 \\
0 \arrow[r] &
f^{-1}(M_i) \arrow[r, "f"] &
M_i \arrow[r, "g"] &
g(M_i) \arrow[r] &
0.
\end{tikzcd}
\]
By the short five lemma, $i$ is an isomorphism, so $M_n = M_i$.
\item Same as~\ref{fin_wed:prop:noet}. \qedhere
\end{enumerate}
\end{proof}

\datum{2023-10-12}

\begin{definicija}
A ring $R$ is \emph{left-noetherian}\index{noetherian!ring} if it
is noetherian as a left $R$-module. We analogously define
\emph{right-noetherian}, \emph{left-artinian}\index{artinian!ring}
and \emph{right-artinian} rings.

A ring $R$ is \emph{noetherian}, if it is both left-noetherian and
right-noetherian. We similarly define \emph{artinian} rings.
\end{definicija}

\begin{opomba}
A ling $R$ is left-noetherian if and only if each left ideal of $R$
is finitely generated.
\end{opomba}

\begin{trditev}
If $R$ is a notherian ring and $M$ is a finitely generated
$R$-module, $M$ is noetherian.
\end{trditev}

\begin{proof}
As $M$ is finitely generated, there exists an endomorphism
$\varphi \colon R^n \to M$ for some $n \in \N$. Consider the short
exact sequence
\[
\begin{tikzcd}
0 \arrow[r] & R \arrow[r] & R^n \arrow[r] & R^{n-1} \arrow[r] & 0.
\end{tikzcd}
\]
By induction on $n$, $R^n$ is noetherian. As $M$ is a quotient of
$R^n$, $M$ is also noetherian.
\end{proof}

\newpage

\subsection{Simple modules}

\begin{definicija}
A nontrivial $R$-module $M$ is \emph{simple}\index{simple module}
if it has no proper nontrivial submodules. An $R$-module $M$ is
\emph{cyclic}\index{cyclic module} with generator $m \in M$ if
$M = R \cdot m$.
\end{definicija}

\begin{trditev}
For $R$-modules $M$, the following are equivalent:

\begin{enumerate}[i)]
\item The module $M$ is simple.
\item The module $M$ is cyclic and its every non-zero element is a
generator.
\item We have $M \cong \kvoc{R}{I}$ for some maximal left ideal
$I \edn R$.
\end{enumerate}
\end{trditev}

\begin{proof}
Suppose that $M$ is simple. Then for every
$m \in M \setminus \set{0}$, $Rm \leq M$ is a nontrivial submodule.
It follows that $m$ is a generator.

Suppose now that every non-zero element is a generator. Define the
homomorphism $\phi \colon R \to M$ with $\phi(r) = rm$. Set
$I = \ker \phi = \ann(m)$. By the isomorphism theorem, we have
$Rm = M \cong \kvoc{R}{I}$. There is bijective correspondence
between ideals $I \edn J \edn R$ and submodules of $M$. As any
element of a proper submodule cannot generate $M$, $I$ must be
maximal.

Suppose now that $M \cong \kvoc{R}{I}$ for some maximal $I \edn R$
and suppose that $M' \leq M$ is a submodule. It follows that $M'$
corresponds to a left ideal $J$ such that $I \edn J \edn R$. Thus,
$J = I$ or $J = R$, or equivalently, $M' = M$ or $M' = (0)$.
\end{proof}

\begin{posledica}
\label{fin_dim_alg:psl:vec_simp}
Let $D$ be a division ring and $V$ be an $n$-dimensional vector
space over $D$. Let $R = \End_D(V)$. Then, $V$ is a simple
$R$-module.
\end{posledica}

\begin{proof}
For every $v \in V \setminus \set{0}$ we have $Rv = V$.
\end{proof}

\begin{izrek}[Schur's lemma]
\index{Schur's lemma}
Let $M$ and $N$ be simple $R$-modules and $f \colon M \to N$ a
homomorphism. Then $f$ is either and isomorphism or the zero map.
In particular, $\End_R(M)$ is a division ring.
\end{izrek}

\begin{proof}
Note that $\ker f \leq M$ and $\im f \leq N$. The conclusion
follows.
\end{proof}

\begin{trditev}
Let $D$ be a division ring and $V$ a $D$-module. Then,
$D \cong \End_R(V)$, where $R = \End_D(V)$.
\end{trditev}

\begin{proof}
Define a homomorphism $\Psi \colon D \to \End_R(V)$ as
$\Psi(d) = (f \mapsto df)$. It is clear that $\Psi$ is injective.
Now let $T \in \End_R(V)$ be an arbitrary endomorphism. Choose a
$v \in V \setminus \set{0}$. For any $w \in V$ there exists an
endomorphism of $V$ that sends $w$ to $v$, therefore,
$V = R \cdot v$. Every $R$-endomorphism is therefore determined
by its image on $v$. To prove that $\Psi$ is surjective, it is
hence enough to show that $Tv = d \cdot v$ for some $d \in D$.

Let $p \in R$ be a projection onto $Dv$. It is easy to check that
\[
Tv = T(p(v)) = p(T(v)) \in Dv. \qedhere
\]
\end{proof}

\begin{lema}
A finite dimensional division algebra $D$ over an algebraically
closed field $k$ is $k$ itself.
\end{lema}

\begin{proof}
Note that, for $\alpha \in D$, $\kvoc{k(\alpha)}{k}$ is a finite
field extension, but as $k$ is algebraically closed,
$k(\alpha) = k$.
\end{proof}

\newpage

\subsection{Semisimple modules}

\begin{definicija}
A module is \emph{semisimple}\index{semisimple!module} if it is a
direct sum of simple modules.
\end{definicija}

\begin{trditev}
If an $R$-module $M$ is a sum of simple submodules $M_i$ for
$i \in I$, then $M$ is semisimple. Moreover, there exists a subset
$I' \subseteq I$ such that
\[
M = \bigoplus_{i \in I'} M_i.
\]
\end{trditev}

\begin{proof}
Set
\[
\mathcal{I} =
\setb{J \subseteq I}{\text{$\br{M_j}_{j \in J}$ is independent}}.
\]
As $\mathcal{I}$ is a non-empty set and every chain in $M$ has an
upper bound, we can apply Zorn's lemma. Let $I'$ be a maximal
element of $\mathcal{I}$. Note that
\[
M' = \bigoplus_{i \in I'} M_i \leq M.
\]
If $M' \cap M_i = \set{0}$ for some $i \in I$, the set $I'$ is not
maximal as we can take $I' \cup \set{i}$. Therefore,
$M' \cap M_i = M_i$ for all $i$ as $M_i$ are simple modules. It
follows that $M' = M$.
\end{proof}

\begin{posledica}
If $M$ is semisimple, then so is every submodule and quotient of
$M$. Furthermore, every submodule of $M$ is a direct summand.
\end{posledica}

\begin{proof}
Let
\[
M = \bigoplus_{i \in I} M_i
\]
be a direct sum of simple modules and $M' \leq M$. The module
$\kvoc{M}{M'}$ is then generated by the images $\oline{M}_i$ of
$M_i$ under the quotient map. If $\oline{M}_i \ne \set{0}$, we have
$\oline{M}_i \cong M_i$ since $M_i$ is simple. Therefore,
$\kvoc{M}{M'}$ is a sum of modules $M_i$, and as such semisimple.
As we can write
\[
M = \br{\bigoplus_{i \in I'} M_i} \oplus M',
\]
we can write
\[
M' = \bigoplus_{i \in I \setminus I'} M_i. \qedhere
\]
\end{proof}

\begin{trditev}
Let $M$ be a module such that every submodule of $M$ is a direct
summand.\footnote{We call this the \emph{complement property}.}
Then $M$ is semisimple.
\end{trditev}

\begin{proof}
Let $M' \leq M$ be a non-zero cyclic submodule, say $M' = Rm$ for
$m \ne 0$. Suppose $M'$ is not simple. By Zorn's lemma, there
exists a maximal submodule $M'' \leq M'$ with $m \not \in M''$. The
module $\kvoc{M'}{M''}$ is therefore simple. As $M'$ also has the
complement property, we can write $M' = M'' \oplus S$ for some
$S \leq M'$. Since $S \cong \kvoc{M'}{M''}$, it is a simple
submodule. In both cases, we have found a simple submodule of $M$.

Let $M_1$ be the sum of all simple submodules of $M$. Then there
exists a submodule $M_2 \leq M$, such that $M = M_1 \oplus M_2$.
If $M_2 \ne \set{0}$, by the same argument as above, $M_2$ has a
simple module. This is of course not possible.
\end{proof}

\newpage

\subsection{Endomorphism ring of a semisimple module}

\begin{trditev}
Let $M$ be an $R$-module, $S = \End_R(M)$ and $p, m, n \in \N$.
There is a canonical isomorphism of abelian groups
\[
\Hom_R(M^n, M^m) \cong S^{m \times n},
\]
such that the composition
\[
\Hom_R(M^n, M^m) \times \Hom_R(M^p, M^n) \to \Hom_R(M^p, M^m)
\]
corresponds to matrix multiplication. In particular,
$\End_R(M^n) \cong S^{n \times n} = M_n(S)$ is an isomorphism of
rings.
\end{trditev}

\begin{proof}
The isomorphism is given by the map
$f \mapsto [\pi_i \circ f \circ \iota_j]_{i,j}$.
\end{proof}

\begin{opomba}
For $r \in R$ the map $T_r \colon R \to R$ given by $T_r(x) = xr$
is $R$-linear. We can therefore define a homomorphism
$\Phi \colon R \to \End_R(R)$ by $\Phi(r) = T_r$. As
$\Phi$ is injective and $f = T_{f(1)}$, we have
$\End_R(R) \cong R^{\mathsf{op}}$.
\end{opomba}

\begin{posledica}
For a division ring $D$, we have
$\End_D(D^n) = M_n(D^{\mathsf{op}})$.
\end{posledica}	

\begin{definicija}
A semisimple module has \emph{finite length}\index{finite length}
if it is a finite direct sum of simple modules.
\end{definicija}

\begin{trditev}
\label{fin_dim_alg:td:mat_prod}
If $M$ is a semisimple $R$-module of finite length, then
$\End_R(M)$ is isomorphic to a finite product of matrix rings over
division rings.
\end{trditev}

\begin{proof}
Let
\[
M \cong \bigoplus_{i=1}^k M_i^{n_i}
\]
for distinct simple modules $M_i$. By Schur's lemma, we can write
\[
\End_R(M) =
\End_R \br{\bigoplus_{i=1}^k M_i} =
\prod_{i=1}^k \End_R \br{M_i^{n_i}} =
\prod_{i=1}^k M_{n_i} \br{\End_R(M_i)}. \qedhere
\]
\end{proof}

\newpage

\subsection{Semisimple rings}

\begin{definicija}
A ring $R$ is \emph{semisimple}\index{semisimple!ring} if it is a
semisimple left $R$-module.
\end{definicija}

\begin{izrek}
Let $R$ be a ring. The following statements are equivalent:

\begin{enumerate}[i)]
\item The ring $R$ is semisimple.
\item Every $R$-module is semisimple.
\item Every short exact sequence of $R$-modules split.
\end{enumerate}
\end{izrek}

\begin{proof}
Suppose that $R$ is semisimple. As all $R$-modules are quotients of
a free module $R^I$, which is semisimple, all $R$-modules are
semisimple.

Suppose that every $R$-module is semisimple. As those have the
complement property, every short exact sequence splits.

Suppose that every short exact sequence splits and let $I \leq R$
be a submodule over $R$. As
\[
\begin{tikzcd}
0 \arrow[r] & I \arrow[r] & R \arrow[r] &
\kvoc{R}{I} \arrow[r] & 0.
\end{tikzcd}
\]
is a short exact sequence, it splits, so $I$ is a direct summand
of $R$. It follows that $R$ has the complement property, therefore,
it is semisimple.
\end{proof}

\begin{posledica}
Suppose that $R$ is a semisimple ring. Then $R$ as an $R$-module
has finite length and any simple $R$-module is isomorphic to a
simple component of $R$.
\end{posledica}

\begin{proof}
We can write
\[
R = \bigoplus_{i \in I} M_i
\]
for simple $R$-modules $M_i$. By considering $1 \in R$, we see that
$I$ is a finite set.

Let $M$ be a simple $R$-module. As we have $M = R \cdot m$, there
exist maps $M_i \to M$. As $R \to M$ is surjective, at least one
of those maps is non-zero and therefore an isomorphism by Schur's
lemma.
\end{proof}

\begin{trditev}
Let $D$ be a division ring and $V$ be an $n$-dimensional vector
space over $D$. Then $R = \End_D(V)$ is semisimple.
\end{trditev}

\begin{proof}
The map $f \mapsto \br{f(e_1), f(e_2), \dots, f(e_n)}$ is an
isomorphism of $R$-modules $R$ and $V^n$. As $V$ is simple by
corollary~\ref{fin_dim_alg:psl:vec_simp}, $R$ is semisimple.
\end{proof}

\newpage

\subsection{Wedderburn structure theorem}

\begin{izrek}[Wedderburn]
\index{Wedderburn's theorem}
Every semisimple ring $R$ is isomorphic to a finite product of
matrix rings over division rings. If $R$ is also commutative, it is
a finite direct products of fields.
\end{izrek}

\begin{proof}
By proposition~\ref{fin_dim_alg:td:mat_prod}, we can write
\[
R^{\mathsf{op}} \cong \End_R(R) \cong \prod_{i=1}^k M_{n_i}(D_i).
\]
It follows that
\[
R \cong
\br{\prod_{i=1}^k M_{n_i}(D_i)}^{\mathsf{op}} =
\prod_{i=1}^k M_{n_i} \br{D_i^{\mathsf{op}}}. \qedhere
\]
\end{proof}
