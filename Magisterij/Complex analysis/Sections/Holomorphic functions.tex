\section{Holomorphic functions}

\subsection{Properties of holomorphic functions}

\datum{2023-10-3}

\begin{definicija}
Let $\Omega \subseteq \C$ be an open subset. A function
$f \colon \Omega \to \C$ is
\emph{complex differentiable}\index{function!complex differentiable}
in a point $a \in \Omega$ if the limit
\[
\lim_{z \to a} \frac{f(z) - f(a)}{z - a}
\]
exists.
\end{definicija}

\begin{opomba}[Cauchy-Riemann equations]
\index{Cauchy-Riemann equations}
Denoting $u = \Re f$ and $v = \Im f$ where $f$ is real
differentiable in $a$, $f$ is complex differentiable in $a$ if and
only if $\parc{u}{x} = \parc{v}{y}$ and
$\parc{u}{y} = - \parc{v}{x}$
\end{opomba}

\begin{definicija}
\emph{Wirtinger derivatives}\index{Wirtinger derivatives} are
defined as
\[
\parc{}{z} =
\frac{1}{2} \cdot \br{\parc{}{x} - i \parc{}{y}}
\quad \text{and} \quad
\parc{}{\oline{z}} =
\frac{1}{2} \cdot \br{\parc{}{x} + i \parc{}{y}}.
\]
\end{definicija}

\begin{opomba}
A function $f$ is complex differentiable in $a$ if and only if
\[
\parc{f}{\oline{z}} = 0.
\]
In that case, we also have
\[
\parc{f}{z}(a) = f'(a).
\]
\end{opomba}

\begin{definicija}
Let $\Omega \subseteq \C$ be an open subset. A function
$f \colon \Omega \to \C$ is
\emph{holomorphic in $a$}\index{function!holomorphic} if it is
complex differentiable in an open neighbourhood of $a$. The
function $f$ is \emph{holomorphic} if it is holomorphic in every
point of $\Omega$. We denote the set of holomorphic functions in
$\Omega$ as $\mathcal{O}(\Omega)$.
\end{definicija}

\begin{izrek}[Inhomogeneous Cauchy integral formula]
\index{Cauchy integral formula}
Let $\Omega \subseteq \C$ be a bounded domain with
$\mathcal{C}^1$-smooth boundary and
$f \in \mathcal{C}^1(\Omega) \cap \mathcal{C} \br{\oline{\Omega}}$.
Then, for all $z \in \Omega$, we have
\[
f(z) =
\frac{1}{2 \pi i} \olint_{\partial \Omega} \frac{f(w)}{w - z}\,dw +
\frac{1}{2 \pi i} \liint_\Omega \parc{f}{\oline{w}} \cdot
\frac{1}{w - z}\,dw \wedge d\oline{w}
\]
\end{izrek}

\begin{proof}
As $\Omega$ is an open set, there exists an $\varepsilon > 0$ such
that $\oline{\dsk(z, \varepsilon)} \subseteq \Omega$. Define a new
domain
$\Omega_\varepsilon =
\Omega \setminus \oline{\dsk(z, \varepsilon)}$.

We now apply Stokes' theorem to $\omega = \frac{f(w)}{w-z}\,dw$ on
$\Omega_\varepsilon$. As
$d\omega =
\parc{f}{\oline{w}} \cdot \frac{1}{w-z}\,d\oline{w} \wedge dw$, we
have
\[
\olint_{\partial \Omega_\varepsilon} \frac{f(w)}{w-z}\,dw =
\liint_{\Omega_\varepsilon} \parc{f}{\oline{w}} \cdot
\frac{1}{w-z}\,d\oline{w} \wedge dw.
\]
Note that
\[
\olint_{\partial \Omega_\varepsilon} \frac{f(w)}{w-z}\,dw =
\olint_{\partial \Omega} \frac{f(w)}{w-z}\,dw -
\olint_{\partial \dsk(z, \varepsilon)} \frac{f(w)}{w-z}\,dw.
\]
In the limit, we have
\[
\lim_{\varepsilon \to 0}
\olint_{\partial \dsk(z, \varepsilon)} \frac{f(w)}{w - z}\,dw =
\lim_{\varepsilon \to 0}
\int_0^{2 \pi}
\frac{f \br{z + \varepsilon e^{it}}}{\varepsilon e^{it}} \cdot
\varepsilon i e^{it}\,dt =
2 \pi i f(z)
\]
by continuity. Also note that
\[
\lim_{\varepsilon \to 0}
\liint_{\Omega_\varepsilon} \parc{f}{\oline{w}} \cdot
\frac{1}{w-z}\,d\oline{w} \wedge dw =
\liint_{\Omega \setminus \set{z}} \parc{f}{\oline{w}} \cdot
\frac{1}{w - z}\,d\oline{w} \wedge dw =
\liint_{\Omega} \parc{f}{\oline{w}} \cdot
\frac{1}{w - z}\,d\oline{w} \wedge dw.
\]
Applying the limit to the Stokes' theorem equation, it follows that
\[
\frac{1}{2 \pi i}
\olint_{\partial \Omega} \frac{f(w)}{w-z}\,dw - f(z) =
-\frac{1}{2 \pi i} \liint_{\Omega} \parc{f}{\oline{w}} \cdot
\frac{1}{w-z}\,dw \wedge d\oline{w}. \qedhere
\]
\end{proof}

\begin{izrek}[Power series expansion]
Let $\Omega \subseteq \C$ be an open subset,
$f \in \mathcal{O}(\Omega)$ and $a \in \Omega$. The function $f$
can be developed into a power series about $a$ that converges
absolutely and uniformly to $f$ in compacts inside $\dsk(a, r)$,
where $r$ is the radius of convergence. For
\[
c_k = \frac{f^{(k)}(a)}{k!} =
\frac{1}{2 \pi i} \olint_{\partial \dsk(a, \rho)}
\frac{f(w)}{(w-z)^{k+1}}\,dw
\]
we have
\[
f(z) = \sum_{k=0}^\infty c_k \cdot (z-a)^k.
\]
\end{izrek}

\begin{opomba}
The converse is also true -- any complex power series defines a
holomorphic function inside its radius of convergence.
\end{opomba}

\begin{opomba}
The radius of convergence is given by the formula
\[
\frac{1}{r} = \limsup_{k \to \infty} \sqrt[k]{\abs{c_k}}.
\]
\end{opomba}

\begin{izrek}[Identity]
\index{identity theorem}
Let $\Omega \subseteq \C$ be a domain and
$f \in \mathcal{O}(\Omega)$ a holomorphic function. Let
$A \subseteq \Omega$ be a subset such that $f(z) = 0$ for all
$z \in A$. If $A$ has an accumulation point in $\Omega$, then
$f(z) = 0$ for all $z \in \Omega$.
\end{izrek}

\begin{proof}
Let $a \in \Omega$ be an accumulation point of $A$. By continuity,
we have $f(a) = 0$. We can now write
\[
f(z) = \sum_{k=k_0}^\infty c_k (z - a)^k,
\]
where we assume $c_{k_0} \ne 0$. But now
$g(z) = \frac{f(z)}{(z-a)^{k_0}}$ is also holomorphic. Again, by
continuity, we must have $g(a) = 0$, which is a contradiction. It
follows that $c_k = 0$ for all $k \in \N_0$. It follows that the
set $\Int \setb{z \in \Omega}{f(z) = 0}$ is non-empty. By the same
argument as above, it has an empty boundary and is therefore equal
to $\Omega$.
\end{proof}

\datum{2023-10-4}

\begin{izrek}[Open mapping]
\index{open mapping theorem}
Let $\Omega \subseteq \C$ be a domain and
$f \in \mathcal{O}(\Omega)$ a function. If $f$ is not constant, it
is an open map.
\end{izrek}

\begin{proof}
We first prove the following lemma:

\begin{lema*}
Let $\Omega \subseteq \C$ be a domain and
$f \in \mathcal{O}(\Omega)$. Suppose that for $a \in \Omega$ and
$r > 0$ we have $\oline{\dsk(a, r)} \subseteq \Omega$. If
\[
\abs{f(a)} < \min_{\partial \dsk(a, r)} \abs{f},
\]
then $f$ has a zero in $\dsk(a, r)$.
\end{lema*}

\begin{proof}[Proof (lemma)]
Assume otherwise. From the inequality it follows that $f$ has no
zeroes on the boundary either. By continuity, $f$ has no zero on an
open set $V$ with $\dsk(a, r) \subseteq V$. We can therefore define
$g \in \mathcal{O}(V)$ with $g(z) = \frac{1}{f(z)}$. We now have
\[
g(a) =
\frac{1}{2 \pi i}
\olint_{\partial \dsk(a, r)} \frac{g(z)}{z-a}\,dz =
\frac{1}{2 \pi i}
\int_0^{2 \pi}
\frac{g \br{a + r \cdot e^{it}}}{re^{it}} \cdot r i e^{it}\,dt =
\frac{1}{2 \pi} \int_0^{2 \pi}  g \br{a + r e^{it}}\,dt.
\]
We can therefore get a bound on $\abs{g(a)}$ as
\[
\abs{g(a)} \leq \max_{\partial \dsk(a, r)} \abs{g},
\]
but as the condition on $f$ can be rewritten as
\[
\abs{g(a)} > \max_{\partial \dsk(a, r)} \abs{g},
\]
we have reached a contradiction.
\end{proof}

Let $U \subseteq \Omega$ be an open set and $w_0 \in f(U)$. Choose
a $z_0 \in U$ such that $f(z_0) = w_0$. Choose a $\rho > 0$ such
that $\dsk(z_0, \rho) \subseteq U$ and $z_0$ is the only pre-image
of $w_0$ in $\dsk(z_0, 2\rho)$.\footnote{If such a disk does not
exist, $f$ is constant by the identity theorem.}

Since $\partial \dsk(z_0, \rho)$ is a compact set and
\[
\abs{f(z) - w_0} > 0
\]
for all $z \in \partial \dsk(z_0, \rho)$, we can choose some
$\varepsilon > 0$ such that
\[
\abs{f(z) - w_0} > 2 \varepsilon
\]
holds on the boundary of the disk. Choose a
$w \in \dsk(w_0, \varepsilon)$. As we have
\[
\abs{f(z) - w} >
\abs{f(z) - w_0} - \abs{w_0 - w} \geq
\varepsilon
\]
on the boundary and
\[
\abs{f(z_0) - w} = \abs{w_0 - w} < \varepsilon,
\]
by the above lemma, $f(z_0) - w$ has a root on $\dsk(z, \rho)$.
\end{proof}

\begin{izrek}[Maximum principle]
\index{maximum principle}
Let $\Omega \subseteq \C$ be a domain. If the modulus $\abs{f}$ of
a function $f \in \mathcal{O}(\Omega)$ attains a local maximum, the
function $f$ is constant.
\end{izrek}

\begin{proof}
Suppose that $f$ is non-constant and that its modulus attains a
local maximum at $z \in \Omega$. As $f$ is an open map, it also
attains the value $(1 + \varepsilon) \cdot f(z)$, which is a
contradiction as the modulus then equals
$(1 + \varepsilon) \cdot \abs{f(z)} > \abs{f(z)}$.
\end{proof}

\begin{izrek}[Maximum principle]
Let $\Omega \subseteq \C$ be a bounded domain and assume that
$f \in \mathcal{O}(\Omega) \cap \mathcal{C} \br{\oline{\Omega}}$.
Then, the maximum of $\abs{f}$ is attained in the boundary
$\partial \Omega$.
\end{izrek}

\begin{proof}
As $\oline{\Omega}$ is compact, $f$ attains a global maximum on
this set. If the maximum is attained in the interior, $f$ is
constant, therefore it is also attained on the boundary.
\end{proof}

\datum{2023-10-10}

\begin{definicija}
A function $f \colon \Omega \setminus \set{a} \to \C$ is
\emph{locally bounded}\index{locally bounded} near $a$ if there
exists an open neighbourhood $U \subseteq \Omega$ of $a$ such that
$\eval{f}{U \setminus \set{a}}{}$ is bounded.
\end{definicija}

\begin{izrek}[Riemann removable singularity theorem]
\index{Riemann!removable singularity theorem}
Let $\Omega \subseteq \C$ be an open subset, $a \in \Omega$ and
$f \in \mathcal{O} \br{\Omega \setminus \set{a}}$. If $f$ is
locally bounded near $a$, then there exists a unique function
$F \in \mathcal{O}(\Omega)$ such that
$\eval{F}{\Omega \setminus \set{a}}{} = f$.
\end{izrek}

\begin{proof}
Define the function $F \colon \Omega \to \C$ as
\[
F(z) = \begin{cases}
f(z) & z \in \Omega \setminus \set{a}, \\
\displaystyle
\frac{1}{2 \pi i} \olint_{\partial \dsk(a, \rho)}
\frac{f(w)}{w-a}\,dw & z = a.
\end{cases}
\]
It remains to check that $F$ is complex differentiable at $a$.
Indeed, for $z \in \dsk(a, \rho)$ we have
\begin{align*}
\lim_{z \to a} \frac{F(z) - F(a)}{z-a} &=
\lim_{z \to a} \frac{1}{z-a} \olint_{\partial \dsk(a, \rho)}
\br{\frac{f(w)}{w-z} - \frac{f(w)}{w-a}}\,dw
\\
&=
\lim_{z \to a} \frac{1}{2 \pi i} \cdot \frac{1}{z-a} \cdot
\olint_{\partial \dsk(a, \rho)}
f(w) \cdot \frac{z-a}{(w-z)(w-a)}\,dw
\\
&=
\frac{1}{2 \pi i}
\olint_{\partial \dsk(a, \rho)} \frac{f(w)}{(w-a)^2}\,dw,
\end{align*}
which exists. Uniqueness follows from the identity theorem.
\end{proof}

\begin{izrek}[Schwarz lemma]
\index{Schwarz!lemma}
Let $f \colon \dsk \to \dsk$ be a holomorphic function with
$f(0) = 0$. Then, $\abs{f'(0)} \leq 1$ and the inequality
$\abs{f(z)} \leq \abs{z}$ holds for all $z \in \dsk$. If
$\abs{f'(0)} = 1$ or $\abs{f(z)} = \abs{z}$ holds for any
$z \ne 0$, then $f(z) = \beta z$ for some
$\beta \in \partial \dsk$.
\end{izrek}

\begin{proof}
We can write
\[
f(z) = \sum_{k=1}^\infty c_k z^k.
\]
We define
\[
g(z) = \frac{f(z)}{z} = \sum_{k=1}^\infty c_k z^{k-1}.
\]
The radius of convergence for both series is at least $1$. Now
apply the maximum principle for $g$ on the domain $\dsk(\rho)$. We
get
\[
\sup_{z \in \dsk(\rho)} \abs{g(z)} \leq
\max_{\abs{z} = \rho} \abs{g(z)} =
\frac{1}{\rho} \max_{\abs{z} = \rho} \abs{f(z)} <
\frac{1}{\rho}.
\]
In the limit as $\rho \to 1$, it follows that
\[
\sup_{z \in \dsk} \abs{g(z)} \leq 1.
\]
It immediately follows that $\abs{f'(0)} = \abs{g(0)} \leq 1$. Also
note that
\[
\frac{\abs{f(z)}}{\abs{z}} \leq \frac{1}{\rho},
\]
which in the limit gives
\[
\abs{f(z)} \leq \abs{z}.
\]
Suppose we have $\abs{f(z_0)} = \abs{z_0}$ for some $z_0 \ne 0$. As
then $\abs{g(z_0)} = 1$, it follows that $g$ is constant, therefore
$f(z) = \beta z$ for some $\beta \in \partial \dsk$. If we have
$\abs{f'(0)} = 0$, the same argument works for $z_0 = 0$.
\end{proof}

\newpage

\subsection{The \texorpdfstring{$\oline{\partial}$}{d} equation}

\begin{lema}
\label{hol:lm:bound_diff}
Let $g \in \mathcal{C}^\infty(\C)$ be a function with compact
support. Then there exists a function
$f \in \mathcal{C}^\infty(\C)$ such that $\parc{f}{\oline{z}} = g$.
\end{lema}

\begin{proof}
Let
\[
f(z) = \frac{1}{2 \pi i}
\liint_\C \frac{g(w)}{w-z}\,dw \wedge d\oline{w}.
\]
As
\[
dw \wedge d\oline{w} = -2ri\,dr \wedge d\varphi
\]
holds for polar coordinates centered at $z$, we can express the
integral as
\[
f(z) = -\frac{1}{\pi}
\liint_\C \frac{r g \br{z + r e^{i \varphi}}}{r e^{i \varphi}}\,
dr \wedge d\varphi.
\]
We can further simplify the integral, as there exists some $R$
such that $\eval{g}{\C \setminus \dsk(z, R)}{} = 0$. We get
\[
f(z) = -\frac{1}{\pi} \liint_{\dsk(z, R)}
g \br{z + r e^{i \varphi}} e^{-i \varphi}\,dr \wedge d\varphi,
\]
which obviously converges. The function $f$ is therefore well
defined. As we are integrating a smooth function on a compact set,
the function $f$ is smooth as well.

For $u = r e^{i \varphi}$, we have
\begin{align*}
\parc{f}{\oline{z}}(z) &=
-\frac{1}{\pi} \liint_{\dsk(z, R)} \parc{}{\oline{z}}
g \br{z + r e^{i \varphi}} e^{-i \varphi}\,dr \wedge d\varphi
\\
&=
\frac{1}{2 \pi i} \liint_{\dsk(0, R)} \parc{}{\oline{z}}
g(u + z) \frac{1}{u}\,du \wedge d\oline{u}
\\
&=
\frac{1}{2 \pi i} \liint_{\dsk(0, R)}
\parc{g}{\oline{u}}(u+z) \frac{1}{u}\,du \wedge d\oline{u}
\\
&=
\frac{1}{2 \pi i} \liint_{\dsk(z, R)}
\parc{g}{\oline{w}}(w) \frac{1}{w - z}\,dw \wedge d\oline{w}.
\end{align*}
Now we can apply the inhomogeneous Cauchy integral formula. We get
\[
g(z) = \frac{1}{2 \pi i} \olint_{\partial \dsk(z, R)}
\frac{g(w)}{w - z}\,dw + \frac{1}{2 \pi i} \liint_{\dsk(z, R)}
\parc{g}{\oline{w}}(w) \frac{1}{w - z}\,dw \wedge d\oline{w}.
\]
by the choice of $R$, we get
\[
\parc{f}{\oline{z}}(z) = g(z). \qedhere
\]
\end{proof}

\begin{lema}
Given bounded domain $U \subset V \subset \R^n$ such that
$\partial U \cap \partial V = \emptyset$, there exists a smooth
function $\chi \colon \R^n \to [0, 1]$ such that
$\eval{\chi}{U}{} = 1$ and $\supp \chi \subseteq V$.
\end{lema}

\begin{proof}
There is a unit partition on the sets $V$ and
$\R^n \setminus \oline{U}$.
\end{proof}

\datum{2023-10-11}

\begin{lema}
Let $\Omega \subseteq \C$ be an open subset. Let
$h_j \colon \Omega \to \C$ be holomorphic functions. If the
sequence $(h_j)_{j \in \N}$ converges uniformly on compact sets,
the limit is also holomorphic on $\Omega$.
\end{lema}

\begin{izrek}[Dolbeault lemma]
\index{Dolbeaut lemma}
Let $g \in \mathcal{C}^\infty(\dsk(R))$ for some
$R \in (0, \infty]$. Then there exists a function
$f \in \mathcal{C}^\infty(\dsk(R))$ such that
$\parc{f}{\oline{z}} = g$.
\end{izrek}

\begin{proof}
Define discs $X_j$ as follows:

\begin{enumerate}[i)]
\item If $R = \infty$, set $X_j = \dsk(j)$.
\item If $R < \infty$, set $X_j = \dsk \br{R - \frac{1}{j}}$ (for
large enough $j$).
\end{enumerate}

Applying the above lemma, define functions $\chi_j$ with
$\eval{\chi_j}{X_j}{} = 1$ and $\supp \chi_j \subseteq X_{j+1}$ and
set
\[
g_j =
\begin{cases}
\chi_j \cdot g & z \in \dsk(R), \\
       0       & z \not \in \dsk(R).
\end{cases}
\]
This is of course a smooth function, so by
lemma~\ref{hol:lm:bound_diff} there exists a function
$f_j \in \mathcal{C}^\infty(\C)$ with
\[
\parc{f_j}{\oline{z}} = g_j.
\]

We inductively construct a new sequence
$\widetilde{f}_j \in \mathcal{C}^\infty(\C)$ such that
\[
\parc{\widetilde{f}_j}{\oline{z}} = g
\]
on $X_j$ and
\[
\norm{\widetilde{f}_j - \widetilde{f}_{j-1}}_{X_{j-2}} \leq 2^{-j}.
\]
Set $\widetilde{f}_1 = f_1$. Observe the function
$F = f_{j+1} - \widetilde{f}_j$ on $X_j$. By construction, we have
$\parc{F}{\oline{z}} = 0$ on $X_j$. It follows that $F$ can be
developed into a power series
\[
F = \sum_{k=0}^\infty c_k z^k
\]
on $X_j$. As power series converge uniformly on compact sets, there
exists some polynomial $p \in \C[z]$ such that
\[
\norm{F - p}_{X_{j-1}} \leq 2^{-j}.
\]
Now just set $\widetilde{f}_{j+1} = f_{j+1} - p$.

Let $z \in \dsk(R)$ be arbitrary. By construction, it is contained
in some $X_{j_0}$, therefore, $\widetilde{f}_j$ is defined for
$j \geq j_0$. As $\br{\widetilde{f}_j(z)}_{j \geq j_0}$ is a
Cauchy sequence, we can define
\[
f(z) = \lim_{j \to \infty} \widetilde{f}_j(z).
\]
But as
\[
f - \widetilde{f}_j =
\sum_{k=j}^\infty \br{\widetilde{f}_{j+1} - \widetilde{f}_j}
\]
is a sum of holomorphic functions that converges uniformly, the
function $f - \widetilde{f}_j$ is a holomorphic function.
Therefore, $f$ is smooth and satisfies $\parc{f}{\oline{z}} = g$.
\end{proof}

\newpage

\subsection{Meromorphic functions}

\datum{2023-10-17}

\begin{definicija}
Let $\Omega \subset \C$ be an open subset. We call a function $f$
\emph{meromorphic}\index{meromorphic function} of $\Omega$ if there
exists $A \subset \Omega$ such that
$f \in \mathcal{O}(\Omega \setminus A)$, $A$ has no accumulation
points in $\Omega$ and for all $a \in A$ there exists some
$k \in \N$ such that
\[
\lim_{z \to a} f(z) \cdot (z-a)^k \ne 0
\]
exists. We call $A$ the set of \emph{poles}\index{pole} of the
function $f$. We denote the set of meromorphic functions on
$\Omega$ with $\mathcal{M}(\Omega)$.
\end{definicija}

\begin{izrek}
Let $0 \leq r < r \leq \infty$. Suppose that
$f \in \mathcal{O}(D_{R,r}(a))$ is a holomorphic function, where
\[
D_{R,r}(a) = \setb{z \in \C}{r < \abs{z-a} < R}.
\]
Then there exists a uniquely determined
\emph{Laurent series}\index{Laurent series}
\[
\sum_{k \in \Z} c_k (z-a)^k
\]
that converges to $f$ uniformly and absolutely on compact subsets
of $D_{R,r}(a)$. We have
\[
c_k = \frac{1}{2 \pi i} \olint_{\partial \dsk(a, \rho)}
\frac{f(w)}{(w - a)^k}\,dw
\]
for $r < \rho < R$.
\end{izrek}

\begin{definicija}
Let
\[
\sum_{k \in \Z} c_k (z-a)^k
\]
be a Laurent series. The series
\[
\sum_{k=-\infty}^{-1} c_k (z-a)^k
\]
is called the \emph{principle part}\index{principle part}.
\end{definicija}

\begin{lema}
Let $f \in \mathcal{O}(\Omega \setminus \set{a})$ be a holomorphic
function. Then $f$ is meromorphic on $\Omega$ if and only if $f$
has a finite principle part in $a$.
\end{lema}

\begin{proof}
Suppose that $f$ is meromorphic on $\Omega$. If $a$ is a removable
singularity, $f$ is holomorphic in $a$, therefore the principle
part is trivial. Otherwise, set $m \in \N$ such that
\[
\lim_{z \to a} (z-a)^m f(z) \ne 0
\]
exists and set $g(z) = (z-a)^m f(z)$. As $g$ is bounded near $a$,
we can extend it to $\Omega$ by the Riemann removable singularity
theorem. The power series of $g$ corresponds to a finite Laurent
series of $f$.

The converse is obvious.
\end{proof}

\begin{izrek}
If $f \in \mathcal{M}(\C)$ is a meromorphic function, there exist
entire functions $g$ and $h$ such that $f = \frac{g}{h}$.
\end{izrek}

\begin{definicija}
Let $\Omega \subseteq \C$ be an open set. An
\emph{additive Cousin problem}\index{additive Cousin problem} on
$\Omega$ is an open cover $\set{U_j}_{j \in J}$ of $\Omega$ and
functions $f_j \in \mathcal{M}(U_j)$ such that
$\eval{f_j - f_k}{U_j \cap U_k}{}$ is holomorphic for all
$j, k \in J$. A function $f \in \mathcal{M}(\Omega)$ is a solution
to the additive Cousin problem if $\eval{f}{U_j}{} - f_j$ is
holomorphic for all $j \in J$.
\end{definicija}

\begin{definicija}
Let $\Omega \subseteq \C$ be an open subset. A
\emph{generalized additive Cousin problem}\index{additive Cousin problem!generalized}
is an open cover $\set{U_j}_{j \in J}$ of $\Omega$ and functions
$f_{j,k} \in \mathcal{O}(U_j \cap U_k)$ for each $(j,k) \in J^2$,
such that

\begin{enumerate}[i)]
\item $f_{j,k} = -f_{k,j}$ on $U_j \cap U_k$ for all
$(j, k) \in J^2$ and
\item $f_{j,k} + f_{k, \ell} + f_{\ell, j} = 0$ on
$U_j \cap U_k \cap U_\ell$ for all $(j, k, \ell) \in J^3$.
\end{enumerate}

A solution to the generalized additive Cousin problem is given by
functions $f_j \in \mathcal{O}(U_j)$ for each $j \in J$ such that
$F_{j,k} = f_j - f_k$ for each $(j, k) \in J^2$.
\end{definicija}

\datum{2023-10-18}

\begin{lema}[Partition of unity]
Let $\Omega \subseteq \C$ be an open set and $\set{U_j}_{j \in J}$
be an open cover of $\Omega$. Then there exists a partition of
unity subordinate to $\set{U_j}_{j \in J}$.
\end{lema}

\begin{lema}
Given a generalized additive Cousin problem on
$\Omega \subseteq \C$, there exist functions
$g_j \in \mathcal{C}^\infty(U_j)$ such that $f_{j,k} = g_j - g_k$
for all $(j, k) \in J^2$.
\end{lema}

\begin{proof}
Let $\set{(V_a, \chi_a)}_{a \in A}$ be a partition of unity,
subordinate to $\set{U_j}_{j \in J}$. For all $a \in A$ choose a
$j(a) \in J$ such that $V_a \subseteq U_{j(a)}$. For all $k \in J$,
define
\[
g_k = -\sum_{a \in A} \chi_a \cdot f_{j(a), k}.
\]
This is of course a smooth function on $U_k$. Now note that
\[
g_k - g_\ell =
\sum_{a \in A} \chi_a \cdot \br{-f_{j(a),k} + f_{j(a),\ell}} =
\sum_{a \in A} \chi_a \cdot f_{k, \ell} =
f_{k, \ell}. \qedhere
\]
\end{proof}

\begin{trditev}
The generalized additive Cousin problem is solvable for
$\Omega = \dsk(r)$ and $\Omega = \C$.
\end{trditev}

\begin{proof}
Let $f_{j,k} = g_j - g_k$ for $g_j \in \mathcal{C}^\infty(U_j)$.
Note that
\[
\parc{g_j}{\oline{z}} = \parc{g_k}{\oline{z}},
\]
therefore,
\[
\eval{h}{U_j}{} = \parc{g_j}{\oline{z}}
\]
induces a smooth function $h \colon \Omega \to \C$. By the
Dolbeault lemma, there exists a function
$g \in \mathcal{C}^\infty(\Omega)$ such that
$\parc{g}{\oline{z}} = h$. It is clear that $f_j = g_j - g$ solves
the generalized additive Cousin problem.
\end{proof}

\begin{trditev}
The additive Cousin problem is solvable for $\Omega = \dsk(r)$ and
$\Omega = \C$.
\end{trditev}

\begin{proof}
An additive Cousin problem induces a generalized additive Cousin
problem for functions $f_{j,k} = f_j - f_k$. Let $g_j$ be a
solution to the generalized problem. As
$f_j - f_k = f_{j,k} = g_j - g_k$ on $U_j \cap U_k$, we can define
a function $f \in \mathcal{M}(\Omega)$ with
$\eval{f}{U_j}{} = f_j - g_j$. This function is of course well
defined. As $\eval{f}{U_j}{} - f_j = g_j \in \mathcal{O}(U_j)$,
this function indeed solves the additive Cousin problem.
\end{proof}

\begin{izrek}[Mittag-Leffler]
\index{Mittag-Leffler theorem}
Let $(a_k)_{k \in \N}$ be a sequence without repetition and
accumulation points. Let
\[
f_k(z) = \sum_{\ell = -m_k}^{-1} c_{k,\ell} (z-a_k)^\ell
\]
be finite principal parts. Then there exists a meromorphic function
$f \in \mathcal{M}(\C)$ with poles in $(a_k)_{k \in \N}$ such that
$f$ has principle part $f_k$ in $a_k$ for each $k \in \N$.
\end{izrek}

\begin{proof}
For each $a_k$ choose a disk $U_k$ containing no other $a_k$. Also
set $U_0 = \C \setminus \setb{a_k}{k \in \N}$ and $f_0 = 0$. As
$\setb{U_k}{k \in \N_0}$ is an open cover of $\C$, there exists a
meromorphic function $f \in \mathcal{M}(\C)$ that solves the
corresponding additive Cousin problem. It is easy to see that the
principle parts of $f$ at $a_k$ are precisely $f_k$.
\end{proof}

\newpage

\subsection{Sequences of holomorphic functions}

\datum{2023-10-24}

\begin{definicija}
A family of functions $\mathcal{F}$ from $\Omega$ to $\C$ is
\emph{locally bounded}\index{locally bounded functions}, if for all
$p \in \Omega$ there exist a $\rho > 0$ and $M > 0$ such that
\[
\sup_{f \in \mathcal{F}} \sup_{z \in \dsk(p, \rho)} \abs{f(z)} < M.
\]
\end{definicija}

\begin{lema}
\label{hol:lm:loc_equi}
Let $\Omega \subseteq \C$ be an open subset and
$\mathcal{F} \subseteq \mathcal{O}(\Omega)$ a locally bounded
family of functions. Then for all $p \in \Omega$ there exists a
$\rho > 0$ such that $\mathcal{F}$ is equi-continuous on
$\Omega \cap \dsk(p, \rho)$.
\end{lema}

\begin{proof}
Fix $p \in \Omega$ and choose $r > 0$ such that
$D = \oline{\dsk(p, 2r)} \subseteq \Omega$. For any $z, w \in D$
and $f \in \mathcal{F}$ we have
\[
f((z) - f(w) =
\frac{1}{2 \pi i} \olint_{\partial D} \frac{f(\xi)}{\xi-z}\,d\xi -
\frac{1}{2 \pi i} \olint_{\partial D} \frac{f(\xi)}{\xi-w}\,d\xi =
\frac{z-w}{2 \pi i} \olint_{\partial D}
\frac{f(\xi)}{(\xi-z)(\xi-w)}\,d\xi.
\]
Note that the family $\mathcal{F}$ is bounded on every compact.
Therefore, we can write
\[
\sup_{f \in \mathcal{F}} \sup_{z \in \partial D} \abs{f(z)} < M.
\]
Now, for $z, w \in \dsk(p, r)$ we have
\[
\abs{f((z) - f(w)} =
\abs{\frac{z-w}{2 \pi i} \olint_{\partial D}
\frac{f(\xi)}{(\xi-z)(\xi-w)}\,d\xi} \leq
\abs{z-w} \cdot \frac{2M}{r}. \qedhere
\]
\end{proof}

\begin{izrek}[Arzelà-Ascoli]
\index{Arzelà-Ascoli theorem}
Let $\Omega \subseteq \C$ be an open subset and let
$\mathcal{F} \subseteq \mathcal{O}(\Omega)$ be an infinite family
such that the following conditions hold:

\begin{enumerate}[i)]
\item $\mathcal{F}$ is point-wise bounded.
\item $\mathcal{F}$ is locally equi-continuous.
\end{enumerate}

Then there $\mathcal{F}$ contains a sequence that converges
uniformly on compacts of $\Omega$.
\end{izrek}

\begin{proof}
Choose a dense countable subset $A \subseteq \Omega$ and enumerate
it as a sequence $(a_k)_{k \in \N}$. Pick any sequence
$(f_n)_{n \in \N} \subseteq \mathcal{F}$ with pairwise distinct
terms. As $\abs{f_n(a_1)} < M$ for all $n$, we can choose a
subsequence $(f_{1,n})_{n \in \N}$ such that $f_{1,n}(a_1)$
converges by Bolzano-Weierstrass.

Similarly, for every $k \in \N$ there exists a subsequence
$(f_{k,n})_{n}$ of $(f_{k-1,n})_n$ such that $(f_{k,n}(a_k))_n$
converges. Now define $F_n = f_{n,n}$. Observe that $(F_n)$
converges at every point in $A$.

Fix a $p \in \Omega$. By local equi-continuity, there exists a
$\rho > 0$ such that for all $\varepsilon > 0$ there exists a
$\delta > 0$ such that $\delta < \rho$ and
$\abs{F_n(z) - F_n(w)} < \frac{\varepsilon}{3}$ for all
$z, w \in \dsk(p, \rho)$ such that $\abs{z-w} < \delta$. Choose an
element $a \in A \cap \dsk(z, \delta)$.\footnote{By compactness of
$\oline{\dsk(p, \rho)}$ we can choose $a$ from a finite set.} Then,we have
\[
\abs{F_n(z) - F_m(z)} \leq
\abs{F_n(z) - F_n(a)} +
\abs{F_n(a) - F_m(a)} +
\abs{F_m(a) - F_m(z)} <
3 \cdot \frac{\varepsilon}{3}.
\]
It follows that $(F_n)$ is locally uniformly convergent, therefore
it converges uniformly on compact sets.
\end{proof}

\begin{izrek}[Montel]
\index{Montel's theorem}
Let $\Omega \subseteq \C$ be an open subset and
$f_n \colon \Omega \to \C$ be a locally bounded sequence of
holomorphic functions. Then $(f_n)_n$ contains a subsequence that
converges uniformly on compacts.
\end{izrek}

\begin{proof}
As the sequence is locally bounded, it is locally equi-continuous
by lemma~\ref{hol:lm:loc_equi}. By Arzelà-Ascoli, there exists a
convergent subsequence.
\end{proof}

\begin{definicija}
Let $\Omega \subseteq \C$ be an open subset. A family of functions
$\mathcal{F} \subseteq \mathcal{O}(\Omega)$ is
\emph{normal}\index{normal family} if every sequence in
$\mathcal{F}$ contains a subsequence that converges uniformly on
compacts.
\end{definicija}

\begin{izrek}[Montel]
\index{Montel's theorem}
Let $\Omega \subseteq \C$ be an open subset. A family
$\mathcal{F} \subseteq \mathcal{O}(\Omega)$ is normal if and only
if it is locally bounded.
\end{izrek}

\obvs

\datum{2023-10-25}

\begin{izrek}[Vitali]
\index{Vitali's theorem}
Let $\Omega \subseteq \C$ be a domain and
$(f_n)_n \subseteq \mathcal{O}(\Omega)$ a locally bounded sequence
of holomorphic functions. The following statements are equivalent:

\begin{enumerate}[i)]
\item The sequence $(f_n)_n$ converges uniformly on compact subsets
of $\Omega$.
\item For a point $p \in \Omega$ and $k \in \N_0$ the sequence
$(f_n^{(k)}(p))_n$ converges.
\item The set
\[
A =
\setb{z \in \Omega}{\lim_{n \to \infty} f_n(z)~\text{converges}}
\]
has an accumulation point in $\Omega$.
\end{enumerate}
\end{izrek}

\begin{proof}
Suppose that the sequence converges uniformly on compact subsets.
Given a $p \in \Omega$, choose a $\delta > 0$ such that
$D = \oline{\dsk(p, \delta)} \subseteq \Omega$. Note that
\[
\abs{g^{(k)}(p)} \leq \frac{k!}{\delta^k} \cdot \norm{g}_D
\]
holds for all holomorphic functions $g$. As
$\norm{f - f_n}$ converges to $0$, the derivatives of $f_n$
converge.

Suppose that the sequence of derivatives converges in each point.
Fix a point $p \in \Delta$ and choose a $\delta > 0$ such that
$D = \oline{\dsk(p, \delta)} \subseteq \Omega$. As the sequence is
locally bounded, there exists a constant $M$ such that
$\norm{f_n}_D \leq M$ holds for all $n \in \N$. We can now develop
power series
\[
f_n(z) = \sum_{k=0}^\infty a_{k,n} (z-p)^k.
\]
They converge uniformly on compact subsets of $\dsk(p, \delta)$.
Note that
\[
a_{k,n} = \frac{f_n^{(k)}(p)}{k!}.
\]
As derivatives converge, we can define the limit
\[
a_k = \lim_{n \to \infty} a_{k,n}.
\]
Now define the formal power series
\[
f(z) = \sum_{k=0}^\infty a_k (z-p)^k.
\]
The Cauchy bounds give us the inequality
\[
\abs{a_{k,n}} =
\frac{\abs{f^{(k)}(p)}}{k!} \leq \frac{M}{\delta^k},
\]
therefore
\[
\limsup_{k \to \infty} \sqrt[k]{\abs{a_k}} \leq
\limsup_{k \to \infty} \frac{\sqrt[k]{M}}{\delta} =
\frac{1}{\delta}.
\]
We conclude that the radius of convergence is at least $\delta$.
Consider some $\rho \in (0, \delta)$ and
$z \in \dsk(p, \rho)$. We have
\begin{align*}
\abs{f_n(z) - f(z)} &\leq
\abs{\sum_{k=0}^m (a_{k,n} - a_k) \cdot (p-z)^k} +
\abs{\sum_{k=m+1}^\infty (a_{k,n} - a_k) \cdot (p-z)^k}
\\
&\leq
\sum_{k=0}^m \abs{a_{n,k} - a_k} \rho^k +
\sum_{k=m+1}^\infty 2M \cdot \frac{\rho^k}{\delta^k}
\\
&=
\sum_{k=0}^m \abs{a_{n,k} - a_k} \rho^k +
2M \cdot \br{\frac{\rho}{\delta}}^{m+1} \cdot
\frac{\delta}{\delta - \rho}
\\
&=
2 \cdot \frac{\varepsilon}{2}
\end{align*}
for large enough $m$ and $n$. It follows that $p$ is an
accumulation point of $A$.

Suppose now that $A$ has an accumulation point in $\Omega$. By
Montel's theorem there exists a subsequence $(f_{n_m})_m$ that
converges uniformly on compact subsets of $\Omega$ to a limit
function $f$. Note that all such subsequences have the same limit
by the identity principle.

Assume that the sequence $(f_n)_n$ does not converge uniformly on
a compact subset $K \subseteq \Omega$. We can therefore construct
another subsequence $(g_n)_n$ of $(f_n)_n$ such that
\[
\norm{g_n - f}_K > \varepsilon
\]
for all $n \in \N$. But note that $(g_n)_n$ also has a convergent
subsequence by Montel's theorem, which is of course a
contradiction, as it cannot converge to $f$.
\end{proof}
