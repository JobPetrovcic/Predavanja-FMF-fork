\section{Ring structure of holomorphic functions}

\subsection{Ideals of holomorphic functions}

\begin{definicija}
Let $\Omega \subseteq \C$ be an open set. A
\emph{divisor}\index{divisor} of a meromorphic function
$f \in \mathcal{M}^*(\Omega)$ is the function
$(f) \colon \Omega \to \Z$, given by
\[
(f)(z) = \begin{cases}
 n & \text{$f$ has a zero of order $n$ in $z$} \\
-n & \text{$f$ has a pole of order $n$ in $z$} \\
 0 & \text{otherwise}.                         \\
\end{cases}
\]
\end{definicija}

\begin{opomba}
The divisor of a product is the sum of divisors,
i.e.~$(f \cdot g) = (f) + (g)$.
\end{opomba}

\begin{definicija}
Let $S \subseteq \mathcal{O}(\Omega)$ be a subset that contains a
non-zero holomorphic function.

Define
\[
d(z) = \min_{f \in S \setminus \set{0}} (f)(z) \in \N_0.
\]
By Weierstraß product theorem there exists a function
$g \in \mathcal{O}(\Omega)$ such that $(g) = d$. We define
$\gcd(S) = g$\index{greatest common divisor}.\footnote{There are of
course multiple possible functions that satisfy this condition, but
their quotients are invertible.}
\end{definicija}

\datum{2023-12-18}

\begin{lema}[Wedderburn]
\index{Wedderburn!lemma}
Let $\Omega \subseteq \C$ be a domain and
$f, g \in \mathcal{O}(\Omega)$ be functions with $\gcd(f, g) = 1$.
Then there exist functions $a, b \in \mathcal{O}(\Omega)$ so that
$a f + b g = 1$. Moreover, we can choose $a$ to be nonwhere
vanishing. 
\end{lema}

\begin{proof}
If $g = 0$ and $f \neq 0$ then $f$ cannot vanish by assumption on
$\gcd$, therefore $a = \frac{1}{f}$ and $b = 1$ suffice. Therefore,
we can assume both $f, g$ and are nonzero. Note that
$Z(f) \cap Z(g)$ is empty, since $(z - p)$ divides $\gcd(f, g)$ for
any $p \in Z(f) \cap Z(g)$. The set $Z(f) \cup Z(g)$ is thus
discrete. Further, for each zero $p$ of $g$, there exists a disk of
radius $\varepsilon$ and a holomorphic function
$f_p \in \mathcal{O}(\dsk(p, \varepsilon))$ such that
\[
f = e^{f_p}.
\]
By the Mittag-Leffler osculation theorem there exists a function
$h \in \mathcal{O}(\Omega)$ such that
$\ord_p(h - f_p) > \ord_p(g)$.

Here we stop for a short observation. Developing into the power
series, we get that $e^{w^n} - 1 = w^n + O(w^{2 n})$. Then, 
\[
\ord_p \br{f - e^h} =
\ord_p \br{e^h \cdot \br{e^{f_p - h} - 1}} =
\ord_p \br{\br{e^{f_p - h} - 1}} =
\ord_p \br{f_p - h} >
\ord_p(g).
\]

Define $k = \frac{f - e^h}{g} \in \mathcal{O}(\Omega)$. We claim
that $a = e^{-h}$ and $b = -k e^{-h}$ satisfy the conditions.
Clearly, $a$ doesn't vanish, and
\[
a f + b g =
e^{-h} f - k e ^{-h} g =
e^{-h}(f - k g) =
e^{- h} \br{f - \frac{f - e^h}{g} g} =
e^{-h} e^h =
1. \qedhere
\]
\end{proof}

\begin{posledica}
For holomorphic functions $f_j \in \mathcal{O}(\Omega)$, where
$j \leq n$, we can write $f = \gcd(f_1, f_2, \dots, f_n)$ as 
\[
f = \sum_{j=1}^n a_j f_j
\]
\end{posledica}

\begin{proof}
We proceed by induction. The base case is just Wedderburn's lemma.
Now let $\hat{f} = \gcd(f_2, f_3, \dots f_n)$, which can be
written as
\[
\hat{f} = \sum_{j=2}^n \hat{a}_j f_j
\]
by the induction hypothesis. Then
$\frac{f_1}{f}, \frac{\hat{f}}{f} \in \mathcal{O}(\Omega)$ are
holomorphic functions with $\gcd$ equal to $1$. We can therefore
apply Wedderburn' lemma to get functions $a$ and $b$ such that
\[
a \frac{f_1}{f} + b \frac{\hat{f}}{f} = 1.
\]
The conclusion follows
\end{proof}

\begin{izrek}
Let $I \edn \mathcal{O}(\Omega)$ be the ideal generated by
holomorphic functions $f_1, f_2, \dots f_n$ on $\Omega$. Then there
exists a holomorphic function $f$ such that $I = (f)$.
\end{izrek}

\begin{proof}
Take $f = \gcd(f_j)$. This function is an element of $I$ by the
previous corollary. Since $f \mid f_j$, this implies that
$I = (f)$.
\end{proof}

\begin{definicija}
Let $\Omega \subseteq \C$ be a domain and
$I \edn \mathcal{O}(\Omega)$ an ideal.

\begin{enumerate}[i)]
\item We call $I$ \emph{closed}\index{closed ideal} if for every
sequence $(f_n)_n \subseteq I$ that converges uniformly on compacts
of $\Omega$ to some function $f$, we also have $f \in I$.
\item We call $p \in \Omega$ a \emph{zero}\index{zero} of $I$ if
$f(p) = 0$ for every $f \in I$.
\end{enumerate}
\end{definicija}

\begin{lema}
Let $\Omega \subseteq \C$ be a domain and
$I \edn \mathcal{O}(\Omega)$ and ideal. Let $p \in \Omega$ be a
point that is not a zero of $I$. Let
$f, g \in \mathcal{O}(\Omega)$ be functions such that $f(z) \ne 0$
for all $z \ne p$. If $f g \in I$, then $g \in I$.
\end{lema}

\begin{proof}
Since $p$ is not a zero of $I$, then there exists a function
$h \in I$ such that $h(p) \ne 0$. Let $n = \ord_p(f)$. If $n=0$,
then $f$ is a unit, so $g \in I$. Otherwise, we have
\[
\frac{f(z)}{z - p}g =
-\frac{1}{h(p)} \cdot
\br{\frac{h - h(p)}{z - p} fg - \frac{f g}{z - p} h} \in I
\]
since $\frac{f}{z - p}$ is holomorphic.
    
We can iterate this process to find $\frac{f}{(z - p)^n} g \in I$.
Since $\frac{f}{(z - p)^n}$ is a unit, $g$ must be an element of
$I$.
\end{proof}

\begin{izrek}
Let $\Omega \subseteq \C$ be a domain and
$I \edn \mathcal{O}(\Omega)$ an ideal. If $I$ has no zeroes and is
closed, then $I = \mathcal{O}(\Omega)$.
\end{izrek}

\begin{proof}
Let $f$ be an arbitrary nonzero element of $I$. By the Weierstraß
product theorem, we can write
\[
f = \prod_{k = 1}^{\infty} f_k,
\]
where each $f_k$ has exactly one zero in $\Omega$, and the tails
\[
\widehat{f}_n = \prod_{k = n}^{\infty} f_k
\]
converge to $1$ uniformly on compacts of $\Omega$. As
$f = \widehat{f}_1 = f_1 \widehat{f}_2$, we can apply the preivous
lemma to find $\widehat{f}_2 \in I$. Inductively,
$\widehat{f}_n \in I$ and since the ideal $I$ is assumed to be
closed, we have
\[
1 = \lim_{k \to \infty} \widehat{f}_k \in I. \qedhere
\]
\end{proof}

