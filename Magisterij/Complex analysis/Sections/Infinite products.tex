\section{Infinite products}

\subsection{Definition and convergence}

\begin{definicija}
\index{infinite product}
Let $(a_k)_k$ be a sequence of complex numbers. The sequence
\[
n \mapsto \prod_{k=1}^n a_k
\]
is called the
\emph{sequence of partial products}\index{partial products} with
factors $a_k$. We denote
\[
p_{m,n} = \prod_{k=m}^n a_k.
\]
We say that the infinite product is \emph{convergent} if there
exists an index $m \in \N$ such that the limit
\[
\widehat{a}_m = \lim_{n \to \infty} p_{m,n}
\]
exists and is non-zero. We then define
\[
\prod_{k=1}^\infty a_k = p_{1, m-1} \cdot  \widehat{a}_m.
\]
as the limit of the infinite product.
\end{definicija}

\begin{opomba}
The limit is uniquely defined.
\end{opomba}

\begin{opomba}
An infinite product is convergent if and only if the product of all
its non-zero factors has a non-zero limit and only finitely many
factors are non-zero.
\end{opomba}

\datum{2023-11-22}

\begin{lema}
Let $(a_k)_k \subseteq \R_{\geq 0}$ be a sequence such that
\[
\sum_{k=1}^\infty (1 - a_k) = \infty.
\]
Then
\[
\lim_{n \to \infty} \prod_{k=p}^n a_k = 0
\]
for all $p \in \N$. In particular, the infinite product is
divergent.
\end{lema}

\begin{proof}
Observe that
\[
0 \leq
\prod_{k=p}^n a_k \leq
\prod_{k=p}^n e^{a_k - 1},
\]
which converges to $0$.
\end{proof}

\begin{definicija}
Let $X \subseteq \C$ be a set.

\begin{enumerate}[i)]
\item A series
\[
\sum_{k=1}^\infty g_k
\]
of continuous functions $g_k \in \mathcal{C}(X)$ is
\emph{normally convergent}\index{normal convergence} if for every
compact $K \subseteq X$ the series
\[
\sum_{k=1}^\infty \norm{g_k}_K
\]
converges.

\item A product
\[
\prod_{k=1}^\infty f_k
\]
of continuous functions $f_k = 1 + g_k \in \mathcal{C}(X)$ is
\emph{normally convergent} if the series
\[
\sum_{k=1}^\infty g_k
\]
is normally convergent.
\end{enumerate}
\end{definicija}

\begin{definicija}
Let $X \subseteq \C$ be a set and $f_k \in \mathcal{C}(X)$ be
continuous functions. Denote
\[
p_{m,n} = \prod_{k=m}^n f_k.
\]
We say that the infinite product
\[
\prod_{k=1}^\infty f_k
\]
converges \emph{uniformly}\index{uniform convergence} on a set
$L \subseteq X$ if there exists an index $m \in \N$ such that
$\eval{f_k}{L}{}$ has no zeroes for $k \geq m$ and
\[
\lim_{n \to \infty} p_{m,n} = \widehat{f}_k
\]
exists, is uniform on $L$ and has no zeroes on $L$. We define
\[
\prod_{k=1}^\infty f_k = p_{1, m-1} \cdot \widehat{f}_m
\]
on $L$.
\end{definicija}

\begin{izrek}[Reordering of infinite products]
\index{reordering theorem}
Let
\[
\prod_{k=1}^\infty f_k
\]
be a normally convergent product in $X \subseteq \C$. Then there
exists a functions $f \colon X \to \C$ such that for all bijections
$\tau \colon \N \to \N$ the product
\[
\prod_{k=1}^\infty f_{\tau(k)}
\]
converges to $f$ uniformly on compacts of $X$. In particular, the
infinite product converges uniformly on compacts.
\end{izrek}

\begin{proof}
Recall that, for $w \in \dsk$, we can define
\[
\log(1+w) = \sum_{k=1}^\infty \frac{(-1)^{k+1}}{k} w^k.
\]
Then,
\[
\abs{\log(1+w)} \leq
\abs{w} \cdot \sum_{k=0}^\infty \abs{w}^k =
\frac{\abs{w}}{1 - \abs{w}}.
\]
In particular, if $\abs{w} \leq \frac{1}{2}$, we have
\[
\abs{\log(1+w)} \leq 2 \abs{w}.
\]
Let $L \subseteq X$ be a compact and write $f_k = 1 + g_k$. For
all $k > N$ we have $\norm{g_k}_L \leq \frac{1}{2}$, therefore we
can write
\[
\log f_k =
\log(1 + g_k) =
\sum_{\ell=1}^\infty \frac{(-1)^{\ell+1}}{\ell} g_k^\ell.
\]
But then
\[
\norm{\log f_k}_L \leq 2 \norm{g_k}_L.
\]
It follows that the series
\[
\sum_{k=N}^\infty \norm{\log f_k}_L
\]
converges. But then the series
\[
h_N = \sum_{k=N}^\infty \log f_k
\]
converges absolutely, and therefore all reorderings of the series
converge as well to the same limit $h_N$.

Observe that
\[
e^{h_N} = \prod_{k=N}^\infty e^{\log f_k} = \prod_{k=N}^\infty f_k.
\]
This product therefore converges uniformly on $L$, independently of
reorderings. We now define
\[
f = \prod_{k=1}^{N-1} f_k \cdot e^{h_N}.
\]
Note that this holds for all reorderings, as they differ from a
suitable one by only finitely many transpositions.
\end{proof}

\newpage

\subsection{Zeroes of infinite products}

\datum{2023-11-28}

\begin{definicija}
Let $\Omega \subseteq \C$ be an open set and
$f \in \mathcal{O}(\Omega)$. The \emph{zero set}\index{zero set} of
$f$ is the set
\[
Z(f) = \setb{z \in \Omega}{f(z) = 0}.
\]
For all $c \in \Omega$, define the
\emph{zero order}\index{zero order} of $f$ in $c$ as follows: if
\[
f(z) = (z - c)^k \cdot g(z)
\]
where $g(c) \ne 0$ is a holomorphic function, then $\ord_c(f) = k$.
\end{definicija}

\begin{opomba}
For non-zero $f \in \mathcal{O}(\Omega)$, the set $Z(f)$ is
discrete in $\Omega$.
\end{opomba}

\begin{opomba}
We have
\[
\ord_c \br{\prod_{k=1}^n f_k} = \sum_{k=1}^n \ord_c(f_k).
\]
\end{opomba}

\begin{lema}
Let $\Omega \subseteq \C$ be a domain and
\[
f = \prod_{k=1}^\infty f_k
\]
be a normally convergent product in $\Omega$, where
$f_k \in \mathcal{O}(\Omega)$ are non-zero holomorphic functions.
Then $f$ is a non-zero function with
\[
Z(f) = \bigcup_{k=1}^\infty Z(f_k)
\]
and
\[
\ord_c(f) = \sum_{k=1}^\infty \ord_c(f_k).
\]
\end{lema}

\begin{proof}
Recall that normally convergent products converge uniformly on
compacts of $\Omega$. In particular, $f$ is a holomorphic function.

Pick a point $c \in \Omega$. By definition of convergence, there
exists some $m \in \N$ such that $\widehat{f}_m(c) \ne 0$. As
$\widehat{f}_m$ is holomorphic as well, we have
\[
f(c) = \br{p_{1, m-1} \cdot \widehat{f}_m}(c),
\]
but then
\[
\ord_c(f) =
\sum_{k=1}^{m-1} \ord_c(f_k) =
\sum_{k=1}^\infty \ord_c(f_k). \qedhere
\]
\end{proof}

\begin{lema}
Let $\Omega \subseteq \C$ be a domain. If
\[
f = \prod_{k=1}^\infty f_k
\]
is a normally convergent product in $\Omega$, where
$f_k \in \mathcal{O}(\Omega)$ are holomorphic functions, then the
sequence $\br{\widehat{f}_n}_n$ converges to $1$ uniformly on
compacts.
\end{lema}

\begin{proof}
Choose $m \in \N$ such that $\widehat{f}_m \ne 0$. Then the set
$Z \br{\widehat{f}_m}$ has no accumulation points in $\Omega$. We
can therefore write
\[
\widehat{f}_n = \frac{\widehat{f}_m}{p_{m,n-1}}
\]
on $\Omega \setminus Z \br{\widehat{f}_m}$. As $p_{m,n-1}$
converges to $\widehat{f}_m$ on compacts of $\Omega$,
\[
\lim_{n \to \infty} \widehat{f}_n = 1
\]
uniformly on compacts of $\Omega \setminus Z \br{\widehat{f}_m}$.
For any compact set $K \subseteq \Omega$, taking $m$ large enough,
we have $Z \br{\widehat{f}_m} \cap K = \emptyset$. The conclusion
follows.
\end{proof}

\begin{definicija}
Let $\Omega \subseteq \C$ be a domain and
$f \in \mathcal{O}(\Omega)$. The meromorphic function
$\frac{f'}{f}$ is called the
\emph{logarithmic derivative}\index{logarithmic derivative} of $f$.
\end{definicija}

\begin{opomba}
For holomorphic functions $f_1, \dots, f_n \in \mathcal{O}(\Omega)$
we have
\[
\br{\prod_{k=1}^n f_k}' \cdot \br{\prod_{k=1}^n f_k}^{-1} =
\sum_{k=1}^n \frac{f_k'}{f_k}.
\]
\end{opomba}

\begin{definicija}
Let $g_k \in \mathcal{M}(\Omega)$ be meromorphic functions. The
series
\[
\sum_{k=1}^\infty g_k
\]
is \emph{normally convergent}\index{normal convergence} in $\Omega$
if for every compact $L \subseteq \Omega$ there exists some
$m \in \N$ such that
\[
\sum_{k=m}^\infty \norm{g_k}_L
\]
converges.
\end{definicija}

\begin{izrek}[Logarithmic differentiation]
Let $\Omega \subseteq \C$ be a domain and
\[
f = \prod_{k=1}^\infty f_k
\]
be a normally convergent product in $\Omega$, where
$f_k \in \mathcal{O}(\Omega)$ are non-zero functions. Then
\[
\sum_{k=1}^\infty \frac{f_k'}{f_k}
\]
is normally convergent in $\Omega$ and
\[
\sum_{k=1}^\infty \frac{f_k'}{f_k} = \frac{f'}{f}.
\]
\end{izrek}

\begin{proof}
As $\widehat{f}_n$ converges to $1$ uniformly on compacts, the
sequence $\br{f_n'}_n$ converges to $0$ uniformly on compacts by
Cauchy estimates. Then for any compact $L$,
$\frac{\widehat{f}_n'}{\widehat{f}_n}$ converges to $0$ as
$\widehat{f}_n$ has no zeroes in $L$ for $n$ large enough. It
follows that
\[
\lim_{n \to \infty} \frac{f'}{f} - \sum_{k=1}^n \frac{f_k'}{f_k} =
\lim_{n \to \infty} \frac{\widehat{f}_{n+1}'}{\widehat{f}_{n+1}} =
0.
\]
Write $f_k = 1 + g_k$ and fix a compact set $L \subseteq \Omega$.
Choose an index $m$ such that we have
$Z \br{\widehat{f}_m} \cap L = \emptyset$ and
\[
\min_{z \in L} \abs{f_k(z)} \geq \frac{1}{2}.
\]
Choose $\varepsilon > 0$ such that
\[
L_\varepsilon =
\setb{z \in \C}{d(z, L) \leq \varepsilon} \subseteq
\Omega.
\]
By the Cauchy estimates, we have
$\norm{g_k'}_L \leq \frac{1}{\varepsilon} \norm{g_k}_L$. But then
\[
\sum_{k=m}^\infty \norm{\frac{f_k'}{f_k}}_L =
\sum_{k=m}^\infty \norm{\frac{g_k'}{f_k}}_L \leq
2 \cdot \sum_{k=m}^\infty \norm{g_k'}_L \leq
\frac{2}{\varepsilon} \cdot \sum_{k=m}^\infty \norm{g_k},
\]
which is convergent by our assumptions.
\end{proof}

\begin{lema}
Let $g$ be meromorphic on $\C$ with poles in $\Z$ with principal
parts $\frac{1}{z-m}$. Moreover, assume that $g$ is an odd
function that satisfies
\[
2 g(2z) = g(z) + g \br{z + \frac{1}{2}}.
\]
Then $g(z) = \pi \cdot \cot(\pi z)$.
\end{lema}

\begin{proof}
Simple calculations show that $\pi \cdot \cot(\pi z)$ is indeed a
solution of the functional equation. Define
$h(z) = g(z) - \pi \cdot \cot(\pi z)$. This another solution of the
functional equation, and an odd function. In particular,
$h(0) = 0$. Observe that the principal parts of $h$ are $0$,
therefore $h \in \mathcal{O}(\C)$ is an entire function.

Suppose that $h$ is not constant. In particular, there exists some
$c \in \partial \dsk(2)$ such that
\[
\abs{h(z)} < \abs{h(c)}
\]
for all $z \in \dsk(2)$. As
$\frac{c}{2}, \frac{c+1}{2} \in \dsk(2)$ , we can write
\[
2 \abs{h(c)} =
\abs{h \br{\frac{c}{2}} + h \br{\frac{c+1}{2}}} \leq
\abs{h \br{\frac{c}{2}}} + \abs{h \br{\frac{c+1}{2}}} <
2 \abs{h(c)},
\]
which is a contradiction. It follows that $h = 0$.
\end{proof}

\begin{posledica}
We have
\[
\pi \cdot \cot(\pi z) =
\frac{1}{z} + \sum_{k=1}^\infty \frac{2z}{z^2 - k^2}.
\]
\end{posledica}

\begin{proof}
Note that
\[
\frac{1}{z} + \sum_{k=1}^\infty \frac{2z}{z^2 - k^2} =
\frac{1}{z} + \sum_{k=1}^\infty \br{\frac{1}{z-k} + \frac{1}{z+k}},
\]
therefore the series has poles in $\Z$ with principal parts
$\frac{1}{z-m}$. It is also an odd function. A calculation shows
that, for
\[
r_n(z) = \frac{1}{z} + \sum_{k=1}^n \frac{2z}{z^2 - k^2},
\]
we have
\[
r_n(z) + r_n \br{z + \frac{1}{2}} =
2 r_{2n}(2z) + \frac{2}{2z + 2n + 1}.
\]
Taking $n \to \infty$, the conclusion follows.
\end{proof}

\begin{izrek}
We have
\[
\sin(\pi z) =
\pi z \cdot \prod_{k=1}^\infty \br{1 - \frac{z^2}{k^2}}.
\]
\end{izrek}

\begin{proof}
The above product is obviously normally convergent, therefore we
can take its logarithmic derivative. A simple calculation shows
that it is equal to $\pi \cot(\pi z)$. As logarithmic derivatives
are equal only for scalar multiples, we only have to check equality
in one point.
\end{proof}

\newpage

\subsection{The Euler gamma function}

\datum{2023-11-29}

\begin{lema}
The infinite product
\[
\prod_{k=1}^\infty \br{1 + \frac{1}{k}} e^{-\frac{z}{k}}
\]
in normally convergent in $\C$.
\end{lema}

\begin{proof}
Write
\begin{align*}
\abs{1 - (1-\omega) e^\omega} &=
\abs{1 - e^\omega + \omega e^\omega}
\\
&=
\abs{-\sum_{k=1}^\infty \frac{\omega^k}{k!} +
\sum_{k=0}^\infty \frac{\omega^{k+1}}{k!}}
\\
&=
\abs{\omega^2 \cdot \sum_{k=1}^\infty
\br{\frac{1}{k!} - \frac{1}{(k+1)!}} \omega^{k-1}}
\\
&\leq
\abs{\omega}^2 \cdot \sum_{k=1}^\infty
\br{\frac{1}{k!} - \frac{1}{(k+1)!}}
\\
&=
\abs{\omega}^2
\end{align*}
for $\abs{\omega} \leq 1$. But then the sum
\[
\sum_{k = \ceil{\abs{z}}}^\infty
\abs{1 - \br{1 + \frac{z}{k}} e^{-\frac{z}{k}}} \leq
\sum_{k = \ceil{\abs{z}}}^\infty \abs{\frac{z^2}{k^2}}
\]
converges normally. The infinite product must then converge
normally in $\C$ as well.
\end{proof}

\begin{lema}
Let
\[
H(z) =
z \cdot \prod_{k=1}^\infty \br{1 + \frac{z}{k}} e^{-\frac{z}{k}}.
\]
Then $H(1) = e^{-\gamma}$, where $\gamma$ is the
\emph{Euler-Mascheroni constant}\index{Euler-Mascheroni constant},
that is
\[
\gamma = \lim_{n \to \infty} \sum_{k=1}^n \frac{1}{k} - \log(n).
\]
\end{lema}

\begin{proof}
First note that
\[
\prod_{k=1}^n \br{1 + \frac{1}{k}} =
\prod_{k=1}^n \frac{k+1}{k} =
n+1.
\]
We therefore have
\[
\prod_{k=1}^n \br{1 + \frac{1}{k}} e^{-\frac{1}{k}} =
\exp \br{\log(n+1) - \sum_{k=1}^n \frac{1}{k}},
\]
therefore
\[
H(1) =
\lim_{n \to \infty}
\exp \br{\log(n+1) - \sum_{k=1}^n \frac{1}{k}} =
e^{-\gamma}. \qedhere
\]
\end{proof}

\begin{lema}
Let $\Delta(z) = e^{\gamma z} H(z)$.

\begin{enumerate}[i)]
\item We have $\Delta(1) = 1$ and $\Delta(z) = z \Delta(z+1)$.
\item We have $\pi \cdot \Delta(z) \Delta(1-z) = \sin(\pi z)$.
\end{enumerate}
\end{lema}

\begin{proof}
Note that $\Delta(1) = 1$ by the previous lemma. Rewrite the
partial products as
\[
z \cdot \prod_{k=1}^n \br{1 + \frac{z}{k}} e^{-\frac{z}{k}} =
\frac{z}{n!} \cdot \prod_{k=1}^n (z+k) \cdot
\exp \br{-z \sum_{k=1}^n \frac{1}{k}}.
\]
We therefore have
\begin{align*}
\Delta(z) &=
\lim_{n \to \infty}
\frac{e^{\gamma z}}{n!} \cdot \prod_{k=0}^n (z+k) \cdot
\exp \br{-z \sum_{k=1}^n \frac{1}{k}}
\\
&=
\lim_{n \to \infty}
\frac{e^{\gamma z}}{n! \cdot n^z} \cdot \prod_{k=0}^n (z+k) \cdot
\exp \br{z \log(n) -z \sum_{k=1}^n \frac{1}{k}}
\\
&=
\lim_{n \to \infty}
\frac{1}{n! \cdot n^z} \cdot \prod_{k=0}^n (z+k).
\end{align*}
We can now calculate
\[
z \cdot \Delta(z+1) =
\lim_{n \to \infty}
z \cdot \frac{1}{n! \cdot n^{z+1}} \cdot \prod_{k=1}^{n+1} (z+k) =
\Delta(z) \cdot \lim_{n \to \infty} \frac{z + n + 1}{n} =
\Delta(z).
\]
It remains to check the equality
$\pi \cdot \Delta(z) \Delta(1-z) = \sin(\pi z)$. We have
\begin{align*}
\pi \cdot \Delta(z) \Delta(1-z) &=
\pi \cdot \Delta(z) \cdot \frac{\Delta(-z)}{-z}
\\
&=
\pi e^{\gamma z} \cdot z \cdot
\prod_{k=1}^\infty \br{1 + \frac{z}{k}} e^{-\frac{z}{k}} \cdot
e^{-\gamma z} \cdot \frac{-z}{-z} \cdot
\prod_{k=1}^\infty \br{1 - \frac{z}{k}} e^{\frac{z}{k}}
\\
&=
\pi z \cdot \prod_{k=1}^\infty \br{1 - \frac{z^2}{k^2}}
\\
&=
\sin(\pi z). \qedhere
\end{align*}
\end{proof}

\begin{definicija}
The \emph{Euler gamma function}\index{Euler gamma function} is
defined as
\[
\Gamma(z) = \frac{1}{\Delta(z)}.
\]
\end{definicija}

\begin{izrek}
The $\Gamma$ function satisfies the following properties:

\begin{enumerate}
\item The function $\Gamma$ is meromorphic with simple poles in
$-\N_0$.
\item We have $\Gamma(1) = 1$.
\item The function $\Gamma$ satisfies $\Gamma(z+1) = z \Gamma(z)$.
\item The function $\Gamma$ satisfies
\[
\Gamma(z) \cdot \Gamma(1-z) = \frac{\pi}{\sin(\pi z)}.
\]
\item We have
\[
\Gamma(z) =
\lim_{n \to \infty}
n! \cdot n^z \cdot \br{\prod_{k=0}^n (z+k)}^{-1}.
\]
\end{enumerate}
\end{izrek}

\obvs

\begin{izrek}
Let $F$ be holomorphic in $\setb{z \in \C}{\Re(z) > 0}$ and assume
$F(z+1) = z \cdot F(z)$. Furthermore, assume that $F$ is bounded
on the strip $1 \leq \Re(z) < 2$ and $F(1) = 1$. Then $F = \Gamma$.
\end{izrek}
