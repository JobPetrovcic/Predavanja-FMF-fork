\section{\texorpdfstring{$C^*$}{C*}-algebras and continuous functional calculus}

\subsection{Spectrum}

\begin{definicija}
Let $A$ be a complex algebra with unity $1$. Define the set
\[
\GL(A) = \setb{a \in A}{\exists b \in A \colon ab = ba = 1}.
\]
The \emph{spectrum}\index{spectrum} of $x \in A$ is the set
\[
\sigma_A(x) =
\setb{\lambda \in \C}{x - \lambda \cdot 1 \not \in \GL(A)}.
\]
\end{definicija}

\begin{trditev}
Let $A$ be a complex algebra with unity $1$ and $x, y \in A$. Then
\[
\sigma_A(xy) \cup \set{0} = \sigma_A(yx) \cup \set{0}.
\]
\end{trditev}

\begin{proof}
By scaling, it is enough to check that
$1 \in \sigma_A(xy) \iff 1 \in \sigma_A(yx)$. Suppose that
$1 - xy \in \GL(A)$. We can check that $1 - yx$ is invertible with
\[
(1-yx)^{-1} = 1 + y (1 - xy)^{-1} x. \qedhere
\]
\end{proof}

\newpage

\subsection{Banach and \texorpdfstring{$C^*$}{C*}-algebras}

\begin{definicija}
A \emph{Banach algebra}\index{Banach algebra} is a Banach space $A$
that is also an algebra, such that
$\norm{xy} \leq \norm{x} \cdot \norm{y}$ holds for all
$x, y \in A$. If a Banach algebra has an unity, we also demand
$\norm{1} = 1$.
\end{definicija}

\begin{definicija}
An \emph{involution}\index{involution} on a Banach algebra $A$ is a
skew-linear map $* \colon A \to A$, satisfying the following for
all $x, y \in A$:

\begin{enumerate}[i)]
\item $(xy)^* = y^* x^*$,
\item $(x^*)^* = x$,
\item $\norm{x^*} = \norm{x}$.
\end{enumerate}

A Banach algebra with involution is called a
\emph{Banach *-algebra}.
\end{definicija}

\begin{definicija}
A Banach *-algebra $A$ that also satisfies
$\norm{x^* x} = \norm{x}^2$ for all $x \in A$ is called a
\emph{$C^*$-algebra}\index{C star algebra@$C^*$-algebra}.
\end{definicija}

\begin{trditev}
Let $A$ be a Banach *-algebra. Then, for all $x \in A$ we have
$(x^*)^{-1} = (x^{-1})^*$ and
$\sigma_A(x^*) = \oline{\sigma_A(x)}$.
\end{trditev}

\begin{trditev}
Let $A$ be a Banach algebra. The following statements are true:

\begin{enumerate}[i)]
\item Let $x \in A$. If $\norm{x} < 1$, then $1 - x \in \GL(A)$ and
\[
(1-x)^{-1} = \sum_{n \in \N_0} x^n.
\]
\item The set $\GL(A)$ is an open subset of $A$ and the map
$x \mapsto x^{-1}$ is continuous on $\GL(A)$.
\end{enumerate}
\end{trditev}

\begin{proof}
Let $y \in \GL(A)$. If $\norm{x-y} < \frac{1}{\norm{y^{-1}}}$, then
\[
\norm{1-xy^{-1}} = \norm{(y-x) y^{-1}} < 1,
\]
therefore, $xy^{-1}$ is invertible. It follows that $x$ is also
invertible, so $\GL(A)$ is open.

Note that
\begin{align*}
\norm{(xy^{-1})^{-1}} &=
\norm{\br{1 - (1 - xy^{-1})}^{-1}}
\\
&\leq
\sum_{n \in \N_0} \norm{1 - xy^{-1}}^n
\\
&\leq
\sum_{n \in \N_0} \norm{y^{-1}}^n \cdot \norm{x-y}^n
\\
&=
\frac{1}{1 - \norm{y^{-1}} \norm{x-y}}.
\end{align*}
It follows that
\begin{align*}
\norm{x^{-1} - y^{-1}} &=
\norm{x^{-1} (y-x) y^{-1}}
\\
&\leq
\norm{y^{-1} (xy^{-1})^{-1}} \cdot
\norm{y^{-1}} \cdot \norm{x-y}
\\
&\leq
\frac{\norm{y^{-1}}^2}{1 - \norm{y^{-1}} \cdot \norm{x-y}} \cdot
\norm{x-y}. \qedhere
\end{align*}
\end{proof}

\begin{trditev}
Let $A$ be a Banach algebra and $x \in A$. Then $\sigma_A(x)$ is a
non-empty compact set.
\end{trditev}

\begin{proof}
Introduction to functional analysis, theorem~6.1.15.
\end{proof}

\begin{izrek}[Gelfald-Mazur]
\index{Gelfald-Mazur theorem}
If $A$ is a Banach algebra that is also a division ring, then
$A = \C$.
\end{izrek}

\begin{proof}
Let $x \in A$ and $\lambda \in \sigma_A(x)$. As
$x - \lambda \cdot 1 \not \in \GL(A) = A \setminus \set{0}$, we
have $x = \lambda \cdot 1 \in \C$.
\end{proof}

\begin{definicija}
If
\[
f(x) = \sum_{j=0}^n a_j x^j
\]
is a polynomial and $a \in A$ an element of an algebra, we define
\[
f(a) = \sum_{j=0}^n a_j a^j \in A.
\]
\end{definicija}

\begin{izrek}[Spectral mapping theorem for polynomials]
\index{spectral!mapping theorem}
Let $A$ be an algebra and $f \in \C[x]$. Then
\[
f(\sigma_A(a)) = \sigma_A(f(a))
\]
holds for all $a \in A$.
\end{izrek}

\begin{proof}
Let $\lambda \in \sigma_A(a)$ and
\[
f(x) = \sum_{j=0}^n a_j x^j.
\]
We can write
\[
f - f(\lambda) =
(x - \lambda) \cdot
\sum_{j=1}^n a_j \sum_{k=0}^{j-1} x^k \lambda^{j-i-k}.
\]
It follows that
\[
f(a) - f(\lambda) =
(a - \lambda) \cdot
\sum_{j=1}^n a_j \sum_{k=0}^{j-1} a^k \lambda^{j-1-k}.
\]
As $a - \lambda$ is not invertible and commutes with the second
factor, it follows that $f(a) - f(\lambda)$ is also not invertible.

Conversely, if $\mu \not \in f(\sigma_A(a))$, we can write
\[
f - \mu = a_n \cdot \prod_{j=1}^n (x - \lambda_j).
\]
As $f(\lambda) - \mu \ne 0$ for all $\lambda \in \sigma_A(a)$, we
have $\lambda_i \not \in \sigma_A(a)$ for all $i$. Therefore, it
follows that $f(a) - \mu \in \GL(A)$.
\end{proof}

\begin{definicija}
Let $A$ be a Banach algebra and $x \in A$. The
\emph{spectral radius}\index{spectral!radius} of $x$ is
\[
r(x) = \sup_{\lambda \in \sigma_A(x)} \abs{\lambda}.
\]
\end{definicija}

\begin{izrek}[Spectral radius formula]
\index{spectral!radius!formula}
Let $A$ be a Banach algebra and $x \in A$. Then, the limit
\[
\lim_{n \to \infty} \sqrt[n]{\norm{x^n}}
\]
exists and
\[
r(x) = \lim_{n \to \infty} \sqrt[n]{\norm{x^n}}.
\]
\end{izrek}

\begin{proof}
Introduction to functional analysis, theorem~6.1.20.
\end{proof}

\begin{definicija}
Let $A$ be a Banach *-algebra and $x \in A$.

\begin{enumerate}[i)]
\item The element $x$ is \emph{normal}\index{normal} if
$x x^* = x^* x$.
\item The element $x$ is \emph{selfadjoint}\index{selfadjoint} if
$x = x^*$.
\item The element $x$ is
\emph{skew selfadjoint}\index{skew selfadjoint} if $x = -x^*$.
\end{enumerate}
\end{definicija}

\begin{posledica}
Let $A$ be a Banach *-algebra and $x \in A$ a normal element. Then
\[
r(x^* x) \leq r(x)^2.
\]
If $A$ is a $C^*$-algebra, then $r(x^* x) = r(x)^2$.
\end{posledica}

\begin{proof}
Note that
\[
r(x^* x) =
\lim_{n \to \infty} \sqrt[n]{\norm{(x^* x)^n}} \leq
\lim_{n \to \infty} \sqrt[n]{\norm{x^n}}^2 = r(x)^2.
\]
If $A$ is a $C^*$-algebra, we have equality.
\end{proof}

\begin{trditev}
Let $A$ be a $C^*$-algebra and $x \in A$ a normal element. Then
\[
r(x) = \norm{x}.
\]
\end{trditev}

\begin{proof}
The statement holds for selfadjoint elements by Introduction to
functional analysis, corollary~6.1.20.1. We can therefore write
\[
\norm{x}^2 = \norm{x^* x} = r(x^* x) = r(x)^2. \qedhere
\]
\end{proof}

\datum{2023-10-25}

\begin{posledica}
Let $A$ and $B$ be $C^*$-algebras and $\Phi \colon A \to B$ a
*-homomorphism.\footnote{$\Phi(x^*) = \Phi(x)^*$ for all
$x \in A$.} Then $\Phi$ is a contraction. Furthermore, if $\Phi$ is
a *-isomorphism, it is isometric.
\end{posledica}

\begin{proof}
Note that $\Phi(\GL(A)) \subseteq \GL(B)$, therefore
$\sigma_B(\Phi(x)) \subseteq \sigma_A(x)$. It follows that
$r(\Phi(x)) \leq r(x)$. Now observe that
\[
\norm{\Phi(x)}^2 =
\norm{\Phi(x) \cdot \Phi(x)^*} =
\norm{\Phi(xx^*)} =
r(\Phi(xx^*)) \leq
r(xx^*) =
\norm{x}^2. \qedhere
\]
\end{proof}

\begin{posledica}
If $A$ is a *-algebra, there exists at most one norm on $A$ such
that $A$ is a $C^*$-algebra.
\end{posledica}

\obvs

\begin{lema}
Let $A$ be a $C^*$-algebra and $x \in A$ a selfadjoint element.
Then $\sigma_A(x) \subseteq \R$.
\end{lema}

\begin{proof}
Let $\lambda = \alpha + i \beta \in \sigma_A(x)$ for some
$\alpha, \beta \in \R$. Define $y = x - \alpha + it$ for some
$t \in \R$. Note that $i \cdot (\beta + t) \in \sigma_A(y)$ and
that $y$ is normal. It follows  that
\[
\abs{i \cdot (\beta + t)}^2 \leq
r(y)^2 =
\norm{y}^2 =
\norm{y y^*} =
\norm{(x - \alpha)^2 + t^2} \leq
\norm{x - \alpha}^2 + t^2.
\]
Rearranging the above inequality, we get
\[
\beta^2 + 2 \beta t \leq \norm{x - \alpha}^2.
\]
As $t \in \R$ was arbitrary, it follows that $\beta = 0$.
\end{proof}

\begin{lema}
Let $A$ be a Banach algebra and $x \in A \setminus \GL(A)$. If the
sequence $(x_n)_n$ of elements of $\GL(A)$ converges to $x$, then
\[
\lim_{n \to \infty} \norm{x_n^{-1}} = \infty.
\]
\end{lema}

\begin{proof}
If the sequence were bounded, we'd have
\[
\lim_{n \to \infty} \norm{1 - x x_n^{-1}} \leq
\lim_{n \to \infty} \norm{x_n - x} \cdot \norm{x_n^{-1}} = 0.
\]
In particular, $\norm{1 - x x_n^{-1}} < 1$ for some $n \in \N$.
It follows that $x x_n^{-1}$ is invertible, therefore $x$ is also
invertible.
\end{proof}

\begin{trditev}
Let $B$ be a $C^*$-algebra and $A \subseteq B$ a unital
$C^*$-subalgebra. Then for all $x \in A$ we have
$\sigma_A(x) = \sigma_B(x)$.
\end{trditev}

\begin{proof}
Note that $\GL(A) \subseteq \GL(B)$. Let $x \in A \setminus \GL(A)$
be a selfadjoint element. Note that $it \not \in \sigma_A(x)$ for
$t \in \R^*$. As $x$ is not invertible in $A$, we have
\[
\lim_{n \to \infty} \norm{(x - it)^{-1}} = \infty
\]
by the previous lemma. Since the inversion map is continuous, it
follows that $x \not \in \GL(B)$.

Let now $x \in A \cap \GL(B)$ be an arbitrary element. We know that
$x^*$ is also invertible, therefore $x^* x \in \GL(B) \cap A$. From
the first part of the proof we now know $x^* x \in \GL(A)$, so
we can write
\[
x^{-1} = (x^* x)^{-1} \cdot x^*. \qedhere
\]
\end{proof}

\newpage

\subsection{Gelfald transform}

\begin{definicija}
Let $A$ be an abelian Banach algebra. The
\emph{spectrum}\index{spectrum} of $A$ is the set
\[
\sigma(A) =
\setb{\varphi \colon A \to \C}
{\text{$\varphi \ne 0$ is a continuous homomorphism}}
\]
with the weak-* topology. We call elements of $\sigma(A)$
\emph{characters}\index{character}.
\end{definicija}

\begin{opomba}
For all characters $\varphi \in \sigma(A)$ we have
$\ker \varphi \cap \GL(A) = \emptyset$.
\end{opomba}

\begin{trditev}
For all $\varphi \in \sigma(A)$ and $x \in A$ we have
$\varphi(x) \in \sigma_A(x)$.
\end{trditev}

\begin{proof}
We have $\varphi(x - \varphi(x)) = 0$.
\end{proof}

\begin{posledica}
We have $\sigma(A) \subseteq (A^*)_1$. In particular, $\sigma(A)$
is a compact Hausdorff space.
\end{posledica}

\begin{proof}
For all characters $\varphi \in \sigma(A)$ we have
$\abs{\varphi(x)} \leq r(x) \leq \norm{x}$, therefore
$\norm{\varphi} \leq 1$.\footnote{In fact, as $\varphi(1) = 1$, we
have $\norm{\varphi} = 1$.}
\end{proof}

\begin{trditev}
Let $A$ be an abelian Banach algebra. The map
$\varphi \mapsto \ker \varphi$ is a bijection between $\sigma(A)$
and maximal ideals $I \edn A$.
\end{trditev}

\begin{proof}
Let $\varphi \in \sigma(A)$ be an arbitrary character. Note that
$\ker \varphi \edn A$. Suppose that
$\ker \varphi \subset I \edn A$ and let
$x \in I \setminus \ker \varphi$. Let
$y = 1 - \frac{x}{\varphi(x)} \in \ker \varphi$. It follows that
\[
1 = y + \frac{1}{\varphi(x)} \cdot x \in I,
\]
therefore $\ker \varphi$ is maximal.

Let $I \edn A$ be a maximal ideal. Note that
$I \cap \GL(A) = \emptyset$ as $I \ne A$. It follows that
$\norm{1 - y} \geq 1$ for all $y \in I$. The set $\oline{I}$ is
again an ideal, but as $1 \not \in \oline{I}$, it follows that
$I = \oline{I}$. Note that $\kvoc{A}{I}$ is an abelian Banach
algebra. Since $I$ was maximal, it is also a division ring. By
Gelfald-Mazur theorem, we have $\kvoc{A}{I} \cong \C$. The
projection $\pi \colon A \to \kvoc{A}{I}$ is therefore a character
with $\ker \pi = I$.
\end{proof}

\begin{posledica}
\label{Cstar:ps:ann}
Let $A$ be an abelian Banach algebra and
$x \in A \setminus \GL(A)$. Then there exists a
$\varphi \in \sigma(A)$ such that $\varphi(x) = 0$.
\end{posledica}

\begin{proof}
By Zorn's lemma, there exists a maximal ideal containing $\skl{x}$.
\end{proof}

\begin{izrek}[Stone-Čech]
\index{Stone-Čech theorem}
Let $X$ be a topological space. For each $x \in X$ define
$\beta_x \colon \mathcal{C}_b(X) \to \C$ as the evaluation
homomorphism. Then the map
$\beta \colon X \to \sigma(\mathcal{C}_b(X))$ given by
$\beta(x) = \beta_x$ is continuous with dense image and the
following universal property: for every continuous map
$\pi \colon X \to K$, where $K$ is a compact Hausdorff topological
space, there exists a unique continuous map
$\beta_\pi \colon \sigma(\mathcal{C}_b(X)) \to K$ such that
$\pi = \beta_\pi \circ \beta$. In particular, if $X$ is a compact
Hausdorff space, then $\beta$ is a homeomorphism.
\[
\begin{tikzcd}[column sep=large, row sep=large]
X \arrow[r, "\beta"] \arrow[dr, "\pi"] &
\sigma(\mathcal{C}_b(X)) \arrow[d, dashed, "\beta_\pi"] \\
&
K
\end{tikzcd}
\]
\end{izrek}

\begin{proof}
Let $(x_i)_I$ be a net with limit $x$. For all
$f \in \mathcal{C}_b(X)$
\[
\lim_{i \in I} \beta_{x_i}(f) =
\lim_{i \in I} f(x_i) =
\beta_x(f),
\]
therefore $\beta$ is continuous. Suppose that
$\varphi \in \sigma(\mathcal{C}_b(X)) \setminus \oline{\beta(X)}$
and denote $I = \ker \varphi$. For all $\psi \in \oline{\beta(X)}$
there exists a $f_\psi \in I$ such that
$f_\psi \not \in \ker \psi$. Therefore there exist $c_\psi > 0$ and
an open neighbourhood $U_\psi$ of $\psi$ such that
$\abs{\widetilde{\psi} \br{f_\psi}} > c_\psi$ for all
$\widetilde{\psi} \in U_\psi$. This is of course an open cover of
$\oline{\beta(X)}$, therefore there exists a finite subcover.
Equivalently, there exist elements $f_1, f_2, \dots f_n \in I$ and
$c > 0$ such that
\[
\sum_{i=1}^n \psi \br{\abs{f_i}^2} > c
\]
for all $\psi \in \oline{\beta(X)}$. In particular, we have
\[
\sum_{i=1}^n \abs{f_i(x)}^2 =
\sum_{i=1}^n \beta_x \br{\abs{f_i}^2} >
c
\]
for all $x \in X$. It follows that the function
\[
\sum_{i=1}^n \abs{f_i}^2 \in I
\]
is invertible, therefore we have $I = \mathcal{C}_b(X)$. This is
of course impossible.

If $X$ is compact and Hausdorff, then $\beta$ is surjective, as its
image is a closed dense set. As $X$ is compact and Hausdorff, it is
normal and therefore $\mathcal{C}_b(X)$ separates points by Tietze.
It follows that $\beta$ is injective as well. As $\beta$ is a map
between compact Hausdorff spaces, it is closed, therefore a
homeomorphism.

Let $\pi \colon X \to K$ be a continuous map, where $K$ is a
compact Hausdorff space. Note that there exists a continuous map
$\pi^* \colon \mathcal{C}(K) \to \mathcal{C}_b(X)$ with
$\pi^*(f) = f \circ \pi$. Similarly, there exists a continuous map
$\widetilde{\pi} \colon
\sigma(\mathcal{C}_b(X)) \to \sigma(\mathcal{C}(K))$ with
$\widetilde{\pi}(\varphi) = \varphi \circ \pi^*$.

As $K$ is a compact Hausdorff space, the map
$\beta^K \colon K \to \sigma(\mathcal{C}(K))$ is a homeomorphism by
the previous part of the proof. Now define
$\beta_\pi \colon \sigma(\mathcal{C}_b(X)) \to K$ by
$\beta_\pi = \br{\beta^K}^{-1} \circ \widetilde{\pi}$. Indeed,
\[
\widetilde{\pi}(\beta_x)(g) =
\beta_x(\pi^*(g)) =
g(\pi(x)) =
\beta_{\pi(x)}^K(g),
\]
therefore $\pi(x) = \beta_\pi(\beta(x))$.
\end{proof}

\begin{definicija}
Let $A$ be an abelian Banach algebra. The
\emph{Gelfald transform}\index{Gelfald!transform} of $A$ is the map
$\Gamma \colon A \to \mathcal{C}(\sigma(A))$ with
$\Gamma(x) = (\varphi \mapsto \varphi(x))$.
\end{definicija}

\begin{izrek}
The Gelfald transform of an abelian Banach algebra $A$ is a
homomorphism and a contraction. For $x \in A$, we have
\[
\Gamma(x) \in \GL(\mathcal{C}(\sigma(A))) \iff x \in \GL(A).
\]
\end{izrek}

\begin{proof}
The Gelfald transform is obviously a homomorphism. Note that
\[
\norm{\Gamma(x)} =
\sup_{\varphi \in \sigma(A)} \norm{\Gamma(x)(\varphi)} =
\sup_{\varphi \in \sigma(A)} \norm{\varphi(x)} \leq
\norm{x}.
\]
If $x$ is invertible, then so is $\Gamma(x)$. Now suppose that
$x \in A$ is not invertible. By corollary~\ref{Cstar:ps:ann} there
exists a character $\varphi \in \sigma(A)$ such that
$\varphi(x) = 0$. As $\Gamma(x)(\varphi) = 0$, the map $\Gamma(x)$
is not invertible either.
\end{proof}

\begin{posledica}
Let $A$ be an abelian Banach algebra. Then
$\sigma(\Gamma(x)) = \sigma(x)$ and $\norm{\Gamma(x)} = r(x)$.
\end{posledica}

\begin{izrek}[Gelfald]
\index{Gelfald!theorem}
Let $A$ be an abelian $C^*$-algebra. Then $\Gamma$ is a
*-isomorphism.
\end{izrek}

\begin{proof}
Let $x \in A$ be a selfadjoint element. We then have
$\sigma(\Gamma(x)) = \sigma(x) \subseteq \R$, therefore
$\oline{\Gamma(x)} = \Gamma(x)$. For an arbitrary $x \in A$, write
$x = a + bi$ for selfadjoint $a, b \in A$. It follows that
\[
\Gamma(x^*) = \Gamma(a - ib) = \oline{\Gamma(x)},
\]
therefore $\Gamma$ is a *-homomorphism.

Since $A$ is abelian, every element is normal. We therefore have
\[
\norm{x} = r(x) = r(\Gamma(x)) = \norm{\Gamma(x)},
\]
so $\Gamma$ is injective and therefore injective. Note that
$\Gamma(A)$ is closed under $*$, closed as a subalgebra in
$\mathcal{C}(\sigma(A))$ and separates points. By
Stone-Weierstrass, $\Gamma(A) = \mathcal{C}(\sigma(A))$.
\end{proof}

\begin{opomba}
If $A$ is a $C^*$-algebra and $x \in A$ a normal element, then the
$C^*$-subalgebra generated by\footnote{Also denoted by $C^*(x)$.}
$x$ is abelian.
\end{opomba}

\begin{posledica}
Let $A$ be an abelian $C^*$-algebra, generated by $x$. Then
$\sigma(A) \approx \sigma(x)$.
\end{posledica}

\begin{proof}
Define a map $\Phi \colon \sigma(A) \to \sigma(x)$ with
$\Phi(\varphi) = \varphi(x) = \Gamma(x)(\varphi)$. Observe that
$\Phi$ is well defined as $\varphi(x) \in \sigma(x)$.
\end{proof}
