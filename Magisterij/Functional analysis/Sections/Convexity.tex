\section{Convexity}

\subsection{Locally convex spaces}

\datum{2023-10-4}

\begin{definicija}
\emph{A topological vector space}\footnote{Also linear topological
space.} $V$ is an $\F$-vector space that is also a topological
space, such both addition and scalar multiplication are continuous.
\end{definicija}

\begin{definicija}
Let $V$ be an $\F$-vectors pace. A map $p \colon V \to \R$ is a
\emph{seminorm}\index{seminorm} if the following holds:

\begin{enumerate}[i)]
\item $\forall x \in V \colon p(x) \geq 0$,
\item $\forall \lambda \in \F, x \in V \colon
p(\lambda x) = \abs{\lambda} p(x)$,
\item $\forall x, y \in V \colon p(x+y) \leq p(x) + p(y)$.
\end{enumerate}
\end{definicija}

\begin{definicija}
Let $V$ be an $\F$-vector space and $\mathcal{P}$ a family of
seminorms on $V$. We define a topology $\mathcal{T}$ on $V$ with
the sets
\[
U(x_0, p, \varepsilon) = \setb{x \in V}{p(x - x_0) < \varepsilon}
\]
as a subbasis.
\end{definicija}

\begin{definicija}
A topological vector space $X$ is a
\emph{locally convex space}\index{locally convex space} if its
topology is generated by a family of seminorms $\mathcal{P}$
satisfying
\[
\bigcap_{p \in \mathcal{P}} \setb{x \in X}{p(x) = 0} = \set{0}.
\]
\end{definicija}

\begin{trditev}
A locally convex space $X$ is Hausdorff.
\end{trditev}

\begin{proof}
Let $x, y \in X$ be distinct points. Let $p \in \mathcal{P}$ be a
seminorm such that $p(x-y) \ne 0$. Then the sets
\[
U = \setb{z \in X}{p(z-x) < \frac{\varepsilon}{2}}
\quad \text{and} \quad
V = \setb{z \in X}{p(z-y) < \frac{\varepsilon}{2}}
\]
split the points $x$ and $y$.
\end{proof}

\begin{opomba}
The converse is also true.
\end{opomba}

\begin{definicija}
A partially ordered set $I$ is
\emph{upward directed}\index{upward directed set} if for all
$i', i'' \in I$ there exists some $i \in I$ such that $i \geq i'$
and $i \geq i''$.
\end{definicija}

\begin{definicija}
A \emph{net}\index{net} is a pair $((I, \leq), x)$, where
$(I, \leq)$ is an upward directed set and $x \colon I \to X$ is a
function. We usually write $(x_i)_{i \in I}$.
\end{definicija}

\begin{opomba}
Let $(X, \mathcal{T})$ be a topological space and $x_0 \in X$.
Partially order the set
\[
\mathcal{U} =
\setb{U \subseteq X}{x_0 \in U \land \text{$U$ is open}}
\]
with reverse inclusion. Then any choice function defines a net
$(x_U)_{U \in \mathcal{U}}$.
\end{opomba}

\begin{definicija}
Let $X$ be a topological space. A net $(x_i)_{i \in I}$
\emph{converges}\index{converging net} so $x \in X$ if for all open
sets $U \subseteq X$ with $x \in U$ there exists some index
$i_0 \in I$ such that for all $i \geq i_0$ we have $x_i \in U$. We
write
\[
\lim_{i \in I} x_i = x.
\]
\end{definicija}

\begin{definicija}
A point $x \in X$ is a \emph{cluster point}\index{cluster point} of
a net $(x_i)_{i \in I}$ if for all open sets $U \subseteq X$ with
$x \in U$ and index $i_0 \in I$ there exists some index
$i \geq i_0$ such that $x_i \in U$.
\end{definicija}

\begin{trditev}
Let $X$ be a topological space and $A \subseteq X$. Then
$x \in \oline{A}$ if and only if there exists a net
$(a_i)_{i \in I}$ in $A$ such that
\[
\lim_{i \in I} a_i = x.
\]
\end{trditev}

\begin{proof}
Suppose a net $(a_i)_{i \in I}$ converges to $x$. For any
neighbourhood $U$ of $x$ and some $i_0 \in I$ we have
$a_{i_0} \in U$. Therefore, $U \cap A \ne \emptyset$.

Assume now that $x \in \oline{A}$. Again, define
\[
\mathcal{U} =
\setb{U \subseteq X}{x_0 \in U \land \text{$U$ is open}}.
\]
There is a choice function $a$ such that $a_U \in A$ for all $U$.
The net $(a_U)_{U \in \mathcal{U}}$ then converges to $x$.
\end{proof}

\begin{trditev}\label{prop:conv:cont_1}
Let $X$ and $Y$ be topological spaces and $f \colon X \to Y$ a map.
Then, $f$ is continuous in $x_0 \in X$ if and only if
\[
\lim_{i \in I} f(x_i) = f(x_0)
\]
for all nets $(x_i)_{i \in I}$ that converge to $x_0$.
\end{trditev}

\begin{proof}
Suppose that $f$ is continuous at $x_0$. Take an open neighbourhood
$U$ of $f(x_0)$. Then there must exist some $i_0 \in I$ such that
for all $i \geq i_0$ we have $x_i \in f^{-1}(U)$, therefore
$f(x_i) \in U$.

Now suppose $f$ is discontinuous at $x_0$. Let
\[
\mathcal{U} =
\setb{U \subseteq X}{x_0 \in U \land \text{$U$ is open}}
\]
and $V \subseteq Y$ be an open set such that $f(x_0) \in V$ and
$x_0$ is not an interior point of $f^{-1}(V)$. Now using the
discontinuity of $f$, for all $U \in \mathcal{U}$ choose
$x_U \in U$ such that $f(x_U) \not \in V$. Trivially the net
$(x_V)_{V \in \mathcal{V}}$ converges to $x_0$, but
\[
\lim_{V \in \mathcal{V}} f(x_V) \ne f(x_0). \qedhere
\]
\end{proof}

\begin{trditev}
The following statements are true:

\begin{enumerate}[i)]
\item A net $(x_i)_{i \in I}$ in a locally convex space converges
to $x_0$ if and only if the net $(p(x_i - x_0))_{i \in I}$
converges to $0$ for all $p \in \mathcal{P}$.
\item The topology in a locally convex space $X$ is the coarsest
topology in which all the maps $x \mapsto p(x - x_0)$ are
continuous for all $x_0 \in X$ and $p \in \mathcal{P}$.
\end{enumerate}
\end{trditev}

\begin{proof}
\phantom{a}
\begin{enumerate}[i)]
\item If $(x_i)_{i \in I}$ converges to $x_0$, just apply the
proposition~\ref{prop:conv:cont_1}. Suppose that all the nets
$(p(x_i - x_0))_{i \in I}$ converge to $0$. Choose an open set from
the local basis of $x_0$. It is given by
\[
U =
\setb{x \in X}{\forall k \leq n \colon p_k(x - x_0) < \varepsilon}.
\]
But as all nets $(p_k(x_i - x_0))_{i \in I}$ converge to $0$, there
is some index $i_k \in I$ such that for all $i \geq i_k$ we have
$p_k(x_i - x_0) < \varepsilon$. Now just take $i_0$ to be an upper
bound of $i_k$. For all $i \geq i_0$ we then have $x_i \in U$.
\item Obvious. \qedhere
\end{enumerate}
\end{proof}

\begin{definicija}
For all $f \in X^*$ define a seminorm $p_f \colon X \to \R$ as
$p_f(x) = \abs{f(x)}$. The family
$\mathcal{P} = \setb{p_f}{f \in X^*}$ induces the
\emph{weak topology}\index{weak topology} on $X$. We denote the
weak topology with $\sigma(X, X^*)$.
\end{definicija}

\begin{opomba}
The space $X$ with the topology $\sigma(X, X^*)$ is a locally
convex space by the Hahn-Banach theorem.\footnote{Introduction to
functional analysis, corollary 2.2.5.2.}
\end{opomba}

\begin{definicija}
Let $X$ be a normed space. For all $x \in X$ we define a seminorm
$p_x \colon X^* \to \R$ as $p_x(f) = \abs{f(x)}$. The family
$\mathcal{P} = \setb{p_x}{x \in X}$ induces the
\emph{weak-* topology}\index{weak-* topology} on $X^*$. We denote
the weak-* topology with $\sigma(X^*, X)$.
\end{definicija}

\begin{opomba}
The weak topology on $X^*$ is finer than the weak-* topology, as
$X$ can be isometrically mapped into $X^{**}$ with the map
$x \mapsto (f \mapsto f(x))$.
\end{opomba}

\newpage

\subsection{Banach-Alaoglu theorem}

\begin{izrek}[Banach-Alaoglu]
\index{Banach-Alaoglu theorem}
Let $X$ be a normed space. Then the closed unit ball in $X^*$
\[
(X^*)_1 = \setb{f \in X^*}{\norm{f} \leq 1}
\]
is compact in the weak-* topology on $X^*$.
\end{izrek}

\begin{proof}
Assign a disk to all $x \in X$ as
$D_x = \setb{z \in \F}{\abs{z} \leq \norm{x}}$ with the euclidean
topology. Define
\[
P = \prod_{x \in X} D_x
\]
with the product topology. The space $P$ is then compact by
Tychonoff's theorem. Now define the map $\Phi \colon (X^*)_1 \to P$
with $\Phi(f) = (f(x))_{x \in X}$. This map is injective.

Let $(f_i)_{i \in I}$ be a net in $(X^*)_1$ that weak-* converges
to $f \in X^*$. Equivalently, we have
\[
\lim_{i \in I} f_i(x) = f(x)
\]
for all $x \in X$. By the definition of the product topology we
have
\[
\lim_{i \in I} \Phi(f_i) = \Phi(f).
\]
Therefore, $\Phi$ is continuous. Analogously,
$\Phi^{-1} \colon \im \Phi \to (X^*)_1$ is continuous.

Suppose that $(\Phi(f_i))_{i \in I}$ converges to some $p \in P$.
By the definition of the product topology this means that
$f_i(x)$ converges to $p_x$ for all $x \in X$. Define a map
$f \colon X \to \F$ given by $f(x) = p_x$. Then, $f$ is linear and
bounded with $\norm{f} \leq 1$. Thus $p = \Phi(f) \in \im(\Phi)$,
therefore, $\Phi((X^*)_1)$ is closed. As $(X^*)_1$ is homeomorphic
to its image which is compact, it is also compact.
\end{proof}

\begin{posledica}
Every Banach space $X$ is isometrically isomorphic to a closed
subspace $\mathcal{C}(K)$ for some compact Hausdorff space $K$.
\end{posledica}

\begin{proof}
Choose $K = (X^*)_1$ with the weak-* topology. By Banach-Alaoglu,
$K$ is compact and Hausdorff. Now define the map
$\Delta \colon X \to K$ with $\Delta(x) = (f \mapsto f(x))$.
Now observe that
\[
\norm{\Delta(x)}_\infty =
\sup_{g \in K} \abs{\Delta(x)(g)} =
\sup_{g \in K} \abs{g(x)} =
\norm{x}
\]
by Hahn-Banach.\footnote{Introduction to functional analysis,
corollary 2.2.5.1.}
\end{proof}

\newpage

\subsection{Minkowski gauge}

\datum{2023-10-11}

\begin{definicija}
Let $X$ be a $\F$-vector space. A set $A \subseteq$ is

\begin{enumerate}[i)]
\item \emph{balanced}\index{balanced set}, if for all $x \in A$
and $\alpha \in \F$ with $\abs{\alpha} \leq 1$ we have
$\alpha x \in A$,
\item \emph{absorbing}\index{absorbing set}, if for all $x \in X$
there exists some $\varepsilon > 0$ such that for all
$t \in (0, \varepsilon)$ we have $tx \in A$,
\item \emph{absorbing in $a \in A$} if $A - a$ is absorbing.
\end{enumerate}
\end{definicija}

\begin{izrek}
Let $X$ be a $\F$-vector space and $V \subseteq X$ a convex,
balanced and absorbing in each of its points. Then there exists a
unique seminorm $p$ such that
\[
V = \setb{x \in X}{p(x) < 1}.
\]
\end{izrek}

\begin{proof}
As $A$ is convex, we can define the
\emph{Minkowski gauge}\index{Minkowski gauge}
\[
p_V(x) = \inf \setb{t \geq 0}{x \in tV}.
\]
It is of course well defined, as $A$ is absorbing. We can check
that
\begin{align*}
p_V(\alpha x) &=
\inf \setb{t \geq 0}{x \in \frac{t}{\alpha} V}
\\
&=
\inf \setb{t \geq 0}{x \in \frac{t}{\abs{\alpha}} V}
\\
&=
\abs{\alpha} \cdot \inf
\setb{\frac{t}{\abs{\alpha}} \geq 0}
{x \in \frac{t}{\abs{\alpha}} V}
\\
&=
\abs{\alpha} p_V(x)
\end{align*}
as $A$ is balanced. Therefore, $p_V$ is homogeneous. As $p_V$ is
sublinear,\footnote{Introduction to functional analysis,
proposition~2.3.3.} it is a seminorm. It follows
that\footnote{Introduction to functional analysis, remark~2.3.4.1.}
\[
V = \setb{x \in X}{p_V(x) < 1}.
\]
Suppose that
\[
V = \setb{x \in X}{q(x) < 1}
\]
for some seminorm $q \ne p_V$. But then we have $p_V(x) \ne q(x)$
for some $x \in X$, therefore there exists some $t \in \R$ such
that $p_V(tx) > 1 > q(tx)$ or $q(tx) > 1 > p_V(tx)$.
\end{proof}

\newpage

\subsection{Applications of the Hahn-Banach theorem}

\begin{izrek}[Hahn-Banach]
\index{Hahn-Banach theorem}
Suppose $X$ is a locally convex space and $A, B \subseteq X$ are
disjoint convex sets. If $B$ is compact, there exists a functional
$f \in X^*$ that separates $A$ from $B$ -- there exist
$\alpha, \beta \in \R$ such that for all $a \in A$ and $b \in B$ we
have
\[
\Re f(a) \leq \alpha < \beta \leq \Re f(b).
\]
\end{izrek}

\begin{izrek}
Suppose $X$ is a locally convex space and $A \subseteq X$ is a
convex space. Then the closure of $A$ is the same as the closure
in the weak topology.
\end{izrek}

\begin{proof}
The set $\oline{A}$ is of course a subset of the closure of $A$ in
the weak topology. Now choose a point $x \not \in \oline{A}$. There
exists a functional $f \in X^*$ and numbers $\alpha, \beta \in \R$
such that
\[
\Re f(a) \leq \alpha < \beta \leq \Re f(x)
\]
for all $a \in \oline{A}$. But then
\[
\oline{A} \subseteq
\setb{y \in X}{\Re f(y) < \alpha} =
\br{\Re f}^{-1} \br{(-\infty, \alpha]} =
C,
\]
where $C$ is closed in the weak topology. It follows that the
closure of $A$ in the weak topology is a subset of $C$. As
$x \not \in C$, we get the desired equality.
\end{proof}

\begin{posledica}
A convex set is a locally convex space if and only if it is weakly
closed.
\end{posledica}

\begin{trditev}
Let $X$ be a topological vector space and $f \colon X \to \F$ a
linear functional. The following statements are equivalent:

\begin{enumerate}[i)]
\item The functional $f$ is continuous.
\item The functional $f$ is continuous in $0$.
\item The functional $f$ is continuous in some point $x_0 \in X$.
\item The set $\ker f$ is closed.
\item The function $x \mapsto \abs{f(x)}$ is a continuous seminorm.
\end{enumerate}

If $X$ is a locally compact space and $\mathcal{P}$ is the family
of seminorms defining the topology on $X$, the above conditions are
also equivalent to
\[
\abs{f(x)} \leq \sum_{k=1}^r \alpha_k p_k(x)
\]
for some $\alpha_k \in \R^+$ and $p_k \in \mathcal{P}$.
\end{trditev}

\begin{proof}
The proof of the equivalence of the first 5 statements is the same
as for normed spaces. Suppose now that
\[
\abs{f(x)} \leq \sum_{k=1}^r \alpha_k p_k(x).
\]
Let $(x_i)_{i \in I}$ be a net in $X$ that converges to $0$. Then
\[
0 \leq \abs{f(x_i)} \leq \sum_{k=1}^r \alpha_k p_k(x_i),
\]
which converges to $0$. It follows that $f$ is continuous at $0$.

Now suppose that $f$ is continuous at $0$. The set
\[
f^{-1} \br{\spr(0, 1)} = \setb{x \in X}{\abs{f(x)} < 1}
\]
contains an open neighbourhood $B$ of the point $0$. We can write
\[
B = \bigcap_{j=1}^r U(0, p_j, \varepsilon).
\]
Take $x \in X$. For $\delta > 0$ be such that
\[
p_j \br{x \cdot \frac{\varepsilon}{\delta + \sum p_j(x)}} =
\frac{\varepsilon}{\delta + \sum p_j(x)} \cdot p_j(x) <
\varepsilon,
\]
therefore,
\[
\abs{f \br{x \cdot \frac{\varepsilon}{\delta + \sum p_j(x)}}} < 1,
\]
which can be rearranged to
\[
\abs{f(x)} <
\frac{1}{\varepsilon} \cdot \sum_{j=1}^r p_j(x) +
\frac{\delta}{\varepsilon}.
\]
Taking a limit, we get the desired inequality.
\end{proof}

\begin{izrek}[Riesz-Markov]
\index{Riesz-Markov theorem}
Let $X$ be a compact Hausdorff space and
$\Phi \in \mathcal{C}(X)^*$. Then there exists a unique regular
Borel measure $\mu$ such that
\[
\Phi(f) = \lint_X f\,d\mu
\]
for all $f \in \mathcal{C}(X)$. Furthermore, we have
$\norm{\Phi} = \norm{\mu} = \abs{\mu}(X)$.
\end{izrek}

\begin{trditev}
Let $X$ be a completely regular space. Endow the space
$\mathcal{C}(X)$ with the topology induced by the seminorms
$\setb{p_K}{\text{$K \subseteq X$ is compact}}$. If
$L \in \mathcal{C}(X)^*$, then there exists a compact set
$K \subseteq X$ and a regular Borel measure on $K$ such that
\[
L(f) = \lint_K f\,d\mu
\]
for all $f \in \mathcal{C}(X)$. Conversely, every such $(K, \mu)$
defines a functional $L \in \mathcal{C}(X)^*$.
\end{trditev}

\begin{proof}
Suppose that
\[
L(f) = \lint_K f\,d\mu
\]
for some compact set $K$ and measure $\mu$. Then we have
\[
\abs{L(f)} =
\abs{\lint_K f\,d\mu} \leq
\norm{\mu} \cdot \sup_K \abs{f} = \norm{\mu} \cdot p_K(f),
\]
so $L$ is continuous.

Let now $L \in \mathcal{C}(X)^*$. We can therefore write
\[
\abs{L(f)} \leq \sum_{k=1}^r \alpha_k p_{K_j}(f)
\]
for some compact sets $K_j$. We can simplify the above to
\[
\abs{L(f)} \leq \alpha \cdot p_K(f),
\]
where
\[
K = \bigcup_{j=1}^r K_j.
\]
Note that if we have $f \in \mathcal{C}(X)$ and
$\eval{f}{K}{} = 0$, it follows that $L(f) = 0$. Now define
$F \colon \mathcal{C} \to \F$ as follows; for any
$g \in \mathcal{C}$ choose an extension
$\widetilde{g} \in \mathcal{C}(X)$ of $g$ and set
\[
F(g) = L \br{\widetilde{g}}.
\]
This map is well defined by the above observation. We can check
that $F$ is indeed linear. Note that
\[
\abs{F(g)} =
\abs{L \br{\widetilde{g}}} \leq
\alpha \cdot p_K \br{\widetilde{g}} =
\alpha \cdot \norm{g}_{\infty, K},
\]
therefore, $F$ is continuous. By the Riesz-Markov theorem there
exists a regular Borel measure $\mu$ on $K$ such that
\[
F(g) = \lint_K g\,d\mu.
\]
If $f \in \mathcal{C}(X)$, we have
$g = \eval{f}{K}{} \in \mathcal{C}(K)$, so
\[
L(f) = F(g) = \lint_K g\,d\mu. \qedhere
\]
\end{proof}

\newpage

\subsection{Krein-Milman theorem}

\begin{definicija}
Let $X$ be a vector space and $C \subseteq X$ a convex subset.

\begin{enumerate}[i)]
\item A non-empty convex subset $F \subseteq C$ is a
\emph{face}\index{face} if for all $t \in (0, 1)$ and $x, y \in C$
satisfying $tx + (1-t)x \in F$, we also have $x, y \in F$.
\item A point $x \in C$ is an
\emph{extreme point}\index{extreme point} if $\set{x} \subseteq C$
is a face. We denote the set of extreme points of $C$ by $\ext C$.
\end{enumerate}
\end{definicija}

\begin{definicija}
For a vector space $X$ and $A \subseteq X$ define the
\emph{convex hull}\index{convex hull} of $A$ as
\[
\co A =
\setb{\sum_{i=1}^n \alpha_i x_i}
{n \in \N \land \alpha_j \in \R_{\geq 0} \land
\sum_{i=1}^n \alpha_i = 1 \land x_i \in A}.
\]
If $X$ is a topological vector space, define the
\emph{closed convex hull}\index{closed convex hull} as
\[
\oline{\co}\,A = \oline{\co A}.
\]
\end{definicija}

\begin{trditev}
The set $\co A$ is the smallest convex set that contains $A$. The
set $\oline{\co}\,A$ is the smallest closed set that contains $A$.
\end{trditev}

\begin{proof}
The only nontrivial part of the proof is convexity of the set
$\oline{\co}\,A$. Let $(x_i)_{i \in I}$ and $(y_i)_{i \in I}$ be
two nets that converge to $x$ and $y$, where
$x, y \in \oline{\co}\,A$. For any $t \in (0, 1)$ we have
\[
tx + (1-t)y =
\lim_{i \in I} \br{t x_i + (1-t) y_i} \in
\oline{\co}\,A. \qedhere
\]
\end{proof}

\begin{lema}
\label{conv:lm:cls_face}
Let $X$ be a topological vector space and $C \subseteq X$ be a
non-empty compact convex subset. Then for any $\phi \in X^*$ the
set
\[
F = \setb{x \in C}{\Re \phi(x) = \min_C \Re \phi}
\]
is a closed face of $C$.
\end{lema}

\begin{proof}
As $C$ is a compact set, the set $F$ is obviously non-empty. Also
note that, as a preimage of a closed point, $F$ is a closed set.
Convexity of $F$ follows from linearity of $\phi$. Suppose that
$tx + (1-t)y \in F$. As
\[
\min_C \Re \phi =
\Re \phi(tx + (1-t)y) =
t \phi(x) + (1-t) \phi(y) \geq
\min_C \Re \phi,
\]
it follows that $x, y \in F$. By definition, $F$ is a face.
\end{proof}

\begin{izrek}[Krein-Milman]
\index{Krein-Milman theorem}
Let $X$ be a locally convex space and $C \subseteq X$ a non-empty
convex compact subset. Then
\[
C = \oline{\co}\, \br{\ext C}.
\]
\end{izrek}

\datum{2023-10-18}

\begin{proof}
Let $\mathcal{F} = \set{\text{closed faces in $C$}}$ be a set,
ordered with $\supseteq$. As $C \in \mathcal{F}$, this is a
non-empty set. As each increasing chain in $\mathcal{F}$ has its
intersection\footnote{The intersection is non-empty as $C$ is
compact.} as an upper bound, we can apply Zorn's lemma and find a
maximal element $F_0 \in \mathcal{F}$.

Suppose there are distinct elements $x, y \in F_0$. By Hahn-Banach,
there exists a functional $\phi \in X^*$ such that
$\Re \phi(x) < \Re \phi(y)$. Now let
\[
F_1 = \setb{z \in F_0}{\Re \phi(z) = \min_{F_0} \Re \phi}.
\]
As $F_1 \subset F_0$ is a closed face in $F_0$ by
lemma~\ref{conv:lm:cls_face}, it is a closed face in $C$. This is a
contradiction, so $\abs{F_0} = 1$. Therefore,
$\ext C \ne \emptyset$.

It is clear that
$\oline{\co}\,(\ext C) \subseteq C = \oline{\co}\,C$. Suppose that
$x \in C \setminus \oline{\co}\,(\ext C)$. By Hahn-Banach, there
exists a functional $\psi \in X^*$ such that
\[
\Re \psi(x) < \min_{\oline{\co}\,(\ext C)} \Re \psi.
\]
Let
\[
F = \setb{z \in C}{\Re \psi(z) = \min_C \Re \psi}
\]
be a closed face in $C$. As there exists some
$z \in \ext F \subseteq \ext C$, we have
\[
\min_C \Re \psi =
\Re \psi(z) =
\min_{\oline{\co}\,(\ext C)} \Re \psi >
\Re \psi(x) \geq
\min_C \Re \psi.
\]
Such $x$ therefore cannot exist.
\end{proof}

\begin{trditev}
The space $c_0$ is not the dual space of a Banach space.
\end{trditev}

\begin{proof}
Let $X$ be a Banach space. By Banach-Alaoglu, the set $(X^*)_1$ is
compact, therefore, $(X^*)_1 = \oline{\co}\,(\ext (X^*)_1)$ by
Krein-Milman. In particular, $(X^*)_1$ has extreme points.

Let $x \in (c_0)_1$. There exists some $N \in \N$ such that
$\abs{x_n} < \frac{1}{2}$ for all $n > N$. Now define
$y, z \in c_0$ with $y_n = z_n = x_n$ for $n \leq N$ and
\[
y_n = x_n + \frac{1}{2^n}, \quad z_n = x_n - \frac{1}{2^n}
\]
for $n > N$. Clearly, $x = y + z$, therefore,
$x \not \in \ext (c_0)_1$. It follows that $(c_0)_1$ has no extreme
points.
\end{proof}

\begin{izrek}[Milman]
\index{Milman theorem}
Let $X$ be a locally convex space and $K \subseteq K$ a compact
space. Suppose that $\oline{\co}\,(K)$ is compact. Then
$\ext (\oline{\co}\,K) \subseteq K$.
\end{izrek}

\begin{proof}
Assume that there exists some
$x_0 \in \ext(\oline{\co}\,K) \setminus K$. Then there exists a
basis neighbourhood $V$ of $0$ in $X$ such that
$x_0 \not \in K + \oline{V}$. Now write
\[
K \subseteq \bigcup_{x \in K} (x + V).
\]
As $K$ is compact, we can write
\[
K \subseteq \bigcup_{i=1}^n (x_i + V)
\]
for some points $x_i \in K$. Now form
\[
K_i = \oline{\co}\,(K \cap (x_j + V)).
\]
Note that $K_j$ is convex and compact as it is a subset of
$\oline{\co}\,K$. We also have
\[
K_j \subseteq x_j + \oline{V}.
\]
Note that
\[
K \subseteq \bigcup_{i=1}^n K_i.
\]
Let
\[
\Sigma = \setb{t \in [0,1]^n}{\sum_{i=1}^n t_i = 1}.
\]
Define the map
\[
f \colon \Sigma \times \prod_{i=1}^n K_i \to X
\]
with
\[
f(t, k) = \sum_{i=1}^n t_i k_i.
\]
Note that $C = \im f$. As
\[
C \subseteq \co \br{\bigcup_{i=1}^n K_i},
\]
the set $C$ is convex. As it is the image of a compact set, it is
also compact. Because $K_j \subseteq C$ for all $j$, it follows
that
\[
C = \co \br{\bigcup_{i=1}^n K_i}.
\]
It follows that
\[
\oline{\co}\,K \subseteq
\oline{\co}\,\br{\bigcup_{i=1}^n K_i} =
\co \br{\bigcup_{i=1}^n K_i}.
\]
We can therefore deduce
\[
\oline{\co}\,K = \co \br{\bigcup_{i=1}^n K_i}.
\]
As $x_0$ is an element of this set, we can write
\[
x_0 = \sum_{i=1}^n t_i y_i
\]
for $t_i \in [0,1]$ and $y_i \in K_i$. As $x_0$ is an extreme
point, we must have $y_j = x_0$ for some $j$, therefore
$x_0 \in K_j \subseteq x_j + \oline{V} \subseteq K + \oline{V}$,
which is a contradiction.
\end{proof}

\begin{opomba}
In finite-dimensional vector spaces, the convex hull of a compact
set is compact.
\end{opomba}

\begin{opomba}
The set $\ext C$ is not always closed.
\end{opomba}
