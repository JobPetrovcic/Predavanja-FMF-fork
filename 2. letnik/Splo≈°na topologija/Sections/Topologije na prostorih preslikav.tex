\section{Topologije na prostorih preslikav}

\epigraph{">Kot nek quidditch."<}{-- prof.~dr.~Petar Pavešić}

\datum{2021-12-21}

\subsection{Prostori preslikav}

\begin{definicija}
Za $A \subseteq X$ in $U \subseteq Y$ označimo
\[
\skl{A,U} = \setb{f \in \mathcal{C}(X,Y)}{f(A) \subseteq U}.
\]
\end{definicija}

\begin{trditev}
Naj bo $\tau_p$ topologija na $\mathcal{C}(X,Y)$, ki jo definira
predbaza
\[
\setb{\skl{\set{x},U}}{x \in X \land U \in \tau_Y}.
\]
Tedaj je $f$ limita zaporedja $f_i$ v $\tau_p$ natanko tedaj, ko
$f_i$ po točkah konvergirajo k $f$.
\end{trditev}

\obvs

\begin{opomba}
Topologiji $\tau_p$ pravimo
\emph{topologija konvergence po točkah}.
\end{opomba}

\begin{definicija}
Topologiji, ki jo generira predbaza
\[
\setb{\skl{K,U}}
{K \subseteq X \land \text{$K$ je kompakt} \land U \in \tau_Y}
\]
pravimo
\emph{kompaktno-odprta topologija}\index{Topologija!Kompaktno-odprta}.
Prostor zveznih preslikav s kompaktno-odprto topologijo označimo z
$\widehat{\mathcal{C}}(X,Y)$.
\end{definicija}

\begin{trditev}
Če je $Y$ metričen, lahko za bazo kompaktno-odprte topologije na
$\mathcal{C}(X,Y)$ vzamemo
\[
\setb{\skl{f,K,\varepsilon}}
{f \in \mathcal{C}(X,Y) \land K \subseteq X
\land \text{$K$ je kompakten} \land \varepsilon > 0},
\]
kjer je
\[
\skl{f,K,\varepsilon} =
\setb{g \in \mathcal{C}(X,Y)}
{\forall x \in K \colon d(f(x),g(x)) < \varepsilon}.
\]
\end{trditev}

\begin{proof}
Najprej dokažimo, da je zgornja množica baza. Očitno je pokritje.
Naj bo
$g \in
\skl{f_1,K_1,\varepsilon_1} \cap \skl{f_2,K_2,\varepsilon_2}$.
dovolj je pokazati, da ima $g$ okolico, vsebovano v tem preseku.
Definiramo kompakt $K = K_1 \cup K_2$ in
\[
\varepsilon =
\min \set{
\varepsilon_1 - \max_{x \in K_1} \set{d(f_1(x), g(x))},
\varepsilon_2 - \max_{x \in K_2} \set{d(f_2(x), g(x))}
}.
\]
Ni težko videti, da je $\skl{g,K,\varepsilon}$ vsebovana v preseku.

Opazimo, da je
$\skl{x \mapsto y_0,K,\varepsilon} \subseteq \skl{K,U}$, zato je
topologija iz zgornje baze kvečjemu močnejša od kompaktno-odprte.

Opazimo, da je
\[
U_c = \setb{x \in K}{d(f(x),f(c)) < \frac{\varepsilon}{2}}
\]
odprto pokritje $K$, zato ima končno podpokritje. Sledi, da je
\[
\bigcap_{n=1}^N \skl{
\overline{U}_{c_n},
K\left(f(c_n), \frac{\varepsilon}{2}\right)
}
\]
odprta okolica $f$ v kompaktno-odprti topologiji, ki pa je po
trikotniški neenakosti vsebovana v $\skl{f,K,\varepsilon}$.
\end{proof}

\begin{opomba}
Če je $X$ kompakten, topologiji iz zgornje baze pravimo
\emph{topologija enakomerne konvergence}. Drugače ji pravimo
\emph{topologija enakomerne konvergence na kompaktih}.
\end{opomba}

\begin{trditev}
Preslikava $i \colon Y \to \widehat{\mathcal{C}}(X,Y)$, ki vsakemu
$y \in Y$ priredi preslikavo $x \mapsto y$, je vložitev. Če je $Y$
Hausdorffov, je vložitev zaprta.
\end{trditev}

\begin{proof}
Za poljubno predbazično okolico velja $\skl{K,U} \cap i(Y) = i(U)$,
zato je $i$ res bijekcija med topologijama.

Naj bo sedaj $Y$ Hausdorffov in $f$ nekonstantna. Sledi, da
obstajata $x_1$ in $x_2$ z različnima slikama, ki ju lahko ločimo z
$U_1$ in $U_2$. Sledi, da je
$\skl{\set{x_1},U_1} \cap \skl{\set{x_2},U_2}$ odprta okolica $U$,
ki ne vsebuje konstantnih preslikav, zato je komplement $i(Y)$
odprt.
\end{proof}

\begin{trditev}
Prostor $\widehat{\mathcal{C}}(X,Y)$ je Hausdorffov natanko tedaj,
ko je $Y$ Hausdorffov, in regularen natanko tedaj, ko je $Y$
regularen.
\end{trditev}

\begin{proof}
Če je $\widehat{\mathcal{C}}(X,Y)$ Hausdorffov, pa je po prejšnji
trditvi tudi $Y$ Hausdorffov. Podoben razmislek velja za
regularnost.

Če je $Y$ Hausdorffov in $f, g \in \mathcal{C}(X,Y)$ funkciji, za
kateri je $f(x) \ne g(x)$, sta $\skl{\set{x},U}$ in
$\skl{\set{x},V}$ okolici, ki ju ostro ločita, pri čemer sta $U$ in
$V$ množici, ki ostro ločita $f(x)$ in $g(x)$.

Naj bo $f \in \mathcal{C}(X,Y)$ in $\skl{K,V}$ njena okolica.
Dovolj je pokazati, da obstaja taka odprta množica $U$, da je
\[
f(K) \subseteq U \subseteq \overline{U} \subseteq V,
\]
saj bo tedaj
$f \in \skl{K,U} \subseteq \overline{\skl{K,U}} \subseteq
\skl{K,\overline{U}} \subseteq \skl{K,V}$.\footnote{Druga inkluzija
velja, saj za $f \not \in \skl{K, \overline{U}}$ obstaja $x \in K$,
za katerega je $f(x) \not \in \overline{U}$. Sledi, da obstaja
okolica $V$ točke $f(x)$, disjunktna $\overline{U}$, in je
$\skl{\set{x},V}$ okolica $f$, disjunktna $\skl{K,U}$.}
Zaradi regularnosti lahko vsako točko $y \in f(K)$ ostro ločimo od
$V^{\mathsf{c}}$. Vse množice, ki ločijo točko od $V^{\mathsf{c}}$,
tvorijo pokritje $f(K)$. Ker je $K$ kompakt, je tudi $f(K)$
kompakt, zato obstaja končno podpokritje. Ni težko videti, da
njegova unija ustreza pogoju za množico $U$.
\end{proof}

\newpage

\subsection{Preslikave na normalnih prostorih}

\datum{2022-1-4}

\begin{lema}[Urison]\index{Lema!Urison}
Prostor $X$ je $T_4$ natanko tedaj, ko za vsak par zaprtih
disjunktnih množic $A, B \subseteq X$ obstaja preslikava
$f \colon X \to [0,1]$, za katero je $f(A)=0$ in $f(B)=1$.
\end{lema}

\begin{proof}
Če taka preslikava obstaja, prasliki intervalov
$\left[0,\frac{1}{2}\right)$ in $\left(\frac{1}{2},1\right]$ ločita
$A$ in $B$. Naj bo sedaj prostor $T_4$.

$T_4$ je ekvivalentna temu, da za vsako zaprto množico $A$ in
odprto množico $U$ obstaja odprta množica $V$, za katero je
$A \subseteq V \subseteq \overline{V} \subseteq U$. Induktivno
lahko izberemo take množice $U_{\frac{n}{2^m}}$, da je
$U_a \subseteq U_b \iff a \leq b$ ter $A \subset U_x \subseteq B$.
Sedaj lahko vrednosti $f$ preprosto določimo z bisekcijo, oziroma
\[
f(x) = \begin{cases}
\inf \setb{r}{x \in U_r}, & x \not\in B \\
1                         & x \in B.
\end{cases}
\]
Ni težko preveriti, da je $f$ res zvezna -- pri
$\frac{1}{2^n} < \varepsilon$ za okolico $x$ vzamemo ustrezen pas
">širine"< $\frac{1}{2^n}$.
\end{proof}

\begin{opomba}
Če je $X$ metričnen, je primer Urisonove funkcije
\[
f(x) = \frac{d(x,A)}{d(x,A) + d(x,B)}.
\]
\end{opomba}

\begin{posledica}
Če je $X$ normalen, realne funkcije ločijo točke.
\end{posledica}

\begin{posledica}
V normalnem 2-števnem prostoru $X$ obstaja števna družina funkcij v
$\mathcal{C}(X,[0,1])$, ki loči točke $X$.
\end{posledica}

\begin{izrek}[Urison]\index{Izrek!Urison}
Vsak normalen 2-števen prostor je metrizabilen.
\end{izrek}

\begin{proof}
Naj bo $\setb{f_n \colon X \to [0,1]}{n \in \N}$ števna družina
Urisonovih funkcij, ki loči točke $X$. Naj bo\footnote{$l^2$
označuje kvadratno sumabilna zaporedja z metriko
$d(x,y) = \displaystyle \sqrt{\sum_{n=1}^\infty (x_n-y_n)^2}$.}
$f \colon X \to l^2$ funkcija, podana s predpisom
\[
f(x) = \left(\frac{1}{n} f_n(x)\right)_{n \in \N}.
\]
$f$ je dobro definirana, saj je zaporedje kvadratno sumabilno po
Weierstrassovem kriteriju. Dokažimo, da je $f$ vložitev. Opazimo,
da je $f$ injektivna, saj loči točke.

$f$ je zvezna -- naj bo $x \in X$, $\varepsilon > 0$ in $N \in \N$
tako naravno število, da je
\[
\sum_{n=N+1}^\infty \frac{1}{n^2} < \frac{\varepsilon^2}{2}.
\]
Množice
\[
U_n = \setb{y \in X}{(f_n(x)-f_n(y))^2 < \frac{\varepsilon^2}{2N}}
\]
so odprte, zato je odprta tudi
\[
U = \bigcap_{n=1}^N U_n.
\]
Za $y \in U$ tako sledi
\[
d(f(x),f(y))^2 =
\sum_{n=1}^\infty \frac{1}{n^2} (f_n(x)-f_n(y))^2 \leq
N \cdot \frac{\varepsilon^2}{2N} + \frac{\varepsilon^2}{2},
\]
zato je $d(f(x),f(y)) < \varepsilon$.

Preverimo še, da je tudi inverz zvezen. Naj bo $U$ poljubna okolica
točke $x \in X$. Obstajata bazični okolici $B$ in $B'$, za kateri
je
\[
x \in B \subseteq \overline{B} \subset B' \subseteq U.
\]
Naj bo $f_n$ Urisonova funkcija, prirejena paru $\overline{B}$ in
$X \setminus B'$. Za vse $y$, za katere je
$d(f(x),f(y)) < \frac{1}{n}$, je $f_n(y) < 1$, saj je $f_n(x) = 0$.
Sledi, da je $y \in B' \subseteq U$, zato je $f^{-1}$ zvezen.

Sledi, da je $X$ homeomorfen podprostoru $l^2$, zato je
metrizabilen.
\end{proof}

\begin{posledica}
2-števen prostor je metrizabilen natanko tedaj, ko je regularen.
\end{posledica}

\begin{lema}
Naj bo $A$ zaprta podmnožica normalnega prostora $X$ in
$f \colon A \to [-c,c]$. Tedaj obstaja taka preslikava
$h \colon X \to \left[-\frac{c}{3},\frac{c}{3}\right]$, za katero
je $\abs{f(a) - h(a)} \leq \frac{2}{3} c$ za vse $a \in A$.
\end{lema}

\begin{proof}
Naj bosta
\[
A_+ = \setb{a \in A}{f(a) \geq \frac{c}{3}}
\quad \text{in} \quad
A_- = \setb{a \in A}{f(a) \leq -\frac{c}{3}}.
\]
Naj bo sedaj $g$ Urisonova funkcija množic $A_+$ in $A_-$. Za $h$
preprosto vzamemo $\frac{2}{3}c \cdot g - \frac{1}{3}c$.
\end{proof}

\begin{izrek}[Tietze]\label{iz:tietze}\index{Izrek!Tietze}
Naj bo $A$ zaprta podmnožica normalnega prostora $X$. Tedaj lahko
vsako preslikavo $f \colon A \to J$, kjer je $J$ interval,
razširimo  do preslikave $F \colon X \to J$.
\end{izrek}

\begin{figure}[H]
\[
\begin{tikzcd}[column sep=large, row sep=large]
A \arrow[r, "f"] \arrow[d, hook'] & J \\
X \arrow[ur, "F"']                &
\end{tikzcd}
\]
\caption{Izrek~\ref{iz:tietze}.}
\end{figure}

\begin{proof}
Brez škode za splošnost naj bo $J=[-1,1]$. Po zgornji lemi obstaja
preslikava
$h_1 \colon X \to \left[-\frac{1}{3},\frac{1}{3}\right]$, za katero
funkcija $f_1 = f - h_1$ slika elemente $A$ v interval
$\left[-\frac{2}{3},\frac{2}{3}\right]$. Sedaj lahko induktivno
skonstruiramo funkcije $h_n$ in $f_n$, nato pa preprosto vzamemo
\[
F = \sum_{n=1}^\infty h_n,
\]
ki je zvezna po Weierstrassovem kriteriju. Ker je
\[
\abs{f_n(x)} \leq \left(\frac{2}{3}\right)^2
\]
za vse $x \in A$, je $F(x)=f(x)$ na $A$.
\end{proof}

\begin{definicija}
Prostor $E$ je
\emph{absolutni ekstenzor}\index{Topološki prostor!Absolutni ekstenzor},\footnote{Ekstenzorjev
in retraktov na predavanjih nismo obdelali. Ta del zapiskov je
povzet po knjigi Splošna topologija \cite{stop}.}
če lahko vsako preslikavo $f \colon A \to E$ iz zaprtega
podprostora $A$ normalnega prostora $X$ razširimo do preslikave, ki
je definirana na celem $X$.
\end{definicija}

\begin{definicija}
Podprostor $A \subseteq X$ je
\emph{retrakt}\index{Topološki prostor!Retrakt} prostora $X$, če
obstaja preslikava $r \colon X \to A$ z lastnostjo $r(a) = a$ za
vse $a \in A$. Taki preslikavi pravimo \emph{retrakcija}.
\end{definicija}

\begin{opomba}
Ker je $A$ množica ujemanja $\id$ in $i \circ r$, je retrakt
Hausdorffovega prostora vedno zaprt podprostor.
\end{opomba}

\begin{trditev}
Retrakt absolutnega ekstenzorja je absolutni ekstenzor.
\end{trditev}

\begin{proof}
Razširitev funkcije komponiramo z retrakcijo.
\end{proof}

\begin{definicija}
\emph{Nosilec}\index{Preslikava!Nosilec} preslikave
$f \colon X \to \R$ je množica $\overline{f^{-1}(\R^*)}$.
\end{definicija}

\begin{definicija}
\emph{Razčlenitev enote}, podrejena pokritju $\setb{U_i}{i \leq n}$
prostora $X$, je družina preslikav $\rho_i \colon X \to I$, za
katere je nosilec $\rho_i$ vsebovan v $U_i$ za vse $i$ in je
\[
\sum_{i=1}^n \rho_i(x) = 1
\]
za vse $x \in X$.
\end{definicija}

\begin{izrek}
Za vsako odprto pokritje $\setb{U_i}{i \leq n}$ normalnega prostora
$X$ obstaja neka podrejena razčlenitev enote.
\end{izrek}

\begin{proof}
Najprej dokažimo, da za vsako odprto pokritje
$\setb{U_i}{i \leq n}$ obstaja pokritje $\setb{V_i}{i \leq n}$, za
katerega velja $\overline{V_i} \subseteq U_i$.

Ker je $X$ normalen, obstaja tak $V_1$, da je
\[
X \bigsetminus \bigcup_{i=2}^n U_i \subseteq V_1 \subseteq
\overline{V_1} \subseteq U_1.
\]
Sledi, da lahko v pokritju $U_1$ zamenjamo z $V_1$. Ta postopek
nadaljujemo, dokler nismo zamenjali vseh množic.

Sledi, da obstajata odprti pokritji $\setb{V_i}{i \leq n}$ in
$\setb{W_i}{i \leq n}$, za kateri velja
\[
\overline{W_i} \subseteq V_i \subseteq
\overline{V_i} \subseteq U_i.
\]
Naj bodo $f_i$ Urisonove funkcije za množice $X \setminus V_i$
in $\overline{W_i}$. Ker je $\setb{W_i}{i \leq n}$ pokritje, je
\[
f = \sum_{i=1}^n f_i
\]
na celem $X$ neničelna, zato so preslikave
\[
\rho_i = \frac{f_i}{f}
\]
dobro definirane. Ker je nosilec $\rho_i$ vsebovan v
$\overline{V_i} \subseteq U_i$, tvorijo $\rho_i$ iskano razčlenitev
enote.
\end{proof}

\newpage

\subsection{Stone-Weierstrassov izrek}

\datum{2022-1-11}

\begin{izrek}[Weierstrass]\index{Izrek!Weierstrass}
Vsako zvezno funkcijo lahko poljubno dobro aproskimiramo s
polinomi.
\end{izrek}

\begin{proof}[Analitični dokaz]
Naj bo
\[
B_{n,i}(x) = \binom{n}{i} x^i (1-x)^{n-i}
\]
Berstainova baza. Velja
\[
\sum_{i=0}^n B_{n,i}(x) = 1.
\]
Za zvezno funkcijo $f \colon [0,1] \to \R$ naj bo
\[
f_n(x) = \sum_{i=0}^n f\left(\frac{i}{n}\right) B_{n,i}(x)
\]
$n$-ti Bernsteinov polinom.

Opazimo, da je
\begin{align*}
f_n(x) - f(x) &=
\sum_{i=0}^n f\left(\frac{i}{n}\right) B_{n,i}(x) -
f(x) \cdot \sum_{i=0}^n B_{n,i}(x)
\\
&=
\sum_{i=0}^n
\left(f\left(\frac{i}{n}\right) - f(x)\right) B_{n,i}(x).
\intertext{Sledi, da je}
\abs{f_n(x) - f(x)} &\leq
\sum_{i=0}^n \abs{f\left(\frac{i}{n}\right) - f(x)} B_{n,i}(x).
\end{align*}
Ker je $f$ zvezna, ima maksimum $M$ in je enakomerno zvezna. Za vse
$\varepsilon > 0$ tako obstaja $\delta > 0$, da za
$\abs{x-y} < \delta$ velja $\abs{f(x)-f(y)}<\frac{\varepsilon}{2}$.
Tako dobimo, da je\footnote{Uporabimo znano identiteto
$\displaystyle
\sum_{i=0}^n \left(x-\frac{i}{n}\right)^2 B_{n,i}(x) =
\frac{x \cdot (1-x)}{n}$.}
\begin{align*}
&\sum_{i=0}^n \abs{f\left(\frac{i}{n}\right) - f(x)} B_{n,i}(x)
\\
=&
\sum_{\abs{x-\frac{i}{n}} < \delta}
\abs{f\left(\frac{i}{n}\right) - f(x)} B_{n,i}(x) +
\sum_{\abs{x-\frac{i}{n}} \geq \delta}
\abs{f\left(\frac{i}{n}\right) - f(x)} B_{n,i}(x)
\\
\leq&
\frac{\varepsilon}{2} + 
2M \cdot \sum_{\abs{x-\frac{i}{n}} \geq \delta} B_{n,i}(x)
\\
\leq&
2M \cdot
\sum_{i=0}^n \frac{\left(x-\frac{i}{n}\right)}{\delta^2} B_{n,i}(x)
\\
=&
\frac{2M}{\delta^2} \cdot \frac{x \cdot (1-x)}{n}
\\
\leq&
\frac{M}{2n \cdot \delta^2}. \qedhere
\end{align*}
\end{proof}

\begin{izrek}[Stone-Weierstrass]\index{Izrek!Stone-Weierstrass}
Če je $A \subseteq \mathcal{C}(X)$ unitalna podalgebra, ki loči
točke $X$, potem je $A$ gosta v $\mathcal{C}(X)$ s kompaktno-odprto
topologijo.
\end{izrek}

\begin{proof}
Pomagamo si z naslednjo lemo:

\begin{lema*}
$f(t) = \sqrt{t}$ je na $[0,1]$ enakomerna limita polinomov.
\end{lema*}

\begin{proof}
$f$ razvijemo v Taylorjevo vrsto okoli $1$ in opazimo, da
konvergira za $t=0$ in je konvergenca na $[0,1]$ enakomerna.
\end{proof}

Iz leme sledi, da za $f \in A$ velja
$\abs{f} = \sqrt{f^2} \in \overline{A}$. Sledi, da je za
$f,g \in A$ tudi
$\max \set{f(x),g(x)}, \min \set{f(x),g(x)} \in \overline{A}$.

Ker $A$ loči točke, za vsaka $x, y \in X$ obstaja tak
$f \in \mathcal{C}(X)$, da je $f(x) = a$ in $f(y) = b$, saj lahko
funkcijo, ki loči $x$ in $y$, afino transformiramo.

Naj bo $f \in \mathcal{C}(X)$ poljubna. Pokazati moramo, da za
poljubna $K$ in $\varepsilon$ obstaja
$g \in \overline{A} \cap \skl{f,K,\varepsilon}$.

Za vsaka $u,v \in K$ naj bo $h_{u,v} \in A$ taka funkcija, da na
$u$ in $v$ sovpada z $f$. Za fiksen $u$ množice
\[
U_v = \setb{x \in K}{h_{u,v}(x) < f(x) + \varepsilon}
\]
tvorijo odprto pokritje. Sledi, da obstaja končno podpokritje.
Definiramo preslikave $h_u(x) = \min \set{h_{u,v_i}(x)}$, ki so v
$\overline{A}$.

Naj bo sedaj
\[
V_u = \setb{x \in K}{h_u(x) > f(x) - \varepsilon}.
\]
Tudi te množice tvorijo odprto pokritje, zato obstaja končno
podpokritje, katerih maksimum po točkah $h$ je v $\overline{A}$.

Ni težko videti, da je $h \in \skl{f,K,\varepsilon}$.
\end{proof}

\begin{posledica}
Polinomi so gosti v $\mathcal{C}(X)$ za $X \subseteq \R$.
\end{posledica}

\begin{posledica}
Fourierovi polinomi so gosti v $\mathcal{C}([-\pi,\pi])$.
\end{posledica}
