% ---- 5. 10. 2021 ----

\section{Prostori in preslikave}

\subsection{Topološki prostori}

\begin{okvir}
\begin{definicija}
Naj bo $X$ množica. \emph{Topologija}\index{Topologija} na $X$ je družina $\mathcal{T}$ podmnožic $X$, ki zadošča pogojem:

\begin{enumerate}[i)]
\item $\emptyset, X\in\mathcal{T}$,
\item poljubna unija elementov $\mathcal{T}$ je element $\mathcal{T}$,
\item poljuben končen presek elementov $\mathcal{T}$ je element $\mathcal{T}$.
\end{enumerate}

\emph{Topološki prostor}\index{Topološki prostor} je par $(X,\mathcal{T})$. Elementom $\mathcal{T}$ pravimo \emph{odprte množice}\index{Topologija!Odprte množice}.
\end{definicija}
\end{okvir}

\begin{opomba}
V metričnih prostorih $(X,d)$ odprte množice\footnote{Tu vzamemo definicijo odprtih množic v metričnih prostorih.} tvorijo topologijo $\mathcal{T}_d$.
\end{opomba}

\begin{definicija}
Topološki prostor $(X,\mathcal{T})$ je \emph{metrizabilen}\index{Topološki prostor!Metrizabilen}, če obstaja taka metrika $d$ na $X$, da je $\mathcal{T}=\mathcal{T}_d$ pri zgornjih oznakah.
\end{definicija}

\begin{opomba}
Za metriko $d'(x,x')=\min\set{d(x,x'),1}$ velja $\mathcal{T}_d=\mathcal{T}_{d'}$.
\end{opomba}