\section{Prostori in preslikave}

\epigraph{">Splošna topologija je LMN na steroidih."<}{-- Luka Horjak}

\subsection{Topološki prostori}

\datum{2021-10-5}

\begin{okvir}
\begin{definicija}
Naj bo $X$ množica. \emph{Topologija}\index{Topologija} na $X$ je
družina $\tau$ podmnožic $X$, ki zadošča pogojem:

\begin{enumerate}[i)]
\item $\emptyset, X \in \tau$,
\item poljubna unija elementov $\tau$ je element
$\tau$,
\item poljuben končen presek elementov $\tau$ je element
$\tau$.
\end{enumerate}

\emph{Topološki prostor}\index{Topološki prostor} je par
$(X,\tau)$. Elementom $\tau$ pravimo
\emph{odprte množice}\index{Topologija!Odprte, zaprte množice},
njihovim komplementom pa \emph{zaprte}.
\end{definicija}
\end{okvir}

\begin{opomba}
V metričnih prostorih $(X,d)$ odprte množice\footnote{Tu vzamemo
definicijo odprtih množic v metričnih prostorih.} tvorijo
topologijo $\tau_d$.
\end{opomba}

\begin{definicija}
Topološki prostor $(X,\tau)$ je
\emph{metrizabilen}\index{Topološka lastnost!Metrizabilnost}, če
obstaja taka metrika $d$ na $X$, da je
$\tau = \tau_d$ pri zgornjih oznakah.
\end{definicija}

\begin{opomba}
Za metriko $d'(x,x') = \min \set{d(x,x'),1}$ velja
$\tau_d = \tau_{d'}$.
\end{opomba}

\newpage

\subsection{Zvezne preslikave}

\datum{2021-10-12}

\begin{okvir}
\begin{definicija}
Funkcija $f \colon (X,\tau)\to(X',\tau')$ je
\emph{zvezna}\index{Funkcija!Zvezna}, če je praslika vsake odprte
množice odprta, oziroma
\[
V\in\tau' \implies f^{-1}(V) \in \tau.
\]
Zveznim funkcijam pravimo \emph{preslikave}.
\end{definicija}
\end{okvir}

\begin{opomba}
Zvezne preslikave med metričnimi prostori so zvezne tudi glede na z
metrikami porojene topologije.
\end{opomba}

\begin{opomba}
Identiteta $\id \colon (X,\tau) \to (X,\tau')$ je
zvezna natanko tedaj, ko je $\tau' \subseteq \tau$.
Pravimo, da je topologija $\tau$
\emph{finejša}\index{Topologija!Finejša, bolj groba},
$\tau'$ pa
\emph{bolj groba}.
\end{opomba}

\begin{trditev}
Kompozitum zveznih preslikav je zvezna preslikava.
\end{trditev}

\begin{proof}
Naj bosta $f \colon (X,\tau) \to (X',\tau')$ in
$g \colon (X',\tau') \to (X'',\tau'')$ zvezni. Sledi
\[
V \in \tau'' \implies g^{-1}(V) \in \tau' \implies
(g \circ f)^{-1}(V) \in \tau. \qedhere
\]
\end{proof}

\begin{opomba}
Množico vseh zveznih preslikav med $(X,\tau)$ in
$(Y,\tau')$ označimo z $\mathcal{C}((X,\tau),
(Y,\tau'))$, oziroma $\mathcal{C}(X,Y)$.
\end{opomba}

\begin{izrek}\label{iz:1}
Naslednje izjave so ekvivalentne:

\begin{enumerate}[i)]
\item $f \colon (X,\tau) \to (Y,\tau')$ je zvezna,
\item $V \in \tau' \implies f^{-1}(V) \in \tau$,
\item $B^\mathsf{c} \in \tau' \implies
(f^{-1}(B))^\mathsf{c} \in \tau$,
\item $\forall A \subseteq X \colon f(\overline{A}) \subseteq
\overline{f(A)}$.
\end{enumerate}
\end{izrek}

\begin{proof}
Prvi dve trditvi sta očitno ekvivalentni po definiciji zveznosti.
2.\ in 3.\ trditev sta očitno ekvivalentni, saj velja
$f^{-1}(B^\mathsf{c}) = f^{-1}(B)^\mathsf{c}$. Dokažimo še
ekvivalenco 3.\ in 4.\ trditve.

Naj bo $A\subseteq X$ poljubna in predpostavimo, da velja 3.\
trditev. Sledi, da je
\[
A \subseteq f^{-1}(f(A)) \subseteq f^{-1}(\overline{f(A)}).
\]
Desna stran je zaprta množica, zato je
$\overline{A} \subseteq f^{-1}(\overline{f(A)})$ in
\[
f(\overline{A}) \subseteq \overline{f(A)}.
\]
Sedaj predpostavimo, da velja 4.\ točka. Naj bo $B$ poljubna zaprta
podmnožica $Y$. Potem je
\[
f(\overline{f^{-1}(B)}) \subseteq \overline{f(f^{-1}(B))}
\subseteq \overline{B} = B.
\]
Sledi, da je
\[
\overline{f^{-1}(B)} \subseteq f^{-1}(B),
\]
zato je $f^{-1}(B)$ zaprta.
\end{proof}

\newpage

\subsection{Homeomorfizmi}

\datum{2021-10-19}

\begin{okvir}
\begin{definicija}
Funkcija $f \colon X \to X'$ določa
\emph{homeomorfizem}\index{Topologija!Homeomorfizem}
med prostoroma $(X,\tau)$ in $(X',\tau')$, če je $f$
bijekcija in obenem inducirana bijekcija
$f \colon \tau \to \tau'$. Pišemo
$(X,\tau) \approx (X',\tau)$ in pravimo, da sta
prostora \emph{homeomorfna}\index{Topološki prostor!Homeomorfen}.
\end{definicija}
\end{okvir}

\begin{definicija}
Funkcija je \emph{odprta}\index{Funkcija!Odprta, zaprta}, če je
slika vsake odprte podmnožice $X$ odprta v $X'$. Simetrično
definiramo \emph{zaprte} funkcija.
\end{definicija}

\begin{trditev}
Naslednje izjave so ekvivalentne:

\begin{enumerate}[i)]
\item $f \colon X \to X'$ je homeomorfizem,
\item $f$ je bijekcija, $f$ in $f^{-1}$ sta zvezni,
\item $f$ je zvezna, odprta bijekcija,
\item $f$ je zvezna, zaprta bijekcija.
\end{enumerate}
\end{trditev}

\obvs

\begin{definicija}
\emph{Topološka lastnost}\index{Topološka lastnost} je vsaka
lastnost topologije, ki se ohranja pri homeomorfizmih.
\end{definicija}

\begin{opomba}
Kompaktnost in povezanost sta topološki lastnosti, polnost pa ne.
\end{opomba}

\begin{definicija}
Definiramo naslednje množice:

\begin{enumerate}[i)]
\item $B^n = \setb{\vv{x} \in \R^n}{\norm{\vv{x}} \leq 1}$
-- zaprta enotska krogla
\item $\mathring{B}^n = \setb{\vv{x} \in \R^n}{\norm{\vv{x}} < 1}$
-- odprta enotska krogla
\item $S^{n-1} = \setb{\vv{x} \in \R^n}{\norm{\vv{x}} = 1}$
-- enotska sfera
\end{enumerate}
\end{definicija}

\begin{trditev}
Velja $\mathring{B}^n \approx \R^n$.
\end{trditev}

\begin{proof}
Vzamemo homeomorfizem
\[
f(\vv{x}) = \frac{\vv{x}}{1 - \norm{\vv{x}}}. \qedhere
\]
\end{proof}

\begin{trditev}
Velja $S^{n-1} \setminus \set{(0,0,\dots,1)} \approx \R^{n-1}$.
\end{trditev}

\begin{proof}
Naredimo inverzijo v točki $(0,0,\dots,1)$.
\end{proof}

\begin{opomba}
Zgornji preslikavi pravimo
\emph{stereografska projekcija}\index{Preslikava!Stereografska projekcija}.
\end{opomba}

\begin{opomba}
Posebej velja $S^2 \approx \C \cup \set{\infty}$. Temu prostoru
pravimo
\emph{Riemannova sfera}\index{Topološki prostor!Riemannova sfera}.
\end{opomba}

\newpage

\subsection{Baze in predbaze}

\datum{2021-10-26}

\begin{okvir}
\begin{definicija}
Družina odprtih množic $\mathcal{B} \subseteq \tau$ je
\emph{baza}\index{Topologija!Baza} topologije $\tau$, če lahko vsak
element $\tau$ dobimo kot unijo elementov $\mathcal{B}$.
\end{definicija}
\end{okvir}

\begin{trditev}
Naj bo $\mathcal{B}$ baza topologije $\tau$ množice $x$.
$A \subseteq X$ je odprta natanko tedaj, ko za vsak $x \in A$
obstaja $B \in \mathcal{B}$, za katero je $x \in B$ in
$B \subseteq A$.
\end{trditev}

\obvs

\begin{trditev}
Naj bo $f \colon (X,\tau) \to (X',\tau')$, $\mathcal{B}$ baza
$\tau$ in $\mathcal{B}'$ baza $\tau'$. Potem velja

\begin{enumerate}[i)]
\item $f$ je zvezna natanko tedaj, ko je
$f^{-1}(\mathcal{B}') \subseteq \tau$ in
\item $f$ je odprta natanko tedaj, ko je
$f(\mathcal{B}) \subseteq \tau'$.
\end{enumerate}
\end{trditev}

\obvs

\begin{definicija}
Naj bo $\mathcal{B}_x \subseteq \tau$ neka poddružina topologije,
katere elementi vsebujejo $x$. Pravimo, da je $\mathcal{B}_x$
\emph{lokalna baza}\index{Topologija!Baza!Lokalna} okolice pri $x$,
če za vsak $U \in \tau$, ki vsebuje $x$, obstaja
$B \in \mathcal{B}_x$, za katero je $B \subseteq U$.
\end{definicija}

\begin{trditev}
Naj bo $\mathcal{B}$ družina podmnožic $X$ in $\tau$ množica unij
elementov $\mathcal{B}$. Potem je $\tau$ topologija natanko tedaj,
ko je $\mathcal{B}$ pokritje $X$ in velja
\[
\forall B_1, B_2 \in \mathcal{B}, x \in B_1 \cap B_2 \;
\exists B \in \mathcal{B} \colon x \in B \land
B \subseteq B_1 \cap B_2.
\]
\end{trditev}

\obvs

\begin{definicija}
Naj bo $\mathcal{P}$ neko pokritje množice $X$. Družina
$\mathcal{B}$ vseh končnih presekov elementov $\mathcal{P}$ je
baza. Družini $\mathcal{P}$ pravimo
\emph{predbaza}\index{Topologija!Baza!Predbaza} baze $\mathcal{B}$.
\end{definicija}

\begin{trditev}
Naj bo $f \colon (X,\tau) \to (X',\tau')$ in $\mathcal{P}'$
predbaza za $\tau'$. Potem je $f$ zvezna natanko tedaj, ko je
$f^{-1}(\mathcal{P}') \subseteq \tau$.
\end{trditev}

\obvs

\begin{trditev}
Preslikava $f \colon Z \to X_1 \times \dots \times X_n$ je zvezna
za produktno topologijo\footnote{Produktna topologija je
topologija, ki jo dobimo iz baze
$\setb{U_1 \times \dots \times U_n}
{\forall i \colon U_i \in \tau_i}$.}\index{Topologija!Produktna}
natanko tedaj, ko so zvezne vse komponente
$f$.
\end{trditev}

\begin{proof}
Očitno je trditev dovolj dokazati za $n=2$. Naj bo
$f \colon Z \to X \times Y$ zvezna. Potem sta komponenti kompozitum
projekcije z $f$, ki sta obe zvezni.

Če sta obe komponenti zvezni, pa ni težko videti, da so praslike
pasov odprte, ti pa tvorijo predbazo.
\end{proof}

\begin{definicija}
Prostor je \emph{1-števen}\index{Topološka lastnost!1, 2-števnost},
če za vsak $x$ obstaja števna lokalna baza pri $x$. Prostor je
\emph{2-števen}, če ima števno bazo.
\end{definicija}

\begin{trditev}
Naj bo $X$ 1-števen topološki prostor. Potem za vsak $A \in X$
velja, da je $\overline{A} = L(A) =
\setb{x}{\text{$x$ je limita zaporedja v $A$}}$.
\end{trditev}

\begin{proof}
Za vsak $x \in \overline{A}$ obstaja števna lokalna baza, zato
lahko za vsak člen zaporedja preprosto izberemo poljuben
\[
a_n \in A \cap \bigcap_{i=1}^n B_i. \qedhere
\]
\end{proof}

\begin{trditev}
Naj bo $X$ 1-števen topološki prostor. $f \colon X \to Y$ je zvezna
natanko tedaj, ko za vsak $A \subseteq X$ velja
$f(L(A)) \subseteq L(f(A))$.
\end{trditev}

\begin{proof}
Uporabimo prejšnjo trditev in izrek~\ref{iz:1}.
\end{proof}

\datum{2021-11-2}

\begin{definicija}
Množica $A \subseteq X$ je \emph{gosta}\index{Gosta množica}, če je
$\overline{A} = X$.
\end{definicija}

\begin{definicija}
Prostor $(X, \tau)$ je
\emph{separabilen}\index{Topološka lastnost!Separabilnost}, če v
njem obstaja števna gosta množica.
\end{definicija}

\begin{izrek}
Metrični prostor je 2-števen natanko tedaj, ko je separabilen.
\end{izrek}

\begin{proof}
Vsak 2-števen prostor je očitno separabilen -- iz vsake bazične
okolice vzamemo po eno točko.

Naj bo $Q$ števna gosta podmnožica $X$. Sedaj preprosto vzamemo
\[
\mathcal{B} = \setb{\mathcal{K}(q,r)}{q \in Q, r \in \Q}. \qedhere
\]
\end{proof}

\newpage

\subsection{Podprostori}

\begin{definicija}
Naj bo $(X, \tau)$ topološki prostor in $A \subseteq X$. Množica
\[
\tau_A = \setb{A \cap U}{U \in \tau}
\]
je topologija na $A$, ki ji pravimo \emph{inducirana} ali
\emph{podedovana topologija}\index{Topologija!Inducirana}. Prostor
$(A, \tau_A)$ je
\emph{podprostor}\index{Topološki prostor!Podprostor} prostora
$(X, \tau)$.
\end{definicija}

\begin{trditev}
Zaprte množice v $A$ so preseki $A$ z zaprtimi množicami $X$.
\end{trditev}

\obvs

\begin{trditev}
Naj bo $\mathcal{B}$ baza $\tau$. Potem je
\[
\mathcal{B}_A = \setb{B \cap A}{B \in \mathcal{B}}
\]
baza inducirane topologije.
\end{trditev}

\obvs

\begin{definicija}
Topološka lastnost je
\emph{dedna}\index{Topologija!Dedna lastnost}, če se prenaša na
podprostore.
\end{definicija}

\begin{opomba}
Separabilnost ni dedna lastnost, a se deduje na odprte podprostore.
\end{opomba}

\datum{2021-11-9}

\begin{trditev}
Naj bo $B \subseteq A \subseteq X$. Potem je
\[
\Cl_A B = A \cap \Cl_X B, \quad
\Int_A B \supseteq A \cap \Int_X B
\quad \text{in} \quad
\Fr_A B \subseteq A \cap \Fr_X B.
\]
\end{trditev}

\begin{proof}
Velja
\begin{align*}
\Cl_A B &=
\bigcap\setb{F}
{B \subseteq F \subseteq A \land \text{$F$ je zaprta v $A$}}
\\
&= \bigcap\setb{A \cap F}
{B \subseteq F \subseteq X \land \text{$F$ je zaprta v $X$}}
\\
&= A \cap \Cl_X B.
\end{align*}
Podobno je
\begin{align*}
\Int_A B &=
\bigcup\setb{U}
{U \subseteq A \land U \subseteq B \land
\text{$U$ je odprta v $A$}}
\\
&= \bigcup\setb{A \cap U}
{U \subseteq X \land U \cap A \subseteq B \land
\text{$U$ je odprta v $X$}}
\\
&\supseteq A \cap \left(\bigcup\setb{U}
{U \subseteq X \land U \subseteq B \land
\text{$U$ je odprta v $X$ in $B$}}\right)
\\
&= A \cap \Int_X B.
\end{align*}
Zadnja inkluzija je direktna posledica prejšnje enakosti in
inkluzije.
\end{proof}

\begin{trditev}
Če je $B$ odprta v $A$ in $A$ odprta v $X$, je $B$ odprta v $X$.
Podobno, če je $B$ zaprta v $A$ in $A$ zaprta v $X$, je $B$ zaprta
v $X$.
\end{trditev}

\obvs

\begin{trditev}
Inkluzija $i \colon (A, \tau_A) \hookrightarrow (X, \tau)$ je
zvezna.
\end{trditev}

\obvs

\begin{trditev}
Zožitev zvezne funkcije na podprostor je zvezna.
\end{trditev}

\begin{proof}
Velja $\eval{f}{A}{} = f \circ i$.
\end{proof}

\begin{izrek}\label{iz:2}
Naj bo $\set{X_\lambda}$ odprto pokritje za $X$. Potem velja
\[
\text{$A$ je odprta v $X$} \iff \forall \lambda \colon
\text{$A \cap X_\lambda$ je odprta v $X_\lambda$}.
\]
Če je $\set{X_\lambda}$ zaprto in lokalno končno pokritje za $X$,
velja
\[
\text{$A$ je zaprta v $X$} \iff \forall \lambda \colon
\text{$A \cap X_\lambda$ je zaprta v $X_\lambda$}.
\]
\end{izrek}

\begin{proof}
Če je $A \cap X_\lambda$ odprta v $X$ za vse $\lambda$, je $A$
odprta v $X$, saj je
\[
A = \bigcup_\lambda A \cap X_\lambda.
\]
Naj bo sedaj $A \cap X_\lambda$ zaprta v $X$ za vse $\lambda$.\footnote{
Obe preostali implikaciji sta direktna posledica definicije
inducirane topologije.} Naj bo $x \in X \setminus A$ in $U$ okolica
$x$, ki seka končno mnogo elementov $\set{X_\lambda}$ -- naj bodo
to $X_{\lambda_1}, \dots, X_{\lambda_n}$. Sledi, da je
\[
x \in U \bigsetminus \bigcup_{i=1}^n A \cap X_{\lambda_i},
\]
kar je odprta okolica $x$, ki ne seka $A$.
\end{proof}

\begin{izrek}
Naj bo $\set{X_\lambda}$ odprto ali lokalno končno zaprto pokritje
$X$ in $f \colon X \to Y$ funkcija. Potem velja
\[
\text{$f$ je zvezna} \iff \forall \lambda \colon
\text{$\eval{f}{X_\lambda}{}$ je zvezna}.
\]
\end{izrek}

\begin{proof}
Naj bo $\set{X_\lambda}$ odprto pokritje. Za odprto množico
$U \subseteq Y$ velja
\[
f^{-1}(U) =
\bigcup_\lambda f^{-1}(U) \cap X_\lambda =
\bigcup_\lambda \left(\eval{f}{X_\lambda}{}\right)^{-1}(U),
\]
kar je odprto v $X$.

Naj bo $\set{X_\lambda}$ lokalno končno zaprto pokritje. Za zaprto
množico $F \subseteq Y$ velja
\[
f^{-1}(F) \cap X_\lambda =
\left(\eval{f}{X_\lambda}{}\right)^{-1}(F),
\]
kar je zaprto v $X$. Po izreku~\ref{iz:2} je torej $f^{-1}(F)$
zaprta.
\end{proof}

\begin{definicija}
Preslikava $f \colon X \to Y$ je
\emph{vložitev}\index{Preslikava!Vložitev}, če je
$f \colon X \to f(X)$ homeomorfizem.
\end{definicija}

\begin{trditev}
Naj bo $f \colon X \to Y$ injektivna preslikava.

\begin{enumerate}[i)]
\item Če je $f(X)$ odprt v $Y$, je $f$ vložitev natanko tedaj, ko
je odprta.
\item Če je $f(X)$ zaprt v $Y$, je $f$ vložitev natanko tedaj, ko
je zaprta.
\end{enumerate}
\end{trditev}

\begin{proof}
Če je $f(X)$ odprt podprostor, so podmnožice $f(X)$ odprte natanko
tedaj, ko so odprte v $Y$. Sledi, da je $f \colon X \to f(X)$
odprta natanko tedaj, ko je odprta $f \colon X \to Y$. Podobno
velja, če je $f(X)$ zaprt.
\end{proof}
