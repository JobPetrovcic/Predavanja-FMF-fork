\section{Prostori in preslikave}

\subsection{Topološki prostori}

\datum{2021-10-5}

\begin{okvir}
\begin{definicija}
Naj bo $X$ množica. \emph{Topologija}\index{Topologija} na $X$ je
družina $\mathcal{T}$ podmnožic $X$, ki zadošča pogojem:

\begin{enumerate}[i)]
\item $\emptyset, X \in \mathcal{T}$,
\item poljubna unija elementov $\mathcal{T}$ je element
$\mathcal{T}$,
\item poljuben končen presek elementov $\mathcal{T}$ je element
$\mathcal{T}$.
\end{enumerate}

\emph{Topološki prostor}\index{Topološki prostor} je par
$(X,\mathcal{T})$. Elementom $\mathcal{T}$ pravimo
\emph{odprte množice}\index{Topologija!Odprte, zaprte množice},
njihovim komplementom pa \emph{zaprte}.
\end{definicija}
\end{okvir}

\begin{opomba}
V metričnih prostorih $(X,d)$ odprte množice\footnote{Tu vzamemo
definicijo odprtih množic v metričnih prostorih.} tvorijo
topologijo $\mathcal{T}_d$.
\end{opomba}

\begin{definicija}
Topološki prostor $(X,\mathcal{T})$ je
\emph{metrizabilen}\index{Topološki prostor!Metrizabilen}, če
obstaja taka metrika $d$ na $X$, da je
$\mathcal{T} = \mathcal{T}_d$ pri zgornjih oznakah.
\end{definicija}

\begin{opomba}
Za metriko $d'(x,x') = \min \set{d(x,x'),1}$ velja
$\mathcal{T}_d = \mathcal{T}_{d'}$.
\end{opomba}

\newpage

\subsection{Zvezne preslikave}

\datum{2021-10-12}

\begin{okvir}
\begin{definicija}
Preslikava $f \colon (X,\mathcal{T})\to(X',\mathcal{T}')$ je
\emph{zvezna}\index{Preslikava!Zvezna}, če je praslika vsake odprte
množice odprta, oziroma
\[
V\in\mathcal{T}' \implies f^{-1}(V) \in \mathcal{T}.
\]
\end{definicija}
\end{okvir}

\begin{opomba}
Zvezne preslikave med metričnimi prostori so zvezne tudi glede na z
metrikami porojene topologije.
\end{opomba}

\begin{opomba}
Identiteta $\id \colon (X,\mathcal{T}) \to (X,\mathcal{T}')$ je
zvezna natanko tedaj, ko je $\mathcal{T}' \subseteq \mathcal{T}$.
Pravimo, da je topologija $\mathcal{T}$
\emph{finejša}\index{Topologija!Finejša, bolj groba},
$\mathcal{T}'$ pa
\emph{bolj groba}.
\end{opomba}

\begin{trditev}
Kompozitum zveznih preslikav je zvezna preslikava.
\end{trditev}

\begin{proof}
Naj bosta $f \colon (X,\mathcal{T}) \to (X',\mathcal{T}')$ in
$g \colon (X',\mathcal{T}') \to (X'',\mathcal{T}'')$ zvezni. Sledi
\[
V \in \mathcal{T}'' \implies g^{-1}(V) \in \mathcal{T}' \implies
(g \circ f)^{-1}(V) \in \mathcal{T}. \qedhere
\]
\end{proof}

\begin{opomba}
Množico vseh zveznih preslikav med $(X,\mathcal{T})$ in
$(Y,\mathcal{T}')$ označimo z $\mathcal{C}((X,\mathcal{T}),
(Y,\mathcal{T}'))$, oziroma $\mathcal{C}(X,Y)$.
\end{opomba}

\begin{izrek}
Naslednje izjave so ekvivalentne:

\begin{enumerate}[i)]
\item $f \colon (X,\mathcal{T}) \to (Y,\mathcal{T}')$ je zvezna,
\item $V \in \mathcal{T}' \implies f^{-1}(V) \in \mathcal{T}$,
\item $B^\mathsf{c} \in \mathcal{T}' \implies
(f^{-1}(B))^\mathsf{c} \in \mathcal{T}$,
\item $\forall A \subseteq X \colon f(\overline{A}) \subseteq
\overline{f(A)}$.
\end{enumerate}
\end{izrek}

\begin{proof}
Prvi dve trditvi sta očitno ekvivalentni po definiciji zveznosti.
2. in 3. trditev sta očitno ekvivalentni, saj velja
$f^{-1}(B^\mathsf{c}) = f^{-1}(B)^\mathsf{c}$. Dokažimo še
ekvivalenco 3. in 4. trditve.

Naj bo $A\subseteq X$ poljubna in predpostavimo, da velja 3.
trditev. Sledi, da je
\[
A \subseteq f^{-1}(f(A)) \subseteq f^{-1}(\overline{f(A)}).
\]
Desna stran je zaprta množica, zato je
$\overline{A} \subseteq f^{-1}(\overline{f(A)})$ in
\[
f(\overline{A}) \subseteq \overline{f(A)}.
\]
Sedaj predpostavimo, da velja 4. točka. Naj bo $B$ poljubna zaprta
podmnožica $Y$. Potem je
\[
f(\overline{f^{-1}(B)}) \subseteq \overline{f(f^{-1}(B))}
\subseteq \overline{B} = B.
\]
Sledi, da je
\[
\overline{f^{-1}(B)} \subseteq f^{-1}(B),
\]
zato je $f^{-1}(B)$ zaprta.
\end{proof}

\newpage

\subsection{Homeomorfizmi}

\datum{2021-10-19}

\begin{okvir}
\begin{definicija}
Funkcija $f \colon X \to X'$ določa
\emph{homeomorfizem}\index{Topologija!Homeomorfizem}
med prostoroma $(X,\mathcal{T})$ in $(X',\mathcal{T}')$, če je $f$
bijekcija in obenem inducirana bijekcija
$f \colon \mathcal{T} \to \mathcal{T}'$. Pišemo
$(X,\mathcal{T}) \approx (X',\mathcal{T})$ in pravimo, da sta
prostora \emph{homeomorfna}\index{Topološki prostor!Homeomorfen}.
\end{definicija}
\end{okvir}

\begin{definicija}
Preslikava je \emph{odprta}\index{Preslikava!Odprta, zaprta}, če je
slika vsake odprte podmnožice $X$ odprta v $X'$. Simetrično
definiramo \emph{zaprte} preslikave.
\end{definicija}

\begin{trditev}
Naslednje izjave so ekvivalentne:

\begin{enumerate}[i)]
\item $f \colon X \to X'$ je homeomorfizem,
\item $f$ je bijekcija, $f$ in $f^{-1}$ sta zvezni,
\item $f$ je zvezna, odprta bijekcija,
\item $f$ je zvezna, zaprta bijekcija.
\end{enumerate}
\end{trditev}

\obvs

\begin{definicija}
\emph{Topološka lastnost}\index{Topologija!Topološka lastnost} je
vsaka lastnost topologije, ki se ohranja pri homeomorfizmih.
\end{definicija}

\begin{opomba}
Kompaktnost in povezanost sta topološki lastnosti, polnost pa ne.
\end{opomba}

\begin{definicija}
Definiramo naslednje množice:

\begin{enumerate}[i)]
\item $B^n = \setb{\vv{x} \in \R^n}{\norm{\vv{x}} \leq 1}$
-- zaprta enotska krogla
\item $\mathring{B}^n = \setb{\vv{x} \in \R^n}{\norm{\vv{x}} < 1}$
-- odprta enotska krogla
\item $S^{n-1} = \setb{\vv{x} \in \R^n}{\norm{\vv{x}} = 1}$
-- enotska sfera
\end{enumerate}
\end{definicija}

\begin{trditev}
Velja $\mathring{B}^n \approx \R^n$.
\end{trditev}

\begin{proof}
Vzamemo bijekcijo
\[
f(\vv{x}) = \frac{\vv{x}}{1 - \norm{\vv{x}}}. \qedhere
\]
\end{proof}

\begin{trditev}
Velja $S^{n-1} \setminus \set{(0,0,\dots,1)} \approx \R^{n-1}$.
\end{trditev}

\begin{proof}
Naredimo inverzijo v točki $(0,0,\dots,1)$.
\end{proof}

\begin{opomba}
Zgornji preslikavi pravimo
\emph{stereografska projekcija}\index{Preslikava!Stereografska
projekcija}.
\end{opomba}

\begin{opomba}
Posebej velja $S^2 \approx \C \cup \set{\infty}$. Temu prostoru
pravimo
\emph{Riemannova sfera}\index{Topološki prostor!Riemannova sfera}.
\end{opomba}
