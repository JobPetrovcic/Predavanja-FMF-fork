\section{Kombinatorika}

\subsection{Osnovna načela kombinatorike}

\datum{2022-2-14}

\begin{trditev}[Načelo produkta]
\index{Načelo produkta, vsote, enakosti}
Naj bodo $A_1, \dots, A_n$ končne množice. Tedaj je
\[
\abs{\prod_{i=1}^n A_i} = \prod_{i=1}^n \abs{A_i}.
\]
\end{trditev}

\begin{trditev}[Načelo vsote]
Naj bodo $A_1, \dots, A_n$ končne, paroma disjunktne množice. Tedaj
je
\[
\abs{\bigcup_{i=1}^n A_i} = \sum_{i=1}^n \abs{A_i}.
\]
\end{trditev}

\begin{trditev}[Načelo enakosti]
Če obstaja bijekcija med končnima množicama $A$ in $B$, je
\[
\abs{A} = \abs{B}.
\]
\end{trditev}

\begin{definicija}
Označimo
\[
[n] = \setb{i \in \N}{i \leq n}.
\]
\end{definicija}

\begin{trditev}
Za \emph{Eulerjev fi}\index{Eulerjev fi}
\[
\varphi(n) = \abs{\setb{i \in [n]}{(i,n) = 1}}
\]
velja rekurzivna formula
\[
\sum_{d \mid n} \varphi(d) = n.
\]
\end{trditev}

\begin{proof}
Na dva načina izračunamo moč množice
\[
\setb{\frac{i}{n}}{i \in [n]}. \qedhere
\]
\end{proof}

\begin{izrek}[Dirichletovo načelo]
\index{Izrek!Dirichletovo načelo}
Če je $n > m$, potem ne obstaja injektivna preslikava
$f \colon [n] \to [m]$.
\end{izrek}

\obvs

\begin{trditev}[Načelo dvojnega preštevanja]
\index{Načelo dvojnega preštevanja}
Če dva izraza predstavljata število elementov iste množice, sta
enaka.
\end{trditev}

\begin{definicija}
Definiramo padajočo in naraščajočo potenco
\[
k^{\underline{n}} = \prod_{i=0}^{n-1} (k - i)
\quad \text{in} \quad
k^{\overline{n}} = \prod_{i=0}^{n-1} (k + i).
\]
\end{definicija}

\begin{trditev}
Za množico $N$ z $n$ elementi in množico $K$ s $k$ elementi velja

\begin{enumerate}[i)]
\item $\abs{K^N} = k^n$
\item $\abs{\setb{f \colon N \to K}{\text{$f$ je injektivna}}} =
k^{\underline{n}}$
\item Število bijekcij med $N$ in $K$ je $n!$, če je $n = k$, sicer
pa $0$.
\end{enumerate}
\end{trditev}

\obvs

\begin{opomba}
Če imata končni množici enako moč, je bijektivnost preslikave med
njima ekvivalentna tako injektivnosti kot surjektivnosti.
\end{opomba}

\newpage

\subsection{Binomski koeficienti in binomski izrek}

\datum{2022-2-21}

\begin{definicija}
Naj bo $x \in \C$ in $k \in \N_0$.
\emph{Binomski koeficient}\index{Binomski koeficient} števil $x$ in
$k$ je število
\[
\binom{x}{k} = \frac{x^{\underline{k}}}{k!}.
\]
\end{definicija}

\begin{trditev}
Če je $n \in \N_0$ in $k \leq n$, potem je
\[
\binom{n}{k} = \frac{n!}{k! \cdot (n-k)!}.
\]
\end{trditev}

\obvs

\begin{definicija}
Naj bo $X$ končna množica. Označimo
\[
\binom{X}{k} = \setb{A \subseteq X}{\abs{A} = k}.
\]
\end{definicija}

\begin{trditev}
Za končno množico $X$ velja
\[
\abs{\binom{X}{k}} = \binom{\abs{X}}{k}.
\]
\end{trditev}

\begin{proof}
Na dva načina preštejemo število urejenih $k$-teric. Dobimo, da je
\[
k! \cdot \abs{\binom{X}{k}} = \abs{X}^{\underline{k}}. \qedhere
\]
\end{proof}

\begin{trditev}
Če je $1 \leq k \leq n$, je
\[
\binom{n}{k} = \binom{n-1}{k-1} + \binom{n-1}{k}.
\]
\end{trditev}

\begin{proof}
Preštejmo število $k$-elementnih podmnožic $[n]$ na dva načina. Za
število $1$ imamo dve možnosti -- lahko je v podmnožico ali ne.
Sedaj preštejemo načine, na katere lahko izmed preostalih $n-1$
elementov izberemo ustrezno število. V prvem primeru dobimo
$\binom{n-1}{k-1}$ podmnožic, v drugem pa $\binom{n-1}{k}$.
\end{proof}

\begin{posledica}
Binomski koeficienti so ravno števila v
\emph{Pascalovem trikotniku}\index{Pascalov trikotnik}:
\[
\begin{array}{ccccccccccc}
  &   &   &    & 1 &    &   &   &   \\
  &   &   & 1  &   & 1  &   &   &   \\
  &   & 1 &    & 2 &    & 1 &   &   \\
  & 1 &   & 3  &   & 3  &   & 1 &   \\
1 &   & 4 &    & 6 &    & 4 &   & 1
\end{array}
=
\begin{array}{ccccccccccc}
             &              &              &              & \binom{0}{0} &              &              &              &              \\
             &              &              & \binom{1}{0} &              & \binom{1}{1} &              &              &              \\
             &              & \binom{2}{0} &              & \binom{2}{1} &              & \binom{2}{2} &              &              \\
             & \binom{3}{0} &              & \binom{3}{1} &              & \binom{3}{2} &              & \binom{3}{3} &              \\
\binom{4}{0} &              & \binom{4}{1} &              & \binom{4}{2} &              & \binom{4}{3} &              & \binom{4}{4}
\end{array}
\]
\end{posledica}

\begin{izrek}[Binomski]\index{Izrek!Binomski}
Za vse $n \in \N_0$ za $a, b \in K$, kjer je $K$ komutativen
kolobar, velja
\[
(a+b)^n = \sum_{k=0}^n \binom{n}{k} a^k b^{n-k}.
\]
\end{izrek}

\begin{proof}
Uporabimo distributivnost in preštejemo število pojavitev vsakega
monoma.
\end{proof}

\newpage

\subsection{Izbori}

\begin{definicija}
Naj bo $N$ množica z $n$ elementi.

\begin{enumerate}[i)]
\item \emph{Urejen izbor s ponavljanjem}\index{Izbor} je vsaka
urejena $k$-terica elementov $N$.
\item \emph{Urejen izbor brez ponavljanja} je vsaka $k$-terica
paroma različnih elementov $N$.
\item \emph{Neurejen izbor brez ponavljanja} je vsaka podmnožica
$k$ elementov množice $N$.
\item \emph{Neurejen izbor s ponavljanjem} je vsaka multimnožica
s $k$ elementi iz $N$.
\end{enumerate}
\end{definicija}

\begin{opomba}
Urejenim izborom pravimo tudi \emph{variacije}, neurejenim pa
\emph{kombinacije}.
\end{opomba}

\begin{trditev}
Število neurejenih izborov s ponavljanjem je enako
\[
\binom{n+k-1}{k}.
\]
\end{trditev}

\begin{proof}
Uporabimo strategijo pik in pregrad. Naj $x_k$ označuje število
pojavitev elementa $k$ v multimnožici. Tedaj lahko vsak izbor
predstavimo kot
\[
\underbrace{\bullet \bullet \dots \bullet}_{x_1} \mid
\underbrace{\bullet \bullet \dots \bullet}_{x_2} \mid
\dots \mid
\underbrace{\bullet \bullet \dots \bullet}_{x_n}.
\]
Vsak izbor natanko ustreza enemu zapisu s pikami in pregradami, teh
pa je ravno $\binom{n+k-1}{k}$. Izmed $n+k-1$ elementov moramo
namreč izbrati $k$ pik, preostalih $n-1$ mest pa zasedejo pregrade.
\end{proof}

\newpage

\subsection{Permutacije in permutacije s ponavljanjem}

\begin{definicija}
\emph{Permutacija}\index{Permutacija} množice $A$ je bijekcija
$f \colon A \to A$. Označimo
\[
S_A = \setb{f \colon A \to A}{\text{$f$ je bijekcija}},
\]
v posebnem primeru
\[
S_n = S_{[n]}.
\]
\end{definicija}
