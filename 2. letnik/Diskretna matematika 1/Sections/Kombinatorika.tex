\section{Kombinatorika}

\subsection{Osnovna načela kombinatorike}

\datum{2022-2-14}

\begin{trditev}[Načelo produkta]
\index{Načelo produkta, vsote, enakosti}
Naj bodo $A_1, \dots, A_n$ končne množice. Tedaj je
\[
\abs{\prod_{i=1}^n A_i} = \prod_{i=1}^n \abs{A_i}.
\]
\end{trditev}

\begin{trditev}[Načelo vsote]
Naj bodo $A_1, \dots, A_n$ končne, paroma disjunktne množice. Tedaj
je
\[
\abs{\bigcup_{i=1}^n A_i} = \sum_{i=1}^n \abs{A_i}.
\]
\end{trditev}

\begin{trditev}[Načelo enakosti]
Če obstaja bijekcija med končnima množicama $A$ in $B$, je
\[
\abs{A} = \abs{B}.
\]
\end{trditev}

\begin{definicija}
Označimo
\[
[n] = \setb{i \in \N}{i \leq n}.
\]
\end{definicija}

\begin{trditev}
Za \emph{Eulerjev fi}\index{Eulerjev fi}
\[
\varphi(n) = \abs{\setb{i \in [n]}{(i,n) = 1}}
\]
velja rekurzivna formula
\[
\sum_{d \mid n} \varphi(d) = n.
\]
\end{trditev}

\begin{proof}
Na dva načina izračunamo moč množice
\[
\setb{\frac{i}{n}}{i \in [n]}. \qedhere
\]
\end{proof}

\begin{izrek}[Dirichletovo načelo]
\index{Izrek!Dirichletovo načelo}
Če je $n > m$, potem ne obstaja injektivna preslikava
$f \colon [n] \to [m]$.
\end{izrek}

\obvs

\begin{trditev}[Načelo dvojnega preštevanja]
\index{Načelo dvojnega preštevanja}
Če dva izraza predstavljata število elementov iste množice, sta
enaka.
\end{trditev}

\begin{definicija}
Definiramo padajočo in naraščajočo potenco
\[
k^{\underline{n}} = \prod_{i=0}^{n-1} (k - i)
\quad \text{in} \quad
k^{\overline{n}} = \prod_{i=0}^{n-1} (k + i).
\]
\end{definicija}

\begin{trditev}
Za množico $N$ z $n$ elementi in množico $K$ s $k$ elementi velja

\begin{enumerate}[i)]
\item $\abs{K^N} = k^n$
\item $\abs{\setb{f \colon N \to K}{\text{$f$ je injektivna}}} =
k^{\underline{n}}$
\item Število bijekcij med $N$ in $K$ je $n!$, če je $n = k$, sicer
pa $0$.
\end{enumerate}
\end{trditev}

\obvs

\begin{opomba}
Če imata končni množici enako moč, je bijektivnost preslikave med
njima ekvivalentna tako injektivnosti kot surjektivnosti.
\end{opomba}

\newpage

\subsection{Binomski koeficienti in binomski izrek}

\datum{2022-2-21}

\begin{definicija}
Naj bo $x \in \C$ in $k \in \N_0$.
\emph{Binomski koeficient}\index{Binomski koeficient} števil $x$ in
$k$ je število
\[
\binom{x}{k} = \frac{x^{\underline{k}}}{k!}.
\]
\end{definicija}

\begin{trditev}
Če je $n \in \N_0$ in $k \leq n$, potem je
\[
\binom{n}{k} = \frac{n!}{k! \cdot (n-k)!}.
\]
\end{trditev}

\obvs

\begin{definicija}
Naj bo $X$ končna množica. Označimo
\[
\binom{X}{k} = \setb{A \subseteq X}{\abs{A} = k}.
\]
\end{definicija}

\begin{trditev}
Za končno množico $X$ velja
\[
\abs{\binom{X}{k}} = \binom{\abs{X}}{k}.
\]
\end{trditev}

\begin{proof}
Na dva načina preštejemo število urejenih $k$-teric. Dobimo, da je
\[
k! \cdot \abs{\binom{X}{k}} = \abs{X}^{\underline{k}}. \qedhere
\]
\end{proof}

\begin{trditev}
Če je $1 \leq k \leq n$, je
\[
\binom{n}{k} = \binom{n-1}{k-1} + \binom{n-1}{k}.
\]
\end{trditev}

\begin{proof}
Preštejmo število $k$-elementnih podmnožic $[n]$ na dva načina. Za
število $1$ imamo dve možnosti -- lahko je v podmnožico ali ne.
Sedaj preštejemo načine, na katere lahko izmed preostalih $n-1$
elementov izberemo ustrezno število. V prvem primeru dobimo
$\binom{n-1}{k-1}$ podmnožic, v drugem pa $\binom{n-1}{k}$.
\end{proof}

\begin{posledica}
Binomski koeficienti so ravno števila v
\emph{Pascalovem trikotniku}\index{Pascalov trikotnik}:
\[
\begin{array}{ccccccccccc}
  &   &   &    & 1 &    &   &   &   \\
  &   &   & 1  &   & 1  &   &   &   \\
  &   & 1 &    & 2 &    & 1 &   &   \\
  & 1 &   & 3  &   & 3  &   & 1 &   \\
1 &   & 4 &    & 6 &    & 4 &   & 1
\end{array}
=
\begin{array}{ccccccccccc}
             &              &              &              & \binom{0}{0} &              &              &              &              \\
             &              &              & \binom{1}{0} &              & \binom{1}{1} &              &              &              \\
             &              & \binom{2}{0} &              & \binom{2}{1} &              & \binom{2}{2} &              &              \\
             & \binom{3}{0} &              & \binom{3}{1} &              & \binom{3}{2} &              & \binom{3}{3} &              \\
\binom{4}{0} &              & \binom{4}{1} &              & \binom{4}{2} &              & \binom{4}{3} &              & \binom{4}{4}
\end{array}
\]
\end{posledica}

\begin{izrek}[Binomski]\index{Izrek!Binomski}
Za vse $n \in \N_0$ za $a, b \in K$, kjer je $K$ komutativen
kolobar, velja
\[
(a+b)^n = \sum_{k=0}^n \binom{n}{k} a^k b^{n-k}.
\]
\end{izrek}

\begin{proof}
Uporabimo distributivnost in preštejemo število pojavitev vsakega
monoma.
\end{proof}

\newpage

\subsection{Izbori}

\begin{definicija}
Naj bo $N$ množica z $n$ elementi.

\begin{enumerate}[i)]
\item \emph{Urejen izbor s ponavljanjem}\index{Izbor} je vsaka
urejena $k$-terica elementov $N$.
\item \emph{Urejen izbor brez ponavljanja} je vsaka $k$-terica
paroma različnih elementov $N$.
\item \emph{Neurejen izbor brez ponavljanja} je vsaka podmnožica
$k$ elementov množice $N$.
\item \emph{Neurejen izbor s ponavljanjem} je vsaka multimnožica
s $k$ elementi iz $N$.
\end{enumerate}
\end{definicija}

\begin{opomba}
Urejenim izborom pravimo tudi \emph{variacije}, neurejenim pa
\emph{kombinacije}.
\end{opomba}

\begin{trditev}
Število neurejenih izborov s ponavljanjem je enako
\[
\binom{n+k-1}{k}.
\]
\end{trditev}

\begin{proof}
Uporabimo strategijo pik in pregrad. Naj $x_k$ označuje število
pojavitev elementa $k$ v multimnožici.\footnote{Definicija v
naslednjem razdelku.} Tedaj lahko vsak izbor predstavimo kot
\[
\underbrace{\bullet \bullet \dots \bullet}_{x_1} \mid
\underbrace{\bullet \bullet \dots \bullet}_{x_2} \mid
\dots \mid
\underbrace{\bullet \bullet \dots \bullet}_{x_n}.
\]
Vsak izbor natanko ustreza enemu zapisu s pikami in pregradami, teh
pa je ravno $\binom{n+k-1}{k}$. Izmed $n+k-1$ elementov moramo
namreč izbrati $k$ pik, preostalih $n-1$ mest pa zasedejo pregrade.
\end{proof}

\newpage

\subsection{Permutacije in permutacije s ponavljanjem}

\begin{definicija}
\emph{Permutacija}\index{Permutacija} množice $A$ je bijekcija
$f \colon A \to A$. Označimo
\[
S_A = \setb{f \colon A \to A}{\text{$f$ je bijekcija}},
\]
v posebnem primeru
\[
S_n = S_{[n]}.
\]
\end{definicija}

\datum{2022-2-28}

\begin{definicija}
\emph{Multimnožica}\index{Multimnožica} je urejen par $(U, \mu)$,
kjer je $U$ množica in $\mu \colon U \to \N_0$.
\end{definicija}

\begin{definicija}
\emph{Permutacija multimnožice}\index{Multimnožica!Permutacija}
$M = (U, \mu)$ je vsako zaporedje elementov $(x_1, \dots, x_n)$,
kjer so vsi $x_i \in U$ in se vsak element $x \in U$ v zaporedju
pojavi $\mu(x)$ krat.
\end{definicija}

\begin{trditev}
Število permutacij multimnožice
$\set{1^{\alpha_1}, \dots, k^{\alpha_k}}$, kjer je
$\alpha_1 + \dots + \alpha_k = n$, je natanko
\[
\frac{n!}{\alpha_1! \cdots \alpha_k!}.
\]
\end{trditev}

\begin{proof}
Velja
\[
\prod_{i=1}^{n}
\binom{n - \alpha_1 - \dots - \alpha_{i-1}}{\alpha_i} =
\frac{n!}{\alpha_1! \cdots \alpha_k!}. \qedhere
\]
\end{proof}

\begin{opomba}
Zgornjemu številu pravimo
\emph{multinomski koeficient}\index{Multinomski koeficient} in ga
označimo z
\[
\frac{n!}{\alpha_1! \dots \alpha_k!} =
\binom{n}{\alpha_1, \dots, \alpha_k}.
\]
\end{opomba}

\begin{izrek}[Multinomski]\index{Izrek!Multinomski}
Za $n \geq 0$ velja
\[
\left(\sum_{i=1}^k x_i\right)^n =
\sum_{\alpha_1+\dots+\alpha_k = n}
\binom{n}{\alpha_1, \dots, \alpha_k} \prod_{i=1}^k x_i^{\alpha_i}.
\]
\end{izrek}

\begin{proof}
Enak kot pri binomskem.
\end{proof}

\newpage

\subsection{Kompozicije in razčlenitve}

\begin{definicija}
\emph{Kompozicija}\index{Kompozicija} naravnega števila $n$ je
zaporedje $\lambda = (\lambda_1, \dots, \lambda_l)$, kjer so
$\lambda_i \in \N$ in velja
\[
\sum_{i=1}^l \lambda_i = n.
\]
\end{definicija}

\begin{trditev}
Število kompozicij števila $n$ je $2^{n-1}$. Število kompozicij
dolžine $k$ je $\binom{n-1}{k-1}$.
\end{trditev}

\begin{proof}
Pike in pregrade.
\end{proof}

\begin{definicija}
\emph{Šibka kompozicija}\index{Kompozicija!Šibka} naravnega števila
$n$ je kompozicija, v kateri dovolimo, da so nekateri členi enaki
$0$.
\end{definicija}

\begin{trditev}
Število šibkih kompozicij števila $n$ dolžine $k$ je
$\binom{n+k-1}{k-1}$.
\end{trditev}

\begin{proof}
Pike in pregrade.
\end{proof}

\begin{definicija}
\emph{Razčlenitev}\index{Razčlenitev} števila $n \in \N$ je vsaka
padajoča kompozicija števila $n$. Število razčlenitev dolžine $k$
označimo z $p_k(n)$, število razčlenitev dolžine največ $k$ pa z
$\overline{p}_k(n)$.
\end{definicija}

\begin{trditev}
Veljajo naslednje zveze:

\begin{enumerate}[i)]
\item $p_k(n) = \overline{p}_k(n-k)$
\item
$\overline{p}_k(n) = \overline{p}_{k-1}(n) + \overline{p}_k(n-k)$
\item $p_k(n) = p_{k-1}(n-1) + p_k(n-k)$
\end{enumerate}
\end{trditev}

\obvs

\newpage

\subsection{Stirlingova števila}

\begin{definicija}
Za $1 \leq k \leq n$ je
\emph{Stirlingovo število 1.~vrste}\index{Stirlingovo število 1.~vrste}
$c(n,k)$ število permutacij $S_n$, ki jih lahko zapišemo kot
produkt $k$ ciklov. Posebej naj bo $c(n,0) = 0$ za $n \geq 1$ in
$c(0,0) = 1$.
\end{definicija}

\datum{2022-3-7}

\begin{trditev}
Velja rekurzivna formula
\[
c(n, k) = c(n-1, k-1) + (n-1) \cdot c(n-1, k).
\]
\end{trditev}

\begin{proof}
Permutacij, v katerih je $x$ fiksna točka, je $c(n-1, k-1)$,
drugače pa na $n-1$ načinov izberemo element, v katerega se
preslika $x$, nato pa jih razporedimo v $k$ ciklov.
\end{proof}

\begin{izrek}
Velja
\[
x^{\overline{n}} = \sum_k c(n, k) x^k.
\]
\end{izrek}

\begin{proof}
Indukcija po $n$.
\end{proof}

\begin{definicija}
Družina nepraznih množic $\set{X_i}_{i \in I}$ je
\emph{razdelitev}\index{Množica!Razdelitev}, če za vse različne
$i, j \in I$ velja $X_i \cap X_j = \emptyset$ in
\[
\bigcup_{i \in I} X_i = X.
\]
Množicam $X_i$ pravimo \emph{blok razdelitve}
\end{definicija}

\begin{definicija}
Za $1 \leq k \leq n$ je
\emph{Stirlingovo število 2.~vrste}\index{Stirlingovo število 2.~vrste}
$S(n, k)$ število razdelitev $n$-množice v $k$ blokov.
\end{definicija}

\begin{trditev}
Velja rekurzivna zveza
\[
S(n,k) = S(n-1, k-1) + k \cdot S(n-1, k).
\]
\end{trditev}

\begin{proof}
Če je element $x$ v nekem bloku sam, imamo $S(n-1, k-1)$ možnosti,
drugače pa na $k$ načinov izberemo blok, v katerega ga dodamo.
\end{proof}

\begin{izrek}
Velja
\[
x^n = \sum_k S(n, k) \cdot x^{\underline{k}}.
\]
\end{izrek}

\begin{proof}
Dovolj je izrek dokazati za naravna števila. Naj bo $A = [x]^n$.
Velja $\abs{A} = x^n$. Po drugi strani pa lahko $A$ razdelimo glede
na število različnih elementov $k$, teh pa je
$S(n,k) \cdot x^{\underline{k}}$.
\end{proof}

\begin{trditev}
Število surjekcij $f \in K^N$ je
\[
k! \cdot S(n, k).
\]
\end{trditev}

\obvs

\begin{definicija}
\emph{Bellovo število}\index{Bellovo število} $B(n)$ je število
vseh razdelitev množice z $n$ elementi v bloke, oziroma
\[
B(n) = \sum_k S(n, k).
\]
\end{definicija}

\begin{trditev}
Velja rekurzivna formula
\[
B(n+1) = \sum_k \binom{n}{k} B(k).
\]
\end{trditev}

\begin{proof}
Naj bo $x$ v bloku velikosti $k+1$. Tedaj je razdelitev natanko
\[
\binom{n}{k} \cdot B(n-k). \qedhere
\]
\end{proof}

\begin{definicija}
Za $1 \leq k \leq n$ je \emph{Lahova števila}\index{Lahovo število}
$L(n, k)$ število razdelitev $n$-množice v $k$ linearno urejenih
blokov.
\end{definicija}

\begin{trditev}
Velja rekurzivna zveza
\[
L(n, k) = L(n-1, k-1) + (n+k-1) \cdot L(n-1, k).
\]
\end{trditev}

\begin{proof}
Podobno kot prejšnji. % cba
\end{proof}

\begin{izrek}
Velja
\[
L(n, k) =
\frac{(n-1)!}{(k-1)!} \binom{n}{k} =
\frac{n!}{k!} \binom{n-1}{k-1}.
\]
\end{izrek}

\begin{proof}
Elemente uredimo na $n!$ načinov, nato pa izberemo $k-1$ elementov,
ki so prvi v svojem bloku. Pri tem smo vsako razdelitev šteli
$k!$-krat.
\end{proof}

\begin{izrek}
Velja
\[
x^{\overline{n}} = \sum_k L(n, k) x^{\underline{k}}.
\]
\end{izrek}

\newpage

\subsection{Dvanajstera pot}

\datum{2022-3-14}

\begin{izrek}
Tabela al neki idk, idgaf, this is bs
\begin{table}[!ht]
\centering
\caption{Kill me pls}
\begin{tabular}{|c|c|c|c|}
\hline 
$f \colon N \to K$ & Poljubna & Injektivna & Surjektivna \\ 
\hline 
Ločimo elemente $N$ in $K$ & $k^n$ & $k^{\underline{n}}$ & $k! \cdot S(n, k)$ \\ 
\hline 
Ločimo elemnete $K$ & $\binom{n+k-1}{n}$ & $\binom{k}{n}$ & $\binom{n-1}{k-1}$ \\ 
\hline 
Ločimo elemente $N$ & $\displaystyle\sum_{i=1}^k S(n, i)$ & $\begin{cases} 1, & n \leq k \\ 0, & n > k \end{cases}$ & $S(n, k)$ \\ 
\hline 
Ne ločimo elementov & $\overline{p}_k(n)$ & $\begin{cases} 1, & n \leq k \\ 0, & n > k \end{cases}$ & $p_k(n)$ \\ 
\hline 
\end{tabular} 
\end{table}
\end{izrek}

\obvs

\newpage

\subsection{Načelo vključitev in izključitev}

\begin{izrek}[Načelo vključitev in izključitev]
\index{Izrek!Načelo vključitev in izključitev}
Velja
\[
\abs{\bigcup_{i=1}^n A_i} =
\sum_{\substack{X \in \mathcal{P}([n]) \\ X \ne \emptyset}}
(-1)^{\abs{X} + 1} \cdot \abs{\bigcap_{i \in X} A_i}.
\]
\end{izrek}

\begin{proof}
Vsak element štejemo natanko
\[
\sum_{i=1}^k (-1)^{i+1} \binom{k}{i} = 1
\]
krat.
\end{proof}

\begin{izrek}
Velja
\[
\varphi(n) = n \cdot \prod_{p \mid n} \left(1 - \frac{1}{p}\right).
\]
\end{izrek}

\obvs

%\begin{posledica}
%Velja
%\[
%S(n, k) =
%\frac{1}{k!} \cdot \sum_{i=0}^k (-1)^i \binom{k}{i} (k-i)^n.
%\]
%\end{posledica}
