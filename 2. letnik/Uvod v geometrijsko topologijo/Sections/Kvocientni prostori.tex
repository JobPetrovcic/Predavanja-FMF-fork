\section{Kvocientni prostori}

\subsection{Kvocientna topologija}

\datum{2022-2-15}

\begin{definicija}
Naj bo $X$ množica. Relacija $\sim$ na $X$ je
\emph{ekvivalenčna}\index{Ekvivalenčna relacija}, če je refleksivna,
simetrična in tranzitivna.
\end{definicija}

\begin{definicija}
\emph{Kvocientna množica}\index{Kvocientna množica} je množica vseh
ekvivalenčnih razredov relacije $\sim$, oziroma
\[
\kvoc{X}{\sim} = \setb{[x]}{x \in X}.
\]
\end{definicija}

\begin{definicija}
\emph{Kvocientna projekcija}\index{Kvocientna projekcija} je
preslikava $q \colon x \mapsto [x]$.
\end{definicija}

\begin{definicija}
Naj bo $X$ topološki prostor z ekvivalenčno relacijo $\sim$.
\emph{Kvocientna topologija}\index{Topologija!Kvocientna}
$\tau_\sim$ je najmočnejša topologija na $\kvoc{X}{\sim}$, za
katero je kvocientna projekcija zvezna, oziroma
\[
\tau_\sim = \setb{V \subseteq \kvoc{X}{\sim}}{q^{-1}(V) \in \tau}.
\]
\end{definicija}

\begin{opomba}
Odprtost in zaprtost sta invariantni za $q^{-1}$.
\end{opomba}

\begin{opomba}
Kvocientna projekcija ni nujni odprta/zaprta.
\end{opomba}

\begin{definicija}
Naj bo $X$ topološki prostor in $q$ kvocientna projekcija. Za
množico $A$ definiramo \emph{nasičenje}\index{Nasičenje} kot
\[
q^{-1}(q(A)) \subseteq X.
\]
\end{definicija}

\begin{trditev}
Pri zgornjih oznakah je $q(A)$ odprta\footnote{Enako velja za
zaprtost.} natanko tedaj, ko je njeno nasičenje odprto. $q$ je
odprta natanko tedaj, ko je nasičenje vsake odprte množice odprto.
\end{trditev}

\obvs

%\begin{figure}[H]
%\[
%\begin{tikzcd}[column sep=large, row sep=large]
%X \arrow[r, "f"] \arrow[d, "q"'] & Y \\
%\kvoc{X}{\sim} \arrow[ur, dashrightarrow, "\approx"'] &
%\end{tikzcd}
%\]
%\caption{To je bilo na tabli, nekoč v prihodnosti bo najbrž
%pomembno.}
%\end{figure}
