\section{Kvocientni prostori}

\subsection{Kvocientna topologija}

\datum{2022-2-15}

\begin{definicija}
Naj bo $X$ množica. Relacija $\sim$ na $X$ je
\emph{ekvivalenčna}\index{Ekvivalenčna relacija}, če je refleksivna,
simetrična in tranzitivna.
\end{definicija}

\begin{definicija}
\emph{Kvocientna množica}\index{Kvocientna množica} je množica vseh
ekvivalenčnih razredov relacije $\sim$, oziroma
\[
\kvoc{X}{\sim} = \setb{[x]}{x \in X}.
\]
\end{definicija}

\begin{definicija}
\emph{Kvocientna projekcija}\index{Kvocientna projekcija} je
preslikava $q \colon x \mapsto [x]$.
\end{definicija}

\begin{definicija}
Naj bo $X$ topološki prostor z ekvivalenčno relacijo $\sim$.
\emph{Kvocientna topologija}\index{Topologija!Kvocientna}
$\tau_\sim$ je najmočnejša topologija na $\kvoc{X}{\sim}$, za
katero je kvocientna projekcija zvezna, oziroma
\[
\tau_\sim = \setb{V \subseteq \kvoc{X}{\sim}}{q^{-1}(V) \in \tau}.
\]
\end{definicija}

\begin{opomba}
Odprtost in zaprtost sta invariantni za $q^{-1}$.
\end{opomba}

\begin{opomba}
Kvocientna projekcija ni nujni odprta/zaprta.
\end{opomba}

\begin{definicija}
Naj bo $X$ topološki prostor in $q$ kvocientna projekcija. Za
množico $A$ definiramo \emph{nasičenje}\index{Nasičenje} kot
\[
q^{-1}(q(A)) \subseteq X.
\]
\end{definicija}

\begin{trditev}
Pri zgornjih oznakah je $q(A)$ odprta\footnote{Enako velja za
zaprtost.} natanko tedaj, ko je njeno nasičenje odprto. $q$ je
odprta natanko tedaj, ko je nasičenje vsake odprte množice odprto.
\end{trditev}

\obvs

\newpage

\subsection{Kvocientne preslikave}

\datum{2022-2-18}

\begin{definicija}
Naj bo $f \colon X \to Y$ preslikava. Preslikavi
$\overline{f} \colon \kvoc{X}{\sim} \to Y$, ki deluje po predpisu
$\overline{f}([x]) = f(x)$, pravimo
\emph{inducirana preslikava}\index{Preslikava!Inducirana}.
\end{definicija}

\begin{skica}{Inducirana preslikava.}
\[
\begin{tikzcd}[column sep=large, row sep=large]
X \arrow[r, "f"] \arrow[d, "q"']                      & Y \\
\kvoc{X}{\sim} \arrow[ur, dashrightarrow, "\overline{f}"'] &
\end{tikzcd}
\]
\end{skica}

\begin{trditev}
Naj bo $f \colon X \to Y$ preslikava, konstantna na ekvivalenčnih
razredih.\footnote{
$\forall x, y \in X \colon x \sim y \implies f(x) = f(y)$.}

\begin{enumerate}[i)]
\item $f$ določa inducirano preslikavo.
\item Če je $f$ zvezna, je tudi $\overline{f}$ zvezna.
\item $\overline{f}$ je surjektivna natanko tedaj, ko je $f$
surjektivna.
\item $\overline{f}$ je injektivna natanko tedaj, ko $f$ loči
ekvivalenčne razrede.
\end{enumerate}
\end{trditev}

\begin{proof}
Dokažimo drugo trditev. $\overline{f}$ je zvezna natanko tedaj, ko
je za vsako odprto množico $V \subseteq Y$ množica
$\overline{f}^{-1}(V)$ odprta, oziroma
\[
q^{-1}\left(\overline{f}^{-1}(V)\right) \in \tau,
\]
velja pa $q^{-1}\left(\overline{f}^{-1}(V)\right) = f^{-1}(V)$.
\end{proof}

\begin{definicija}
Surjektivna preslikava $f \colon X \to Y$ je
\emph{kvocientna}\index{Preslikava!Kvocientna}, če za vsako
množico $V \subseteq Y$ velja
\[
V \in \tau_Y \iff f^{-1}(V) \in \tau_X.
\]
\end{definicija}

\begin{opomba}
Če je $f$ kvocientna, je njena inducirana preslikava homeomorfizem,
zato se obnaša kot kvocientna projekcija.
\end{opomba}

\begin{opomba}
Če je $f$ surjektivna, je kvocientna natanko tedaj, ko za vsako
množico $V \subseteq Y$ velja
\[
V^{\mathsf{c}} \in \tau_Y \iff f^{-1}(V)^{\mathsf{c}} \in \tau_X.
\]
\end{opomba}

\begin{izrek}
Naj bo $q \colon X \to \kvoc{X}{\sim}$ kvocientna projekcija in
$f \colon X \to Y$ kvocientna preslikava, konstantna na
ekvivalenčnih razredih, ki loči ekvivalenčne razrede. Potem je
$\overline{f} \colon \kvoc{X}{\sim} \to Y$ homeomorfizem.
\end{izrek}

\obvs

\begin{lema}
Naj bo $f \colon X \to Y$ zvezna in surjektivna. Če je $f$ odprta
ali zaprta, je kvocientna.
\end{lema}

\begin{proof}
Naj bo $f$ zaprta. Dokažimo, da je za vsako zaprto množico
$f^{-1}(Z)$ tudi $Z$ zaprta. Ker je $f$ zaprta, je tudi
$f\left(f^{-1}(Z)\right) $ zaprta, velja pa
\[
f\left(f^{-1}(Z)\right) = Z,
\]
saj je $f$ surjektivna.
\end{proof}

\begin{opomba}
Če je $f \colon X \to Y$ zvezna, $X$ kompakten in $Y$ Hausdorffov,
je $f$ zaprta.
\end{opomba}

\begin{trditev}
Naj bosta $f \colon X \to Y$ in $g \colon Y \to Z$ preslikavi.

\begin{enumerate}[i)]
\item Če sta $f$ in $g$ kvocientni, je $g \circ f$ kvocientna.
\item Če je $g \circ f$ kvocientna in sta $f$ in $g$ zvezni, je $g$
kvocientna.
\end{enumerate}
\end{trditev}

\begin{proof}
Če sta $f$ in $g$ kvocientni, je očitno tak tudi njun kompozitum.

Če je $g \circ f$ kvocientna, je $g$ surjektivna. Velja pa
\[
g^{-1}(V) \in \tau_Y \implies
f^{-1}\left(g^{-1}(V)\right) \in \tau_X \implies
V \in \tau_Z. \qedhere
\]
\end{proof}

\begin{zgled}
Naj bo $X = S^1 \times S^1$ torus in
$A = S^1 \times \set{1} \cup \set{1} \times S^1$. Tedaj je
$\kvoc{X}{A} \approx S^2$.
\end{zgled}

\begin{proof}
Ker sta na spodnjem diagramu $f$ in $q_1$ kvocientni, je $F$
kvocientna in $\overline{F}$ homeomorfizem.
\[
\begin{tikzcd}[column sep=large, row sep=large]
I^2 \arrow[rr, "f"] \arrow[dd, "q_2"'] \arrow[ddrr, red, "F = q_1 \circ f"] & & X \arrow[dd, "q_1"]\\
\\
\kvoc{I^2}{\partial I^2} \arrow[rr, dashrightarrow, "\overline{F}"'] & & \kvoc{X}{A}
\end{tikzcd} \qedhere
\]
\end{proof}

\begin{opomba}
Če je $h \colon X \to Y$ homeomorfizem, $\sim_X$ in $\sim_Y$ pa
ekvivalenčni relaciji, za kateri velja
$x \sim_X y \iff h(x) \sim_Y h(y)$, velja
\[
\kvoc{X}{\sim_X} \approx \kvoc{Y}{\sim_Y}.
\]
\end{opomba}

\begin{definicija}
Topološka lastnost $L$ je
\emph{deljiva}\index{Topološka lastnost!Deljiva}, če se s
poljubnega topološkega prostora prenese na vsak njegov kvocientni
prostor.
\end{definicija}

\begin{trditev}
Naslednje topološke lastnosti so deljive:

\begin{multicols}{2}
\begin{enumerate}[i)]
\item Trivialnost
\item Diskretnost
\item Separabilnost
\item Povezanost (s potmi)
\item Lokalna povezanost (s potmi)
\item Kompaktnost
\end{enumerate}
\end{multicols}
\end{trditev}

\begin{proof}
Dokažimo 5.~točko -- naj bo $f \colon X \to Y$ kvocientna, kjer je
$X$ lokalno povezan (s potmi). Ekvivalentno so komponente vsake
odprte množice odprte.

Naj bo $V \subseteq Y$ odprta s komponentami $V_\lambda$, torej
\[
V = \bigcap_\lambda  V_\lambda.
\]
Sledi, da je
\[
f^{-1}(V) = \bigcap_\lambda f^{-1}\left( V_\lambda \right)
\]
odprta. Naj bodo $W_\mu$ njene komponente. Ker je
$f(W_\mu) \subseteq V$ povezana, je vsebovana v neki komponenti
$V_\lambda$. Sledi, da je $f^{-1}\left( V_\lambda \right)$ unija
odprtih množic (komponent), zato je odprta.
\end{proof}

\begin{trditev}
Naslednje topološke lastnosti niso deljive:

\begin{multicols}{2}
\begin{enumerate}[i)]
\item Separacijske lastnosti
\item $1$ in $2$-števnost
\item Lokalna kompaktnost
\item Metrizabilnost
\item Popolna nepovezanost
\end{enumerate}
\end{multicols}
\end{trditev}

\begin{proof}
Dokažimo, da Hausdorffova lastnost ni dedna. Vzemimo
$X = \R \times \set{0, 1}$ in $(x,0) \sim (x,1) \iff x > 0$.
Opazimo, da ne moremo ostro ločiti slik točk $(0,0)$ in $(0,1)$.
Primer deluje tudi za metrizabilnost.

Protiprimer za $1$-števnost je naslednji: Naj bo
$X = \N \times [0,1]$ in $A = \N \times \set{0}$. Tedaj prostor
$\kvoc{X}{A}$ ni $1$-števen -- naj bo $\setb{V_n}{n \in \N}$ števna
lokalna baza točke $q((1,0))$. Sedaj skonstruiramo odprto množico
\[
W = \bigcup_{n \in \N} \left(n \times [0, a_n)\right),
\]
kjer $n \times [0, a_n)$ ni vsebovana v $V_n$. Sledi, da $V_n$ ni
lokalna baza.
\end{proof}

\begin{izrek}[Aleksandrov]\index{Izrek!Aleksandrov}
Če je $X$ poljuben metrični kompakt, obstaja zvezna surjekcija
$f \colon C \to X$.\footnote{$C$ označuje \emph{Cantorjevo
množico}.}
\end{izrek}

\begin{opomba}
Z $X = [0, 1]$ dobimo protiprimer za deljivost popolne
nepovezanosti. $f$ je namreč kvocientna, saj slika iz kompakta v
Hausdorffov prostor.
\end{opomba}

\begin{trditev}
Naj bo $R$ razdelitev prostora $X$. Tedaj velja
\[
\kvoc{X}{R} \in T_1 \iff \text{Elementi $R$ so zaprti v $X$}.
\]
\end{trditev}

\obvs

\newpage

\subsection{Topološke grupe}

\begin{definicija}
\emph{Topološka grupa}\index{Topološka grupa} $G$ je grupa, ki je
hkrati topološki prostor z zveznim množenjem in invertiranjem.
\end{definicija}

\datum{2022-3-4}

\begin{definicija}
Direktni produkt topoloških grup je direktni produkt grup s
produktno topologijo.
\end{definicija}

\begin{definicija}
Naj bo $G$ topološka grupa in $a \in G$.
\emph{Leva translacija}\index{Topološka grupa!Translacija} je
preslikava
\[
L_a \colon g \mapsto ag.
\]
Simetrično definiramo \emph{desno translacijo}.
\end{definicija}

\begin{trditev}
Translacije so homeomorfizmi.
\end{trditev}

\begin{proof}
Translaciji sta očitno bijektivni. Opazimo, da sta zvezni, saj sta
definirani z množenjem. Velja $L_a^{-1} = L_{a^{-1}}$, zato je tudi
inverz zvezen.
\end{proof}

\begin{definicija}
Topološki prostor $X$ je
\emph{homogen}\index{Topološki prostor!Homogen}, če za vsaka dva
elementa $a, b \in X$ obstaja tak homeomorfizem $h \colon X \to X$,
da je $h(a) = b$.
\end{definicija}

\begin{posledica}
Topološka grupa $G$ je homogen prostor.
\end{posledica}

\begin{definicija}
Naj bo $X$ topološki prostor in $G$ topološka grupa.
\emph{Levo delovanje}\index{Topološka grupa!Delovanje} na prostoru
$X$ je zvezna preslikava $\varphi \colon G \times X \to X$, pri
čemer za vse $a, b \in G$ in $x \in X$ velja

\begin{enumerate}[i)]
\item $1 \cdot x = x$ in
\item $a(bx) = (ab)x$.
\end{enumerate}
\end{definicija}

\begin{opomba}
Za vse $a \in G$ je preslikava $x \mapsto ax$ homeomorfizem.
\end{opomba}

\begin{trditev}
Delovanje grupe $G$ na prostoru $X$ določa ekvivalenčno relacijo
\[
x \sim y \iff \exists g \in G \colon gx = y.
\]
Pripadajoči kvocientni prostor imenujemo
\emph{prostor orbit}\index{Topološki prostor!Prostor orbit} in ga
označimo z
\[
\kvoc{X}{\sim} = \kvoc{X}{G}.
\]
\end{trditev}

\obvs

\begin{definicija}
\emph{Orbita}\index{Topološka grupa!Orbita} točke $x$ je množica
\[
[x] = G \cdot x = \setb{g \cdot x}{g \in G}.
\]
\end{definicija}

\begin{definicija}
Za element $x \in X$ grupi
\[
G_x = \setb{g \in G}{g \cdot x = x}
\]
pravimo
\emph{stabilizatorska podgrupa}\index{Stabilizatorska podgrupa}.
\end{definicija}

\begin{opomba}
Obstaja bijekcija $G \cdot x \to \kvoc{G}{G_x}$.
\end{opomba}

\begin{trditev}
Naj bo $G$ topološka grupa, ki deluje na topološki prostor $X$.
Potem je kvocientna projekcija
\[
q \colon X \to \kvoc{X}{G}
\]
odprta.
\end{trditev}

\begin{proof}
Naj bo $V \in \tau_X$. Dokazati moramo, da je njeno nasičenje
odprto. Velja pa
\begin{align*}
q^{-1}\left(q(V)\right)
&=
\setb{y \in X}{\exists x \in V \colon y \sim x}
\\
&=
\setb{g \cdot x}{x \in V, g \in G}
\\
&=
\bigcup_{g \in G} \setb{g \cdot x}{x \in V}
\\
&= \bigcup_{g \in G} L_g(V). \qedhere
\end{align*}
\end{proof}
