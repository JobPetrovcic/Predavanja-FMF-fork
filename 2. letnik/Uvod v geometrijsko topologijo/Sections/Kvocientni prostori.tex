\section{Kvocientni prostori}

\subsection{Kvocientna topologija}

\datum{2022-2-15}

\begin{definicija}
Naj bo $X$ množica. Relacija $\sim$ na $X$ je
\emph{ekvivalenčna}\index{Ekvivalenčna relacija}, če je refleksivna,
simetrična in tranzitivna.
\end{definicija}

\begin{definicija}
\emph{Kvocientna množica}\index{Kvocientna množica} je množica vseh
ekvivalenčnih razredov relacije $\sim$, oziroma
\[
\kvoc{X}{\sim} = \setb{[x]}{x \in X}.
\]
\end{definicija}

\begin{definicija}
\emph{Kvocientna projekcija}\index{Kvocientna projekcija} je
preslikava $q \colon x \mapsto [x]$.
\end{definicija}

\begin{definicija}
Naj bo $X$ topološki prostor z ekvivalenčno relacijo $\sim$.
\emph{Kvocientna topologija}\index{Topologija!Kvocientna}
$\tau_\sim$ je najmočnejša topologija na $\kvoc{X}{\sim}$, za
katero je kvocientna projekcija zvezna, oziroma
\[
\tau_\sim = \setb{V \subseteq \kvoc{X}{\sim}}{q^{-1}(V) \in \tau}.
\]
\end{definicija}

\begin{opomba}
Odprtost in zaprtost sta invariantni za $q^{-1}$.
\end{opomba}

\begin{opomba}
Kvocientna projekcija ni nujni odprta/zaprta.
\end{opomba}

\begin{definicija}
Naj bo $X$ topološki prostor in $q$ kvocientna projekcija. Za
množico $A$ definiramo \emph{nasičenje}\index{Nasičenje} kot
\[
q^{-1}(q(A)) \subseteq X.
\]
\end{definicija}

\begin{trditev}
Pri zgornjih oznakah je $q(A)$ odprta\footnote{Enako velja za
zaprtost.} natanko tedaj, ko je njeno nasičenje odprto. $q$ je
odprta natanko tedaj, ko je nasičenje vsake odprte množice odprto.
\end{trditev}

\obvs

\newpage

\subsection{Kvocientne preslikave}

\datum{2022-2-18}

\begin{definicija}
Naj bo $f \colon X \to Y$ preslikava. Preslikavi
$\overline{f} \colon \kvoc{X}{\sim} \to Y$, ki deluje po predpisu
$\overline{f}([x]) = f(x)$, pravimo
\emph{inducirana preslikava}\index{Preslikava!Inducirana}.
\end{definicija}

\begin{skica}{Inducirana preslikava.}
\[
\begin{tikzcd}[column sep=large, row sep=large]
X \arrow[r, "f"] \arrow[d, "q"']                      & Y \\
\kvoc{X}{\sim} \arrow[ur, dashrightarrow, "\overline{f}"'] &
\end{tikzcd}
\]
\end{skica}

\begin{trditev}
Naj bo $f \colon X \to Y$ preslikava, konstantna na ekvivalenčnih
razredih.\footnote{
$\forall x, y \in X \colon x \sim y \implies f(x) = f(y)$.}

\begin{enumerate}[i)]
\item $f$ določa inducirano preslikavo.
\item Če je $f$ zvezna, je tudi $\overline{f}$ zvezna.
\item $\overline{f}$ je surjektivna natanko tedaj, ko je $f$
surjektivna.
\item $\overline{f}$ je injektivna natanko tedaj, ko $f$ loči
ekvivalenčne razrede.
\end{enumerate}
\end{trditev}

\begin{proof}
Dokažimo drugo trditev. $\overline{f}$ je zvezna natanko tedaj, ko
je za vsako odprto množico $V \subseteq Y$ množica
$\overline{f}^{-1}(V)$ odprta, oziroma
\[
q^{-1}\left(\overline{f}^{-1}(V)\right) \in \tau,
\]
velja pa $q^{-1}\left(\overline{f}^{-1}(V)\right) = f^{-1}(V)$.
\end{proof}

\begin{definicija}
Surjektivna preslikava $f \colon X \to Y$ je
\emph{kvocientna}\index{Preslikava!Kvocientna}, če za vsako
množico $V \subseteq Y$ velja
\[
V \in \tau_Y \iff f^{-1}(V) \in \tau_X.
\]
\end{definicija}

\begin{opomba}
Če je $f$ kvocientna, je njena inducirana preslikava homeomorfizem,
zato se obnaša kot kvocientna projekcija.
\end{opomba}

\begin{opomba}
Če je $f$ surjektivna, je kvocientna natanko tedaj, ko za vsako
množico $V \subseteq Y$ velja
\[
V^{\mathsf{c}} \in \tau_Y \iff f^{-1}(V)^{\mathsf{c}} \in \tau_X.
\]
\end{opomba}

\begin{izrek}
Naj bo $q \colon X \to \kvoc{X}{\sim}$ kvocientna projekcija in
$f \colon X \to Y$ kvocientnta preslikava, konstantna na
ekvivalenčnih razredih, ki loči ekvivalenčne razrede. Potem je
$\overline{f} \colon \kvoc{X}{\sim} \to Y$ homeomorfizem.
\end{izrek}

\obvs

\begin{lema}
Naj bo $f \colon X \to Y$ zvezna in surjektivna. Če je $f$ odprta
ali zaprta, je kvocientna.
\end{lema}

\begin{proof}
Naj bo $f$ zaprta. Dokažimo, da je za vsako zaprto množico
$f^{-1}(Z)$ tudi $Z$ zaprta. Ker je $f$ zaprta, je tudi
$f\left(f^{-1}(Z)\right) $ zaprta, velja pa
\[
f\left(f^{-1}(Z)\right) = Z,
\]
saj je $f$ surjektivna.
\end{proof}

\begin{opomba}
Če je $f \colon X \to Y$ zvezna, $X$ kompakten in $Y$ Hausdorffov,
je $f$ zaprta.
\end{opomba}

\begin{trditev}
Naj bosta $f \colon X \to Y$ in $g \colon Y \to Z$ preslikavi.

\begin{enumerate}[i)]
\item Če sta $f$ in $g$ kvocientni, je $g \circ f$ kvocientna.
\item Če je $g \circ f$ kvocientna in sta $f$ in $g$ zvezni, je $g$
kvocientna.
\end{enumerate}
\end{trditev}

\begin{proof}
Če sta $f$ in $g$ kvocientni, je očitno tak tudi njun kompozitum.

Če je $g \circ f$ kvocientna, je $g$ surjektivna. Velja pa
\[
g^{-1}(V) \in \tau_Y \implies
f^{-1}\left(g^{-1}(V)\right) \in \tau_X \implies
V \in \tau_Z. \qedhere
\]
\end{proof}
