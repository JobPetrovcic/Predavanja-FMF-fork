\section{Mnogoterosti}

\epigraph{"Če imaš selotejp, lahko vse strgaš."}
{-- Maša Žaucer}

\subsection{Topološke mnogoterosti}

\datum{2022-4-29}

\begin{definicija}
Naj bo $n \in \N_0$.
\emph{Topološka mnogoterost}\index{Mnogoterost} dimenzije $n$ je
Hausdorffov, 2-števen topološki prostor $M$, v katerem ima vsaka
točka v $M$ odprto okolico $V \subseteq M$, homeomorfno $\R^n$ ali
$\R_+^n$.\footnote{Označimo
$\R_+^n = \setb{x \in \R^n}{x_n \geq 0}$.} Okolico $V$ imenujemo
\emph{evklidska okolica}, homeomorfizem med $V$ in $\R^n$ ali
$\R_+^n$ pa \emph{karta}.
\end{definicija}

\begin{opomba}
Ta definicija je dobra po invarianci odprtih množic.
\end{opomba}

\begin{opomba}
Ekvivalentno lahko zahtevamo, da je $V$ homeomorfna neki odprti
množici v $\R_+^n$.
\end{opomba}

\begin{opomba}
Vsaka točka v $M$ ima bazo evklidskih okolic.
\end{opomba}

\begin{definicija}
\emph{Notranjost}\index{Mnogoterost!Notranjost, rob} mnogoterosti
$M$ je množica
\[
\Int M = \setb{x \in M}
{\text{obstaja evklidska okolica $x$, homeomorfna $\R^n$}}.
\]
\emph{Rob} mnogoterosti $M$ je množica
$\partial M = M \setminus \Int M$.
\end{definicija}

\begin{trditev}
Mnogoterosti so metrizabilne.
\end{trditev}

\begin{proof}
Ker je mnogoterost lokalno evklidski prostor, je lokalno kompakten
in 2-števen. Iz lokalne kompaktnosti in Hausdorffove lastnosti
sledi regularnost, ker pa je prostor 2-števen, je metrizabilen.
\end{proof}

\begin{definicija}
Mnogoterost $M$ je \emph{sklenjena}\index{Mnogoterost!Sklenjena},
če je $\partial M = \emptyset$ in je $M$ kompaktna.
\end{definicija}

\begin{trditev}
Naj bo $n \in \N_0$ in $\F \in \set{\R, \C, \HH}$. Projektivni
prostor $\proj{\F}{n}$ je sklenjena mnogoterost dimenzije $dn$,
kjer je $d = \dim_{\R} \F$.
\end{trditev}

\begin{proof}
Naj bo $q \colon \F^{n+1} \setminus \set{0} \to \proj{\F}{n}$
kvocientna projekcija. Ker je odprta, je $\proj{\F}{n}$ 2-števen
prostor. Množice
\[
V_i =
\setb{(x_0, \dots, x_n) \in \F^{n+1} \setminus \set{0}}{x_i \ne 0}
\]
tvorijo pokritje prostora, zato je $q(V_i)$ odprto pokritje za
$\proj{\F}{n}$. Dovolj je tako pokazati, da je
$q(V_i) \approx \F^n$, oziroma, da je $q(V_0) \approx \F^n$.

Naj bo $f \colon V_0 \to \F^n$ preslikava, podana z
\[
f(x_0, x_1, \dots, x_n) = x_0^{-1} \cdot (x_1, \dots, x_n).
\]
Ker je $\eval{q}{V_0}{}$ kvocientna, $f$ pa je konstantna na
ekvivalenčnih razredih, dobimo inducirano zvezno preslikavo
$\overline{f}$ iz $q(V_0)$ v $\F^n$. Dovolj je tako preveriti, da
ima $\overline{f}$ zvezen inverz, kar je res, saj elementu $\F^n$
dodamo na prvo mesto $1$ in ga slikamo s $q$. Sledi, da je
$\proj{\F}{n}$ lokalno evklidski prostor brez roba.

Preveriti je treba še lastnost $T_2$. Če je $a \in V_i$ za vse $i$,
jo lahko od vsake točke ločimo znotraj neke množice $V_j$. Ker je
$\proj{\F}{n}$ homogen, pa lahko poljubno točko slikamo v tak $a$,
zato lahko ločimo vse točke.
\end{proof}

\newpage

\subsection{Konstrukcije mnogoterosti}

\begin{izrek}
Naj bosta $M$ in $N$ $n$-mnogoterosti, $V \subseteq \Int M$ odprta
množica in $f \colon V \to N$ zvezna in injektivna preslikava.
Potem je $f$ odprta in je $f(V) \subseteq \Int N$.
\end{izrek}

\begin{proof}
Naj bo $W \subseteq V$ poljubna odprta množica. Naj bo $y \in f(W)$
in $x = f^{-1}(y)$. Ker je $y \in N$, ima evklidsko okolico $U$.
Točka $x$ ima zato evklidsko okolico
\[
U' \subseteq f^{-1}(U) \cap W \subseteq V,
\]
ki je homeomorfna $\R^n$.

Po Brouwerju je kompozicija $f$ s homeomorfizmi odprta preslikava,
zato je njena slika odprta v $\R^n$. Točka $y$ ima torej okolico,
homeomorfno $\R^n$, zato je notranja točka v $N$.
\end{proof}

\begin{izrek}
Naj bo $M$ $n$-mnogoterost z nepraznim robom. Potem je njen rob
$\partial M$ $(n-1)$-mnogoterost s praznim robom. Če je $M$
kompaktna, je $\partial M$ sklenjena.
\end{izrek}

\begin{proof}
Naj bo $x \in \partial M$. Sledi, da obstaja homeomorfizem $h$,
ki okolico $U$ točke $x$ preslika v $\R_+^n$. Ker pa je
$\eval{h}{U \cap \Int M}{}$ zvezna in injektivna, je odprta, zato
je
\[
h(U \cap \Int M) \subseteq \Int \R_+^n =
\R^{n-1} \times (0, \infty).
\]
Podobno sklepamo za inverzno preslikavo. Sledi, da je
\[
h(U \cap \partial M) = \R^{n-1} \times \set{0},
\]
zato je $\partial M$ res $(n-1)$-mnogoterost brez roba. Če je $M$
kompaktna, je tudi $\partial M$ kompaktna, in zato sklenjena.
\end{proof}

\begin{opomba}
Če je $M$ povezana, je tudi $M \setminus \partial M$ povezana.
\end{opomba}

\begin{izrek}
Rob ima globalni obrobek -- obstaja vložitev
$\varphi \colon \partial M \times [0, 1] \to M$, za katero je
$\varphi(x, 0) = x$.
\end{izrek}

\datum{2022-5-6}

\begin{izrek}
Naj bo $M$ $m$-mnogoterost in $N$ $n$-mnogoterost. Tedaj je
$M \times N$ $(m+n)$-mnogoterost z notranjostjo
\[
\Int (M \times N) = \Int M \times \Int N
\]
in robom
\[
\partial (M \times N) =
\partial M \times N \cup M \times \partial N.
\]
\end{izrek}

\begin{proof}
Lastnosti $T_2$ in 2-števnost sta multiplikativni, zato nam teh ni
treba preverjati.

Naj bosta $x \in M$ in $y \in N$ notranji točki z evklidskima
okolicama $U_x$ in $U_y$. Tedaj je $U_x \times U_y$ evklidska
okolica točke $(x, y)$. Če je ena izmed $x$ in $y$ robna, je
$U_x \times U_y \approx \R_+^{n+m}$. Če sta obe robni, pa velja
\[
U_x \times U_y \approx
\R^{n+m-2} \times [0, \infty) \times [0, \infty) \approx
\R^{n+m-2} \times \R_+^2 \approx
\R_+^{n+m}. \qedhere
\]
\end{proof}

\begin{definicija}
Naj bo $L$ $\ell$-mnogoterost in $N$ $n$-mnogoterost, pri čemer
velja $L \subseteq N$ in $\ell \leq n$.

\begin{enumerate}[i)]
\item Mnogoterost $L$ je
\emph{prav vložena}\index{Mnogoterost!Prav vložena} v $N$, če velja
\[
L \cap \partial N = \partial L.
\]
\item Mnogoterost $L$ je
\emph{lokalno ploščata}\index{Mnogoterost!Lokalno ploščata} ali
\emph{podmnogoterost} v $N$, če je prav vložena in za vsak
$x \in L$ obstajata evklidska okolica $V$ za $x$ v $N$ in
homeomorfizem $h$ med $V$ in $\R^n$ ali $\R_+^n$, za katerega velja
\[
h(V \cap L) =
h(V) \cap \left(\set{0}^{n - \ell} \times \R^\ell\right).
\]
\end{enumerate}
\end{definicija}

\begin{izrek}
Naj bosta $N_1$ in $N_2$ $n$-mnogoterosti ter
$L_1 \subseteq \partial N_1$ in $L_2 \subseteq \partial N_2$ zaprti
$(n-1)$-mnogoterosti z lokalno ploščatima robovoma,
$h \colon L_1 \to L_2$ pa homeomorfizem. Tedaj je zlepek
$N_1 \cup_h N_2$ $n$-mnogoterost z robom
\[
\partial (N_1 \cup_h N_2) =
(\partial N_1 \setminus \Int L_1)
\cup_{\eval{h}{\partial L_1}{}}
(\partial N_2 \setminus \Int L_2).
\]
\end{izrek}

\begin{proof}
Podoben kot trditev~\ref{td:zlp}.
\end{proof}

\begin{lema}
Naj bosta $x$ in $y$ točki v $\Int B^n$. Poljuben homeomorfizem
$\varphi \colon S^{n-1} \to S^{n-1}$ lahko razširimo do
homeomorfizma $\Phi \colon B^n \to B^n$, za katerega velja
$\Phi(x) = y$.
\end{lema}

\begin{izrek}
Naj bo $M$ povezana $n$-mnogoterost. Potem za poljubna
$x, y \in \Int M$ obstaja homeomorfizem $h \colon M \to M$, za
katerega je $h(x) = y$.
\end{izrek}

\begin{proof}
Definiramo ekvivalenčno relacijo $\sim$ tako, da je $x \sim y$
natanko tedaj, ko obstaja tak homeomorfizem. Ker je $\Int M$
odprta in povezana, je dovolj pokazati, da so ekvivalenčni razredi
odprti.

Naj bo $x \in \Int M$ poljuben in naj bo $U \approx \R^n$
njegova evklidska okolica, $h$ pa homeomorfizem. Naj bo
$V = h^{-1}(\Int B^n)$ in $y \in V$ poljubna. Po zgornji lemi
obstaja homeomorfizem $\Phi \colon h^{-1}(B^n) \to h^{-1}(B^n)$, ki
je na robu enak $\id$ in preslika $x$ v $y$. Če ga na $M$ razširimo
z identiteto, dobimo iskan homeomorfizem.
\end{proof}

\begin{definicija}
Naj bosta $M$ in $N$ $n$-mnogoterosti ter $D \subseteq \Int M$ in
$E \subseteq \Int N$ množici, homeomorfni $B^n$. Naj bo
$h \colon \partial D \to \partial E$ homeomorfizem lokalno
ploščatih sfer. Zlepku
\[
M \mathbin{\#} N = (M \setminus \Int D) \cup_h (N \setminus \Int E)
\]
pravimo \emph{povezana vsota}\index{Mnogoterost!Povezana vsota}
mnogoterosti $M$ in $N$.
\end{definicija}

\begin{opomba}
Tudi $M \mathbin{\#} N$ je $n$-mnogoterost.
\end{opomba}

\newpage

\subsection{Kompaktne ploskve}

\datum{2022-5-13}

\begin{definicija}
Naj bo $T = S^1 \times S^1$ torus in $P = \proj{\R}{2}$ projektivna
ravnina. Induktivno definiramo $1T = T$, $1P = P$, za $n \geq 2$ pa
\[
nT = (n-1)T \mathbin{\#} T
\quad \text{in} \quad
nP = (n-1)P \mathbin{\#} P.
\]
Posebej označimo $0T = S^2$.
\end{definicija}

\begin{trditev}
Naj bo
\[
K = \coprod_{i=1}^n K_i
\]
disjunktna unija mnogokotnikov v ravnini in naj bo $\sim$
ekvivalenčna relacija na $K$, ki identificira nekatere pare
mnogokotnikov z linearnimi homeomorfizmi. Tedaj je $\kvoc{K}{\sim}$
disjunktna unija kompaktnih ploskev. Pri tem so robne točke natanko
tiste točke s stranic, ki niso identificirane z nobeno drugo.
\end{trditev}

\begin{proof}
Vemo že, da je $\kvoc{K}{\sim}$ 2-števen in Hausdorffov, saj je
zlepek takih prostorov. Dovolj je tako preveriti obstoj evklidskih
okolic.

Oglejmo si poljuben $x \in \Int K_i$. Ta točka ima evklidsko
okolico $U$ v $K$, njena slika $q(U)$ pa je okolica za $[x]$. Velja
pa $q(U) \approx U$, saj so ekvivalenčni razredi točk enojci.
Podobno velja za točke na stranicah, ki jih ne identificiramo.

Če je $x$ točka na stranici, ki se identificira z neko drugo, se v
$\kvoc{K}{\sim}$ okolici originalov po robu zlepita v odprto
evklidsko množico.

Za oglišča dobimo zaporedje identifikacij, inducirano iz
identifikacij stranic. Če je to zaporedje ciklično, dobimo $\R^2$,
sicer pa $\R_+^2$.
\end{proof}

\begin{opomba}
Tako ploskev imenujemo \emph{poliedrska ploskev}.
\end{opomba}

\begin{izrek}[Radó]
Vsaka sklenjena kompaktna ploskev je homeomorfna neki poliedrski.
\end{izrek}

\begin{definicija}
Naj bo
\[
M = \kvoc{\coprod_{i=1}^n K_i}{\sim}
\]
poliedrska ploskev, predstavljena kot kvocient disjunktne unije
mnogokotnikov, v katerih identificiramo nekatere pare stranic.
Ploskev $M$ je \emph{orientabilna}\index{Mnogoterost!Orientabilna},
če lahko mnogokotnike $K_i$ orientiramo tako, da so orientirani
glede na vsak par identificiranih stranic.
\end{definicija}

\begin{izrek}
Orientabilnost je topološka lastnost.
\end{izrek}

\begin{trditev}
Naj bo $M$ poliedrska ploskev in $N$ njena poliedrska podploskev.
Če je $M$ orientabilna, je orientabilna tudi $N$.
\end{trditev}

\begin{proof}
Izberemo enako orientacijo.
\end{proof}

\begin{posledica}
Ploskev je neorientabilna natanko tedaj, ko vsebuje Möbiusov trak.
\end{posledica}

\begin{proof}
Če je ploskev orientabilna, so tudi podploskve orientabilne, torej
ne vsebuje Möbiusovega traku. Če ploskev ni orientabilna,
identificiramo stranice, dokler še lahko izbiramo skladno
orientacijo. Ko je ne moremo, dobimo Möbiusov trak tako, da
izberemo poljuben trak, ki povezuje neskladni stranici.
\end{proof}

\begin{trditev}
Naj bosta $M$ in $N$ orientabilni ploskvi. Tedaj je tudi
$M \mathbin{\#} N$ orientabilna.
\end{trditev}

\begin{proof}
Ker sta orientabilni $M$ in $N$, sta taki tudi $M \setminus \Int D$
in $N \setminus \Int E$, orientaciji pa določata orientaciji
$\partial D$ in $\partial E$. Če se ti ujemata, je
$M \mathbin{\#} N$ orientabilna, sicer pa lahko preprosto obrnemo
orientacijo ene izmed njih.
\end{proof}

\begin{definicija}
Naj bo
\[
M = \kvoc{\coprod_{i=1}^n K_i}{\sim}
\]
poliedrska ploskev, predstavljena kot kvocient disjunktne unije
mnogokotnikov, v katerih identificiramo nekatere pare stranic.
2-celice ploskve $M$ so slike odprtih mnogokotnikov s kvocientno
projekcijo, 1-celice slike stranic, 0-celice pa slike oglišč. Z
$N_i$ označimo število $i$-celic ploskve $M$.

\emph{Eulerjeva karakteristika}\index{Mnogoterost!Eulerjeva karakteristika}
ploskve $M$ je število
\[
\chi(M) = N_2 - N_1 + N_0.
\]
\end{definicija}

\begin{trditev}
Eulerjeva karakteristika ploskve je topološka invarianta.
\end{trditev}

\begin{trditev}
Ploskve $nT$ za $n \in \N_0$ ali $mP$ za $m \in \N$ med seboj niso
homeomorfne.
\end{trditev}

\begin{proof}
Ploskve $mP$ niso orientabilne, $nT$ pa so. Velja še
\[
\chi(nT) = 2 - 2n
\quad \text{in} \quad
\chi(mP) = 2 - m. \qedhere
\]
\end{proof}

\datum{2022-5-27}

\begin{trditev}
Naj bo $M$ poliedrska ploskev in $A, B \subseteq M$ uniji celic, za
kateri velja $M = A \cup B$. Potem je
\[
\chi(M) = \chi(A) + \chi(B) - \chi(A \cap B).
\]
\end{trditev}

\begin{trditev}
Število $\chi(S^2)$ je neodvisno od poliedrske strukture in je
enako $2$.
\end{trditev}

\begin{proof}
Indukcija po številu povezav.
\end{proof}

\begin{izrek}
Naj bo $M$ poljubna sklenjena ploskev. Tedaj je $M$ homeomorfna
natanko eni izmed ploskev $nt$ za $n \in \N_0$ ali $mP$ za
$m \in \N$.
\end{izrek}

\begin{proof}
Naj bo $M$ poljubna sklenjena poliedrska ploskev. Brez škode za
splošnost jo lahko predstavimo kot kvocient pravilnega
mnogokotnika, saj je povezana. Predpostavimo še, da se stranice
zlepijo z linearnimi homeomorfizmi.

Okoli vsakega oglišča si izberemo izsek majhnega diska in ga
odstranimo. Dobimo kompaktno ploskev z robom $N$, ki sestoji iz več
krožnic. Z dodajanjem diskov dobimo nazaj $M$.

Ploskev $N$ lahko predstavimo kot disk s prilepljenimi trakovi, ki
jih imenujemo ročaji. Te ločimo na orientabilne in neorientabilne.
Ločimo dva primera:

\begin{enumerate}[i)]
\item Vsi ročaji so orientabilni. Vidimo, da lahko poljuben
prepleten par ročajev najprej izoliramo, vsak tak prepleten par pa
nam da en torus. Preostale ročaje lahko ignoriramo, saj se
">prilepijo"< na disk.
\item Obstaja neorientabilen ročaj. Vidimo, da ga lahko izoliramo.
Dobimo torej izolirane neorientabilne ročaje, pare prepletenih
orientabilnih ročajev in nekaj neprepletenih orientabilnih ročajev.
Prepleten par lahko spremenimo v dva neorientabilna. \qedhere
\end{enumerate}
\end{proof}

\begin{opomba}
Velja
\[
P \# T \approx 3 P.
\]
\end{opomba}
