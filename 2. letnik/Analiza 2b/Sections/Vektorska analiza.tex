\section{Vektorska analiza}

\subsection{Skalarna in vektorska polja}

\begin{definicija}
Naj bo $D \subseteq \R^3$ odprta. Funkcijam oblike
$\mathcal{U} \colon D \to \R$ pravimo
\emph{skalarno polje}\index{Polje}. Preslikavam
$\vv{R} \colon D \to \R^3$ oblike pravimo \emph{vektorsko polje}.
\end{definicija}

\begin{definicija}
\emph{Standardna baza}\index{Prostor!Standardna baza} je množica
\[
\set{\vv{i}, \vv{j}, \vv{k}} = \set{\vv{e_1}, \vv{e_2}, \vv{e_3}}.
\]
\end{definicija}

\begin{definicija}
Pravimo, da je baza $\set{\vv{p}, \vv{q}, \vv{r}}$
\emph{pozitivno orientirana}\index{Prostor!Orientacija baze}, če je
\[
[\vv{p}, \vv{q}, \vv{r}] > 0.
\]
Če je mešani produkt negativen, pravimo, da je baza
\emph{negativno orientirana}.
\end{definicija}

\begin{opomba}
Standardna baza je pozitivno orientirana.
\end{opomba}

\begin{opomba}
Baza je pozitivno orientirana natanko tedaj, ko je
\[
\vv{p} \times \vv{q} = \vv{r}.
\]
\end{opomba}

\datum{2022-3-2}

\begin{definicija}
\emph{Smerni odvod}\index{Polje!Smerni odvod} skalarnega polja
$\mathcal{U}$ v smeri vektorja $\vv{s}$ v točki $p$ je limita
\[
\lim_{t \to 0}
\frac{\mathcal{U}(\vv{p} + t \vv{s}) - \mathcal{U}(\vv{p})}{t} =
\frac{\partial \mathcal{U}}{\partial \vv{s}}(\vv{p}),
\]
če obstaja.
\end{definicija}

\begin{opomba}
Če je $\mathcal{U} \in \mathcal{C}^1(D)$, velja
\[
\frac{\partial \mathcal{U}}{\partial \vv{s}}(\vv{p}) =
(D\mathcal{U})(\vv{p}) \cdot \vv{s} =
\grad \mathcal{U} \cdot \vv{s}.
\]
\end{opomba}

\begin{definicija}
Operator \emph{nabla}\index{Polje!Nabla} je operator
\[
\vv{\nabla} = \left(
\frac{\partial}{\partial x},
\frac{\partial}{\partial y},
\frac{\partial}{\partial z}
\right).
\]
\end{definicija}

\begin{trditev}
Naj bo $\mathcal{U} \in \mathcal{C}^1(D)$. V točki $\vv{p} \in D$
skalarno polje najhitreje narašča v smeri gradienta, najhitreje pa
pada v nasprotni smeri.
\end{trditev}

\obvs

\begin{opomba}
V smereh, pravokotnih na gradient, se $\mathcal{U}$ najpočasneje
spreminja.
\end{opomba}

\begin{definicija}
Naj bo $\vv{R}$ vektorsko polje.
\emph{Divergenca}\index{Polje!Divergenca} polja je sled odvoda,
oziroma
\[
\dv \vv{R} = X_x + Y_y + Z_z = \vv{\nabla} \cdot \vv{R}.
\]
\end{definicija}

\begin{definicija}
Naj bo $\vv{R}$ vektorsko polje. \emph{Rotor}\index{Polje!Rotor}
polja je produkt\footnote{Abuse of notation, nablo uporabimo po
komponentah.}
\[
\rot \vv{R} = \vv{\nabla} \times \vv{R}.
\]
\end{definicija}

\begin{trditev}
Naj bo $D$ odprta podmnožica $\R^3$,
$\mathcal{U} \in \mathcal{C}^2(D)$ skalarno in
$\vv{R} \in \mathcal{C}^2(D)$ vektorsko polje. Tedaj velja

\begin{enumerate}[i)]
\item $\rot(\grad \mathcal{U}) =
\vv{\nabla} \times \vv{\nabla} \mathcal{U} = \vv{0}$ in
\item $\dv(\rot \mathcal{U}) =
\vv{\nabla} \cdot \left(\vv{\nabla} \times \vv{R}\right) = 0$.
\end{enumerate}
\end{trditev}

\begin{proof}
Velja
\[
\vv{\nabla} \times \vv{\nabla} \cdot \mathcal{U} =
(u_{zy} - u_{yz}, u_{xz} - u_{zx}, u_{yx} - u_{xy}) =
\vv{0}
\]
in
\[
\vv{\nabla} \cdot \left(\vv{\nabla} \times \vv{R}\right) =
Z_{yx} - Y_{zx} + X_{zy} - Z_{xy} + Y_{xz} - X_{yz} =
0. \qedhere
\]
\end{proof}

\begin{definicija}
Vektorsko polje je \emph{potencialno}\index{Polje!Potencialno}, če
obstaja tako skalarno polje $\mathcal{U} \in \mathcal{C}^1(D)$, da
je $\vv{R} = \grad U$. Polju $\mathcal{U}$ pravimo
\emph{potencial}.
\end{definicija}

\begin{definicija}
Naj bo $\mathcal{U}$ skalarno polje.
\emph{Laplaceov operator}\index{Polje!Laplaceov operator} je
\[
\Delta \mathcal{U} =
\dv \grad \mathcal{U} =
\mathcal{U}_{xx} + \mathcal{U}_{yy} + \mathcal{U}_{zz}.
\]
\end{definicija}

\begin{definicija}
Funkcijam, ki rešijo enačbo
\[
\Delta \mathcal{U} = 0,
\]
pravimo \emph{harmonične funkcije}.
\end{definicija}

\begin{definicija}
Množica $D \subseteq \R^3$ je
\emph{konveksna}\index{Množica!Konveksna}, če za poljubni točki
$\vv{a}, \vv{b} \in D$ in $t \in [0,1]$ tudi
\[
t \vv{a} + (1-t) \vv{b} \in D.
\]
\end{definicija}

\begin{definicija}
Množica $D \subseteq \R^3$ je
\emph{zvezdasta}\index{Množica!Zvezdasta}, če obstaja taka točka
$\vv{a} \in D$, da je za vse $\vv{b} \in D$ in $t \in [0,1]$ tudi
\[
t \vv{a} + (1-t) \vv{b} \in D.
\]
\end{definicija}
