\section{Laplaceova transformacija}

\epigraph{">Ta teden bom končal skripto."<}{-- Luka Horjak}

\subsection{Definicija}

\begin{definicija}
Naj bo $f \colon [0, \infty) \to \C$ odsekoma zvezna funkcija.
Funkciji
\[
\mathscr{L}(f)(z) = F(z) = \int_0^\infty e^{-zt} f(t)\,dt
\]
pravimo
\emph{Laplaceova transformiranka}\index{Laplaceova transformacija}
funkije $f$.
\end{definicija}

\begin{definicija}
Preslikava $f \colon [0, \infty) \to \C$ je funkcija
\emph{eksponentnega naraščanja}, če obstajata taka $M \geq 0$ in
$k \in \R$, da je
\[
\abs{f(t)} \leq M e^{kt}.
\]
\end{definicija}

\begin{trditev}
Če za funkcijo $f$ velja
\[
\abs{f(t)} \leq M e^{kt},
\]
njena Laplaceova transformiranka obstaja za vse $z \in \C$, za
katere je $\Re z > k$.
\end{trditev}

\begin{proof}
Naj bo $\varepsilon > 0$ in $\Re z \geq k + \varepsilon$. Tedaj je
\[
\abs{\int_0^\infty e^{-zt} f(t)\,dt} \leq
\int_0^\infty e^{-t \cdot \Re z} M e^{kt}\,dt \leq
M \cdot \int_0^\infty e^{-\varepsilon t}\,dt. \qedhere
\]
\end{proof}

\begin{opomba}
Opazimo, da ta integral konvergira enakomerno na
$\Re z \geq k + \varepsilon$, zato je $F$ zvezna na $\Re z > k$.
\end{opomba}

\begin{trditev}
Če za funkcijo $f$ velja
\[
\abs{f(t)} \leq M e^{kt},
\]
je $F$ holomorfna za $\Re z > k$.
\end{trditev}

\begin{proof}
Integral
\[
-\int_0^\infty e^{-zt} t f(t)\,dt
\]
konvergira enakomerno.
\end{proof}

\begin{trditev}
Če Laplaceova transformiranka obstaja za nek $z_0 \in \C$, obstaja
tudi za vse $z$, za katere je $\Re z > \Re z_0$.
\end{trditev}

\begin{proof}
Dovolj je trditev dokazati za zvezne funkcije. Naj bo
\[
\Phi(t) = \int_0^t e^{-z_0s} f(s)\,ds.
\]
Tedaj je
\[
\Phi'(t) = e^{-z_0t} f(t)
\]
in
\[
\mathscr{L}(f)(z_0) = \lim_{t \to \infty} \Phi(t).
\]
Funkcija $\Phi$ je torej omejena na $[0, \infty)$.

Naj bo $\Re z > \Re z_0$. Tedaj je
\begin{align*}
\int_0^t e^{-zs} f(s)\,ds &=
\int_0^t e^{-(z-z_0)s} \Phi'(s)\,ds
\\
&=
\eval{e^{-(z-z_0)s} \Phi(s)}{0}{t} +
(z-z_0) \int_0^t \Phi(s) e^{-(z-z_0)s}\,ds
\\
&=
e^{-(z-z_0)t} \Phi(t) + (z-z_0) \int_0^t \Phi(s) e^{-(z-z_0)s}\,ds.
\end{align*}
Sedaj preprosto vzamemo limito, ki obstaja.
\end{proof}

\begin{posledica}
Velja
\[
\mathscr{L}(f) \in \mathcal{O}(\setb{z \in \C}{\Re z > \sigma(f)}).
\]
\end{posledica}

%PAZI
\begin{definicija}
\emph{Abscisa konvergence}\index{Laplaceova transformacija!Abscisa konvergence}
je število
\[
\sigma(f) = \inf \setb{\Re z}{\text{$\mathscr{L}(f)(z)$ obstaja}}.
\]
\end{definicija}

\newpage

\subsection{Lastnosti}

\begin{trditev}
Za Laplaceovo transformacijo veljajo naslednje lastnosti:

\begin{enumerate}[i)]
\item Transformacija je linearna.
\item Velja
\[
\mathscr{L}(e^{\alpha t} f(t))(z) = \mathscr{L}(f)(z - \alpha).
\]
\item Za $k > 0$ in $\Re z > \sigma(f)$ je
\[
\mathscr{L}(f(t-k))(z) = e^{-kz} \mathscr{L}(f)(z).
\]
\item Za $k > 0$ je
\[
\mathscr{L}(f(kt))(z) =
\frac{1}{k} \mathscr{L}(f)\left(\frac{z}{k}\right).
\]
\item Za vsak $n \in \N$ in $\Re z > \sigma(f)$ je
\[
\mathscr{L}\left(f^{(n)}\right)(z) =
(-1)^n \mathscr{L}(t^n f(t))(z).
\]
\item Naj bo $f$ $n$-krat zvezno odvedljiva na $[0, \infty)$ in naj
transformiranke $f, f', f'', \dots, f^{(n)}$ obstajajo za
$\Re z > k$. Tedaj za $\Re z > k$ velja
\[
\mathscr{L}\left(f^{(n)}\right)(z) =
z^n \mathscr{L}(f)(z) - \sum_{i=0}^{n-1} f^{(i)}(0) z^{n-1-i}.
\]
\end{enumerate}
\end{trditev}

\begin{proof}
Dokažimo zadnjo točko. Velja
\[
\int_0^\infty f'(t) e^{-zt}\,dt =
\eval{f(t) e^{-zt}}{0}{\infty} + z \int_0^\infty f(t) e^{-zt}\,dt =
-f(0) + z \mathscr{L}(f)(z). \qedhere
\]
\end{proof}

\begin{definicija}
Naj bosta $f, g \colon [0, \infty) \to \C$ funkciji.
\emph{Konvolucija}\index{Laplaceova transformacija!Konvolucija}
funkcij $f$ in $g$ je funkcija
\[
(f*g)(t) = \int_0^t f(s) g(t-s)\,ds.
\]
\end{definicija}

\begin{trditev}
Naj bosta $f$ in $g$ eksponentnega naraščanja, torej
\[
\abs{f(t)}, \abs{g(t)} \leq M e^{kt}.
\]
Tedaj velja
\[
\mathscr{L}(f*g)(z) = \mathscr{L}(f)(z) \cdot \mathscr{L}(g)(z).
\]
\end{trditev}

\begin{izrek}
Naj bo $f \colon [0, \infty) \to \C$ zvezna in odsekoma zvezno
odvedljiva. Naj $\mathscr{L}(f)$ obstaja za $\Re z > k$ in naj za
nek $x > k$ obstaja integral
\[
\int_0^\infty e^{-xt} \abs{f(t)}\,dt.
\]
Tedaj za vsak $t > 0$ velja
\[
f(t) = \lim_{R \to \infty} \frac{1}{2 \pi i}
\int_{x - iR}^{x + iR} e^{zt} \mathscr{L}(f)(z)\,dz.
\]
\end{izrek}
