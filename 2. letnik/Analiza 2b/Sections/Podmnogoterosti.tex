\section{Podmnogoterosti v \titleRn{n}}

\subsection{Definicija}

\datum{2022-2-14}

\begin{definicija}
Neprazna podmnožica $M \subseteq \R^{n+m}$ je
\emph{gladka podmnogoterost}\index{Podmnogoterost} dimenzije $n$ in
kodimenzije $m$, če za vsako točko $a \in M$ obstaja odprta okolica
$U$ točke $a$ v $\R^{n+m}$ in take funkcije
\[
F_1, \dots, F_m \in \mathcal{C}^1(U),
\]
da ima preslikava $F = (F_1, \dots, F_m)$ rang $m$ na $U$ in velja
\[
M \cap U = \setb{x \in U}{F(x) = 0}.
\]
\end{definicija}

\begin{opomba}
Preslikavi $F$ pravimo
\emph{definicijska funkcija}\index{Podmnogoterost!Definicijska funkcija}.
\end{opomba}

\begin{opomba}
Dovolj je že, da ima $F$ rang $m$ v točki $a$.
\end{opomba}

\begin{trditev}
Neprazna podmnožica $M \subseteq \R^{n+m}$ je podmnogoterost
dimenzije $n$ natanko tedaj, ko za vsako točko $a \in M$ obstaja
njena odprta okolica $U$ v $\R^{n+m}$ in taka permutacija $\sigma$
njenih koordinat, da je $M \cap U$ graf neke $\mathcal{C}^1$
preslikave $\varphi \colon D \to \R^m$, kjer je $D \subseteq \R^n$
odprta, oziroma
\[
M \cap U = \setb{x_{\sigma(1)}, \dots, x_{\sigma(n)},
\varphi\left(x_{\sigma(n+1)}, \dots, x_{\sigma(n+m)}\right)}
{\left(x_{\sigma(1)}, \dots, x_{\sigma(n)}\right) \in D}.
\]
\end{trditev}

\begin{proof}
Če je $M$ podmnogoterost, preprosto uporabimo izrek o implicitni
preslikavi. Če velja predpostavka iz trditve, pa definiramo
\[
F_j(x_1, \dots, x_{n+m}) = x_{n+j} - \varphi_j(x_1, \dots, x_n).
\]
Sledi, da velja
\[
F = 0 \iff (x_1, \dots, x_{n+m}) =
(x_1, \dots, x_n, \varphi(x_1, \dots, x_n))
\]
in
\[
\frac{\partial F_i}{x_{n+j}} = \delta_{i,j},
\]
zato je $\rang DF = m$.
\end{proof}

\begin{trditev}
Naj bo $M \subseteq \R^{n+m}$ podmnogoterost dimenzije $n$. Potem
za vsak $a \in M$ obstaja okolica $U$ točke $a$ v $\R^{n+m}$ in
preslikava $\Phi \in \mathcal{C}^1(D)$ ranga $n$, kjer je
$D \subseteq \R^n$ odprta, da je $\Phi(D) = M \cap U$.
\end{trditev}

\begin{proof}
Vzamemo
\[
\Phi = (x_1, \dots, x_n, \varphi(x_1, \dots, x_n)). \qedhere
\]
\end{proof}

\begin{trditev}
Naj bo $\Phi \colon D \to \R^{n+m}$ $\mathcal{C}^1$ preslikava
ranga $n$. Potem za vsak $t_0 \in D$ obstaja njegova okolica $V$,
da je $\Phi(V)$ podmnogoterost dimenzije $n$ v $\R^{n+m}$.
\end{trditev}

\datum{2022-2-16}

\begin{proof}
Obstaja $n \times n$ poddeterminanta $D\Phi(t_0)$, različna od $0$.
Sledi, da lahko s permutacijo koordinat zapišemo
$\Phi = (\Phi_1, \Phi_2)$, kjer je $\Phi_1 \colon V \to \Phi(V)$
difeomorfizem. Sedaj si oglejmo preslikavo
\[
(x_1, \dots, x_n) \in \Phi(V) \mapsto \Phi \circ \Phi_1^{-1}.
\]
Opazimo, da za $\varphi = \Phi_2 \circ \Phi_1^{-1}$ velja
\[
\Phi(V) =
\left(\Phi \circ \Phi_1^{-1}\right)(x_1, \dots, x_n) =
\setb{x_1, \dots, x_n, \varphi\left(x_1, \dots, x_n\right)}
{(x_1, \dots, x_n) \in \Phi(V)}. \qedhere
\]
\end{proof}

\begin{trditev}
Naj bo $M \subseteq \R^{n+m}$ podmnogoterost dimenzije $n$. Potem
za vsako točko $a \in M$ obstaja njena okolica
$U \subseteq \R^{n+m}$ in difeomorfizem $\Phi \colon U \to V$, za
katera je\footnote{$M$ lahko ">izravnamo"<.}
\[
\Phi(U \cap M) = V \cap \left(\R^n \times \set{0}^m\right).
\]
\end{trditev}

\begin{proof}
Lokalno je $M$ graf nad enem izmed $n$ dimenzionalnih podprostorov.
Za
\[
\Phi(x,y) = (x, y - \varphi(x))
\]
je tako
\[
\Phi^{-1}(x,z) = (x, z + \varphi(x)),
\]
ki je diferennciabilna. Sledi, da je
\[
\Phi(M \cap U) = D \times \set{0}^m. \qedhere
\]
\end{proof}

\newpage

\subsection{Tangentni prostor}

\begin{definicija}
Naj bo $M \subseteq \R^n$ podmnogoterost in $a \in M$.
\emph{Tangentni prostor}\index{Podmnogoterost!Tangentni prostor} na
$M$ v točki $a$ je množica
\[
T_aM = \setb{\dot{\gamma}(t_0)}
{\gamma \colon (\alpha, \beta) \to M \land
\gamma \in \mathcal{C}^1
\land t_0 \in (\alpha, \beta) \land
\gamma(t_0) = a}.
\]
\end{definicija}

\begin{trditev}
Naj bo $M \subseteq \R^{n+m}$ $n$-dimenzionalna podmnogoterost in
$a \in M$. Potem je $T_aM$ $n$-dimenzionalni vektorski podprostor
prostora $\R^{n+m}$.
\end{trditev}

\begin{proof}
Obstaja okolica $U$ točke $a$, za katero je $M \cap U$ graf nad
enim izmed $n$-dimen\-zionalnih koordinatnih podprostorov. Vsaka
krivulja na $M \cap U$ je oblike
\[
t \mapsto (x(t), \varphi(x(t))).
\]
Naj bo pri tem $x(t_0) = x_0$ in $a = (x_0, \varphi(x_0))$.

Odvod zgornje preslikave je enak
\[
\begin{bmatrix}
\dot{x}(t) \\
D\varphi(x(t)) \cdot \dot{x}(t)
\end{bmatrix},
\]
kar je v $a$ enako
\[
\begin{bmatrix}
I \\
D\varphi(x_0)
\end{bmatrix}
\cdot \dot{x}(t_0).
\]
Ker pa je $\dot{x}(t_0)$ poljuben\footnote{Vzamemo
$t \mapsto x_0 + t \cdot v$.} vektor v $\R^n$, je tangentni prostor
kar
\[
\Im
\begin{bmatrix}
I \\
D\varphi(x_0)
\end{bmatrix}. \qedhere
\]
\end{proof}

\begin{opomba}
Običajno si tangentni prostor predstavljamo kot afin podprostor
$a + T_aM$.
\end{opomba}

\begin{posledica}
Naj bo $M \subseteq \R^{n+m}$ podmnogoterost, v okolici $U$ točke
$a \in M$ podana z definicijskimi funkcijami $F_i$. Tedaj je
\[
T_aM = \ker(DF)(a).
\]
\end{posledica}

\begin{proof}
Lokalno je $M \cap U$ graf oblike
\[
M \cap U = \setb{(x, \varphi(x))}
{x \in D \subseteq \R^n \land \varphi \colon D \to \R^m}.
\]
Vemo, da je
\[
T_aM = \Im
\begin{bmatrix}
I \\
D\varphi(x_0)
\end{bmatrix}
\]
in
\[
F(x, \varphi(x)) = 0
\]
na $D$. Sedaj z odvajanjem dobimo
\[
F_x \cdot I + F_y \cdot D\varphi = 0,
\]
oziroma
\[
(DF)(a) \cdot
\begin{bmatrix}
I \\
D\varphi
\end{bmatrix},
\]
torej
\[
T_aM \leq \ker(DF)(a).
\]
Ker sta dimenziji enaki, sta to enaka podprostora.
\end{proof}

\begin{opomba}
Gradienti definicijskih funkcij so pravokotni na $T_aM$.
\end{opomba}

\begin{posledica}
Naj bo $\Phi \colon D \to \R^{n+m}$, kjer je $D \subseteq \R^n$,
$\mathcal{C}^1$ preslikava ranga $n$. Naj bo $t_0 \in D$ in
$M \subseteq \R^{n+m}$ podmnogoterost, za katera je
\[
M \cap U = \Phi(V),
\]
kjer je $V$ okolica $t_0$ in $U$ okolica $\Phi(t_0)$. Tedaj je
\[
T_{\Phi(t_0)}M = \Im(D\Phi)(t_0).
\]
\end{posledica}

\begin{proof}
Lokalno v oklici $a = \Phi(t_0)$ je $M \cap U = F^{-1}(\set{0})$,
kjer je $\rang(DF)(a) = m$. Velja, da je
\[
T_aM = \ker(DF)(a)
\quad \text{in} \quad
F(\Phi(t)) = 0.
\]
Sledi, da je
\[
DF(a) \cdot D\Phi(t_0) = 0,
\]
oziroma
\[
\Im(D\Phi)(t_0) \leq \ker(DF)(a).
\]
S primerjanjem dimenzij vidimo, da sta podprostora enaka.
\end{proof}

\datum{2022-2-17}

\begin{opomba}
Podmnogoterostim, katerim dodamo robne točke, pravimo
\emph{mnogoterosti z robom}.
\end{opomba}

\newpage

\subsection{Krivulje v \titleRn{3}}

\begin{definicija}
\emph{Krivulja}\index{Krivulja} je enodimenzionalna podmnogoterost.
\end{definicija}

\begin{trditev}
Povezane krivulje lahko parametriziramo globalno.
\end{trditev}

\begin{opomba}
Vsaka krivulja ima neskončno mnogo regularnih parametrizacij. Velja
\[
\vv{\rho}' = \dot{\vv{r}}(h) \cdot h'.
\]
\end{opomba}

\begin{definicija}
Naj bo $\Gamma$ krivulja v $\R^3$ in
$\vv{r} \colon [\alpha, \beta] \to \Gamma$ njena regularna
parametrizacija. Naj bo $D$ delitev intervala $[\alpha, \beta]$.
Naj bo
\[
\ell(D) =
\sum_{i=1}^n d(\vv{r}(t_{i-1}),\vv{r}(t_i)).
\]
\emph{Dolžina krivulje}\index{Krivulja!Dolžina} je limita
\[
\lim_{\max \Delta t \to 0} \ell(D).
\]
\end{definicija}

\begin{trditev}
Naj bosta $\vv{r}\colon[a,b]\to\R^3$ in
$\vv{\rho}\colon[\alpha,\beta]\to\R^3$ regularni parametrizaciji
poti $\Gamma$. Potem je
\[
\int_a^b \norm{\dot{\vv{r}}(t)}\;dt =
\int_\alpha^\beta \norm{\dot{\vv{\rho}}(t)}\;dt
\]
\end{trditev}

\begin{proof}
Uporabimo izrek o vpeljavi nove spremenljivke.
\end{proof}

\begin{trditev}
Naj bo $\vv{r} \in \mathcal{C}^1([\alpha, \beta])$. Tedaj je
dolžina krivulje enaka
\[
\int_\alpha^\beta \sqrt{\dot{x}^2 + \dot{y}^2 + \dot{z}^2}\;dt =
\int_\alpha^\beta \norm{\dot{\vv{r}}(t)}\;dt.
\]
\end{trditev}

\begin{proof}
Dokaz je enak dokazu izreka 5.4.6.\ v zapiskih Analize 1 prvega
letnika.
\end{proof}

\begin{definicija}
Naj bo $\vv{r} \colon [\alpha, \beta] \to \Gamma$ regularna
parametrizacija krivulje $\Gamma$. Naj bo
\[
S(t) = \int_\alpha^t \norm{\dot{\vv{r}}(\tau)}\;d\tau.
\]
Za inverzno preslikavo\footnote{Ta obstaja, saj je $S' > 0$.}
$T=S^{-1}$ parametrizacijo
\[
s \mapsto \vv{r}(T(s))
\]
imenujemo
\emph{naravna parametrizacija}\index{Krivulja!Naravna parametrizacija}.
\end{definicija}

\begin{trditev}
Odvod naravne parametrizacije je normiran.
\end{trditev}

\begin{proof}
Velja
\[
\frac{d}{ds} \left(\vv{r}(T(s))\right) =
\dot{\vv{r}}(T(s)) \cdot T'(s) =
\frac{\dot{\vv{r}}(T(s))}{\dot{S}(T(s))} =
\frac{\dot{\vv{r}}(T(s))}{\norm{\dot{\vv{r}}(T(s))}}. \qedhere
\]
\end{proof}

\begin{definicija}
\emph{Normalna ravnina}\index{Krivulja!Normalna, pritisnjena ravnina}
krivulje $\vv{r}$ v točki $t$ je ravnina skozi $\vv{r}(t)$ in
normalo $\dot{\vv{r}}(t)$.
\end{definicija}

\begin{definicija}
Naj bo $\vv{r}$ naravna $\mathcal{C}^2$ parametrizacija.
\emph{Spremljajoči trieder} v točki $t$ so vektorji\footnote{To ni
nujno dobra definicija.}
\[
\vv{T} = \dot{\vv{r}}(t),
\quad
\vv{N} = \frac{\vv{T}'}{\norm{\vv{T}'}}
\quad \text{in} \quad
\vv{B} = \vv{T} \times \vv{N}.
\]
\end{definicija}

\begin{opomba}
Vektorju $\vv{N}$ pravimo \emph{vektor glavne normale}, vektorju
$\vv{B}$ pa vektor \emph{binormale}.
\end{opomba}

\begin{definicija}
\emph{Pritisnjena ravnina} v točki $s$ je ravnina, ki jo razpenjata
tangentni in normalni vektor ter gre skozi $\vv{r}(s)$.
\end{definicija}

\begin{trditev}
Pritisnjena ravnina v točki $s$ je ravnina, ki se najbolje prilega
krivulji v okolici $\vv{r}(s)$.
\end{trditev}

\begin{proof}
Velja
\[
\vv{n} \cdot (\vv{r}(s+h)-\vv{r}(s)) =
\vv{n} \cdot \vv{r}'(s) \cdot h +
\vv{n} \cdot \vv{r}''(s) \cdot \frac{h^2}{2} +
\vv{n} \cdot \vv{o}(h^3).
\]
Ta izraz bo najmanjši, ko bo $\vv{n} \parallel \vv{B}(s)$.
\end{proof}

\begin{trditev}
Za regularno $\mathcal{C}^2$ parametrizacijo $\vv{r}$ krivulje
$\Gamma$ velja
\[
\vv{T} = \frac{\dot{\vv{r}}(t)}{\norm{\dot{\vv{r}}(t)}},
\quad \text{in} \quad
\vv{N} =
\frac{\dot{\vv{r}} \times
\left(\ddot{\vv{r}} \times \dot{\vv{r}}\right)}
{\norm{\dot{\vv{r}}} \cdot
\norm{\ddot{\vv{r}} \times \dot{\vv{r}}}}.
\]
\end{trditev}

\begin{proof}
Uporabimo verižno pravilo.
\end{proof}

\newpage

\subsection{Ukrivljenost krivulj}

\datum{2022-2-21}

\begin{definicija}
\emph{Fleksijska ukrivljenost}\index{Krivulja!Fleksijska, torzijska ukrivljenost}
je definirana kot
\[
\kappa(s) = \norm{\vv{r}''(s)} = \norm{\vv{T}'(s)}.
\]
\end{definicija}

\begin{opomba}
Velja
\[
\kappa = \frac{\norm{\dot{\vv{r}} \times \ddot{\vv{r}}}}{\norm{\dot{\vv{r}}}^3}.
\]
\end{opomba}

\begin{definicija}
\emph{Pritisnjena krožnica}\index{Krivulja!Pritisnjena krožnica} je
krožnica z radijem
\[
\rho(s) = \frac{1}{\kappa(s)}
\]
in središčem v točki
\[
\vv{r}(s) + \rho(s) \cdot \vv{N}(s),
\]
ki leži v pritisnjeni ravnini.
\end{definicija}

\begin{opomba}
Pritisnjena krožnica je parametrizirana s
\[
\varphi \mapsto \vv{r} + \rho \cdot \vv{N} + \rho \cdot
\left(\vv{T} \cdot \cos \varphi + \vv{N} \cdot \sin \varphi\right).
\]
\end{opomba}

\begin{definicija}
\emph{Torzijska ukrivljenost}\footnote{Tudi \emph{zvitost}.}
krivulje $\vv{r} \in \mathcal{C}^3$ je definirana kot
\[
\omega(s) = \norm{\vv{B}'(s)}.
\]
\end{definicija}

\begin{opomba}
Velja
\[
\omega =
\frac{\left[\dot{\vv{r}},\ddot{\vv{r}},\dddot{\vv{r}}\right]}
{\norm{\dot{\vv{r}} \times \ddot{\vv{r}}}^2}.
\]
\end{opomba}

\begin{trditev}
Velja
\[
\vv{N}' = - \kappa \cdot \vv{T} + \omega \cdot \vv{B}.
\]
\end{trditev}

\begin{proof}
Opazimo, da velja $\vv{N}' \perp \vv{N}$, saj je $\vv{N} = 1$.
Sedaj preprosto odvajamo zvezi
\[
\vv{N} \cdot \vv{T} = 0
\quad \text{in} \quad
\vv{N} \cdot \vv{B} = 0. \qedhere
\]
\end{proof}

\begin{izrek}[Frenet-Serretov sistem]
\index{Izrek!Frenet-Serretov sistem}
Velja
\[
\begin{bmatrix}
\vv{T} \\
\vv{N} \\
\vv{B}
\end{bmatrix}' =
\begin{bmatrix}
0       & \kappa  & 0      \\
-\kappa & 0       & \omega \\
0       & -\omega & 0
\end{bmatrix}
\cdot
\begin{bmatrix}
\vv{T} \\
\vv{N} \\
\vv{B}
\end{bmatrix}.
\]
\end{izrek}
