\section{Podmnogoterosti v \titleRn{n}}

\subsection{Definicija}

\datum{2022-2-14}

\begin{definicija}
Neprazna podmnožica $M \subseteq \R^{n+m}$ je
\emph{gladka podmnogoterost}\index{Podmnogoterost} dimenzije $n$ in
kodimenzije $m$, če za vsako točko $a \in M$ obstaja odprta okolica
$U$ točke $a$ v $\R^{n+m}$ in take funkcije
\[
F_1, \dots, F_m \in \mathcal{C}^1(U),
\]
da ima preslikava $F = (F_1, \dots, F_m)$ rang $m$ na $U$ in velja
\[
M \cap U = \setb{x \in U}{F(x) = 0}.
\]
\end{definicija}

\begin{opomba}
Preslikavi $F$ pravimo
\emph{definicijska funkcija}\index{Podmnogoterost!Definicijska funkcija}.
\end{opomba}

\begin{opomba}
Dovolj je že, da ima $F$ rang $m$ v točki $a$.
\end{opomba}

\begin{trditev}
Neprazna podmnožica $M \subseteq \R^{n+m}$ je podmnogoterost
dimenzije $n$ natanko tedaj, ko za vsako točko $a \in M$ obstaja
njena odprta okolica $U$ v $\R^{n+m}$ in taka permutacija $\sigma$
njenih koordinat, da je $M \cap U$ graf neke $\mathcal{C}^1$
preslikave $\varphi \colon D \to \R^m$, kjer je $D \subseteq \R^n$
odprta, oziroma
\[
M \cap U = \setb{x_{\sigma(1)}, \dots, x_{\sigma(n)},
\varphi\left(x_{\sigma(n+1)}, \dots, x_{\sigma(n+m)}\right)}
{\left(x_{\sigma(1)}, \dots, x_{\sigma(n)}\right) \in D}.
\]
\end{trditev}

\begin{proof}
Če je $M$ podmnogoterost, preprosto uporabimo izrek o implicitni
preslikavi. Če velja predpostavka iz trditve, pa definiramo
\[
F_j(x_1, \dots, x_{n+m}) = x_{n+j} - \varphi_j(x_1, \dots, x_n).
\]
Sledi, da velja
\[
F = 0 \iff (x_1, \dots, x_{n+m}) =
(x_1, \dots, x_n, \varphi(x_1, \dots, x_n))
\]
in
\[
\frac{\partial F_i}{x_{n+j}} = \delta_{i,j},
\]
zato je $\rang DF = m$.
\end{proof}

\begin{trditev}
Naj bo $M \subseteq \R^{n+m}$ podmnogoterost dimenzije $n$. Potem
za vsak $a \in M$ obstaja okolica $U$ točke $a$ v $\R^{n+m}$ in
preslikava $\Phi \in \mathcal{C}^1(D)$ ranga $n$, kjer je
$D \subseteq \R^n$ odprta, da je $\Phi(D) = M \cap U$.
\end{trditev}

\begin{proof}
Vzamemo
\[
\Phi = (x_1, \dots, x_n, \varphi(x_1, \dots, x_n)). \qedhere
\]
\end{proof}

\begin{trditev}
Naj bo $\Phi \colon D \to \R^{n+m}$ $\mathcal{C}^1$ preslikava
ranga $n$. Potem za vsak $t_0 \in D$ obstaja njegova okolica $V$,
da je $\Phi(V)$ podmnogoterost dimenzije $n$ v $\R^{n+m}$.
\end{trditev}

%\begin{proof}
%Obstaja $n \times n$ poddeterminanta $D\Phi(t_0)$, različna od $0$.
%Sledi, da lahko s permutacijo koordinat zapišemo
%$\Phi = (\Phi_1, \Phi_2)$, kjer je $\Phi_1 \colon V \to \Phi(V)$
%difeomorfizem. Sedaj si oglejmo preslikavo
%\[
%(x_1, \dots, x_n) \in \Phi(V) \mapsto \Phi \circ \Phi_1^{-1}.
%\]
%Opazimo, da za $\varphi = \Phi_2 \circ \Phi_1^{-1}$ velja
%\[
%\Phi(V) =
%\left(\Phi \circ \Phi_1^{-1}\right)(x_1, \dots, x_n) =
%\setb{x_1, \dots, x_n, \varphi\left(x_1, \dots, x_n\right)}
%{(x_1, \dots, x_n) \in \Phi(V)}. \qedhere
%\]
%\end{proof}
%
%\begin{trditev}
%Naj bo $M \subseteq \R^{n+m}$ podmnogoterost dimenzije $n$. Potem
%za vsako točko $a \in M$ obstaja njena okolica
%$U \subseteq \R^{n+m}$ in difeomorfizem $\Phi \colon U \to V$, za
%katera je
%\[
%\Phi(U \cap M) = V \cap \left(\R^n \times \set{0}^m\right).
%\]
%\end{trditev}
%
%\begin{proof}
%Naj bo
%\[
%\Phi(x_1, \dots, x_{n+m}) =
%(x_1, \dots, x_{n+1} - \varphi_1(x_1, \dots), \dots).
%\]
%\end{proof}
