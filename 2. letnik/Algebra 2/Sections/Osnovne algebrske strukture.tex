\section{Osnovne algebrske strukture}

\subsection{Linearne operacije}

\datum{2021-10-1}
\begin{definicija}
\emph{Binarna operacija}\index{Binarna operacija} $*$ na neprazni
množici $S$ je preslikava $*\colon S\times S\to S$. Po dogovoru
namesto $*(x,y)$ pišemo $x*y$.
\end{definicija}

\begin{definicija}
Naj bo $*$ binarna operacija na $S$. Element $e\in S$ je
\emph{nevtralni element}\index{Binarna operacija!Nevtralni element}
ali \emph{enota}, če za vsak $x\in S$ velja
\[
x*e=e*x=x.
\]
\end{definicija}

\begin{definicija}
Naj bo $*$ binarna operacija na $S$. Element $e\in S$ je \emph{levi
nevtralni element}, če za vsak $x\in S$ velja
\[
e*x=x.
\]
Podobno je $e$ \emph{desni nevtralni element}, če za vsak $x\in S$
velja
\[
x*e=x.
\]
\end{definicija}

\begin{trditev}
Veljajo naslednje trditve:

\begin{enumerate}[i)]
\item Če je $e'$ levi in $e''$ desni nevtralni element, je
$e'=e''=e$, kjer je $e$ nevtralni element.
\item Če nevtralni element obstaja, je enolično določen.
\item Levih/desnih nevtralnih elementov je lahko več.
\end{enumerate}
\end{trditev}

\begin{proof}
Za prvo točko preprosto opazimo, da je
\[
e'=e'*e''=e''.
\]
Sledi, da je $e'$ levi in desni nevtralni element, torej je $e'=e$.

Druga točka je direktna posledica prve. Če sta $e$ in $f$ nevtralna
elementa, je namreč $e$ levi, $f$ pa desni nevtralni element, zato
je $e=f$.

Za dokaz tretje trditve si oglejmo operaciji
$*_1,*_2\colon\N\to\N$, ki delujeta s predpisi $x*_1y=x$ in
$x*_2y=y$ za vse naravne $x$ in $y$. Vidimo, da so vsa naravna
števila desni nevtralni element prve in levi nevtralni element
druge operacije.
\end{proof}

\begin{definicija}
Operacija $*$ na $S$ je:
\begin{enumerate}[i)]
\item \emph{asociativna}\index{Binarna operacija!Asociativna}, če
za vse $a,b,c\in S$ velja $a*(b*c)=(a*b)*c$,
\item \emph{komutativna}\index{Binarna operacija!Komutativna}, če
za vse $a,b\in S$ velja $a*b=b*a$.
\end{enumerate}
\end{definicija}

\begin{definicija}
Naj bo $T\subseteq S$ in $*$ operacija na $S$. Množica $T$ je
\emph{zaprta}\index{Binarna operacija!Zaprta množica} za $*$, če za
vse $t_1,t_2\in T$ velja $t_1*t_2\in T$. Pravimo, da je $*$
\emph{notranja}\index{Binarna operacija!Notranja, zunanja}\footnote{
\emph{Zunanja} binarna operacija je preslikava
$*\colon K\times S\to S$.} binarna operacija za $T$.
\end{definicija}

\newpage

\subsection{Polgrupe in monoidi}

\begin{definicija}
\emph{Algebrske strukture}\index{Algebrska struktura} so množice,
opremljene z eno ali več binarnimi operacijami, ki izpolnjujejo
določene aksiome.
\end{definicija}

\begin{definicija}
Množica $S$ z operacijo $*$ je
\emph{polgrupa}\index{Algebrska struktura!Polgrupa, monoid},
če je $*$ asociativna. Polgrupam z nevtralnim elementom pravimo
\emph{monoid}.
\end{definicija}

\begin{opomba}
Če je $S$ polgrupa, oklepajev ni potrebno postavljati.
\end{opomba}

\begin{opomba}
V polgrupah z $x^n$ označujemo $\underbrace{x*\dots *x}_{n}$.
\end{opomba}

\begin{definicija}
Naj bo $(S,*)$ monoid z enoto $e$.

\begin{enumerate}[i)]
\item $y\in S$ je
\emph{levi inverz}\index{Binarna operacija!Inverz} $x\in S$, če je
$y*x=e$.
\item $z\in S$ je \emph{desni inverz} $x\in S$, če je $x*z=e$.
\item $w\in S$ je \emph{inverz} $x\in S$, če je $x*w=w*x=e$.
\end{enumerate}

Pravimo, da je $x$ \emph{obrnljiv}, če ima inverz.
\end{definicija}

\datum{2021-10-8}
\begin{trditev}
Naj bo $S$ monoid. Če obstajata taka $l,d\in S$, da za nek $x\in S$
velja $lx=xd=e$, velja $l=d$.
\end{trditev}

\obvs

\begin{posledica}
Obrnljiv element ima samo en inverz. Če je $x$ obrnljiv, je
$xy=e\iff yx=e$.
\end{posledica}

\begin{opomba}
Inverz elementa $x$ označimo z $x^{-1}$. Očitno je
$(x^{-1})^{-1}=x$. Označimo še $x^{-n}=(x^{-1})^n=(x^n)^{-1}$ in
$x^0=e$.
\end{opomba}

\begin{trditev}
Če sta $x,y\in S$ obrnljiva, je obrnljiv tudi $xy$ z inverzom
$y^{-1}x^{-1}$.
\end{trditev}

\obvs

\begin{trditev}
Naj bo $x\in S$ obrnljiv. Potem za vse $y,z\in S$ velja
\[
xy=xz\implies y=z\quad\text{in}\quad yx=zx\implies y=z.
\]
\end{trditev}

\obvs

\newpage

\subsection{Grupe}

\begin{definicija}
Monoidu, v katerem so vsi elementi obrnljivi, pravimo
\emph{grupa}\index{Algebrska struktura!Grupa}.
\end{definicija}

\begin{definicija}
Grupi, v kateri je operacija komutativna, pravimo
\emph{Abelova}\index{Algebrska struktura!Grupa!Abelova}.
\end{definicija}

\begin{definicija}
Grupa $G$ je \emph{končna}\index{Algebrska struktura!Grupa!Končna},
če obstaja tak $n\in\N$, da je $\abs{G}=n$. Številu $n$ pravimo
\emph{red}\index{Algebrska struktura!Grupa!Red} grupe $G$.
\end{definicija}

\begin{trditev}
Naj bo $S$ monoid. Potem je
$\setb{x}{x\in S\land\text{$x$ je obrnljiv}}$ grupa.
\end{trditev}

\begin{definicija}
Grupam reda $1$ pravimo
\emph{trivialne grupe}\index{Algebrska struktura!Grupa!Trivialna}.
\end{definicija}

\begin{definicija}
\emph{Simetrična grupa}\index{Algebrska struktura!Grupa!Simetrična}
množice $X$ je množica
\[
\Sim(X)=\setb{f}{f\colon X\to X\land\text{$f$ je bijektivna}}
\]
z operacijo kompozitum. Če je $\abs{X}=n$, označimo $\Sim(X)=S_n$.
\end{definicija}

\begin{opomba}
V nadaljevanju namesto z $e$ enoto označimo z $1$. Za operacije
pišemo $\cdot$, razen v Abelovih grupah, kjer jo označimo s $+$.
\end{opomba}

\newpage

\subsection{Kolobarji in polja}

\datum{2021-10-15}

\begin{definicija}
Množica $K$ z binarnima operacijama seštevanja in množenja je
\emph{kolobar}\index{Algebrska struktura!Kolobar}, če velja:

\begin{enumerate}[i)]
\item za seštevanje je $K$ Abelova grupa,
\item za množenje je $K$ monoid in
\item veljata leva in desna distributivnost.
\end{enumerate}
\end{definicija}

\begin{trditev}
V poljubnem kolobarju $K$ velja:

\begin{enumerate}[i)]
\item $\forall x \in K \colon 0x = x0 = 0$,
\item $\forall x, y \in K \colon (-x)y = x(-y) = -(xy)$,
\item $\forall x, y, z \in K \colon (x-y)z = xz - yz$,
\item $\forall x, y \in K \colon (-x)(-y) = xy$ in
\item $\forall x \in K \colon (-1)x = x(-1) = -x$.
\end{enumerate}
\end{trditev}

\obvs

\begin{definicija}
Kolobarju s komutativnim množenjem pravimo
\emph{komutativen kolobar}\index{Algebrska struktura!Kolobar!Komutativen}.
\end{definicija}

\begin{definicija}
Element $x$ kolobarja $K$ je
\emph{delitelj niča}\index{Algebrska struktura!Kolobar!Delitelj niča},
če je $x \ne 0$ in obstaja tak $y \ne 0$ iz $K$, da je $xy = 0$ ali
$yx = 0$.
\end{definicija}

\begin{definicija}
Neničelnemu kolobarju, v katerem je vsak neničelni element
obrnljiv, pravimo \emph{obseg}\index{Algebrska struktura!Obseg}.
Komutativnemu obsegu pravimo
\emph{polje}\index{Algebrska struktura!Polje}.
\end{definicija}

\begin{trditev}
Obrnljiv element kolobarja ni delitelj niča.
\end{trditev}

\obvs

\begin{posledica}
Obseg je kolobar brez deliteljev niča.
\end{posledica}

\newpage

\subsection{Vektorski prostori in algebre}

\begin{definicija}
Naj bo $\F$ polje. Množica $V$ s seštevanjem in množenjem s
skalarjem je
\emph{vektorski prostor}\index{Algebrska struktura!Vektorski prostor}
nad $\F$, če velja:

\begin{enumerate}[i)]
\item $V$ je Abelova grupa za seštevanje,
\item $\forall \lambda \in \F, u,v \in V \colon
\lambda(u+v) = \lambda u + \lambda v$,
\item $\forall \lambda, \mu \in \F, v \in V \colon
(\lambda + \mu)v = \lambda v + \mu v$,
\item $\forall \lambda, \mu \in \F, v \in V \colon
\lambda(\mu v) = (\lambda \mu)v$ in
\item $\forall v \in V \colon 1v = v$.
\end{enumerate}
\end{definicija}

\begin{definicija}
Naj bo $\F$ polje. Množica $A$ s seštevanjem in množenjem ter
množenjem s skalari iz $\F$ imenujemo
\emph{algebra}\index{Algebrska struktura!Algebra} nad $\F$, če
velja:

\begin{enumerate}[i)]
\item $A$ je vektorski prostor nad $\F$ za seštevanje in množenje s
skalarji,
\item $A$ je za seštevanje in množenje kolobar in
\item $\forall \lambda \in \F, x,y \in A \colon
\lambda(xy) = (\lambda x)y = x(\lambda y)$.
\end{enumerate}
\end{definicija}

\newpage

\subsection{Podstrukture}

\datum{2021-10-22}

\begin{definicija}
Podmnožica $H$ grupe $G$ je
\emph{podgrupa}\index{Algebrska struktura!Podstruktura} grupe $G$,
če je grupa isto operacijo.\footnote{S tem je seveda mišljena
skrčitev operacije na $H \times H$.} Pišemo $H \leq G$.
\end{definicija}

\begin{opomba}
Za vsako podgrupo $H$ je $1 \in H$.
\end{opomba}

\begin{trditev}
Za neprazno podmnožico $H$ grupe $G$ so naslednje izjave
ekvivalentne:

\begin{enumerate}[i)]
\item $H \leq G$
\item $\forall x, y \in H \colon xy^{-1} \in H$
\item $\forall x, y \in H \colon xy \in H \land x^{-1} \in H$
\end{enumerate}
\end{trditev}

\obvs

\begin{definicija}
\emph{Center}\index{Algebrska struktura!Grupa!Center} grupe $G$ je
množica
\[
Z(G) = \setb{c \in G}{\forall x \in G \colon cx = xc}.
\]
\end{definicija}

\begin{opomba}
Velja $Z(G) \leq G$.
\end{opomba}

\begin{definicija}
Naj bo $H \leq G$ podgrupa. $aHa^{-1}$ za $a \in G$ je
\emph{konjugirana podgrupa}\index{Algebrska struktura!Grupa!Konjugirana podgrupa}\footnote{
Elementa $x,y \in G$ sta si \emph{konjugirana}, če je $y=axa^{-1}$
za nek $a \in G$.} grupe $G$.
\end{definicija}

\begin{definicija}
Podmnožica $L$ kolobarja $K$ je \emph{podkolobar}, če je kolobar za
isti operaciji in vsebuje enoto $1$ kolobarja $K$.
\end{definicija}

\begin{trditev}
Podmnožica $L$ kolobarja $K$ je podkolobar natanko tedaj, ko
$1 \in L$ in velja
$\forall x,y \in L \colon xy \in L \land x-y \in L$.
\end{trditev}

\obvs

\begin{definicija}
\emph{Center}\index{Algebrska struktura!Kolobar!Center} kolobarja
$K$ je množica
\[
Z(K) = \setb{c \in K}{\forall x \in K \colon cx = xc}.
\]
\end{definicija}

\begin{opomba}
Center kolobarja je podkolobar.
\end{opomba}

\begin{definicija}
Podmnožica $U$ vektorskega prostora $V$ je \emph{podprostor}
prostora $V$, če je za isti operaciji tudi sama vektorski prostor.
\end{definicija}

\begin{trditev}
Za podmnožico $U$ vektorskega prostora $V$ so naslednje izjave
ekvivalentne:

\begin{enumerate}[i)]
\item $U$ je podprostor $V$
\item $\forall u,v \in U, \lambda \in \F \colon
u+v \in U \land \lambda u \in U$
\item $\forall u,v \in U, \lambda, \mu \in \F \colon
\lambda u + \mu v \in U$
\end{enumerate}
\end{trditev}

\obvs

\begin{definicija}
Podmnožica $B$ algebre $A$ je \emph{podalgebra}, če je algebra za
iste operacije in vsebuje enoto $A$.
\end{definicija}

\begin{trditev}
$B$ je podalgebra $A$ natanko tedaj, ko je $B$ podkolobar in
podprostor.
\end{trditev}

\begin{definicija}
Podmnožica $\F$ polja $\mathbb{E}$ je \emph{podpolje}, če je $\F$
polje za isti operaciji. Polju $\mathbb{E}$ pravimo
\emph{razširitev}\index{Algebrska struktura!Polje!Razširitev} polja
$\F$.
\end{definicija}

\begin{trditev}
Podmnožica $\F$ polja $\mathbb{E}$ je podpolje natanko tedaj, ko
velja $1 \in \F$,
\[
\forall x,y \in \F \colon x-y \in \F \land xy \in \F
\]
in $\forall x \in \F \colon x \ne 0 \implies x^{-1} \in \F$.
\end{trditev}

\obvs

\begin{trditev}
Presek podstruktur neke strukture je znova podstruktura.
\end{trditev}

\newpage

\subsection{Generatorji}

\begin{definicija}
Naj bo $G$ grupa in $X$ neka njena podmnožica. Z $\skl{X}$ označimo
najmanjšo podgrupo $G$, ki vsebuje $X$ in jo imenujemo podgrupa,
\emph{generirana z $X$}, elementom $X$ pa pravimo
\emph{generatorji}\index{Grupa!Generatorji}.
\end{definicija}

\begin{opomba}
Podobno definiramo generatorje ostalih algebrskih struktur.\footnote{
Notacijo $\skl{X}$ uporabljamo le pri grupah.}
\end{opomba}

\begin{trditev}
Velja
\[
\skl{X} = \setb{\prod_{i=1}^n y_i}{\forall i \colon
y_i \in X \lor y_i^{-1} \in X}.
\]
\end{trditev}

\obvs

\begin{definicija}
Grupa $G$ je
\emph{končno generirana}\index{Algebrska struktura!Grupa!Končno generirana},
če obstaja končna množica $X$, za katero je $\skl{X} = G$.
\end{definicija}

\begin{opomba}
Grupam, generiranim z enim elementom, pravimo
\emph{ciklične grupe}.
\end{opomba}

\begin{trditev}
Podkolobar, generiran z $X$, je množica\footnote{Oznaka
$\overline{X}$ ni standardna.}
\[
\overline{X} =
\setb{\sum_{i=1}^n \left(n_i \prod_{j=1}^{m_i} x_{i,j}\right)}
{n,m_i \in \N_0, n_i \in \Z, x_{i,j} \in X}.
\]
\end{trditev}

\obvs

\begin{trditev}
Podpolje, generirano z $X$, je množica
\[
\setb{uv^{-1}}{u,v \in \overline{X}}.
\]
\end{trditev}

\begin{proof}
Dovolj je preveriti zaprtost za seštevanje, kar pa lahko zapišemo
kot
\[
ab^{-1} + cd^{-1} = (ad + bc)(bd)^{-1}. \qedhere
\]
\end{proof}

\begin{trditev}
Podprostor, generiran z $X$, je $\Lin X$.
\end{trditev}

\obvs

\begin{trditev}
Podalgebra, generirana z $X$, je množica
\[
\overline{X} =
\setb{
\sum_{i=1}^n \left(\lambda_i \prod_{j=1}^{m_i} x_{i,j}\right)
}{
n,m_i \in \N_0, \lambda_i \in \F, x_{i,j} \in X
}.
\]
\end{trditev}
