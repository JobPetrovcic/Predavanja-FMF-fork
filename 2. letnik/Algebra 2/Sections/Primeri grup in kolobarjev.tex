\section{Primeri grup in kolobarjev}

\subsection{Cela števila}

\begin{izrek}[Osnovni izrek o deljenju]
\index{Izrek!Osnovni izrek o deljenju}
Naj bo $m \in \Z$ in $n \in \N$. Potem obstajata taki enolični
števili $q$ in $r$, za kateri je
\[
m = qn + r \quad \text{in} \quad 0 \leq r < n.
\]
\end{izrek}

\begin{proof}
Naj bo
\[
S = \setb{k \in \Z}{kn \leq m}.
\]
$S$ je navzgor omejena, zato ima največji element $q$, ki ustreza
zgornjim pogojem.
\end{proof}

\begin{posledica}
Podmnožica $H$ aditivne grupe $\Z$ je podgrupa natanko tedaj, ko je
$H$ oblike $n \Z$ za $n \in \N_0$.
\end{posledica}

\begin{proof}
$n \Z$ je očitno grupa za vsak $n$, opazimo pa, da najmanjši
naravni element $H$ deli vse ostale.
\end{proof}

\begin{definicija}
$d \in \N$ je
\emph{največji skupni delitelj}\index{Cela števila!Največji skupni delitelj}
celih števil $m$ in $n$, če $d \mid n$, $d \mid m$ in vsak skupni
delitelj $m$ in $n$ deli tudi $d$. Označimo $d = \gcd(m,n)$.
\end{definicija}

\begin{trditev}
Naj bo $G$ aditivna grupa in $H, K \leq G$. Potem je tudi
\[
H + K = \setb{h + k}{h \in H \land k \in K}
\]
podgrupa $G$.
\end{trditev}

\obvs

\begin{posledica}
Za vse pare celih števil $m$ in $n$, ki nista obe $0$, obstaja
enoličen največji skupni delitelj, ki je oblike
\[
d = mx + ny
\]
za neka $x, y \in \Z$.
\end{posledica}

\begin{proof}
Grupa $n \Z + m \Z$ je grupa oblike $d \Z$.
\end{proof}

\begin{definicija}
Če je $\gcd(m,n) = 1$ pravimo, da sta si $m$ in $n$
\emph{tuji}\index{Cela števila!Tujost}.
\end{definicija}

\begin{lema}[Evklid]\index{Izrek!Evklidova lema}
Naj bo $p \in \mathbb{P}$ in $m, n \in \Z$. Potem velja
\[
p \mid m \cdot n \implies p \mid n \lor p \mid m.
\]
\end{lema}

\begin{proof}
Če $p \nmid m$, je $\gcd(p,m) = 1$, zato obstajata taka $x$ in $y$,
da je
\[
px + my = 1.
\]
Sledi, da je
\[
p \cdot \left(nx + \frac{mn}{p} \right) = n. \qedhere
\]
\end{proof}
