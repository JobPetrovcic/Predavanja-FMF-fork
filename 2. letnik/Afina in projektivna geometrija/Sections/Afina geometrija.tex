\section{Afina geometrija}

\subsection{Afini podprostori v vektorskem prostoru}

\datum{2022-2-15}

\begin{definicija}
    Naj bo $V$ končnorazsežen vektorski prostor nad obsegom $O$, $a \in V$ in $W \leq V$. Množico
    \[
    a + W = \setb{a + x}{x \in W}
    \]
    imenujemo \emph{afin podprostor} v $V$.
    Množica $\mathcal{A}$ je \emph{afin prostor}\index{Afin prostor}, če je afin podprostor v kakšnem vektorskem prostoru.
\end{definicija}

\begin{opomba}
    V nadaljevanju $V$ označuje končnorazsežen vektorski prostor nad komutativnim obsegom $O$.
\end{opomba}

\begin{lema}
    Naj bo $\mathcal{A} = a + W$ afin podprostor. Tedaj je $\mathcal{A} = b + W$ za vse $b \in \mathcal{A}$.
\end{lema}

\begin{proof}
    Po definiciji je $b = a + w$ za nek $w \in W$, torej je $w = b - a$. Za vsak $x \in W$ je
    \[
    a + x = b + (a - b) + x = b - w + x,
    \]
    in ker je $W$ vektorski podprostor je $(x - w) \in W$, torej je $a + x = b + (x - w) \in b + W$. Enako pokažemo drugo smer.
\end{proof}

\begin{posledica}
    Naj bosta $\mathcal{A} = a + U$ in $\mathcal{B} = b + W$ afina podprostora v $V$. Če je $\mathcal{A} \subseteq \mathcal{B}$, je $U \leq W$.
\end{posledica}

\begin{proof}
	Velja
	\[    
    a + U = \mathcal{A} \subseteq \mathcal{B} = b + W = a + W. \qedhere
    \]
\end{proof}

\begin{posledica}
    Naj bo $\mathcal{A}$ afin prostor v $V$. Če je $\mathcal{A} = a + W$ in $\mathcal{A} = a' + W'$, potem je $W = W'$.
\end{posledica}

\begin{definicija}
    \emph{Razsežnost}\index{Afin prostor!Razsežnost} afinega prostora $\mathcal{A} = a + W$ je 
    \[
    \dim \mathcal{A} = \dim U.
    \]
\end{definicija}

\begin{definicija}
    Naj bodo $a_{i} \in \mathcal{A}$ in $\alpha_{i} \in O$ za vse $1 \leq i \leq n$, in naj bo $\sum_{i=1}^{n} \alpha_{i} = 1$. Vsoto 
    \[
    \sum_{i=1}^{n} \alpha_{i} a_{i}
    \]
    imenujemo \emph{afina kombinacija}\index{Afina kombinacija} točk $a_1,\dots, a_n$.
\end{definicija}

\begin{lema}
    Naj bo karakteristika $O$ različna od 2. Poljubna afina kombinacija dveh elementov iz $\mathcal{A}$ je v $\mathcal{A}$ natanko tedaj, ko je poljubna afina kombinacija poljubno elementov iz $\mathcal{A}$ v $\mathcal{A}$.
\end{lema}

\begin{proof}
    Lemo dokažemo z indukcijo po številu sumandov. Primera $n = 1$ in $n = 2$ sta trivialna.
    
    Naj bo $n \geq 3$ in predpostavimo, da velja izrek za vse $m < n$. Ideja dokaza je, da pogledamo vsoto prvih $n - 1$ členov in pametno izpostavimo tak faktor, da postane afina in na njej uporabimo izrek
    in zmanjšamo vsoto na afino kombinacijo dveh elementov, za katero izrek trivialno velja.
    Označimo $\alpha = \alpha_1 + \dots + \alpha_{n-1}$. Sedaj ločimo dva primera:
   
    \begin{enumerate}[i)]
    \item Velja $\alpha \neq 0.$ Sledi, da je
    \[
        \alpha_1 a_1 + \dots + \alpha_{n-1} a_{n-1} + \alpha_n a_n = 
        \underbrace{\alpha \cdot \overbrace{(\alpha^{-1} \cdot \alpha_1 a_1 + \dots + \alpha^{-1} \cdot \alpha_{n-1} a_{n-1})}^{\text{afina kombinacija $n - 1$ elementov}} + \alpha_n a_n}_{\text{afina kombinacija dveh elementov}}.
    \]
    Po indukcijski predpostavki je torej afina kombinacija znova element $\mathcal{A}$.
    \item Velja $\alpha = 0$.
    Brez škode za splošnost je $\alpha_1 + \dots + \alpha_{n-2} \neq 0$, drugače bi bil $\alpha_{n-1} = 0$ in bi imeli kombinacijo $n-1$ elementov, za katero po indukcijski predpostavki izrek drži.
    Dokaz je isti kot zgoraj, le da vzamemo prvih $n-2$ elementov namesto $n-1$ in vsoto zmanjšamo na 3 elemente namesto 2. 
    
    Dovolj je tako pokazati trditev za $n=3$. Ker ima $O$ karakteristiko različno od 2, lahko izberemo taka $\alpha_1$ in $\alpha_2$, da je $\alpha_1 + \alpha_2 \neq 0$,
    saj drugače velja
    \[
    \alpha_1 + \alpha_2 = \alpha_3 + \alpha_2 = \alpha_1 + \alpha_3 = 0,
    \]
    torej velja $\alpha_1 = \alpha_2 = \alpha_3 = 1$ in zato $\alpha_1 + \alpha_2 + \alpha_3 = 1 + 1 + 1 = 1$, oziroma $1 + 1 = 0$, kar je protislovje. Sedaj zaključimo kot v prejšnjem primeru. \qedhere
    \end{enumerate}
\end{proof}

\begin{trditev}
    Naj bo karakteristika $O$ različna od 2. $\mathcal{A} \leq V$ je afin podprostor natanko tedaj, ko poljubna afina kombinacija dveh točk iz $\mathcal{A}$ leži v $\mathcal{A}$.
\end{trditev}

\begin{proof}
    ($\Rightarrow$)
    Predpostavimo, da je $\mathcal{A}$ afin podprostor.
    Naj bo $\mathcal{A} = a + W$ in $a + w_1, a + w_2 \in \mathcal{A}$, kjer sta $w_1, w_2 \in W$, ter naj bosta $\alpha_1, \alpha_2 \in O$ taka, da velja $\alpha_1 + \alpha_2 = 1$. Potem velja
    \begin{align*}
    \alpha_1 a_1 + \alpha_2 a_2 &= \alpha_1 (a + w_1) + \alpha_2 (a + w_2) \\
    &= \alpha_1 a + \alpha_1 w_1 + \alpha_2 a + \alpha_1 w_2 \\
    &= \alpha_1 a + \alpha_2 a + \alpha_1 w_1 + \alpha_1 w_2 \\
    &= \underbrace{(\alpha_1 + \alpha_2)}_{=\: 1} a + \underbrace{(\alpha_1 w_1 + \alpha_1 w_2)}_{\text{leži v $W$}}.
    \end{align*}

    ($\Leftarrow$)
    Sedaj predpostavimo, da poljubna afina kombinacija dveh točk iz $\mathcal{A}$ leži v $\mathcal{A}$.
    $\mathcal{A}$ je afin prostor natanko tedaj, ko obstajata nek $W \leq V$ in $a \in \mathcal{A}$, da je $\mathcal{A} = a + W$, oziroma ko za vsak $v \in \mathcal{A}$ velja $v - a \in W$.
    
    Fiksiramo $a \in A$. Pokazali bomo da je množica $W = \setb{b-a}{b \in \mathcal{A}}$ vektorski prostor.
    Naj bosta $x$ in $y$ poljubna elementa $W$, torej $x = b - a$ in $y = c - a$ za neka $b, c \in \mathcal{A}$, in naj bosta $\alpha, \beta \in O$.
    
    Linearna kombinacija $\alpha x + \beta y$ leži v $W$ natanko tedaj, ko za nek $d \in \mathcal{A}$ velja
    \[
    \alpha x + \beta y = \alpha (b - a) + \beta (c - a) = d - a,
    \]
    oziroma
    \[
    a + \alpha (b - a) + \beta (c - a) = (1 - \alpha - \beta) a + \alpha b + \beta c = d.
    \]
    Ker pa velja $(1 - \alpha - \beta) + \alpha + \beta = 1$, je zgornja vsota afina kombinacija elementov $a$, $b$ in $c$ iz $\mathcal{A}$, torej po predpostavki njihova vsota leži v $\mathcal{A}$.
\end{proof}

\datum{2022-2-22}

\begin{posledica}
    $\mathcal{A}$ je afin podprostor v $V$ natanko tedaj, ko leži poljubna afina kombinacija elementov iz $\mathcal{A}$ v $\mathcal{A}$.
\end{posledica}

\begin{trditev}
    Če je presek $\mathcal{P}$ kake družine afinih podprostorov neprazen, je $\mathcal{P}$ afin podprostor.
\end{trditev}

\begin{proof}
    Naj bo $\mathcal{A}_{\lambda}$ družina afinih podprostorov. Izberemo $a \in \bigcap \mathcal{A}_{\lambda} = \mathcal{P}$. Potem za vsak $\lambda$ velja $\mathcal{A}_{\lambda} = a + W_{\lambda}$ za nek vektorski prostor $W_{\lambda}$. Velja
    \[
    \bigcap \mathcal{A}_{\lambda} = \bigcap \set{a + W_{\lambda}} = a + \bigcap W_{\lambda},
    \]
    kjer je $\bigcap W_{\lambda}$ vektorski prostor, torej je $\mathcal{P}$ afin podprostor.
\end{proof}

\begin{definicija}
    \emph{Afina ogrinjača}\index{Afina ogrinjača} množice $X \subseteq V$ je presek vseh afinih podprostorov, ki vsebujejo $X$. Označimo jo z $\Af (X)$ in je afin prostor po zgornji trditvi.
\end{definicija}

\begin{opomba}
    $\Af (X)$ je po definiciji najmanjši afin podprostor, ki vsebuje $X$.
\end{opomba}

\begin{trditev}
    $\Af (X)$ je enaka množici vseh afinih kombinacij elementov iz $X$.
\end{trditev}

\begin{proof}
    Z $\mathcal{A}$ označimo množico vseh afinih kombinacij elementov iz $X$.

    Ker je $\Af (X)$ afin podprostor, leži poljubna linearna kombinacija elementov iz $\Af (X)$ v $\Af (X)$, torej velja $\mathcal{A} \subseteq \Af (X)$.

    Ker je $X \subseteq \mathcal{A}$, je po zgornji opombi za $\Af (X) \subseteq \mathcal{A}$ dovolj pokazati, da je $\mathcal{A}$ afin podprostor. Poljubna afina kombinacija elementov iz $\mathcal{A}$ je afina kombinacija afinih kombinacij elementov iz $X$,
    kar je spet afina kombinacija elementov iz $X$, torej leži v $\mathcal{A}$.
\end{proof}

\begin{lema}
    Naj bosta $\mathcal{A} = a + W$ in $\mathcal{B} = b + U$ afina podprostora. Tedaj se $\mathcal{A}$ in $\mathcal{B}$ sekata natanko tedaj ko je $b - a \in W + U$.
\end{lema}

\begin{proof}
    ($\Leftarrow$)
    Naj se $\mathcal{A}$ in $\mathcal{B}$ sekata. Potem obstajata taka $w \in W$ in $u \in U$, da za neka $a \in \mathcal{A}$ in $b \in \mathcal{B}$ velja
    \[
    a + w = b + u,    
    \]
    iz česar sledi
    \[
    b - a = w - u,    
    \]
    torej velja $b - a \in W + U$

    ($\Rightarrow$)
    Naj bo $b - a \in W + U$. Potem obstajata taka $w \in W$ in $u \in U$, da je
    \[
    b - a = w + u;    
    \]
    iz česar sledi
    \[
    b + (- u) = a + w,    
    \]
    torej se $\mathcal{A}$ in $\mathcal{B}$ sekata v nekem elementu.
\end{proof}

\begin{lema}
    $\Af ((a + W) \cup (b + U)) = a + W + U + \Lin (\set{b - a})$.
\end{lema}

\begin{proof}
    Naj bo 
    \[
    T = W + U + \Lin (\set{b - a}),
    \]
    $T$ je vektorski prostor, torej je $a + T = a + (b - a) + T = b + T$ afin podprostor.

    Najprej pokažemo $\Af ((a + W) \cup (b + U)) \subseteq a + T$.
    \begin{align*}
        a + W &\subseteq a + T \text{ in} \\
        b + U &\subseteq b + T = a + T,
    \end{align*}
    torej velja
    \[
    \Af ((a + W) \cup (b + U)) \subseteq a + T.
    \]

    Da dokažemo vsebovanost v drugo smer bomo pokazali, da vsak afin prostor $\mathcal{C}$ ki vsebuje $(a + W) \cup (b + U)$, vsebuje tudi $a + T$.

    Naj bo $(a + W) \cup (b + U) \subseteq \mathcal{C}$. Potem sta $a, b \in \mathcal{C}$, torej obstaja nek vektorski prostor $S$, da velja $a + S = b + S = \mathcal{C}$. iz
    \begin{align*}
        a + W \subseteq a + S, \\
        b + U \subseteq b + S
    \end{align*}
    sledi $W \leq S$ in $U \leq S$. Ker je $b \in a + S$ obstaja nek $s \in S$, da je $b = a + s$, torej je $b - a \in S$, iz česar sledi $\Lin (\set{b - a}) \subseteq S$.
    Če združimo vse to, dobimo da je $T \subseteq S$, torej je
    \[
    a + T \subseteq a + S = C.
    \] 
\end{proof}

\begin{trditev}
    Naj bosta $\mathcal{A} = a + W$ in $\mathcal{B} = b + U$ afina podprostora. Velja
    \[
    \mathcal{A} \cap \mathcal{B} = \emptyset \Leftrightarrow \dim \Af ( \mathcal{A} \cap \mathcal{B} ) = \dim (W + U) + 1
    \]
\end{trditev}

\begin{proof}
    \begin{align*}
        \mathcal{A} \cap \mathcal{B} = \emptyset &\Leftrightarrow b - a \notin W + U \\
        &\Leftrightarrow \dim \Af ( \mathcal{A} \cap \mathcal{B} ) = \dim (W + U + \Lin (\set{b-a})) 
    \end{align*}
    in velja
    \begin{align*}
        \dim \Af ( \mathcal{A} \cap \mathcal{B} ) &= \dim (W + U + \Lin (\set{b-a})) \\
        &= \dim (W + U) + \dim \Lin (\set{b-a}) - \dim ((W + U) \cap \Lin (\set{b-a})) \\
        &= \dim (W + U) + 1.
    \end{align*}
\end{proof}

\begin{definicija}
    Afina prostora $\mathcal{A} = a + W$ in $\mathcal{B} = b + U$ sta \emph{vzporedna}\index{Vzporedna}, če je $W \leq V$ ali $V \leq W$, kar označimo z $\mathcal{A} || \mathcal{B}$. 
\end{definicija}

\begin{trditev}
    Naj bosta $\mathcal{A}$ in $\mathcal{B}$ afina podprostora. 

    a) Če se $\mathcal{A}$ in $\mathcal{B}$ sekata, sta vzporedna natanko tedaj, ko je ali $\mathcal{A} \subseteq \mathcal{B}$, ali pa $\mathcal{B} \subseteq \mathcal{A}$.

    b) Če se $\mathcal{A}$ in $\mathcal{B}$ ne sekata, sta vzporedna natanko tedaj, ko velja 
    \[
        \dim \Af (\mathcal{A} \cup \mathcal{B}) = \max (\dim \mathcal{A}, \dim \mathcal{B}) + 1.
    \]
\end{trditev}

\begin{proof}
    a) Naj bosta $\mathcal{A} = a + W$ in $\mathcal{B} = a + U$, kjer je $a \in \mathcal{A} \cap \mathcal{B}$.
    \[
    \mathcal{A} || \mathcal{B} \Leftrightarrow W \leq U \text{ ali } U \leq W \Leftrightarrow \underbrace{a + W \subseteq a + U}_{\mathcal{A} \subseteq \mathcal{B}} \text{ ali } \underbrace{a + U \subseteq a + W}_{\mathcal{B} \subseteq \mathcal{A}}.
    \]

    b) Naj bosta $\mathcal{A} = a + W$ in $\mathcal{B} = b + U$.
    \begin{align*}
        \mathcal{A} || \mathcal{B} &\Leftrightarrow W \leq U \text{ ali } U \leq W \\
        &\Leftrightarrow W + U = U \text{ ali } W + U = W \\
        &\Leftrightarrow \dim (W + U) = \dim U \text{ ali } \dim (W + U) = \dim W \\
        &\Leftrightarrow \dim (W + U) = \max (\dim U, \dim W) \\
        &\Leftrightarrow \dim \Af (\mathcal{A} \cup \mathcal{B}) = \dim (W + U) + 1 = \max (\dim U, \dim W) + 1.
    \end{align*}
\end{proof}

\begin{definicija}
    Množica $\set{x_0, x_1, \dots, x_n}$ je \emph{afino neodvisna}\index{Afino neodvisna}, če je množica \\
     %ta \\ je tukaj ker latex ni znal sam prestaviti enacbe v novo vrstico in se je pritozeval da ven gleda.
    $\set{x_1 - x_0, \dots, x_n - x_0}$ linearno neodvisna.
\end{definicija}

\begin{opomba}
    Definicija je neodvisna od vrstnega reda elementov.
\end{opomba}

\begin{definicija}
    Množica $X$ je \emph{afina baza}\index{Afina baza} afinega prostora $\mathcal{A}$, če je afino neodvisna in velja $\Af (X) = \mathcal{A}$.
\end{definicija}

\begin{izrek}
    Naj bo $\mathcal{A} = a + W$ afin podprostor.

    a) $\set{x_0, \dots, x_n}$ je afina baza za $\mathcal{A}$ natanko tedaj, ko je $\set{x_1 - x_0, \dots, x_n - x_0}$ baza za $W$.

    b) $\set{e_1, \dots, e_n}$ je baza za $W$ natanko tedaj, ko je $\set{a, e_1 + a, \dots, e_n + a}$ afina baza za $\mathcal{A}$
\end{izrek}


\subsection{Semilinearne preslikave}

\begin{definicija}
    Naj bosta $U, V$ vektorska prostora nad istim obsegom $O$. Preslikava $A: U \to V$ je \emph{semi-linearna}\index{Semi-linearna preslikava}, če je

    (i) aditivna: Za vse $x,y \in U$
    \[
    A(x + y) = Ax + Ay.    
    \]

    (ii) semi-homogena: Obstaja nek avtomorfizem $f$ obsega $O$, da je za vsak $x \in U, \alpha \in O$
    \[
    A(\alpha x) = f(\alpha) Ax.
    \]
\end{definicija}

\begin{opomba}
    Obsegi $\R$, $\Q$ in $\F_p$ nimajo netrivialnih avtomorfizmov. $\mathbb{C}$ jih ima neskončno, ampak edini lahek netrivialen primer je konjugacija.
\end{opomba}