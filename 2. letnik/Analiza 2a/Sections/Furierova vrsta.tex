\section{Furierova vrsta}

\epigraph{">Tako kot bi odvijali tele naše babuške nazaj."<}
{-- prof.~dr.~Miran Černe}

\subsection{Hilbertov prostor}

\datum{2021-12-15}

\begin{definicija}
Naj bo $X$ vektorski prostor nad $\R$ (ali $\C$).
\emph{Skalarni produkt}\index{Skalarni produkt} je preslikava
$\skl{\;,\;} \colon X \times X \to \R$ (ali $\C$), za katero je

\begin{enumerate}[i)]
\item $\skl{x,x} \geq 0$,
\item $\skl{x,x} = 0 \iff x = 0$,
\item $\skl{x,y} = \overline{\skl{y,x}}$,
\item $\skl{\lambda x + \mu y, z} =
\lambda \skl{x,z} + \mu \skl{y,z}$.
\end{enumerate}
\end{definicija}

\begin{definicija}
Naj bo $X$ vektorski prostor. \emph{Norma}\index{Norma} je
preslikava $\norm{\;} \colon X \to \R$, za katero je

\begin{enumerate}[i)]
\item $\norm{x} \geq 0$,
\item $\norm{x} = 0 \iff x = 0$,
\item $\norm{\lambda x} = \abs{\lambda} \norm{x}$,
\item $\norm{x+y} \leq \norm{x} + \norm{y}$.
\end{enumerate}
\end{definicija}

\begin{izrek}[Cauchyjeva neenakost]
\index{Izrek!Cauchyjeva neenakost}
Velja
\[
\abs{\skl{x,y}} \leq \norm{x} \norm{y}.
\]
\end{izrek}

\begin{proof}
Glej izrek 7.1.3 v zapiskih Algebre 1.
\end{proof}

\begin{posledica}
Skalarni produkt je zvezna preslikava.
\end{posledica}

\begin{proof}
Velja
\[
\abs{\skl{x,a} - \skl{y,a}} \leq \norm{a} \cdot \norm{x - y}.
\]
\end{proof}

\begin{opomba}
Skalarni produkt inducira normo $\norm{x} = \sqrt{\skl{x,x}}$,
norma pa metriko $d(x,y) = \norm{x - y}$. Posledično skalarni
produkt inducira metriko.
\end{opomba}

\begin{definicija}
\emph{Hilbertov prostor}\index{Hilbertov prostor} je vektorski
prostor s skalarnim produktom, ki je v inducirani metriki poln
metrični prostor.
\end{definicija}

\begin{definicija}
Napolnitev prostora $\mathcal{C}([a,b])$ za
\[
d(f,g) = \sqrt{\int_a^b \abs{f-g}^2\;dx}
\]
označimo z $L^2([a,b])$.
\end{definicija}

\begin{opomba}
Označimo
\[
\mathcal{L}^2([a,b]) =
\setb{f \colon [a,b] \to \R}{\int_a^b \abs{f}^2\;dx < \infty}.
\]
$\mathcal{L}^2([a,b])$ je vektorski prostor. S skalarnim produktom
\[
\skl{f,g} = \int_a^b f(x)g(x)\;dx
\]
je tako $\kvoc{\mathcal{L}^2([a,b])}{\sim}$ Hilbertov. Izkaže se,
da je to ravno $L^2([a,b])$.
\end{opomba}

\begin{definicija}
Naj bo $X$ vektorski prostor s skalarnim produktom. Pravimo, da sta
vektorja $x, y \in X$ \emph{pravokotna}\index{Pravokotnost}, če
velja
\[
\skl{x,y} = 0.
\]
\end{definicija}

\datum{2021-12-21}

\begin{definicija}
\emph{Ortogonalni komplement}\index{Pravokotnost!Ortogonalni komplement}
množice $A$ je podprostor\footnote{Ortogonalni komplement je zaprt,
saj vsebuje vse svoje limite.}
\[
A^\bot = \setb{x \in X}{\forall a \in A \colon \skl{x,a} = 0}.
\]
\end{definicija}

\begin{definicija}
Naj bo $Y \leq X$. \emph{Pravokotna projekcija} vektorja $x$ na $Y$
je vektor $P_Y(x) \in Y$, za katerega je $x-P_Y(x) \in Y^\bot$, če
obstaja.
\end{definicija}

\begin{opomba}
Projekcija je dobro definirana. Če sta $u$ in $v$ projekciji, je
namreč $\skl{u-v,u-v}=0$. 
\end{opomba}

\begin{definicija}
Zaporedje $(e_n)_{n=1}^\infty$ neničelnih vektorjev je
\emph{ortogonalni sistem}\index{Hilbertov prostor!Ortonormiran sistem},
če za vse $i \ne j$ velja
$\skl{e_i,e_j}=0$. Sistem je \emph{ortonormiran}, če velja
$\skl{e_i,e_j}=\delta_{i,j}$.
\end{definicija}

\begin{trditev}[Besselova neenakost]
\index{Izrek!Besselova neenakost}
Naj bo $x \in X$ element vektorskega prostora s skalarnim produktom
in $(e_n)_{n=1}^\infty$ ortonormiran sistem. Tedaj je
\[
\sum_{n=1}^\infty \abs{\skl{x,e_n}}^2 \leq \norm{x}^2.
\]
\end{trditev}

\begin{proof}
Naj bo $Y_n = \Lin(\setb{e_i}{i \leq n})$. Sledi, da ti vektorji
tvorijo ortonormirano bazo $Y_n$. Sledi, da za vse $x \in X$ velja
\[
P_{Y_n}(x) = \sum_{i=1}^n \skl{x,e_i} e_i.
\]
Po Pitagorovem izreku je tako
\[
\norm{x}^2 \geq
\norm{P_{Y_n}(x)}^2 =
\sum_{i=1}^n \abs{\skl{x,e_i}}^2. \qedhere
\]
\end{proof}

\begin{opomba}
Številom $\skl{x,e_n}$ pravimo
\emph{Fourierovi koeficienti}\index{Fourierova vrsta!Fourierovi koeficienti}.
\end{opomba}

\begin{trditev}
Naj bo $X$ Hilbertov prostor. Naj bo $(c_n)_{n=1}^\infty$ zaporedje
števil, za katere je vrsta
\[
\sum_{n=1}^\infty \abs{c_n}^2
\]
konvergentna. Naj bo $(e_n)_{n=1}^\infty$ ortonormiran sistem.
Tedaj obstaja tak $x \in X$, da je
\[
x = \sum_{n=1}^\infty c_n e_n.
\]
\end{trditev}

\begin{proof}
Z $x_n$ označimo $n$-to delno vsoto. Velja
\[
\norm{x_{n+p} - x_n}^2 =
\norm{\sum_{i=n+1}^{n+p} c_ie_i}^2 =
\sum_{i=n+1}^{n+p} \abs{c_i}^2.
\]
Ker je $\left(\abs{c_n}^2\right)$ Cauchyjevo, je tako tudi $(x_n)$
Cauchyjevo, zato ima limito.
\end{proof}

\begin{definicija}
Ortonormiran sistem v Hilbertovem prostoru je
\emph{kompleten}\index{Hilbertov prostor!Ortonormiran sistem!Kompleten},
če za vse $x \in X$ velja
\[
x = \sum_{n=1}^\infty \skl{x,e_n}e_n.
\]
\end{definicija}

\begin{izrek}
\label{iz:kons:kar}
Naj bo $X$ Hilbertov in $(e_n)_{n=1}^\infty$ ortonormiran sistem.
Naslednje izjave so ekvivalentne:

\begin{enumerate}[i)]
\item
\label{iz:kons:1}
$(e_n)_{n=1}^\infty$ je kompleten.
\item
\label{iz:kons:2}
$\forall x,y \in X \colon
\skl{x,y} = \sum_{n=1}^\infty \skl{x,e_n} \overline{\skl{y,e_n}}$.
\item
\label{iz:kons:3}
$\forall x \in X \colon \norm{x}^2 =
\sum_{n=1}^\infty \abs{\skl{x,e_n}}^2$.\footnote{Tej enakosti
pravimo \emph{Parsevalova enakost}.}
\item
\label{iz:kons:4}
$(e_n)_{n=1}^\infty$ ni vsebovan v nobenem strogo večjem
ortonormiranem sistemu.
\item
\label{iz:kons:5}
Edini vektor, ki je pravokoten na vse $e_i$, je $0$.
\item
\label{iz:kons:6}
Končne linearne kombinacije vektorjev $e_i$ so goste v $X$.
\end{enumerate}
\end{izrek}

\begin{proof}
Če velja \ref{iz:kons:1}, dobimo
\[
\skl{x,y} =
\skl{\sum_{n=1}^\infty \skl{x,e_n}e_n,y} =
\sum_{n=1}^\infty \skl{x,e_n} \skl{e_n,y},
\]
saj je $\skl{\;,\;}$ zvezen, zato velja \ref{iz:kons:2}.

Če v \ref{iz:kons:2} vstavimo $x=y$, dobimo \ref{iz:kons:3}.

Denimo, da obstaja normiran $e_0$, ki je pravokoten na vse
ostale. Če vstavimo $e_0$ v \ref{iz:kons:3}, dobimo
$0 = \norm{e_0}^2 = 1$, kar je protislovje.

Če obstaja neničelni $x$, ki je pravokoten na vse v ortonormiranem
sistemu, ga normiramo in dobimo prostislovje s \ref{iz:kons:4}.

Predpostavimo, da velja \ref{iz:kons:5}. Velja
\[
\skl{x-\sum_{i=1}^\infty \skl{x,e_i}e_i,e_n} =
\skl{x,e_n} - \skl{\sum_{i=1}^\infty \skl{x,e_i}e_i,e_n} =
0,
\]
zato sledi \ref{iz:kons:1}.

Dokažimo še ekvivalenco z \ref{iz:kons:6}. Če velja
\ref{iz:kons:1}, lahko zapišemo
\[
x = \sum_{n=1}^\infty \skl{x,e_n} e_n
\]
in vzamemo dovolj veliko delno vsoto. Sedaj predpostavimo še, da
velja \ref{iz:kons:6}. Naj bo $x$ pravokoten na vse $e_i$.
Obstajajo take $\lambda_1,\dots,\lambda_n$, da je
\[
\norm{x-\sum_{i=1}^n \lambda_i e_i} < \varepsilon.
\]
Sledi, da je
\[
\norm{x}^2 =
\skl{x,x} =
\skl{x-\sum_{i=1}^n \lambda_i e_i,x} \leq
\norm{x-\sum_{i=1}^n \lambda_i e_i} \cdot \norm{x}.
\]
Sledi, da je $\norm{x} < \varepsilon$.
\end{proof}

\begin{trditev}
Prostor $\ell^2$ je Hilbertov.\footnote{Vektorski prostor
zaporedij, za katera so vrste kvadratov njihovih členov
konvergentne.}
\end{trditev}

\begin{proof}
Naj bo $\alpha_i$ Cauchyjevo in
\[
d(\alpha_i,\alpha_j) = \sum_{n=-\infty}^\infty \abs{a_n^i-a_n^j}^2.
\]
Sledi, da so zaporedja $\set{\alpha_n^i}_i$ Cauchyjeva, zato imajo
limite $a_n$. Opazimo, da je
\[
\sum_{n=1}^\infty a_n^2
\]
konvergentna, zato je zaporedje element $\ell^2$. Ni težko videti,
da je dobljeno zaporedje limita $\alpha_i$.
\end{proof}

\newpage

\subsection{Klasične Fourierove vrste}

\datum{2021-12-22}

\begin{definicija}
Naj bo $f \in L^2([-\pi,\pi])$. \emph{Klasični Fourierovi
koeficienti}\index{Fourierova vrsta!Klasični koeficienti} so
števila
\[
a_n = \frac{1}{\pi} \int_{-\pi}^\pi f(x)\cos(nx)\;dx
\quad \text{in} \quad
b_n = \frac{1}{\pi} \int_{-\pi}^\pi f(x)\sin(nx)\;dx.
\]
\end{definicija}

\begin{trditev}
Naj bo $f \in L^2([-\pi,\pi])$ funkcija.

\begin{enumerate}[i)]
\item Njena pripadajoča
\emph{Fourierova vrsta}\index{Fourierova vrsta} je
\[
\frac{a_0}{2} + \sum_{n=1}^\infty a_n \cos(nx) + b_n \sin(nx).
\]
\item Velja enakost
\[
\frac{1}{\pi} \int_{-\pi}^\pi \abs{f(x)}^2\;dx =
\frac{\abs{a_0}}{2} + \sum_{n=1}^\infty \abs{a_n}^2 + \abs{b_n}^2.
\]
\label{iz:fou:par}
\end{enumerate}
\end{trditev}

\begin{proof}
\ref{iz:fou:par} je le transformirana Parsevalova enakost.
\end{proof}

\begin{lema}[Riemann-Lebesque]\index{Lema!Riemann-Lebesque}
Velja
\[
\lim_{n \to \infty} a_n = 0
\quad \text{in} \quad
\lim_{n \to \infty} b_n = 0.
\]
\end{lema}

\begin{proof}
Besselova neenakost.
\end{proof}

\begin{lema}
Naj bo $f$ periodična s periodo $p$. Tedaj je
\[
\int_a^{a+p} f(x)\;dx = \int_0^p f(x)\;dx.
\]
\end{lema}

\obvs

\begin{lema}
Velja\footnote{Funkciji $D_N$ pravimo
\emph{Dirichletovo jedro}\index{Jedro!Dirichletovo}.}
\[
D_N(x) =
\frac{1}{2} + \sum_{n=1}^N \cos(nx) =
\frac{1}{2} \frac{\sin\left(\left(N + \frac{1}{2}\right)x\right)}
{\sin\left(\frac{x}{2}\right)}.
\]
\end{lema}

\begin{proof}
Velja
\[
D_N(x) =
\frac{1}{2} + \sum_{n=1}^N \frac{e^{inx} + e^{-inx}}{2} =
\frac{1}{2} \frac{
e^{\left(N+\frac{1}{2}\right)ix}-e^{-\left(N+\frac{1}{2}\right)ix}}
{e^{\frac{1}{2}ix}-e^{-\frac{1}{2}ix}}. \qedhere
\]
\end{proof}

\begin{lema}
Velja
\[
\frac{1}{\pi} \int_{-\pi}^\pi D_N(x)\;dx = 1.
\]
\end{lema}

\obvs

\begin{lema}
Velja
\[
D_N(x) =
\frac{1}{2} \left(
\frac{\cos\left(\frac{x}{2}\right)}{\sin\left(\frac{x}{2}\right)}
\sin(Nx) + \cos(Nx)\right).
\]
\end{lema}

\begin{proof}
Adicijski izrek.
\end{proof}

\begin{izrek}
Naj bo $f \colon \R \to \R$ funkcija s periodo $2\pi$, ki je
odsekoma zvezna\footnote{Na vsakem končnem intervalu ima $f$ končno
mnogo točk nezveznosti in za vse $x \in \R$ obstajata leva in desna
limita v $x$.} in odsekoma odvedljiva.\footnote{Za vsak $x \in \R$
obstajata levi in desni odvod v $x$.} Potem za vse $x \in \R$
pripadajoča Fourierova vrsta konvergira v $x$ in je enaka
\[
\frac{\displaystyle
\lim_{t \uparrow x} f(t) + \lim_{t \downarrow x} f(t)}{2}.
\]
\end{izrek}

\begin{proof}
Ker je $f$ odsekoma zvezna, je v $L^2([-\pi,\pi])$. Naj bo
\begin{align*}
S_N(x) &= \frac{a_0}{2} + \sum_{n=1}^N a_n \cos(nx) + b_n \sin(nx)
\\
&=
\frac{a_0}{2} + \sum_{n=1}^N \frac{1}{\pi}\int_{-\pi}^\pi f(t)
\left(\cos(nt)\cos(nx) + \sin(nt)\sin(nx)\right)dt.
\intertext{Sledi, da je}
S_N(x) &=
\frac{1}{\pi} \int_{-\pi}^\pi f(t)
\left(\frac{1}{2} + \sum_{n=1}^N \cos(n(x-t))\right)dt
\\
&=
\frac{1}{\pi} \int_{-\pi}^\pi f(t) D_N(x-t)\;dt
\\
&=
\frac{1}{\pi} \int_{x-\pi}^{x+\pi} f(x-v) D_N(v)\;dv
\\
&=
\frac{1}{\pi} \int_{-\pi}^{\pi} f(x-v) D_N(v)\;dv
\\
&=
\frac{1}{\pi} \left(\int_{-\pi}^0 f(x-v)D_N(v)\;dv +
\int_0^\pi f(x-v)D_N(v)\;dv\right)
\\
&=
\frac{1}{\pi} \int_0^\pi \left(f(x+v) + f(x-v)\right) D_N(v)\;dv.
\end{align*}
Poglejmo razliko
\[
S_N(x) - \frac{f_+(x) + f_-(x)}{2} =
\frac{1}{\pi} \int_0^\pi (f(x+v)-f_+(x)+f(x-v)-f_-(x))D_N(v)dv.
\]
Velja
\[
\frac{1}{\pi} \int_0^\pi \underbrace{
\frac{f(x+v)-f_+(x)}{v} \cdot
\frac{v}{\sin\left(\frac{v}{2}\right)}
\cos\left(\frac{v}{2}\right)
}_g \sin(Nv)\;dv
= b_N,
\]
kjer je $b_N$ Fourierov koeficient funkcije, ki je za pozitivne $x$
enaka $g$, za negativne pa $0$. Sledi, da je v limiti integral enak
$0$. Podobno izrazimo še ostale člene.
\end{proof}

\datum{2022-1-4}

\begin{definicija}
\emph{Fejérjevo jedro}\index{Jedro!Fejérjevo} je
\[
F_N(x) = \frac{1}{N} \sum_{n=0}^{N-1} D_n(x).
\]
\end{definicija}

\begin{trditev}
Veljajo naslednje trditve:

\begin{enumerate}[i)]
\item Velja enakost
\[
F_N(x) = \frac{1}{N} \left(\frac
{\sin\left(\frac{Nx}{2}\right)}
{\sin\left(\frac{x}{2}\right)}\right)^2.
\]
\item Velja
\[
\frac{1}{\pi} \int_{-\pi}^\pi F_n(x)\;dx = 1.
\]
\item $F_n$ je soda in nenegativna.
\item Za vse $a \in (0,\pi)$ zaporedje $F_N(x)$ na $[a,\pi]$
konvergira enakomerno proti $0$.
\end{enumerate}
\end{trditev}

\begin{proof}
Velja
\begin{align*}
F_N(x) &= \frac{1}{N} \sum_{n=0}^{N-1}
\frac{1}{2} \frac{\sin\left(\left(n + \frac{1}{2}\right)x\right)}
{\sin\left(\frac{x}{2}\right)}
\\
&=
\frac{1}{2N \left(\sin\left(\frac{x}{2}\right)\right)^2} \cdot
\sum_{n=0}^{N-1} \sin\left(\left(n + \frac{1}{2}\right)x\right)
\cdot \sin\left(\frac{x}{2}\right)
\\
&=
\frac{1}{2N \left(\sin\left(\frac{1}{2}\right)\right)^2} \cdot
\sum_{n=0}^{N-1} \left(\cos(nx)-\cos(nx+x)\right)
\\
&=
\frac{1-\cos(Nx)}{2N\left(\sin\left(\frac{1}{2}\right)\right)^2}.
\end{align*}
Druga in tretja trditev sta trivialni. Opazimo še, da velja
\[
F_N(x) \leq
\frac{1}{N} \cdot \left(\frac
{\sin\left(\frac{Nx}{2}\right)}
{\abs{x}}\cdot \pi\right)^2 \leq
\frac{1}{N} \frac{\pi^2}{a^2}. \qedhere
\]
\end{proof}

\begin{izrek}
Naj bo $f$ zvezna funkcija s periodo $2\pi$. Potem \emph{Cesárjeve
delne vsote}
\[
\sigma_N(x) = \frac{1}{N} \sum_{n=0}^{N-1} S_n(x)
\]
konvergirajo k $f$ enakomerno na $\R$.
\end{izrek}

\begin{proof}
Opazimo, da velja
\[
\sigma_N(x) = \frac{1}{\pi} \int_{-\pi}^\pi f(x+y) F_N(y)\;dy.
\]
Naj bo $\varepsilon >0$. Ker je $f$ enakomerno zvezna, obstaja tak
$\delta > 0$, da za vse $\abs{y} < \delta$ velja
\[
\abs{f(x+y) - f(x)} < \frac{\varepsilon}{2}
\]
za vse $x \in \R$. Sledi, da za dovolj velike $N$ velja
\begin{align*}
\abs{\sigma_N(x) - f(x)} &=
\abs{\frac{1}{\pi} \int_{-\pi}^\pi (f(x+y) - f(x)) F_N(y)\;dy}
\\
&\leq
\frac{1}{\pi} \int_{-\pi}^\pi \abs{f(x+y) - f(x)} F_N(y)\;dy
\\
&=
\frac{1}{\pi} \int_{-\delta}^\delta \abs{f(x+y) - f(x)} F_N(y)\;dy
+ \frac{1}{\pi} \lint_{\delta \leq \abs{x} \leq \pi}
\abs{f(x+y) - f(x)} F_N(y)\;dy
\\
&< \frac{\varepsilon}{2} + \frac{\varepsilon}{2} =
\varepsilon. \qedhere
\end{align*}
\end{proof}

\begin{posledica}
Množica
\[
\set{\frac{1}{\sqrt{2\pi}}} \cup
\setb{\frac{1}{\sqrt{\pi}} \cos(nx)}{n \in \N} \cup
\setb{\frac{1}{\sqrt{\pi}} \sin(nx)}{n \in \N}
\]
je kompleten ortonormiran sistem v $L^2([-\pi,\pi])$.
\end{posledica}

\begin{proof}
Z adicijskimi izreki preverimo, da je sistem res ortonormiran.
Po izreku \ref{iz:kons:kar} je dovolj pokazati, da lahko vsako
zvezno funkcijo poljubno dobro aproksimiramo s tem sistemom, saj so
zvezne funkcije goste v $L^2$. Ker smo zgoraj dokazali, da lahko
poljubno dobro aproksimiramo zvezne funkcije (na majhnih
intervalih jih priredimo tako, da se vrednosti v krajiščih
ujemata), ki pa so goste v $L^2$.
\end{proof}

\begin{izrek}[Weierstrass]\index{Izrek!Weierstrass}
Naj bo $f \in \mathcal{C}([a,b])$ in $\varepsilon > 0$. Potem
obstaja tak polinom $p$, da je
\[
\max_{[a,b]} \abs{f(x) - p(x)} < \varepsilon.
\]
\end{izrek}

\begin{proof}
Dovolj je opazovati interval
$\left[-\frac{\pi}{2},\frac{\pi}{2}\right]$. $f$ zvezno razširimo
na $[-\pi,\pi]$ tako, da je $f(-\pi)=f(\pi)=0$. Sedaj lahko $f$
poljubno dobro aproksimiramo s trigonometričnimi polinomi, ki jih
lahko poljubno dobro aproksimiramo s Taylorjevimi polinomi.
\end{proof}

\newpage

\subsection{Fourierova transformacija}

\begin{okvir}
\begin{definicija}
\emph{Fourierova transformacija}\index{Fourierova vrsta!Fourierova transformacija}
je preslikava iz $L^1(\R)$ v $L^\infty(\R)$,\footnote{Skoraj povsod
omejene funkcije.} definirana kot
\[
\mathcal{F}(f)(\xi) = \hat{f}(\xi) =
\int_{-\infty}^\infty e^{-2\pi i \xi t} f(t)\;dt.
\]
Funkciji $\hat{f}$ pravimo
\emph{Fourierova transformiranka $f$}\index{Fourierova vrsta!Fourierova transformiranka}.
\end{definicija}
\end{okvir}

\begin{opomba}
Res velja $\hat{f} \in L^\infty(\R)$, saj je
\[
\abs{\hat{f}(\xi)} \leq
\int_{-\infty}^\infty \abs{e^{-2\pi i \xi t} f(t)}\;dt =
\norm{f}_1.
\]
\end{opomba}

\begin{opomba}
$\mathcal{F}$ je linearna.
\end{opomba}

\begin{opomba}
Fourierova transformacija je na nek način razširitev Fourierove
vrste, ki jo dobimo s kompletnim ortonormiranim sistemom
\[
\setb{\frac{1}{\sqrt{2\pi}} e^{inx}}{n \in \Z}.
\]
Integracijski interval smo razširili na cel $\R$, $n$ pa zamenjali
s poljubnim realnim številom $\xi$.
\end{opomba}

\datum{2022-1-11}

\begin{lema}[Riemann-Lebesque]\index{Lema!Riemann-Lebesque}
Naj bo $f \in L^1(\R)$. Tedaj je
\[
\lim_{\abs{\xi} \to \infty} \hat{f}(\xi) = 0.
\]
\end{lema}

\begin{proof}
Velja, da je $L^1(\R)$ napolnitev prostora
$\mathcal{C}(R) \cap L^1(\R)$ v metriki
\[
d_1(f,g) = \int_{-\infty}^\infty \abs{f(t)-g(t)}\;dt.
\]
Naj bo $f \in L^1(\R)$ in $\varepsilon > 0$. Tedaj obstaja tak
$\tilde{f} \in \mathcal{C}(\R) \cap L^1(\R)$, da je
\[
\int_{-\infty}^\infty \abs{f(t) - \tilde{f}(t)}\;dt <
\frac{\varepsilon}{2}.
\]
Funkcijo $\tilde{f}$ lahko aproksimiramo s funkcijo
\[
g = \chi_{[-N,N]} \cdot \tilde{f}.
\]
Vsak tak $g$ lahko poljubno dobro enakomerno aproksimiramo s
funkcijo
\[
\chi_{[-N,N]} \cdot p.
\]
S pravilom per partes pa dobimo
\begin{align*}
\widehat{\chi_{[-N,N]} p}(\xi) &=
\int_{-N}^N e^{-2 \pi i \xi t} p(t)\;dt
\\
&=
\eval{\frac{-p(t)}{2 \pi i \xi} e^{-2 \pi i \xi t}}{-N}{N} +
\int_{-N}^N \frac{p'(t) e^{-2 \pi i \xi t}}{2 \pi i \xi}\;dt
\end{align*}
Opazimo, da gresta oba člena proti $0$ ko gre $\xi$ proti $\infty$.

Ker je zgornja funkcija poljubno dobra aproksimacija $f$, velja
lema tudi za $f$.
\end{proof}
 
\begin{trditev}
Fourierova transformacija ima naslednje lastnosti:

\begin{enumerate}[i)]
\item $\norm{\mathcal{F}(f)}_\infty \leq \norm{f}_1$.
\item $\mathcal{F}(f)$ je enakomerno zvezna na $\R$.
\item Če je $f$ odvedljiva z absolutno integrabilnim odvodom, je
\[
\mathcal{F}(f')(\xi) = 2\pi \xi \mathcal{F}(f)(\xi).
\]
\item Če je $f$ taka funkcija, da je tudi $t \cdot f(t)$ absolutno
integrabilna, je $\hat{f}$ odvedljiva in velja
\[
\mathcal{F}(f)'(\xi) = -2 \pi i \mathcal{F}(tf(t))(\xi).
\]
\item Naj bo $f$ odsekoma zvezna in odsekoma odvedljiva. Tedaj za
vse $t \in \R$ velja
\[
\lim_{R \to \infty} \int_{-R}^R \hat{f}(\xi)e^{2 \pi i \xi t}\;d\xi
= \frac{f_+(t) + f_-(t)}{2}.
\]
\end{enumerate}
\end{trditev}

\begin{proof}
Naj bo $\varepsilon > 0$. Ker je $f \in L^1(\R)$, obstaja tak
$N > 0$, da je
\[
\int_N^\infty \abs{f(t)}\;dt + \int_{-\infty}^{-N} \abs{f(t)}\;dt
< \frac{\varepsilon}{4}.
\]
Sledi, da je
\begin{align*}
\abs{\hat{f}(\xi_1) - \hat{f}(\xi_2)} &\leq
\int_{-\infty}^\infty
\abs{e^{-2\pi i \xi_1 t} - e^{-2\pi i \xi_2 t}} \abs{f(t)}\;dt
\\
&\leq
\int_{-\infty}^{-N} 2 \abs{f(t)}\;dt +
\int_{-N}^N \abs{e^{-2\pi i (\xi_1-\xi_2) t} - 1} \abs{f(t)}\;dt +
\int_N^\infty 2 \abs{f(t)}\;dt
\\
&<
\frac{\varepsilon}{2} +
\int_{-N}^N \abs{e^{-2\pi i (\xi_1-\xi_2) t} - 1} \abs{f(t)}\;dt.
\end{align*}
Ker je $e^x$ zvezna v $0$, obstaja tak $\delta > 0$, da za vse
$\abs{x} < \delta$ sledi
\[
\abs{e^x - 1} < \frac{\varepsilon}{2 (\norm{f}_1 + 1}.
\]
Če je $\abs{-2\pi i (\xi_1-\xi_2) t} < \delta$, je zgornji izraz
tako manjši od $\varepsilon$.

Tretja točka sledi iz odvajanja po parametru, četrta pa iz pravila
per partes.

Po Fubinijevem izreku je
\begin{align*}
\int_{-R}^R \hat{f}(\xi) e^{2\pi i \xi t}\;d\xi
&=
\int_{-R}^R \left(
\int_{-\infty}^\infty e^{-2\pi i \xi \tau} f(t)\;d\tau
\right) e^{2\pi i \xi t}\;d\xi
\\
&=
\int_{-\infty}^\infty f(\tau) \left(
\int_{-R}^R e^{2\pi i \xi (t-\tau)}\;d\xi
\right)d\tau
\\
&=
\frac{1}{\pi} \int_{-\infty}^\infty f(\tau)
\frac{\sin(2 \pi R (t - \tau))}{t - \tau}\;d\tau
\\
&=
\frac{1}{\pi} \left(
\int_{-\infty}^t f(\tau)
\frac{\sin(2 \pi R (t - \tau))}{t - \tau}\;d\tau
+
\int_t^\infty f(\tau)
\frac{\sin(2 \pi R (t - \tau))}{t - \tau}\;d\tau
\right)
\\
&=
\frac{1}{\pi} \left(
\int_0^\infty f(t - u)
\frac{\sin(2 \pi R u)}{u}\;du
+
\int_0^\infty f(t + u)
\frac{\sin(2 \pi R u)}{u}\;du
\right).
\end{align*}
Sledi, da je
\begin{align*}
&\int_{-R}^R \hat{f}(\xi)e^{2 \pi i \xi t}\;d\xi -
\frac{f_+(t) + f_-(t)}{2}
\\
=
&\frac{1}{\pi} \left(
\int_0^\infty \frac{f(t - u) - f_-(t)}{u} \sin(2 \pi R u)\;du
+
\int_0^\infty \frac{f(t + u) - f_+(t)}{u} \sin(2 \pi R u)\;du
\right).
\end{align*}
Velja pa
\begin{align*}
&\int_0^\infty \frac{f(t + u) - f_+(t)}{u} \sin(2 \pi R u)\;du
\\
=
&\int_0^1 \frac{f(t + u) - f_+(t)}{u} \sin(2 \pi R u)\;du +
\int_1^\infty \frac{f(t + u)}{u} \sin(2 \pi R u)\;du
\\
-
&f_+(t) \int_1^\infty \frac{\sin(2 \pi R u)}{u}\;du.
\end{align*}
Opazimo, da lahko prva dva integrala zapišemo kot
\[
\int_{-\infty}^\infty F(u) \sin(2 \pi R u)\;du,
\]
kjer je $F \in L^1(\R)$. Po Riemann-Lebesqueovi lemi zato
konvergirata proti $0$ ko $R$ limitira v $\infty$.\footnote{Lemo
smo dokazali za $e^{-2 \pi i \xi t}$, a je $\sin$ le linearna
kombinacija takih funkcij.} Opazimo pa še, da je
\[
\lim_{R \to \infty} \int_1^\infty \frac{\sin(2 \pi R u)}{u}\;du =
\lim_{R \to \infty} \int_{2 \pi R}^\infty \frac{\sin(v}{v}\;dv = 0.
\qedhere
\]
\end{proof}

\begin{definicija}
Naj bosta $f,g \in L^1(\R)$ funkciji.
\emph{Konvolucija}\index{Konvolucija} funkcij $f$ in $g$ je
funkcija
\[
(f * g)(t) = \int_{-\infty}^\infty f(s) g(t - s)\;ds.
\]
\end{definicija}

\begin{opomba}
$f * g \in L^1(\R)$, saj je po Fubiniju
\begin{align*}
\int_{-\infty}^\infty \abs{f*g}(t)\;dt &=
\int_{-\infty}^\infty
\abs{\int_{-\infty}^\infty f(s) g(t - s)\;ds}dt
\\
&\leq
\int_{-\infty}^\infty \left(
\int_{-\infty}^\infty \abs{f(s)} \abs{g(t - s)}\;ds
\right)dt
\\
&=
\int_{-\infty}^\infty \abs{f(s)} \left(
\int_{-\infty}^\infty \abs{g(t - s)}\;dt
\right)ds
\\
&=
\norm{f}_1 \cdot \norm{g}_1
\end{align*}
\end{opomba}

\begin{trditev}
Za konvolucijo veljajo naslednje lastnosti:

\begin{enumerate}[i)]
\item $\norm{f*g}_1 \leq \norm{f}_1 \cdot \norm{g}_1$.
\item Komutativnost.
\item Distributivnost.
\item Asociativnost.
\end{enumerate}
\end{trditev}

\begin{proof}
Po Fubiniju je
\begin{align*}
((f*g)*h)(t) &=
\int_{-\infty}^\infty \left(
\int_{-\infty}^\infty f(u) g(s - u)\;du
\right) \cdot h(t - s)\;ds
\\
&=
\int_{-\infty}^\infty f(u) \left(
\int_{-\infty}^\infty g(s - u) h(t - s)\;ds
\right)\;du
\\
&=
\int_{-\infty}^\infty f(u) (g * h)(t - u)\;du
\\
&= (f*(g*h))(t). \qedhere
\end{align*}
\end{proof}

\begin{trditev}
Fourierova transformacija je homomorfizem algeber\footnote{S
$\mathcal{C}_B(\R)$ označujemo omejene zvezne funkcije.}
\[
\mathcal{F} \colon (L^1(\R),*) \to (\mathcal{C}_B(\R),\cdot),
\]
oziroma
\[
\mathcal{F}(f*g)(\xi) =
\mathcal{F}(f)(\xi) \cdot \mathcal{F}(g)(\xi).
\]
\end{trditev}

\begin{proof}
Po Fubiniju je
\begin{align*}
\mathcal{F}(f*g)(\xi) &=
\int_{-\infty}^\infty e^{-2 \pi i \xi t} (f*g)(t)\;dt
\\
&=
\int_{-\infty}^\infty e^{-2 \pi i \xi t} \left(
\int_{-\infty}^\infty f(s) g(t-s)\;ds
\right)dt
\\
&=
\int_{-\infty}^\infty e^{-2 \pi i \xi s} f(s) \left(
\int_{-\infty}\infty e^{-2 \pi i Xi (t - s)} g(t-s)\;dt
\right)ds
\\
&=
\mathcal{F}(f)(\xi) \cdot \mathcal{F}(g)(\xi). \qedhere
\end{align*}
\end{proof}

\datum{2022-1-12}

\begin{izrek}[Plancherel]\index{Izrek!Plancherel}
Za $f \in L^1(\R) \cap L^2(\R) \cap \mathcal{C}^1(\R)$ velja
\[
\norm{f}_2 = \norm{\hat{f}}_2.
\]
Fourierova transformacija je izometrija na prostoru $L^2(\R)$.
\end{izrek}

\begin{proof}
Naj bo $g(t)=\overline{f(-t)}$. Sledi, da je
\[
\hat{g}(\xi) =
\overline{\int_{-\infty}^\infty e^{2 \pi i \xi t} f(-t)\;dt} =
\overline{\hat{f}(\xi)}.
\]
Tako sledi
\begin{align*}
\int_{-\infty}^\infty \abs{f}^2\;dt
&=
\int_{-\infty}^\infty f(t) \overline{f(t)}\;dt
\\
&=
\int_{-\infty}^\infty f(t) g(-t)\;dt
\\
&=
(f*g)(0)
\\
&=
\lim_{R \to \infty} \int_{-R}^R
\widehat{f*g}(\xi) e^{2 \pi i \xi \cdot 0}\;d\xi
\\
&=
\lim_{R \to \infty} \int_{-R}^R \hat{f}(\xi) \hat{g}(\xi)\;d\xi
\\
&=
\lim_{R \to \infty} \int_{-R}^R \abs{\hat{f}(\xi)}^2\;d\xi
\\
&=
\norm{\hat{f}}_2^2. \qedhere
\end{align*}
\end{proof}
