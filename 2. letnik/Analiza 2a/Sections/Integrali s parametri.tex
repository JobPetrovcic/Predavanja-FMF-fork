\section{Integrali s parametri}

\epigraph{">No glej kakšen revček si."<}{-- doc. dr. Gregor Cigler}

\subsection{Eulerjeva gama}

\datum{2021-11-2}

\begin{definicija}
\emph{Eulerjeva funkcija gama}\index{Eulerjeva gama} je funkcija
\[
\Gamma(s) = \int_0^\infty x^{s-1} e^{-x}\;dx.
\]
\end{definicija}

\begin{trditev}
Velja $\Gamma(s+1) = s \cdot \Gamma(s)$.
\end{trditev}

\begin{proof}
Z integriranjem po delih dobimo
\[
\Gamma(s+1) =
\int_0^\infty x^s e^{-x}\;dx =
\eval{-x^s e^{-x}}{0}{\infty} +
s \cdot \int_0^\infty x^{s-1} e^{-x}\;dx =
s \cdot \Gamma(s). \qedhere
\]
\end{proof}

\begin{posledica}
Velja $\Gamma(n+1) = n!$.
\end{posledica}

\begin{definicija}
\emph{Eulerjeva funkcija beta}\index{Eulerjeva beta} je funkcija
\[
B(p,q) = \int_0^1 x^{p-1}(1-x)^{q-1}\;dx.
\]
\end{definicija}

\newpage

\subsection{Zveznost in odvedljivost integralov s parametri}

\begin{definicija}
Množica $Y \subseteq \R^n$ je
\emph{lokalno kompaktna}\index{Množica!Lokalno kompaktna}, če za
vse $y \in Y$ obstaja tak $r>0$, da je
$Y \cap \overline{\mathcal{K}(y,r)}$ kompaktna.
\end{definicija}

\begin{opomba}
Zaprte in odprte množica so lokalno kompaktne.
\end{opomba}

\begin{opomba}
Zvezne funkcije na lokalno kompaktnih množicah so lokalno omejene in enakomerno zvezne.
\end{opomba}

\begin{izrek}
Naj bo $I = [a,b]$ in $Y \subseteq \R^n$ lokalno kompaktna. Naj bo
$f \colon I \times Y \colon \R$ zvezna. Potem je funkcija
$F \colon I \times I \times Y \to \R$, podana s predpisom
\[
F(u,v,y) = \int_u^v f(x,y)\;dx,
\]
zvezna.
\end{izrek}

\begin{proof}
Ker je $Y$ lokalno kompaktna, obstaja tak $r>0$, da je
$A = Y \cap \overline{\mathcal{K}(y_0,r)}$ kompaktna, zato je
$I \times A$ kompaktna in je $f$ na tej množici enakomerno zvezna.
Naj bo $\varepsilon > 0$. Sledi, da obstaja tak $\delta$, da za vse
$y_1,y_2 \in A$, za katere velja $\norm{y_1 - y_2} < \delta$, velja
\[
\abs{f(x,y_1) - f(x,y_2)} < \frac{\varepsilon}{3 \cdot (b-a)}.
\]
Ker je $I \times A$ kompaktna, je $f$ na $I \times A$ omejena z
$M$. Sedaj za
$\abs{u-u_0},\abs{v-v_0} < \frac{\varepsilon}{3(b-a)}$ dobimo
\begin{align*}
&\abs{F(u,v,y) - F(u_0,v_0,y_0)}
\\
= &\abs{\int_u^v f(x,y)\;dx - \int_{u_0}^{v_0} f(x,y_0)\;dx}
\\
\leq &\abs{\int_u^{u_0} f(x,y)\;dx} +
\abs{\int_v^{v_0} f(x,y)\;dx} +
\abs{\int_{v_0}^{u_0} \left(f(x,y) - f(x,y_0)\right)\;dx}
\\
\leq &M \cdot \abs{u-u_0} + M \cdot \abs{v-v_0} +
\frac{\varepsilon}{3(b-a)} \cdot (b-a) < \varepsilon. \qedhere
\end{align*}
\end{proof}

\begin{posledica}
Naj bo $Y \subseteq \R^n$ lokalno kompaktna in
$f \colon [a,b] \times Y \to \R$ zvezna. Potem je
\[
F(y) = \int_a^b f(x,y)\;dx
\]
zvezna na $Y$.
\end{posledica}

\begin{izrek}\label{iz:odv_int}
Naj bodo $a<b$ in $c<d$ realna in $D = [a,b] \times (c,d)$. Naj bo
$f \colon D \to \R$ zvezna in v vsaki točki $(x,y) \in D$ parcialno
odvedljiva po $y$ z zveznim parcialnim odvodom. Tedaj je
\[
F(y) = \int_a^b f(x,y)\;dx
\]
zvezno odvedljiva in velja
\[
F'(y) = \int_a^b f_y(x,y)\;dx.
\]
\end{izrek}

\begin{proof}
Naj bo $y \in (c,d)$ in $[y-r,y+r] \subseteq (c,d)$. Naj bo
$h \ne 0$, $\abs{h} < r$ in $\varepsilon > 0$. Potem je po
Lagrangevem izreku
\begin{align*}
\abs{\frac{F(y+h) - F(y)}{h} - \int_a^b f_y(x,y)\;dx} &=
\abs{\frac{1}{h} \int_a^b \left(f(x,y+h) - f(x,y)\right)\;dx -
\int_a^b f_y(x,y)\;dx}
\\
&= \abs{\int_a^b \left(f_y(x,y^*) - f_y(x,y)\right)\;dx},
\end{align*}
kjer $y^*$ leži med $y$ in $y+h$. Ker je $f_y$ na
$[a,b] \times [y-r,y+r]$ enakomerno zvezna, obstaja tak
$\delta > 0$, da za $\abs{h} < \delta$ velja
\[
\abs{f_y(x,y^*) - f_y(x,y)} < \frac{\varepsilon}{b-a},
\]
zato je
\[
\abs{\int_a^b \left(f(x,y^*) - f_y(x,y)\right)\;dx} < \varepsilon.
\qedhere
\]
\end{proof}

\begin{posledica}
Naj bosta $\alpha,\beta \colon (c,d) \to [a,b]$ zvezno odvedljivi.
Tedaj je
\[
F(y) = \int_{\alpha(y)}^{\beta(y)} f(x,y)\;dx
\]
zvezno odvedljiva na $(c,d)$ z odvodom
\[
F'(y) = \int_{\alpha(y)}^{\beta(y)} f_y(x,y)\;dx +
\beta'(y) f(\beta(y),y) - \alpha'(y) f(\alpha(y),y).
\]
\end{posledica}

\begin{proof}
Naj bo
\[
\Phi(u,v,y) = \int_u^v f(x,y)\;dx.
\]
Velja
\[
\frac{\partial \Phi}{\partial v} = f(v,y),
\quad
\frac{\partial \Phi}{\partial u} = -f(u,y)
\quad \text{in} \quad
\frac{\partial \Phi}{\partial y} = \int_u^v f_y(x,y)\;dx.
\]
Z odvajanjem zveze $F(y) = \Phi(\alpha(y),\beta(y),y)$ dobimo
iskano enakost.
\end{proof}

\begin{posledica}
Naj bo $D \subseteq \R^n$ odprta in
$f \colon [a,b] \times D \to \R$ zvezna. Naj za vsak
$(x,y) \in [a,b] \times D$ obstajajo zvezni parcialni odvodi
$f_{y_j}(x,y)$. Tedaj je
\[
F(y) = \int_a^b f(x,y)\;dx
\]
$\mathcal{C}^1(D)$ z odvodi
\[
F_{y_j}(y) =  \int_a^b f_{y_j}(x,y)\;dx.
\]
\end{posledica}

\datum{2021-11-9}

\begin{izrek}[Fubini]\index{Izrek!Fubini}\label{iz:fub1}
Naj bo $f \colon [a,b] \times [c,d] \to \R$ zvezna. Potem je
\[
\int_c^d \left(\int_a^b f(x,y)\;dx\right) dy =
\int_a^b \left(\int_c^d f(x,y)\;dy\right) dx.
\]
\end{izrek}

\begin{proof}
Naj bo
\[
\Phi(y) = \int_c^y \left(\int_a^b f(x,t)\;dx\right) dt
\]
in
\[
\Psi(y) = \int_a^b \left(\int_c^y f(x,t)\;dt\right) dx.
\]
Dokazujemo, da je $\Phi \equiv \Psi$. Označimo
\[
g(x,y) = \int_c^y f(x,t)\;dt.
\]
Očitno je $\Phi(c) = \Psi(c) = 0$, zato je dovolj dokazati
$\Phi' \equiv \Psi'$. Velja
\[
\Phi'(y) = \int_a^b f(x,y)\;dx,
\]
po izreku \ref{iz:odv_int} pa dobimo še
\[
\Psi'(y) =
\frac{d}{dy} \left(\int_a^b g(x,y)\;dx\right) =
\int_a^b \frac{\partial g}{\partial y}(x,y)\;dx =
\int_a^b f(x,y)\;dx. \qedhere
\]
\end{proof}

\newpage

\subsection{Posplošeni integrali s parametrom}

\begin{definicija}
Integral
\[
F(y) = \int_a^\infty f(x,y)\;dx
\]
\emph{konvergira enakomerno}\index{Integral!Enakomerna konvergenca}
na $Y$, če za vsak $\varepsilon > 0$ obstaja tak $b_0$, da za vse
$b > b_0$ in $y \in Y$ velja
\[
\abs{\int_b^\infty f(x,y)\;dx} < \varepsilon.
\]
\end{definicija}

\begin{opomba}
Ekvivalentno integrali
\[
F_b(y) = \int_a^b f(x,y)\;dx
\]
konvergirajo enakomerno na $Y$ proti $F$.
\end{opomba}

\begin{trditev}
Naj bo $f \colon [a,\infty) \times Y \to \R$ funkcija, ki je za
vsak $y \in Y$ zvezna kot funkcija $x$. Predpostavimo, da obstaja
taka zvezna funkcija $\varphi \colon [a,\infty) \to [0,\infty)$,
da velja:

\begin{enumerate}[i)]
\item $\forall (x,y) \in [a,\infty) \times Y \colon
\abs{f(x,y)} \leq \varphi(x)$ in
\item $\displaystyle \int_a^\infty \varphi(x)\;dx < \infty$
obstaja.
\end{enumerate}

Tedaj
\[
F(y) = \int_a^\infty f(x,y)\;dx
\]
konvergira enakomerno na $Y$.
\end{trditev}

\begin{proof}
Velja
\[
\abs{\int_b^\infty f(x,y)\;dx} \leq
\int_b^\infty \abs{f(x,y)}\;dx \leq
\int_b^\infty \varphi(x)\;dx. \qedhere
\]
\end{proof}

\begin{opomba}
Zveznost in odvedljivost sta lokalni lastnosti, zato je pogosto
dovolj zahtevati \emph{lokalno enakomerno konvergenco}: za $Y$, ki
so lokalno kompaktne, je to ekvivalentno:

\begin{enumerate}[i)]
\item za vse $y \in Y$ obstaja tak $r > 0$, da na
$\mathcal{K}(y,r)$ integral konvergira enakomerno ali
\item na kompaktnih podmnožicah $Y$ imamo enakomerno konvergenco.
\end{enumerate}
\end{opomba}

\begin{izrek}
Naj bo $Y \subseteq \R^n$ lokalno kompaktna in
$f \colon [a,\infty) \times Y \to \R$ zvezna. Če integral
\[
F(y) = \int_a^\infty f(x,y)\;dx
\]
konvergira lokalno enakomerno na $Y$, je $F$ zvezna na $Y$.
\end{izrek}

\begin{proof}
Označimo
\[
F_b(y) = \int_a^b f(x,y)\;dx.
\]
Te funkcije so zvezne na $Y$ in konvergirajo lokalno enakomerno na
$Y$ k $F$. Sledi, da je $F$ lokalno zvezna, zato je zvezna.
\end{proof}

\begin{posledica}
$\Gamma$ je zvezna.
\end{posledica}

\begin{izrek}[Fubini]\index{Izrek!Fubini}
Naj bo $f \colon [a,\infty) \times [c,d] \to \R$ zvezna in
\[
F(y) = \int_a^\infty f(x,y)\;dx
\]
konvergira enakomerno na $[c,d]$. Tedaj velja
\[
\int_c^d \left(F(y)\right)\;dy =
\int_a^\infty \left(\int_c^d f(x,y)\;dy\right) dx.
\]
\end{izrek}

\begin{proof}
Vemo, da
\[
\lim_{b \to \infty} F_b(y) = F(y)
\]
konvergira enakomerno na $[c,d]$. Po izreku \ref{iz:fub1} sledi, da
je
\begin{align*}
\int_c^d F(y)\;dy &=
\lim_{b \to \infty} \int_c^d F_b(y)\;dy
\\
&=
\lim_{b \to \infty} \int_a^b \left(\int_c^d f(x,y)\;dy\right) dx
\\
&=
\int_a^\infty \left(\int_c^d f(x,y)\;dy\right) dx. \qedhere
\end{align*}
\end{proof}

\begin{izrek}[Fubini]
Naj bo $f \colon [a,\infty) \times [c,\infty) \to [0,\infty)$
zvezna. Predpostavimo, da integral
\[
F(y) = \int_a^\infty f(x,y)\;dx
\]
konvergira enakomerno na $[c,\infty)$ in integral
\[
G(x) = \int_c^\infty f(x,y)\;dy
\]
konvergira enakomerno na $[a,\infty)$. Če je
\[
\int_c^\infty F(y)\;dy < \infty,
\]
je tudi
\[
\int_a^\infty G(x)\;dx < \infty,
\]
in velja
\[
\int_c^\infty F(y)\;dy = \int_a^\infty G(x)\;dx.
\]
\end{izrek}

\begin{proof}
Velja
\begin{align*}
\int_a^b G(x)\;dx &=
\int_a^b \left(\int_c^\infty f(x,y)\;dy \right)dx
\\
&=
\int_c^\infty \left(\int_a^b f(x,y)\;dx\right)dy
\\
&\leq
\int_c^\infty F(y)\;dy < \infty. \qedhere
\end{align*}
\end{proof}

\begin{izrek}[Fubini]
Naj bo $f \colon [a,\infty) \times [c,\infty) \to \R$ zvezna.
Predpostavimo, da je
\[
y \mapsto \int_a^\infty \abs{f(x,y)}\;dx
\]
lokalno enakomerno konvergenten na $[c,\infty)$ in
\[
x \mapsto \int_c^\infty \abs{f(x,y)}\;dy
\]
lokalno enakomerno konvergenten na $[a,\infty)$. Naj bo
\[
\int_c^\infty \left(\int_a^\infty \abs{f(x,y)}\;dx\right)dy < \infty.
\]
Tedaj je
\[
\int_c^\infty \left(\int_a^\infty f(x,y)\;dx\right)dy =
\int_a^\infty \left(\int_c^\infty f(x,y)\;dy\right)dx.
\]
\end{izrek}

\begin{proof}
Ker je
\[
\abs{\int_b^\infty f(x,y)\;dx} \leq \int_b^\infty \abs{f(x,y)}\;dx,
\]
tudi
\[
F(y) = \int_a^\infty f(x,y)\;dx
\]
konvergira lokalno enakomerno na $[c,\infty)$ in
\[
G(y) = \int_c^\infty f(x,y)\;dy
\]
konvergira lokalno enakomerno na $[a,\infty)$. Sledi, da je
\[
\lim_{b \to \infty}\int_a^b G(x)\;dx =
\lim_{b \to \infty}
\int_c^\infty F_b(y)\;dy.
\]
Velja
\begin{align*}
\abs{\int_d^\infty F_b(y)\;dy} &\leq
\int_d^\infty \abs{F_b(y)}\;dy
\\
&=
\int_d^\infty \abs{\int_a^b f(x,y)\;dx}\;dy
\\
&\leq
\int_d^\infty \left(\int_a^b \abs{f(x,y)}\;dx\right)dy
\\
&\leq
\int_d^\infty \left(\int_a^\infty \abs{f(x,y)}\;dx\right)dy.
\end{align*}
Za vse $\varepsilon > 0$ lahko najdemo tak $d$, da je
\[
\abs{\int_d^\infty F(y)\;dy} \leq
\int_d^\infty \left(\int_a^\infty \abs{f(x,y)}\;dx\right)dy <
\frac{\varepsilon}{3}.
\]
Sledi, da je
\begin{align*}
&\abs{\int_c^\infty F_b(y)\;dy - \int_c^\infty F(y)\;dy}
\\
\leq&
\abs{\int_d^\infty F_b(y)\;dy} +
\abs{\int_d^\infty F(y)\;dy} +
\abs{\int_c^d F_b(y)\;dy - \int_c^d F(y)\;dy}
\\
<&
\varepsilon. \qedhere
\end{align*}
\end{proof}

\datum{2021-11-10}

\begin{izrek}
Naj bo $f \colon [a,\infty) \times (c,d) \to \R$ zvezna in v vsaki
točki parcialno odvedljiva glede na $y$ z odvodom $f_y$, ki je prav
tako zvezen na $[a,\infty) \times (c,d)$. Naj
\[
F(y) = \int_a^\infty f(x,y)\;dx
\]
obstaja za vse $y \in (c,d)$. Naj
\[
y \mapsto \int_a^\infty f_y(x,y)\;dx
\]
konvergira lokalno enakomerno na $(c,d)$. Potem je
$F \in \mathcal{C}^1((c,d))$ in velja
\[
F'(y) = \int_a^\infty f_y(x,y)\;dx.
\]
\end{izrek}

\begin{proof}
Označimo
\[
G(y) = \int_a^\infty f_y(x,y)\;dx
\]
in naj bo $y_0 \in (c,d)$. Definiramo funkcijo
\[
\Phi(y) = \int_{y_0}^y G(t)\;dt.
\]
Ker je $G$ zvezna, je $\Phi \in \mathcal{C}^1((c,d))$. Sledi, da je
\begin{align*}
\Phi(y) &= \int_{y_0}^y \left(\int_a^\infty f_y(x,t)\;dx\right)dt
\\
&=
\int_a^\infty \left(\int_{y_0}^y f_y(x,t)\;dt\right)dx
\\
&=
\int_a^\infty (f(x,y) - f(x,y_0))\;dx = F(y) - F(y_0). \qedhere
\end{align*}
\end{proof}

\begin{posledica}
Naj bo $D \subseteq \R^n$ odprta in
$f \colon [a,\infty) \times D \to \R$. Naj obstajajo zvezni
parcialni odvodi $f_{y_i}$ za vse $i$. Naj
\[
F(y) = \int_a^\infty f(x,y)\;dx 
\]
obstaja za vse $y \in D$ in naj bodo
\[
G_i = \int_a^\infty f_{y_i}(x,y)\;dx
\]
lokalno enakomerno konvergentni za vse $i$. Tedaj je
$F \in \mathcal{C}^1(D)$ in velja
\[
\frac{\partial F}{\partial y_i}(y) =
\int_a^\infty f_{y_i}(x,y)\;dx.
\]
\end{posledica}

\begin{izrek}
Velja
\[
\int_0^\infty \frac{\sin x}{x}\;dx = \frac{\pi}{2}.
\]
\end{izrek}

\begin{proof}
Poglejmo
\[
F(a) = \int_0^\infty e^{-ax} \frac{\sin x}{x}\;dx.
\]
$F$ konvergira enakomerno na $[c,\infty)$, saj je
\[
\abs{e^{-ax} \frac{\sin x}{x}} \leq e^{-ax} \leq e^{-cx}.
\]
Naj bo $0 \leq \alpha < \beta$, $f(x) = \sin x$ in
$g(x) = \frac{e^{-ax}}{x}$. Po izreku~5.2.25 iz zapiskov Analize 1
obstaja tak $\gamma$, da je
\[
\int_\alpha^\beta \frac{e^{-ax}}{x} \sin x\;dx =
\frac{e^{-a\alpha}}{\alpha} \int_\alpha^\gamma \sin x\;dx =
\frac{e^{-a\alpha}}{\alpha} \eval{(-\cos x)}{\alpha}{\gamma}.
\]
Sledi, da je
\[
\abs{\int_\alpha^\beta \frac{e^{-ax}}{x} \sin x\;dx} \leq
\frac{2}{\alpha},
\]
zato $F$ konvergira enakomerno na $[0,\infty)$ in je zvezna. Sledi
\[
\int_0^\infty \frac{\sin x}{x}\;dx = F(0) = \lim_{a \to 0} F(a).
\]
Kandidat za odvod $F$ je
\[
-\int_0^\infty e^{-ax} \sin x\;dx,
\]
ta integral pa je lokalno enakomerno konvergenten na $(0,\infty)$,
saj za $0 < c \leq a$ velja
\[
\abs{e^{-a} \sin x} \leq e^{-cx}.
\]
Sledi, da je $F$ res odvedljiva z zgornjim odvodom, ki ga lahko
integriramo po delih in dobimo
\[
F'(a) = -\frac{1}{a^2 + 1}.
\]
Sledi, da je
\[
F(a) = -\arctan(a) + C
\]
in
\[
0 \leq \abs{F(a)} \leq \int_0^\infty e^{-ax}\;dx = \frac{1}{a},
\]
zato je
\[
\lim_{a \to \infty} F(a) = 0
\]
in $C = \frac{\pi}{2} = F(0)$.
\end{proof}

\begin{posledica}
Velja
\[
\frac{2}{\pi} \int_0^\infty \frac{\sin (ax)}{x}\;dx = \sgn(a).
\]
\end{posledica}

\begin{trditev}
Za $k \in \N$ velja
\[
\Gamma^{(k)}(s) = \int_0^\infty x^{s-1} (\ln x)^k e^{-x}\;dx.
\]
\end{trditev}

\begin{proof}
Velja, da je $\Gamma$ zvezna in na $(0,\infty)$ konvergira
enakomerno. Velja pa, da
\[
\int_0^1 x^{s-1} (\ln x)^k e^{-x}\;dx
\]
konvergira lokalno enakomerno na $(0,\infty)$,
\[
\int_1^\infty x^{s-1} (\ln x)^k e^{-x}\;dx
\]
pa lokalno enakomerno na $\R$.
\end{proof}

\begin{posledica}
$\Gamma$ je gladka.
\end{posledica}

\begin{posledica}
$\Gamma$ je konveksna.
\end{posledica}

\begin{posledica}
$\ln\Gamma$ je konveksna.
\end{posledica}

\begin{proof}
Velja
\[
\Gamma\left(\frac{s_1 + s_2}{2}\right) =
\int_0^\infty \left(x^{\frac{s_1-1}{2}} e^{\frac{x}{2}}\right)
\cdot \left(x^{\frac{s_1-1}{2}} e^{\frac{x}{2}}\right)dx.
\]
Po Cauchyju\footnote{Vzamemo skalarni produkt
$\displaystyle\skl{f,g} = \abs{\int_0^\infty f(x)g(x)\;dx}$.}
sledi
\[
\Gamma\left(\frac{s_1 + s_2}{2}\right) \leq
\sqrt{\Gamma(s_1)} \cdot \sqrt{\Gamma(s_2)}. \qedhere
\]
\end{proof}

\begin{opomba}
Funkcija $\Gamma$ je enolično določena z naslednjimi lastnostmi:

\begin{enumerate}[i)]
\item $\Gamma(1) = 1$,
\item $\Gamma(s+1) = s \cdot \Gamma(s)$,
\item $\Gamma > 0$,
\item $\Gamma \in \mathcal{C}^1(\R^+)$ in
\item $\ln\Gamma$ je konveksna.
\end{enumerate}
\end{opomba}

\begin{trditev}
Velja
\[
\int_0^{\frac{\pi}{2}} \sin^\alpha(t) \cos^\beta(t)\;dt =
\frac{1}{2} \cdot
B\left(\frac{\alpha+1}{2}, \frac{\beta+1}{2}\right).
\]
\end{trditev}

\begin{proof}
Naredimo substitucijo $x = \sin^2 t$.
\end{proof}

\begin{trditev}
Velja
\[
B(p,q) = \int_0^\infty \frac{t^{p-1}}{(1+t)^{p+q}}\;dt.
\]
\end{trditev}

\begin{proof}
Naredimo substitucijo $t = \frac{x}{1-x}$.
\end{proof}

\begin{izrek}
Velja
\[
B(p,q) = \frac{\Gamma(p) \cdot \Gamma(q)}{\Gamma(p+q)}.
\]
\end{izrek}

\begin{proof}
Velja
\begin{align*}
B(p,q) \cdot \Gamma(p+q) &=
\left(\int_0^\infty \frac{t^{p-1}}{(1+t)^{p+q}}\;dt\right) \cdot
\left(\int_0^\infty x^{p+q-1} e^{-x}\;dx\right)
\\
&=
\int_0^\infty \left(\int_0^\infty x^{p+q-1} e^{-x}\;dx\right)
\frac{t^{p-1}}{(1+t)^{p+q}}\;dt
\\
&=
\int_0^\infty \left(\int_0^\infty
\left(\frac{x}{1+t}\right)^{p+q-1}\frac{t^{p-1} e^{-x}}{1+t}\;dx
\right)dt
\intertext{Sedaj naredimo substitucijo $x = (1+t)u$. Sledi, da je
zgornji izraz enak}
&=
\int_0^\infty \left(\int_0^\infty
u^{p+q-1} t^{p-1} e^{-u} e^{-tu}\;dx \right)dt
\intertext{S Fubinijevim izrekom dobimo}
&=
\int_0^\infty \left(
\int_0^\infty u^{p+q-1} t^{p-1} e^{-u} e^{-tu}\;dt
\right)dx
\\
&=
\int_0^\infty u^{s-1} e^{-u} \left(\int_0^\infty (ut)^{p-1} e^{-ut} u\;dt\right)du
\intertext{Sedaj naredimo še substitucijo $ut=v$}
&=
\int_0^\infty u^{q-1}e^{-u} \Gamma(p)\;du
\\
&=
\Gamma(p) \cdot \Gamma(q). \qedhere
\end{align*}
\end{proof}

\begin{posledica}
Velja
\[
\Gamma\left(\frac{1}{2}\right) = \sqrt{\pi}.
\]
\end{posledica}

\begin{posledica}
Velja
\[
\frac{1}{\sqrt{2\pi}\sigma}
\int_{-\infty}^{\infty} e^{-\frac{(x-a)^2}{2\sigma^2}}\;dx = 1.
\]
\end{posledica}

\begin{trditev}
Velja
\[
\lim_{s \to \infty} \frac{\Gamma(s+1)}{s^s e^{-s}\sqrt{2\pi s}} = 1.
\]
\end{trditev}

\datum{2021-11-16}

\begin{proof}
Razpišimo
\[
\Gamma(s+1) = \int_0^\infty x^s e^{-x}\;dx.
\]
Odvod funkcije $x \mapsto x^s e^{-x}$ je enak $x^{s-1}e^{-x}(s-x)$.
S substitucijo $x = (1+u)s$ dobimo
\[
\Gamma(s+1) =
\int_{-1}^{\infty} (1+u)^s s^s e^{-s} e^{-su}\;du =
\left(s^s e^{-s} \sqrt{s}\right) \sqrt{s} \cdot
\int_{-1}^\infty \left((1+u)e^{-u}\right)^s du.
\]
Sledi, da je
\[
\frac{\Gamma(s+1)}{s^s e^{-s}\sqrt{s}} =
\sqrt{s} \cdot \int_{-1}^\infty \left((1+u)e^{-u}\right)^s du.
\]
Naj bo
\[
\varphi(u) = \frac{\ln((1+u)e^{-u}) + \frac{u^2}{2}}{u^3}.
\]
Potem s substitucijo $v = u \sqrt{s}$ dobimo
\[
\sqrt{s} \cdot \int_{-1}^\infty \left((1+u)e^{-u}\right)^s du =
\sqrt{s} \cdot \int_{-1}^\infty
e^{-\frac{su^2}{2}} \cdot e^{su^3\varphi(u)}\;du =
\int_{-\sqrt{s}}^{\infty}
e^{-\frac{v^2}{2}} e^{\frac{v^3}{\sqrt{s}} \cdot \varphi\left(
\frac{v}{\sqrt{s}}
\right)}\;dv.
\]
Ocenimo lahko
\[
\sqrt{s} \int_1^\infty \left((1+u)e^{-u}\right)du \leq
\sqrt{s} 2^{s-1} e^{-(s-1)} M.
\]
Sledi, da ta integral konvergira k $0$, zato je dovolj izračunati
\[
\lim_{s \to \infty}
\sqrt{s} \int_{-1}^{1} \left((1+u)e^{-u}\right)du =
\lim_{s \to \infty} \sqrt{s} \int_{-1}^{1}
e^{-\frac{su^2}{2}} \cdot e^{su^3\varphi(u)}\;du =
\lim_{s \to \infty} \int_{-\sqrt{s}}^{\sqrt{s}}
e^{-\frac{v^2}{2}} e^{\frac{v^3}{\sqrt{s}} \cdot \varphi\left(
\frac{v}{\sqrt{s}}
\right)}\;dv.
\]
Označimo $s = r^{10}$. Velja, da je
\[
\int_{-r^5}^{-r}
e^{-\frac{v^2}{2}} e^{\frac{v^3}{r^5} \cdot \varphi\left(
\frac{v}{r^5}
\right)}\;dv \leq e^{-\frac{r^2}{2}} e^{-\frac{1}{r^2} \varphi
\left(-\frac{1}{r^4}\right)} r^5,
\]
kar limitira proti 0. Podobno ocenimo integral na mejah od $r$ do
$r^5$. Opazimo še
\[
e^{-\frac{M}{r^2}} \int_{-r}^{r} e^{-\frac{v^2}{2}}\;dv \leq
\int_{-r}^{r}
e^{-\frac{v^2}{2}} e^{\frac{v^3}{r^5} \cdot \varphi\left(
\frac{v}{r^5}
\right)}\;dv \leq
e^{\frac{M}{r^2}} \int_{-r}^{r} e^{-\frac{v^2}{2}}\;dv,
\]
zato integral limitira proti $\sqrt{2\pi}$.
\end{proof}

\newpage

\subsection{Riemannov integral}

\begin{definicija}
\emph{Delitev kvadra}\index{Delitev kvadra} je razdelitev na kvadre
\[
\prod_{j=1}^n \left[x_{l_j-1}^j, x_{l_j}^j\right],
\]
kjer je za vsaj $j$
\[
a_j = x_0^j < x_1^j < \dots < x_{m_j}^j = b_j.
\]
\end{definicija}

\begin{definicija}
Delitev $D'$ je \emph{finejša}\index{Delitev kvadra!Finejša}, če
vsebuje vse delilne točke $D$.
\end{definicija}

\begin{definicija}
Naj bo $K$ kvader in $f \colon K \to \R^n$ omejena funkcija.
Označimo
$\displaystyle m(k) = \inf_k f$ in $\displaystyle M(k) = \sup_k f$.
Naj bo $D$ delitev $K$ s kvadri $K_1,\dots,K_N$.
\emph{Spodnja Darbouxjeva vsota}\index{Darbouxjeva vsota} je
\[
s(f,D) = s(D) = \sum_{i=1}^N m(K_i) \cdot V(K_i),
\]
\emph{zgornja Darbouxjeva vsota} pa
\[
S(f,D) = S(D) = \sum_{i=1}^N M(K_i) \cdot V(K_i).
\]
\end{definicija}

\begin{trditev}
Naj bo $D'$ finejša od $D$. Potem je
\[
s(D) \leq s(D') \leq S(D') \leq S(D).
\]
\end{trditev}

\begin{proof}
Indukcija.
\end{proof}

\begin{trditev}
Naj bosta $D_1$ in $D_2$ delitvi $K$. Potem je
\[
s(D_1) \leq S(D_2).
\]
\end{trditev}

\begin{proof}
Vzamemo delitev, ki je finejša od obeh.
\end{proof}

\begin{posledica}
Obstajata
\[
S = \inf S(D) \quad \text{in} \quad s = \sup s(D).
\]
\end{posledica}

\begin{definicija}
Omejena funkcija $f \colon K \to \R$ je na $K$
\emph{integrabilna po Darbouxxju}\index{Darbouxjev integral}, če
velja
\[
I_D = s = S.
\]
Označimo
\[
I_D = \lint_K f(x)\;dx = \lint_K f(x)\;dV(x) =
\lidotsint_K f(x_1,\dots,x_n)\;dx_n\cdots dx_1.
\]
\end{definicija}

\begin{definicija}
Naj bo $D$ delitev kvadra in
\[
\xi = \setb{\xi_j \in k_j}{\text{$k_j$ je kvader v $D$}}.
\]
\emph{Riemannovo vsoto}\index{Riemannova vsota} definiramo kot
\[
R(f,D,\xi) = \sum_{j=1}^n f(\xi_j) \cdot V(k_j).
\]
\end{definicija}

\begin{definicija}
Funkcija $f$ je na $K$ \emph{Riemannovo integrabilna}, če obstaja
limita njenih Riemannovih vsot
\[
I_R = \lim_{\delta \to 0} R(f,D,\xi),
\]
kjer je $\delta$ največja stranica kvadrov v delitvi.
\end{definicija}

\datum{2021-11-17}

\begin{izrek}
Naj bo $f \colon K \to \R$ omejena. Naslednje trditve so
ekvivalentne:

\begin{enumerate}[i)]
\item $f$ je integrabilna po Darbouxju
\item $f$ je integrabilna po Riemannu
\item za vse $\varepsilon > 0$ obstaja taka delitev $D$, da je
$S(D) - s(D) < \varepsilon$.
\end{enumerate}

Če integrala obstajata, sta enaka.
\end{izrek}

\begin{proof}
Tretja točka je očitno ekvivalentna prvi. Sedaj redpostavimo, da je
$f$ Riemannovo integrabilna z integralom $I_R$. Naj bo
$\varepsilon > 0$. Potem obstaja tak $\delta > 0$, da za vse
delitve, za katere so vsi intervali krajši od $\delta$, in izbore
točk velja
\[
\abs{R(f,D,\xi) - I_R} < \frac{\varepsilon}{3}.
\]
Sedaj točke v $\xi$ izbiramo tako, da se približujemo supremumu na
vsakem kvadru. Sledi, da je v limiti
\[
\abs{S(D) - I_R} \leq \frac{\varepsilon}{3}.
\]
Podobno dobimo
\[
\abs{s(D) - I_R} \leq \frac{\varepsilon}{3},
\]
zato je
\[
S(D) - s(D) < \frac{\varepsilon}{2}.
\]
Sledi, da je $f$ integrabilna po Darbouxu z enakim integralom.

Predpostavimo še, da je $f$ integrabilna po Darbouxju.

\begin{lema*}
Naj bo $D_0$ delitev $K$ in $\varepsilon > 0$. Potem obstaja tak
$\delta > 0$, da za vsako delitev $D$, ki ima vse robove krajše od
$\delta$, velja, da je vsota prostornin tistih kvadrov delitve $D$,
ki niso vsebovani v kakšen kvadru v $D_0$, manjša od $\varepsilon$.
\end{lema*}

\begin{proof}
Vzamemo
\[
\delta = \frac{\varepsilon}{V(K) \cdot \sum \frac{N_i-1}{b_i-a_i}}.
\qedhere
\]
\end{proof}

Naj bo $\varepsilon > 0$ in $\abs{f} \leq M$. Obstaja taka delitev
$D_0$, da je
\[
I_D - \frac{\varepsilon}{2} < s(D_0) \leq
I_D \leq S(D_0) < I_D + \frac{\varepsilon}{2}.
\]
Sedaj uporabimo zgornjo lemo za $\frac{\varepsilon}{2M}$. Naj bo
$D$ delitev s kvadri $K_1,\dots,K_N$, ki imajo vse stranice manjše
od $\delta$, kjer so $K_1,\dots,K_m$ tisti, ki niso vsebovani v
nobenem kvadru $D_0$. Sledi, da je
\[
\abs{\sum_{i=1}^m f(\xi_i) \cdot V(K_i)} \leq
\abs{\sum_{i=1}^m f(\xi_i)} \cdot V(K_i) <
M \frac{\varepsilon}{2M} = \frac{\varepsilon}{2}.
\]
V vseh ostalih kvadrih lahko $f$ omejimo s supremumom in infimumom
kvadrov delitve $D_0$. Sledi, da je
\[
s(D_0) \leq \sum_{i=m+1}^N f(\xi_i) \cdot V(K_i) \leq S(D_0).
\]
S trikotniško neenakostjo dobimo
\[
\abs{R(f,D,\xi) - I_D} \leq
\abs{\sum_{i=1}^m f(\xi_i) \cdot V(K_i)} +
\abs{\sum_{i=m+1}^N f(\xi_i) \cdot V(K_i) - I_D} <
\frac{\varepsilon}{2} + \frac{\varepsilon}{2} = \varepsilon.
\qedhere
\]
\end{proof}

\begin{izrek}
Naj bo $f \colon K \to \R$ zvezna. Potem je $f$ integrabilna.
\end{izrek}

\begin{proof}
$K$ je kompaktna, zato je $f$ enakomerno zvezna na $K$. Naj bo
$\varepsilon > 0$. Potem obstaja tak $\delta > 0$, da velja
\[
\max \setb{\abs{x_i - y_i}}{i \leq n} < \delta \implies
\abs{f(x) - f(y)} \leq \frac{\varepsilon}{V(K)}.
\]
Za delitev $D$ z robovi, ki so krajši od $\delta$, tako dobimo
\begin{align*}
S(D) - s(D) &= \sum_{i=1}^N (M(K_i) - m(K_i)) \cdot V(K_i)
\\
&= \sum_{i=1}^N (f(x_i) - f(y_i)) \cdot V(K_i)
\\
&< \frac{\varepsilon}{V(K)} \cdot V(K) = \varepsilon. \qedhere
\end{align*}
\end{proof}

\begin{trditev}
Veljajo naslednje lastnosti:

\begin{enumerate}[i)]
\item Integrabilne funkcije tvorijo vektorski prostor nad $\R$,
integral pa je linearen funkcional.
\item Če sta $f \leq g$ integrabilni, je
$\lint_K f(x)\;dx \leq \lint_K g(x)\;dx$.
\item $\abs{f}$ je integrabilna in velja
$\abs{\lint_K f(x)\;dx} \leq \lint_K \abs{f(x)}\;dx$.
\end{enumerate}
\end{trditev}

\obvs

\begin{izrek}[Fubini]\index{Izrek!Fubini}
Naj bosta $A \subseteq \R^n$ in $B \subseteq \R^m$ kvadra in
$f \colon A \times B \to \R$ integrabilna. Denimo, da je za vse
$x \in A$ funkcija $f^x \colon y \mapsto f(x,y)$ integrabilna na
$B$. Tedaj je na $A$ integrabilna funkcija
$x \mapsto \lint_B f(x,y)\;dy$ in velja
\[
\liint_{A \times B} f(x,y)\;dx\;dy =
\lint_A \left(\lint_B f(x,y)\;dy\right) dx.
\]
\end{izrek}

\datum{2021-11-23}

\begin{proof}
Naj bo $D_A$ delitev množice $A$ na kvadre $A_1,\dots,A_N$ in $D_B$
delitev množice $B$ na kvadre $B_1,\dots,B_M$. Potem je
$K_{i,j} = A_i \times B_j$ delitev $A \times B$. Naj bo
\[
m_{i,j} = \inf_{K_{i,j}} f
\quad \text{in} \quad
M_{i,j} = \sup_{K_{i,j}} f
\]
Sledi, da je
\[
m_{i,j} \leq \inf_{B_j} f^x(y) \leq \sup_{B_j} f^x(y) \leq M_{i,j}.
\]
Tako dobimo
\[
\sum_{j=1}^M m_{i,j} \cdot V(B_j) \leq
s(f^x,D_B) \leq
S(f^x,D_B) \leq
\sum_{j=1}^M M_{i,j} \cdot V(B_j)
\]
Naj bo $g(x) = \lint_B f^x(y)\;dy$. Sledi, da je
\[
\sum_{j=1}^M m_{i,j} \cdot V(B_j) \leq
\inf_{A_i} g(x) \leq
\sup_{A_i} g(x) \leq
\sum_{j=1}^M M_{i,j} \cdot V(B_j).
\]
Sledi, da je
\[
s(f,D_A \times D_B) \leq
s(g,D_A) \leq
S(g,D_A) \leq
s(f,D_A \times D_B).
\]
Sledi, da je $g$ integrabilna na $A$, integral pa je enak integralu
$f$ na $A \times B$.
\end{proof}

\begin{posledica}
Naj bo $f \colon A \times B \to \R$ zvezna. Tedaj je
\[
\lint_A \left(\lint_B f(x,y)\;dy\right) dx =
\lint_B \left(\lint_A f(x,y)\;dx\right) dy.
\]
\end{posledica}

\newpage

\subsection{Prostornina}

\begin{definicija}
Naj bo $A \subseteq \R^n$ množica, omejena s kvadrom $K$, in
$f \colon A \to \R$ omejena. Naj bo
\[
\widetilde{f}(x) = \begin{cases}
f(x), &x \in A
\\
0, &x \not \in A.
\end{cases}
\]
Pravimo, da je $f$ \emph{integrabilna} na $A$, če obstaja integral
\[
\lint_A f(x)\;dx = \lint_K \widetilde{f}(x)\;dx.
\]
\end{definicija}

\begin{definicija}
\emph{Karakteristična funkcija}\index{Karakteristična funkcija}
množice $A$ je funkcija
\[
\chi_A(x) = \begin{cases}
1, &x \in A
\\
0, &x \not \in A.
\end{cases}
\]
\end{definicija}

\begin{definicija}
Omejena množica $A \subseteq \R^n$ ima
\emph{Jordanovo prostornino}\index{Jordanova prostornina}, če je
$1$ integrabilna na $A$. Tedaj označimo
\[
V(A) = \lint_A 1\;dx.
\]
\end{definicija}

\begin{opomba}
Za $a, b \in \R^n$ velja
\[
V([a,b]) = V([a,b)) = V((a,b]) = V((a,b)).
\]
\end{opomba}

\begin{trditev}
Naj bo $B \subseteq \R^n$ omejena. Množica $B$ ima prostornino $0$
natanko tedaj, ko za vse $\varepsilon > 0$ obstajajo taki kvadri
$Q_1,\dots,Q_N$, da velja
\[
B \subseteq \bigcup_{i=1}^n Q_i
\quad \text{in} \quad
\sum_{i=1}^n V(Q_i) < \varepsilon.
\]
\end{trditev}

\begin{proof}
Če ima $B$ prostornino $0$, preprosto vzamemo tiste kvadre iz
dovolj fine delitve v Darbouxjevi vsoti, ki vsebujejo kakšno točko
iz $B$.

Naj bo $C$ unija kvadrov $Q_i$. Sledi, da je
\[
\inf_D S(\chi_B,D) \leq
\inf_D S(\chi_C,D) \leq
V(C) < \varepsilon. \qedhere
\]
\end{proof}

\begin{trditev}
Naj bodo $B_1,\dots,B_M$ množice s prostornino $0$. Tedaj ima
njihova unija prostornino $0$.
\end{trditev}

\obvs

\begin{trditev}
Naj bo $K \subseteq \R^n$ kvader in $f \colon K \to \R$
integrabilna. Tedaj ima njen graf prostornino $0$ v $\R^{n+1}$.
\end{trditev}

\begin{proof}
Naj bo $\varepsilon > 0$. Potem obstaja taka delitev $D$ kvadra
$K$, da je $S(D) - s(D) < \varepsilon$. Sedaj preprosto izberemo
pokritje
\[
K_j \times [m(K_j), M(K_J)],
\]
kjer so $K_j$ kvadri delitve $D$.
\end{proof}

\datum{2021-11-24}

\begin{trditev}\label{td:3}
Omejena množica $A \subseteq \R^n$ ima prostornino natanko tedaj,
ko je
\[
V(\partial A) = 0.
\]
\end{trditev}

\begin{proof}
Velja, da je
\[
S(\chi_A,D) - s(\chi_A,D) \leq
S(\chi_{\partial A},D) - s(\chi_{\partial A},D),
\]
zato je $V(\partial A) = 0$ zadosten pogoj.

Naj obstaja $V(A)$ in naj bo $\varepsilon > 0$. Sledi, da obstaja
delitev $D$, za katero je
\[
S(\chi_A,D) - s(\chi_A,D) < \frac{\varepsilon}{6^n}.
\]
Sedaj lahko vzamemo tako finejšo delitev, da je razmerje med
najdaljšim in najkrajšim robom kvadrov manjše od $2$. Sledi, da za
vsaka dva kvadra $K_1$ in $K_2$ delitve $D'$ velja
\[
V(K_1) \leq 2^n V(K_2).
\]
Naj bo $K_0$ kvader delitve $D$, ki leži v notranjosti $K$. Vidimo,
da ima okolico, sestavljeno iz $3^n$ sosednjih kvadrov.

Zdaj vzamemo tak $K_0$, da je $K_0 \cap \partial A \ne \emptyset$.
S $\widetilde{K}_0$ označimo unijo njegovih $3^n$ sosednjih
kvadrov. Sledi, da $\widetilde{K}_0$ seka tako $A$ kot
$A^\mathsf{c}$, zato vsaj eden izmed teh $3^n$ kvadrov seka tako
$A$ kot $A^\mathsf{c}$ -- označimo ga z $K_0'$. Sledi, da je
\[
S(\chi_{\partial A},D) =
\sum_{K_0 \cap \partial A \ne \emptyset} V(K_0) \leq
2^n \cdot \sum_{K_0 \cap \partial A \ne \emptyset} V(K_0') \leq
2^n \cdot 3^n \sum_{\substack{
	K' \cap A \ne \emptyset \\
	K' \cap A^\mathsf{c} \ne \emptyset}
} V(K') <
6^n \cdot \frac{\varepsilon}{6^n} = \varepsilon. \qedhere
\]
\end{proof}

\begin{trditev}
Naj ima $A \subseteq K$ prostornino, kjer je $K$ kvader. Tedaj ima
$A^\mathsf{c}$ prostornino in velja
\[
V(A^\mathsf{c}) = V(K) - V(A).
\]
\end{trditev}

\begin{proof}
$1 - \chi_A$ je integrabilna na $K$.
\end{proof}

\begin{trditev}
Naj bo $A \subseteq K$ množica s prostornino $0$ in
$f \colon K \to \R$ omejena in taka, da za vse
$x \in K \setminus A$ velja $f(x) = 0$. Potem je $f$ integrabilna
na $K$ in je
\[
\lint_K f(x)\;dx = 0.
\]
\end{trditev}

\begin{proof}
Naj bo $\varepsilon > 0$ in $M$ tak, da za vse $x \in K$ velja
$\abs{f(x)} \leq M$. Ker je $V(A) = 0$, obstajajo kvadri
$Q_1,\dots,Q_N$ s skupno prostornino največ
$\frac{\varepsilon}{2M}$, ki pokrijejo $A$. Brez škode za splošnost
je $A$ v notranjosti njihove unije. Sedaj poglejmo delitev $D$
kvadra $K$, ki jo dobimo s projekcijo vseh oglišč na vsako
koordinatno os. Dobimo, da je
\begin{align*}
S(f,D) &=
\sum M(K_j) \cdot V(K_j)
\\
&= \sum_{\exists i \colon K_j \subseteq Q_i} M(K_j) \cdot V(K_j)
\\
&\leq M \cdot \sum_{i=1}^N V(Q_i)
\\
&< \frac{\varepsilon}{2}.
\end{align*}
Podobno je
\[
s(f,D) > -\frac{\varepsilon}{2}. \qedhere
\]
\end{proof}

\begin{posledica}
Naj bosta $f,g \colon K \to \R$ omejeni in $f$ integrabilna na $K$.
Če je
\[
V(\setb{x \in K}{f(x) \ne g(x)}) = 0,
\]
je $g$ integrabilna in velja
\[
\lint_K f(x)\;dx = \lint_K g(x)\;dx.
\]
\end{posledica}

\begin{definicija}
Naj bo $A \subseteq \R^n$. Pravimo, da ima $A$
\emph{mero $0$}\index{Mera $0$}, če za vsak $\varepsilon > 0$
obstaja števno mnogo kvadrov $Q_1,Q_2,\dots$, da je
\[
A \subseteq \bigcup_{i=1}^\infty Q_i
\quad \text{in} \quad
\sum_{i=1}^\infty V(Q_i) < \varepsilon.
\]
\end{definicija}

\begin{opomba}
Ekvivalentno lahko za $Q_i$ vzamemo odprte kvadre.
\end{opomba}

\begin{opomba}
Množica z ničelno prostornino ima mero $0$.
\end{opomba}

\begin{trditev}
Števna unija množic z mero $0$ ima mero $0$.
\end{trditev}

\begin{proof}
Prostornino pokritja vsake množice omejimo z
$\frac{\varepsilon}{2^i}$.
\end{proof}

\begin{posledica}
Vsaka števna množica ima mero $0$.
\end{posledica}

\begin{posledica}
Naj bo $f \colon \R^n \to \R$ zvezna. Tedaj ima njen graf mero $0$.
\end{posledica}

\begin{proof}
Velja
\[
G(f) = \bigcup_{j=1}^\infty G\left(\eval{f}{[-j,j]^n}{}\right),
\]
ker pa je $f$ integrabilna, imajo te množice prostornino $0$.
\end{proof}

\begin{trditev}
Naj bo $A \subseteq \R^n$ kompaktna. Tedaj je $V(A) = 0$ natanko
tedaj, ko ima $A$ mero $0$.
\end{trditev}

\begin{proof}
Naj ima $A$ mero $0$ in $\varepsilon > 0$. Tedaj obstajajo taki
kvadri $Q_1,Q_2,\dots$, da je
\[
A \subseteq \bigcup_{i=1}^\infty \Int(Q_i)
\quad \text{in} \quad
\sum_{i=1}^\infty V(Q_i) < \varepsilon.
\]
Ker je $A$ kompaktna, obstaja končno podpokrijte.
\end{proof}

\begin{definicija}
Če lastnost $L(x)$ ne velja samo za $x$ iz množice z mero $0$,
pravimo, da $L$ velja \emph{skoraj povsod}.
\end{definicija}

\begin{opomba}
Če ima $A$ mero $0$, $A$ nima notranjosti.
\end{opomba}

\begin{lema}
Naj bo $f \colon K \to [0,\infty)$ integrabilna. Denimo, da je
\[
\lint_K f(x)\;dx = 0.
\]
Tedaj za vsak $c > 0$ velja
\[
V(\setb{x \in K}{f(x) \geq c}) = 0.
\]
\end{lema}

\begin{proof}
Naj bo $C = \setb{x \in K}{f(x) \geq c}$ in
$S(f,D) < c \cdot \varepsilon$. Tedaj velja
\[
c \sum_{K_i \cap C \ne \emptyset} V(K_i) \leq
\sum_{K_i \cap C \ne \emptyset} M(K_i) \cdot V(K_i) \leq
\sum_{i=1}^n M(K_i) \cdot V(K_i) <
c \cdot \varepsilon. \qedhere
\]
\end{proof}

\begin{posledica}
Naj bo $f \colon K \to [0,\infty)$ integrabilna. Če je
\[
\lint_K f(x)\;dx = 0,
\]
je $f(x) = 0$ skoraj povsod.
\end{posledica}

\begin{proof}
Velja
\[
\setb{x \in K}{f(x) \ne 0} =
\bigcup_{i=1}^\infty \setb{x \in K}{f(x) \geq \frac{1}{i}}.
\qedhere
\]
\end{proof}

\begin{posledica}
Naj bosta $f,g \colon K \to \R$ integrabilni in naj za vse $x$
velja $f(x) \leq g(x)$. Če velja
\[
\lint_K f(x)\;dx = \lint_K g(x)\;dx,
\]
je $f=g$ skoraj povsod.
\end{posledica}

\begin{trditev}
Naj bo $A \subseteq \R^n$ omejena in $f \colon A \to \R$
integrabilna. Če ima $A$ mero $0$, je
\[
\lint_A f(x)\;dx = 0.
\]
\end{trditev}

\begin{proof}
Naj bo
\[
\widetilde{f}(x) = \begin{cases}
f(x), &x \in A
\\
0, &x \not \in A
\end{cases}
\]
in $D$ delitev $K$. Za vsak kvader $K_i$ delitve $D$ velja
$K_i \cap A^\mathsf{c} \ne 0$. Sledi, da je $M(K_i) \geq 0$ in
$m(K_i) \leq 0$, zato je
\[
\inf_D S(\widetilde{f},D) \geq 0 \geq \sup_D s(\widetilde{f},D).
\qedhere
\]
\end{proof}

\datum{2021-11-30}

\begin{izrek}
Naj bosta $A, B \subseteq \R^n$ omejena in $f, g \colon A \to \R$
integrabilni.

\begin{enumerate}[i)]
\item Integrabilne funkcije na $A$ tvorijo vektroski prostor nad
$\R$, integral pa je linearen funkcional.
\item Če je $f(x) \leq g(x)$ na $A$, velja
\[
\lint_A f(x)\;dx \leq \lint_A g(x)\;dx.
\]
\item $\abs{f}$ je integrabilna in velja
\[
\abs{\lint_A f(x)\;dx} \leq \lint_A \abs{f(x)}\;dx.
\]
\item Če ima $A$ prostornino in je $m \leq f(x) \leq M$, je
\[
m \cdot V(A) \leq \lint_A f(x)\;dx \leq M \cdot V(A).
\]
\item Če je $A$ kompaktna, povezana množica s prostornino in $f$
zvezna, obstaja tak $x_0$, da je
\[
\lint_A f(x)\;dx = f(x_0) V(A).
\]
\item Naj bo $f \colon A \cup B \to \R$ integrabilna na $A$ in $B$
ter $V(A \cap B) = 0$. Tedaj je $f$ integrabilna na $A \cup B$ in
velja
\[
\lint_{A \cup B} f(x)\;dx = \lint_A f(x)\;dx + \lint_B f(x)\;dx.
\]
\end{enumerate} 
\end{izrek}

\begin{proof}
Dokažimo 5.\ lastnost. Če je $V(A) = 0$, lahko vzamemo poljuben
$x_0$. V nasprotnem primeru pa velja
\[
\min_A f \leq \frac{1}{V(A)} \lint_A f(x)\;dx \leq \max_A f
\]
in lahko uporabimo izrek o vmesni vrednosti.
\end{proof}

\begin{izrek}[Lebesgue]\index{Izrek!Lebesgue}
Naj bo $K$ kvader in $f \colon K \to \R$ omejena. Potem je $f$
integrabilna natanko tedaj, ko je na $K$ zvezna skoraj povsod.
\end{izrek}

\begin{proof}
Naj bo $f$ zvezna skoraj povsod. Naj bo $E$ taka podmnožica $K$ z
mero $0$, da je $f$ zvezna na $K \setminus E$. Naj bo
$\varepsilon > 0$ in $\abs{f(x)} \leq M$.

Obstajajo taki kvadri $Q_1,Q_2,\dots$, da velja
\[
E \subseteq \bigcup_{i=1}^\infty \Int(Q_i)
\quad \text{in} \quad
\sum_{i=1}^\infty V(Q_i) < \frac{\varepsilon}{4M}.
\]
Naj bo $x \in K \setminus E$. Ker je $f$ tam zvezna, obstaja kvader
$Q_x$, v katerem je
\[
\abs{f(y)-f(z)} < \frac{\varepsilon}{2 V(K)}.
\]
Sledi, da je
\[
K \subseteq
\bigcup_{i=1}^\infty \Int(Q_i) \cup
\bigcup_{x \in K \setminus E} \Int(Q_x).
\]
Ker so kvadri kompaktni sledi, da obstaja končno podpokritje. Ti
kvadri porodijo delitev $D$ kvadra $K$, za katero je vsak izmed
njih unija nekaj kvadrov delitve $D$. Sledi, da je
\begin{align*}
S(D) - s(D) &=
\sum_{i=1}^N (M(K_i)-m(K_i)) V(K_i)
\\
&=
\sum_{\exists l \colon K_i \subseteq Q_l}(M(K_i)-m(K_i)) V(K_i) +
\sum_{\forall l \colon K_i \not\subseteq Q_l}(M(K_i)-m(K_i)) V(K_i)
\\
&<
2M \cdot \frac{\varepsilon}{4M} + \frac{\varepsilon}{2 V(K)} V(K)
\\
&=
\varepsilon.
\end{align*}

Sedaj naj bo $f$ integrabilna. Naj bo $D$ delitev $K$ s kvadri
$K_1,\dots,K_N$, kjer je $K_i = [a_i,b_i]$. Označimo
$K_i' = [a_i,b_i)$. Naj bo
\[
U_D = \sum_{i=1}^N M(K_i) \chi_{K_i'}
\quad \text{in} \quad
L_D = \sum_{i=1}^N m(K_i) \chi_{K_i'}
\]
Sledi, da je $L_D \leq f \leq U_D$ na $K'$. Opazimo še, da je
\[
\lint_K L_D(x)\;dx = s(D)
\quad \text{in} \quad
\lint_K U_D(x)\;dx = S(D).
\]
Naj bo $(D_l)_{l=1}^\infty$ zaporedje vedno finejših delitev $K$,
katerih dolžine robov gredo proti $0$. Sledi, da je
\[
\lim_{l \to \infty} s(D_l) = s = S = \lim_{l \to \infty} S(D_l).
\]
Riemannove vsote namreč konvergirajo proti integralu, zato tudi
$s$ in $S$. Ker pa $L_{D_i}$ po točkah naraščajo in $U_{D_l}$ po
točkah padajo, hkrati pa so omejene, obstajata limiti
\[
L(x) = \lim_{l \to \infty} L_{D_l}
\quad \text{in} \quad
U(x) = \lim_{l \to \infty} U_{D_l},
\]
za kateri velja
\[
L \leq f \leq U.
\]
Za vsako delitev $D$ velja
\[
s(L_{D_l},D) \leq s(L,D) \leq S(L,D) \leq S(U,D) \leq S(U_{D_l},D)
\]
in
\[
s(L_{D_l},D) \leq s(L,D) \leq s(U,D) \leq S(U,D) \leq S(U_{D_l},D).
\]
Ker pa je
\[
\lim_{l \to \infty} \lint_K L_{D_l}(x)\;dx = s
\]
in
\[
\lim_{l \to \infty} \lint_K U_{D_l}(x)\;dx = S
\]
ter vemo, da je $f$ integrabilna, sledi, da sta tudi $L$ in $U$
integrabilni z enakim integralom. Ker je $L \leq U$ sledi, da sta
enaki skoraj povsod.

Naj bo
\[
E =
\setb{x \in K}{L(x) \ne U(x)} \cup \set{\text{meje delitev $D_i$}}.
\]
Tedaj ima $E$ mero $0$. Naj bo $x_0 \in K \setminus E$ in
$\varepsilon > 0$. Sledi, da je $U(x_0) = L(x_0)$ in obstaja tak
$i_0$, da za vse $i \leq i_0$ velja
\[
U_{D_i}(x_0) - L_{D_i}(x_0) < \varepsilon.
\]
Ker pa $x_0$ ne pripada robu delitve $D_j$, je pripadajoč delilni
kvader njegova okolica, v tem kvadru pa velja
\[
U_{D_i}(x)-L_{D_i}(x) = U_{D_i}(x_0)-L_{D_i}(x_0) < \varepsilon.
\]
Sledi, da je $f$ zvezna v $x_0$.
\end{proof}

\begin{opomba}
Trditev \ref{td:3} je posledica Lebesgueovega izreka. $V(A)$ namreč
obstaja natanko tedaj, ko je $\chi_A $ zvezna skoraj povsod,
$\chi_A$ pa ni zvezna natanko na robu $A$, ki je kompaktna.
\end{opomba}

\begin{trditev}
Naj bo $A$ omejena množica s prostornino in $f \colon A \to \R$
omejena. Potem je $f$ integrabilna na $A$ natanko tedaj, ko je $f$
zvezna skoraj povsod na $A$.
\end{trditev}

\begin{proof}
Naj bo $K$ kvader, ki vsebuje $A$, in
\[
\widetilde{f}(x) = \begin{cases}
f(x), &x \in A
\\
0, &x \not \in A.
\end{cases}
\]
$f$ je integrabilna na $A$ natanko tedaj, ko je $\widetilde{f}$
integrabilna na $K$, kar pa je ekvivalentno temu, da je
$\widetilde{f}$ zvezna skoraj povsod na $K$.

Če je $f$ zvezna skoraj povsod na $A$, je $\widetilde{f}$ zvezna
skoraj povsod na $K$, saj so vse točke nezveznosti vsebovane v
zaprtju $A$, njen rob pa ima mero $0$. Če pa je $f$ integrabilna
na $A$, je $\widetilde{f}$ zvezna skoraj povsod na $K$, zato je
zvezna skoraj povsod tudi na $A$.
\end{proof}

\newpage

\subsection{Posledica Fubinijevega izreka}

\begin{trditev}
Naj bo $K$ kvader v $\R^{n}$ in
$\varphi, \psi \colon K \to \R$ zvezni funkciji, za kateri je
$\varphi(x) \leq \psi(x)$ za vse $x \in K$. Naj bo
\[
A = \setb{(x,y) \in K \times \R}{\varphi(x) \leq y \leq \psi(x)}.
\]
Naj bo $f \colon A \to \R$ zvezna. Tedaj je
\[
\lint_A f(x)\;dx =
\lint_K \left( \int_{\varphi(x)}^{\psi(x)} f(x,y)\;dy\right)dx.
\]
\end{trditev}

\begin{proof}
Naj bo $a \leq \varphi \leq \psi \leq b$. Sledi, da je
\[
A \subseteq K \times [a,b] = Q.
\]
Naj bo $\widetilde{f} = f \cdot \chi_A$. Sledi, da je $f$ zvezna na
\[
\setb{(x,y) \in Q}{\varphi(x) < y < \psi(x)}
\]
in na
\[
\setb{(x,y) \in Q}{y < \varphi(x) \lor \psi(x) < y}.
\]
Sledi, da je $\widetilde{f}$ nezvezna kvečjemu na grafih $\varphi$
in $\psi$, ki pa imata mero $0$, zato je integrabilna na $Q$. Za
vse $x$ pa je $y \mapsto f(x,y)$ odsekoma zvezna in zato
integrabilna. Sledi, da je
\[
\liint_A f(x,y)\;dx\;dy =
\liint_Q \widetilde{f}(x,y)\;dx\;dy =
\lint_K \left(\lint_{[a,b]} \widetilde{f}(x,y)\;dy\right)dx =
\lint_K \left( \int_{\varphi(x)}^{\psi(x)} f(x,y)\;dy\right)dx.
\qedhere
\]
\end{proof}

\begin{lema}
Če ima množica $B \subseteq \R^n$ prostornino $0$, ima za vsako
omejeno množico $E \subseteq \R$ množica $B \times E$ prostornino
$0$ v $\R^{n+1}$.
\end{lema}

\begin{proof}
Naj bo $E \subseteq = [a,b]$. Naj bo $\varepsilon > 0$. Ker je
$V(B) = 0$, obstajajo kvadri $Q_1,\dots,Q_N \subseteq \R^n$, za
katere je
\[
B \subseteq \bigcup_{i=1}^N Q_i
\quad \text{in} \quad
\sum_{i=1}^N V(Q_i) < \frac{\varepsilon}{b-a}.
\]
Sedaj preprosto vzamemo kvadre $Q_i \times [a,b]$.
\end{proof}

\begin{trditev}
Naj bo $D \subseteq \R^n$ omejena množica s prostornino. Naj bosta
$\varphi,\psi \colon D \to \R$ omejeni, zvezni funkciji, za kateri
je $\varphi(x) \leq \psi(x)$ za vse $x \in D$. Naj bo
\[
A = \setb{(x,y) \in D \times \R}{\varphi(x) \leq y \leq \psi(x)}.
\]
Naj bo $f \colon A \to \R$ omejena, zvezna funkcija. Tedaj je $f$
integrabilna na $A$ in velja
\[
\lint_A f(x)\;dx =
\lint_D \left( \int_{\varphi(x)}^{\psi(x)} f(x,y)\;dy\right)dx.
\]
\end{trditev}

\begin{proof}
Naj bo $D \subseteq K$ in $a \leq \varphi \leq \psi \leq b$. Potem
je $A \subseteq K \times [a,b]$.

Oglejmo si vse možne točke nezveznosti funkcije
$\widetilde{f} = f \cdot \chi_A$. Te so lahko le na robu $A$, saj
je $\widetilde{f}$ v notranjosti zvezna, v zunanjosti pa enaka $0$.
Velja pa, da je
\[
\partial A \subseteq
\partial D \times [a,b] \cup \Gamma_\varphi \cup \Gamma_\psi.
\]
$\varphi$ in $\psi$ razširimo na $K$ z $0$. Sledi, da imata točke
nezveznosti le na $\partial D$, ki ima prostornino $0$, zato sta
integrabilni na $K$. Sledi, da imata njuna grafa v $\R^{n+1}$
prostornino $0$. Dobimo, da ima $\partial A$ v $\R^{n+1}$
prostornino $0$ in je $\widetilde{f}$ integrabilna na
$K \times [a,b]$.

Za vsak $x$ je funkcija $\widetilde{f}$ integrabilna na $[a,b]$,
saj je tam odsekoma zvezna. Sledi, da je
\begin{align*}
\liint_A f(x,y)\;dx\;dy &=
\lint_K \left(\int_a^b \widetilde{f}(x,y)\;dy\right)dx
\\
&=
\lint_D \left(\int_a^b \widetilde{f}(x,y)\;dy\right)dx
\\
&=
\lint_D \left(
\int_{\varphi(x)}^{\psi(x)} \widetilde{f}(x,y)\;dy\right)dx.
\qedhere
\end{align*}
\end{proof}

\begin{trditev}
Naj bo $A \subseteq \R^n$ omejena. Če ima $A$ prostornino, jo imata
tudi $\Int A$ in $\overline{A}$ in velja
\[
V(A) = V(\Int A) = V(\overline{A}).
\]
\end{trditev}

\begin{proof}
Ker ima $A$ prostornino, je $V(\partial A) = 0$. Sledi, da je tudi
$V(\partial \Int A) = 0$ in $V(\partial\overline{A}) = 0$, saj
velja $\partial \Int A \subseteq \partial A$ in
$\partial \overline{A} \subseteq \partial A$. Sledi, da imata
množici prostornino in je
\[
V(\Int A) \leq V(A) \leq V(\overline{A}),
\]
velja pa
\[
V(\overline{A}) = V(\Int A) + V(\partial A) = V(\Int A). \qedhere
\]
\end{proof}

\begin{opomba}
Podobno ima vsaka množica
$\Int A \subseteq B \subseteq \overline{A}$ prostornino enako
$V(A)$.
\end{opomba}

\begin{trditev}
Naj bo $A \subseteq \R^n$ omejena množica s prostornino in
$f \colon \overline{A} \to \R$ omejena. Če je $f$ integrabilna na
$A$, je integrabilna tudi na $\Int A$ in $\overline{A}$ in velja
\[
\lint_A f(x)\;dx = \lint_{\Int A} f(x)\;dx =
\lint_{\overline{A}} f(x)\;dx.
\]
\end{trditev}

\begin{proof}
Naj bo $\overline{A} \subseteq K$. Velja $V(\partial A) = 0$, saj
ima $A$ prostornino. Ker je $f$ omejena, je tako
\[
\lint_{\partial A} f(x)\;dx = 0.
\]
Dovolj je tako dokazati, da obstaja integral po $\Int A$ in je
\[
\lint_A f(x)\;dx = \lint_{\Int A} f(x)\;dx.
\]
Oglejmo si potencialne točke nezveznosti $\chi_{\Int A} \cdot f$.
Lahko so na $\partial \Int A \subseteq \partial A$, zato imajo mero
$0$, ali pa so to točke nezveznosti $f$ na $\Int A$, ki imajo prav
tako mero $0$, saj je $f$ integrabilna na $A$. Ker ima $\partial A$
prostornino in mero $0$, tako sledi
\[
\lint_A f(x)\;dx =
\lint_{A \setminus \Int A} f(x)\;dx + \lint_{\Int A} f(x)\;dx =
\lint_{\Int A} f(x)\;dx. \qedhere
\]
\end{proof}

\begin{opomba}
Podobno velja za vse množice
$\Int A \subseteq B \subseteq \overline{A}$.
\end{opomba}

\newpage

\subsection{Vpeljava nove spremenljivke}

\begin{lema}
Naj bo $A \subseteq \R^n$ omejena odprta množica in naj bo
$E \subseteq A$ množica z mero $0$. Če je $\Phi \colon A \to \R^n$
$\mathcal{C}^1$ difeomorfizem, ima $\Phi(E)$ mero $0$.
\end{lema}

\begin{proof}
Vsako odprto monžico $A$ lahko zapišemo kot števno unijo zaprtih
krogel. Dovolj je tako za poljubno kroglo
\[
\overline{K(a,R)} \subseteq \overline{K(a,R')} \subseteq A
\]
pokazati, da ima $\Phi(E \cap \overline{K(a,R)})$ mero $0$. Na
$\overline{K(a,R')}$ so vsi odvodi $D\Phi$ enakomerno omejeni.
Sledi, da obstaja tak $c > 0$, da za
$t_1,t_2 \in \overline{K(a,R')}$ velja
\[
\abs{\Phi(t_1) - \Phi(t_2)} \leq c \cdot \abs{t_1 - t_2}.
\]
Naj bo $\varepsilon > 0$. Obstajajo krogle $B_1,B_2,\dots$ v
$\overline{K(a,R')}$, da velja
\[
E \cap \overline{K(a,R)} \subseteq \bigcup_{i=1}^\infty B_i
\quad \text{in} \quad
\sum_{i=1}^\infty V(B_i) < \frac{\varepsilon}{c^n}.
\]
Oglejmo si
\[
\Phi(E \cap \overline{K(a,R)}) \subseteq
\Phi(\bigcup_{i=1}^\infty B_i).
\]
Velja, da je $B_i \subseteq \overline{K(a_i, cr_i)}$, zato je
\[
\Phi(E \cap \overline{K(a,R)}) \subseteq
\Phi(\bigcup_{i=1}^\infty \overline{K(a_i, cr_i)}),
\]
volumen desne strani pa je manjši od $\varepsilon$. Sledi, da ima
$\Phi(E \cap \overline{K(a,R)})$ mero $0$.
\end{proof}

\begin{posledica}
Naj bo $U \subseteq \R^n$ odprta množica in naj bo
$A \subseteq \overline{A} \subseteq U$ množica s prostornino. Naj
bo $\Phi \colon U \to V$ difeomorfizem, kjer je $V$ odprta
podmnožica $\R^n$. Tedaj ima $\Phi(A)$ prostornino.
\end{posledica}

\begin{proof}
Velja $V(\partial A) = 0$. Ker je $\Phi$ difeomorfizem in
$\overline{A} \subseteq U$, je
$\Phi(\partial A) = \partial \Phi(A)$. Sledi, da ima
$\partial \Phi(A)$ mero $0$, ker pa je kompakt, ima tudi
prostornino enako $0$.
\end{proof}

\datum{2021-12-7}

\begin{lema}
Vsaka nesingularna matrika je produkt elementarnih matrik (menjava
vrstic, prištevanje ene vrstice drugi, množenje vrstice s
konstanto).
\end{lema}

\begin{proof}
Gaussova eliminacija.
\end{proof}

\begin{trditev}
Naj bo $A \subseteq \R^n$ množica s prostornino in
$T \colon \R^n \to \R^n$ obrnljiva afina preslikava, oziroma
$T = L + d$, kjer je $L$ obrnljiva linearna. Tedaj je
\[
V(T(A)) = \abs{\det L} \cdot V(A).
\]
\end{trditev}

\begin{proof}
Ker translacija ohranja volumen, je trditev dovolj dokazati za
elementarne preslikave, kar pa je (bolj ali manj) očitno.
\end{proof}

\datum{2021-12-14}

\begin{trditev}
Naj bo $\Phi \colon U \to V$ $\mathcal{C}^1$ difeomorfizem. Naj bo
$E \subseteq U$ kompaktna podmnožica in $\varepsilon > 0$. Potem
obstaja tak $\delta > 0$, da za vsak kvader $K_0$ v $E$, katerega
razmerje med najdaljšim in najkrajšim robom je manjši od $2$ in
vsemi robovi manjšimi od $\delta$ velja
\[
V(\Phi(K_0)) - V(\Phi(a)+D\Phi(a)(K_0-a)) < \epsilon \cdot V(K_0),
\]
kjer je $a$ središče $K_0$.
\end{trditev}

\begin{izrek}
Naj bo $A \subseteq \R^n$ omejena odprta množica s prostornino. Naj
bo $\Phi \colon A \to B$, kjer je $B = \Phi(A) \subseteq \R^n$,
$\mathcal{C}^1$ difeomorfizem. Denimo, da ima $B$ prostornino in da
je $J\Phi = \det D\Phi$ omejena funkcija na $A$. Naj bo
$f \colon B \to \R$ omejena in zvezna skoraj povsod. Potem je
\[
\lint_B f(x)\;dx = \lint_A f(\Phi(t)) \abs{J\Phi(t)}\;dt.
\]
\end{izrek}

\begin{proof}
Kvadri Riemannove vsote, ki sekajo rob $A$, imajo poljubno majhno
prostornino. Dovolj je tako izrek dokazati za kvadre.

Velja pa
\begin{align*}
R((f \circ \Phi) \cdot \abs{J\Phi},D,\xi) &=
\sum_{i=1}^N f(\Phi(a_i)) \cdot \abs{J\Phi} \cdot V(K_i)
\\
&=
\sum_{i=1}^N f(\Phi(a_i)) \cdot \abs{J\Phi} \cdot V(K_i - a_i)
\\
&=
\sum_{i=1}^N f(\Phi(a_i)) \cdot V(\Phi(a_i)+D\Phi(t_i)(K_i-a_i))
\\
&\doteq \sum_{i=1}^N f(\Phi(a_i)) \cdot V(\Phi(K_i))
\\
&\doteq \sum_{i=1}^N \lint_{\Phi(K_i)} f(\Phi(a_i))\;dx
\\
&\doteq \sum_{i=1}^N \lint_{\Phi(K_i)} f(x)\;dx
\\
&= \lint_K f(x)\;dx.
\qedhere
\end{align*}
\end{proof}

\begin{posledica}
Naj bo $\Phi \colon U \to V$ $\mathcal{C}^1$ difeomorfizem odprtih
podmnožic $U, V \subseteq \R^n$ s prostornino. Naj bo
$J \Phi = \det D\Phi$ omejena na $U$. Naj bo $B \subseteq V$,
$A = \Phi^{-1}(B) \subseteq U$ in $f \colon B \to \R$ integrabilna.
Tedaj je
\[
\lint_B f(x)\;dx = \lint_A f(\Phi(t)) \abs{J\Phi(t)}\;dt.
\]
\end{posledica}

\begin{proof}
$f$ razširimo na $V$ z $0$.
\end{proof}

\newpage

\subsection{Posplošeni \texorpdfstring{$n$}{n}-terni integral}

\begin{definicija}
Naj bo $A \subseteq \R^n$ in naj ima $\partial A$ mero $0$. Naj bo
$f \colon A \to \R$ zvezna skoraj povsod. $f$ razširimo na $\R^n$ z
$0$. Naj bo\footnote{$P$ je zaprta podmnožica $\R^n$ z mero $0$.}
\[
P = \setb{x \in \R^n}{\text{$f$ ni omejena v nobeni okolici $x$}}
\]
Z $\mathcal{K}_f$ označimo množico kompaktnih podmnožic
$\R^n \setminus P$ s prostornino.
\end{definicija}

\begin{trditev}
Če je $Q \in \mathcal{K}_f$, obstaja
\[
\lint_Q f(x)\;dx.
\]
\end{trditev}

\begin{proof}
Na $Q$ je $f$ omejena, saj je $Q$ kompaktna. Ker ima $Q$
prostornino in je $f$ zvezna skoraj povsod, integral res obstaja.
\end{proof}

\begin{definicija}
Naj bo $f \geq 0$ na $A$. Definiramo
\[
\lint_A f(x)\;dx = \sup_{Q \in \mathcal{K}} \lint_Q f(x)\;dx.
\]
\end{definicija}

\begin{definicija}
Zaporedje kompaktnih množic $Q_n$ v odprti množici $\Omega$
\emph{izčrpa} $\Omega$, če velja
\[
Q_1 \subseteq \Int Q_2 \subseteq Q_2 \subseteq \dots
\quad \text{in} \quad
\bigcup_{i=1}^\infty Q_i = \Omega.
\]
\end{definicija}

\begin{trditev}
Velja
\[
\lint_A f(x)\;dx =
\lim_{i \to \infty} \lint_{Q_i} f(x)\;dx.
\]
\end{trditev}

\obvs

\begin{definicija}
Funkcija $f$ je
\emph{absolutno integrabilna}\index{Integral!Absolutna integrabilnost},
če je
\[
\lint_A \abs{f}\;dx < \infty.
\]
\end{definicija}

\begin{opomba}
Množica $\mathcal{L}^1(A)$ absolutno integrabilnih funkcij na $A$
je vektorski prostor nad $\R$. Če na $\mathcal{L}^1$ uvedemo
ekvivalenčno relacijo, za katero je $f \sim g$ natanko tedaj, ko je
$f = g$ skoraj povsod, integral na njem inducira normo in s tem
metriko. Označimo $L^1(A) = \kvoc{\mathcal{L}^1(A)}{\sim}$.
\end{opomba}
