\section{Integrali s parametri}

\subsection{Eulerjeva gama}

\datum{2021-11-2}

\begin{definicija}
\emph{Eulerjeva funkcija gama}\index{Eulerjeva gama} je funkcija
\[
\Gamma(s) = \int_0^\infty x^{s-1} e^{-x}\;dx.
\]
\end{definicija}

\begin{trditev}
Velja $\Gamma(s+1) = s \cdot \Gamma(s)$.
\end{trditev}

\begin{proof}
Z integriranjem po delih dobimo
\[
\Gamma(s+1) =
\int_0^\infty x^s e^{-x}\;dx =
\eval{-x^s e^{-x}}{0}{\infty} +
s \cdot \int_0^\infty x^{s-1} e^{-x}\;dx =
s \cdot \Gamma(s). \qedhere
\]
\end{proof}

\begin{posledica}
Velja $\Gamma(n+1) = n!$.
\end{posledica}

\begin{definicija}
\emph{Eulerjeva funkcija beta}\index{Eulerjeva beta} je funkcija
\[
\beta(p,q) = \int_0^1 x^{p-1}(1-x)^{q-1}\;dx
\]
\end{definicija}

\newpage

\subsection{Zveznost in odvedljivost integralov s parametri}

\begin{definicija}
Množica $Y \subseteq \R^n$ je
\emph{lokalno kompaktna}\index{Množica!Lokalno kompaktna}, če za
vse $y \in Y$ obstaja tak $r>0$, da je
$Y \cap \overline{\mathcal{K}(y,r)}$ kompaktna.
\end{definicija}

\begin{opomba}
Zaprte in odprte množica so lokalno kompaktne.
\end{opomba}

\begin{opomba}
Zvezne funkcije na lokalno kompaktnih množicah so lokalno omejene in enakomerno zvezne.
\end{opomba}

\begin{izrek}
Naj bo $I = [a,b]$ in $Y \subseteq \R^n$ lokalno kompaktna. Naj bo
$f \colon I \times Y \colon \R$ zvezna. Potem je funkcija
$F \colon I \times I \times Y \to \R$, podana s predpisom
\[
F(u,v,y) = \int_u^v f(x,y)\;dx,
\]
zvezna.
\end{izrek}

\begin{proof}
Ker je $Y$ lokalno kompaktna, obstaja tak $r>0$, da je
$A = Y \cap \overline{\mathcal{K}(y_0,r)}$ kompaktna, zato je
$I \times A$ kompaktna in je $f$ na tej množici enakomerno zvezna.
Naj bo $\varepsilon > 0$. Sledi, da obstaja tak $\delta$, da za vse
$y_1,y_2 \in A$, za katere velja $\norm{y_1 - y_2} < \delta$, velja
\[
\abs{f(x,y_1) - f(x,y_2)} < \frac{\varepsilon}{3 \cdot (b-a)}.
\]
Ker je $I \times A$ kompaktna, je $f$ na $I \times A$ omejena z
$M$. Sedaj za
$\abs{u-u_0},\abs{v-v_0} < \frac{\varepsilon}{3(b-a)}$ dobimo
\begin{align*}
&\abs{F(u,v,y) - F(u_0,v_0,y_0)}
\\
= &\abs{\int_u^v f(x,y)\;dx - \int_{u_0}^{v_0} f(x,y_0)\;dx}
\\
\leq &\abs{\int_u^{u_0} f(x,y)\;dx} +
\abs{\int_v^{v_0} f(x,y)\;dx} +
\abs{\int_{v_0}^{u_0} \left(f(x,y) - f(x,y_0)\right)\;dx}
\\
\leq &M \cdot \abs{u-u_0} + M \cdot \abs{v-v_0} +
\frac{\varepsilon}{3(b-a)} \cdot (b-a) < \varepsilon. \qedhere
\end{align*}
\end{proof}

\begin{posledica}
Naj bo $Y \subseteq \R^n$ lokalno kompaktna in
$f \colon [a,b] \times Y \to \R$ zvezna. Potem je
\[
F(y) = \int_a^b f(x,y)\;dx
\]
zvezna na $Y$.
\end{posledica}

\begin{izrek}
Naj bodo $a<b$ in $c<d$ realna in $D = [a,b] \times (c,d)$. Naj bo
$f \colon D \to \R$ zvezna in v vsaki točki $(x,y) \in D$ parcialno
odvedljiva po $y$ z zveznim parcialnim odvodom. Tedaj je
\[
F(y) = \int_a^b f(x,y)\;dx
\]
zvezno odvedljiva in velja
\[
F'(y) = \int_a^b f_y(x,y)\;dx.
\]
\end{izrek}

\begin{proof}
Naj bo $y \in (c,d)$ in $[y-r,y+r] \subseteq (c,d)$. Naj bo
$h \ne 0$, $\abs{h} < r$ in $\varepsilon > 0$. Potem je po
Lagrangevem izreku
\begin{align*}
\abs{\frac{F(y+h) - F(y)}{h} - \int_a^b f_y(x,y)\;dx} &=
\abs{\frac{1}{h} \int_a^b \left(f(x,y+h) - f(x,y)\right)\;dx -
\int_a^b f_y(x,y)\;dx}
\\
&= \abs{\int_a^b \left(f(x,y^*) - f_y(x,y)\right)\;dx},
\end{align*}
kjer $y^*$ leži med $y$ in $y+h$. Ker je $f_y$ na
$[a,b] \times [y-r,y+r]$ enakomerno zvezna, obstaja tak
$\delta > 0$, da za $\abs{h} < \delta$ velja
\[
\abs{f_y(x,y^*) - f_y(x,y)} < \frac{\varepsilon}{b-a},
\]
zato je
\[
\abs{\int_a^b \left(f(x,y^*) - f_y(x,y)\right)\;dx} < \varepsilon.
\qedhere
\]
\end{proof}

\begin{posledica}
Naj bosta $\alpha,\beta \colon (c,d) \to [a,b]$ zvezno odvedljivi.
Tedaj je
\[
F(y) = \int_{\alpha(y)}^{\beta(y)} f(x,y)\;dx
\]
zvezno odvedljiva na $(c,d)$ z odvodom
\[
F'(y) = \int_{\alpha(y)}^{\beta(y)} f_y(x,y)\;dx +
\beta(y) f(\beta(y),y) - \alpha(y) f(\alpha(y),y).
\]
\end{posledica}

\begin{proof}
Naj bo
\[
\Phi(u,v,y) = \int_u^v f(x,y)\;dx.
\]
Velja
\[
\frac{\partial \Phi}{\partial v} = f(v,y),
\quad
\frac{\partial \Phi}{\partial u} = -f(u,y)
\quad \text{in} \quad
\frac{\partial \Phi}{\partial y} = \int_u^v f_y(x,y)\;dx.
\]
Z odvajanjem zveze $F(y) = \Phi(\alpha(y),\beta(y),y)$ dobimo
iskano enakost.
\end{proof}

\begin{posledica}
Naj bo $D \subseteq \R^n$ odprta in
$f \colon [a,b] \times D \to \R$ zvezna. Naj za vsak
$(x,y) \in [a,b] \times D$ obstajajo zvezni parcialni odvodi
$f_{y_j}(x,y)$. Tedaj je
\[
F(y) = \int_a^b f(x,y)\;dx
\]
$\mathcal{C}^1(D)$ z odvodi
\[
F_{y_j}(y) =  \int_a^b f_{y_j}(x,y)\;dx.
\]
\end{posledica}
