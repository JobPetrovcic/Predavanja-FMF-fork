\section{Funkcije več spremenljivk}

\subsection{Prostor $\R^n$}

\datum{2021-10-5}

\begin{definicija}
\emph{Prostor $\R^n$} je kartezični produkt
$\underbrace{\R \times \dots \times \R}_n$. Na njem definiramo
seštevanje in množenje s skalarjem po komponentah. S tema
operacijama je $(\R,+,\cdot)$ vektorski prostor nad $\R$. Posebej
definiramo še skalarni produkt
\[
x \cdot y = \sum_{i=1}^n x_iy_i,
\]
ki nam da normo $\norm{x}=\sqrt{x \cdot x}$ in metriko
$d(x,y) = \norm{x-y}$. $(\R^n,d)$ je tako metrični prostor.
\end{definicija}

\begin{definicija}
Naj bosta $a,b \in \R^n$ vektorja, za katera je $a_i \leq b_i$ za
vse $i \in \set{1,\dots,n}$. \emph{Zaprt kvader}\index{Kvader}, ki
ga določata $a$ in $b$, je množica
\[
[a,b] = \setb{x\in\R^n}{\forall i \in \set{1,\dots,n} \colon
a_i \leq x_i \leq b_i}.
\]
Podobno definiramo \emph{odprt kvader} kot
\[
(a,b) = \setb{x\in\R^n}{\forall i \in \set{1,\dots,n} \colon
a_i < x_i < b_i}.
\]
\end{definicija}

\begin{opomba}
Odprte množice v normah $\norm{x}_\infty$ in $\norm{x}_2$ so iste.
\end{opomba}

\begin{izrek}
Množica $K\subseteq\R^n$ je kompaktna natanko tedaj, ko je zaprta
in omejena.\footnote{Za dokaz glej izrek 7.5.6 v zapiskih predmeta
Analiza 1 prvega letnika.}
\end{izrek}

\newpage

\subsection{Zaporedja v $\R^n$}

\begin{trditev}
Zaporedje $\set{a_m}$ v $\R^n$ konvergira natanko tedaj, ko za vse
$1 \leq j \leq n$ konvegira zaporedje koordinat $\set{a_j^m}$.
Tedaj velja
\[
\lim_{m\to\infty} a_m = \left(\lim_{m\to\infty} a_1^m, \dots,
\lim_{m\to\infty} a_n^m\right).
\]
\end{trditev}

\begin{proof}
Predpostavimo, da zaporedje konvergira k točki $a$. Za vsak
$\varepsilon>0$ tako obstaja tak $m_0 \in \N$, da je
$d(a_m,a) < \varepsilon$ za vse $m \geq m_0$. Sledi, da je
\[
\sqrt{\sum_{i=1}^n \left(a_i^m - a_i\right)^2} < \varepsilon,
\]
zato je $\abs{a_j^m - a_j} < \varepsilon$.

Če konvergirajo zaporedja koordinat, pa za vsak $\varepsilon>0$
obstaja tak $m_j \in \N$, da je
$\abs{a_j^m - a_j} < \frac{\varepsilon}{\sqrt{n}}$ za vse
$m \geq m_j$. Naj bo $m_0 = \max\set{m_1, \dots, m_n}$. Potem za
vsak $m \geq m_0$ velja
\[
d(a^m,a) = \sqrt{\sum_{i=1}^n (a_i^m - a_i)^2}
< \sqrt{n \cdot \frac{\varepsilon^2}{n}} = \varepsilon. \qedhere
\]
\end{proof}

\newpage

\subsection{Zveznost preslikav iz $\R^n$ v $\R^m$}

\begin{definicija}
Naj bo $D \subseteq \R^n$ in $f \colon D\to\R^m$ preslikava.
Pravimo, da je $f$ \emph{zvezna}\index{Preslikava!Zvezna} v točki
$a\in D$, če za vsak $\varepsilon > 0$ obstaja tak $\delta > 0$, da
za vsak $x \in D$, za katerega je $\norm{x - a} < \delta$, velja
\[
\norm{f(x) - f(a)} < \varepsilon.
\]
Pravimo, da je $f$ zvezna na $D$, če je zvezna v vsaki točki
$a \in D$.
\end{definicija}

\begin{definicija}
Preslikava $f$ je
\emph{enakomerno zvezna na $D$}\index{Preslikava!Enakomerno zvezna},
če za vsak $\varepsilon > 0$ obstaja tak $\delta > 0$, da za vsaka
$x,y \in D$, za katera je $\norm{x - y} < \delta$, velja
\[
\norm{f(x) - f(y)} < \varepsilon.
\]
\end{definicija}

\begin{opomba}
Če je $m = 1$, pravimo, da je $f$
\emph{funkcija $n$ spremenljivk}\index{Preslikava!Funkcija več spremenljivk}
na $D$. Pišemo $f(x) = f(x_1, \dots, x_n)$.
\end{opomba}

\begin{izrek}
Naj bo $D \subseteq \R^n$ in $f, g \colon D \to \R$ funkciji,
zvezni v točki $a$. Tedaj so v točki $a$ zvezne tudi funkcije
$f + g$, $f - g$ in $\lambda f$ za $\lambda \in \R$ in $f\cdot g$.
Če je $g(x) \ne 0$ na $D$, je tudi $\frac{f}{g}$ zvezna v $a$.
\end{izrek}

\obvs

\begin{opomba}
Seveda so vse konstantne in koordinatne funkcije zvezne
(projekcije). Sledi, da so vse racionalne funkcije zvezne, kjer so
definirane.
\end{opomba}

\begin{opomba}
Kompozitum zvenih preslikav je zvezen.
\end{opomba}

\datum{2021-10-6}
\begin{izrek}
Naj bo $D \subseteq \R^n$ in $f \colon D \to \R$ funkcija, zvezna v
notranji točki $a \in D$. Tedaj je v točki $a$ funkcija $f$ zvezna
kot funkcija vsake spremenljivke posebej.\footnote{To pomeni, da je
funkcija $f_i(t) = f(a_1, \dots, t, \dots, a_n)$ zvezna.}
\end{izrek}

\obvs

\begin{opomba}
Obratno ne velja. Protiprimer je funkcija
\[
f(x,y)=\begin{cases}
\frac{2xy}{x^2+y^2},& x^2+y^2\ne 0 \\
0,& x=y=0.
\end{cases}
\]
\end{opomba}

\newpage

\subsection{Preslikave iz $\R^n$ v $\R^m$}

\begin{trditev}
Naj bo $D \subseteq \R^n$ in $f \colon D \to \R^m$. Označimo
\[
f(x_1, \dots, x_n) =
\left(f_1(x_1, \dots, x_n), \dots, f_m(x_1, \dots, x_n)\right).
\]
Preslikava $f$ je zvezna v $a \in D$ natanko tedaj, ko so vse
funkcije $f_1, \dots, f_m$ zvezne v $a$.
\end{trditev}

\begin{proof}
Če je $f$ zvezna v $a$, za vsak $\varepsilon > 0$ obstaja tak
$\delta > 0$, da iz $\norm{x - a} < \delta$ sledi
$\norm{f(x) - f(a)} < \varepsilon$. Sledi, da je
$\abs{f_j(x) - f_j(a)} < \varepsilon$.

Sedaj predpostavimo, da so vse koordinatne funkcije zvezne. Naj bo
$\varepsilon > 0$. Za vsak $j$ obstaja tak $\delta_j$, da iz
$\norm{x - a} < \delta_j$ sledi
$\abs{f_j(x) - f_j(a)} < \frac{\varepsilon}{\sqrt{m}}$. Naj bo
$\delta = \min\set{\delta_1, \dots, \delta_m}$. Potem za vse
$\norm{x - a} < \delta$ velja
\[
\norm{f(x) - f(a)} =
\sqrt{\sum_{i=1}^n \left(f_i(x) - f_i(a)\right)^2} <
\sqrt{m \cdot \frac{\varepsilon^2}{m}} = \varepsilon. \qedhere
\]
\end{proof}

\begin{posledica}
Vsaka linearna preslikava je zvezna.
\end{posledica}

\begin{trditev}\label{td:1}
Naj bo $A \colon \R^n \to \R^m$ linearna preslikava. Potem obstaja
tak $M \in \R$, da je
\[
\frac{\norm{Ax}}{\norm{x}} \leq M
\]
za vse $x \in \R^n$ in obstaja supremum
\[
\sup_{x \ne 0} \frac{\norm{Ax}}{\norm{x}} =
\sup_{\norm{x} = 1}\norm{Ax} =
\norm{A}.
\]
\end{trditev}

\begin{proof}
Naj bo $A = [a_{i,j}]$ in $C = \max_{i,j}\abs{a_{i,j}}$. Za vsako
komponentno funkcijo $A_i$ je po Cauchyjevi neenakosti
\[
\abs{A_i(x)} \leq
C \cdot \sum_{i=1}^n \abs{x_i} \leq
C\sqrt{n} \cdot \norm{x}.
\]
Sledi, da je
\[
\norm{Ax} =
\sqrt{\sum_{i=1}^m L_i(x)^2} \leq
C \sqrt{nm} \cdot \norm{x}. \qedhere
\]
\end{proof}

\newpage

\subsection{Parcialni odvodi in diferenciabilnost}

\begin{definicija}
Naj bo $a$ notranja točka množice $D \subseteq \R^n$ in
$f \colon D \to \R$ funkcija. Če obstaja limita
\[
\lim_{h \to 0}
\frac{f(a_1, \dots, a_{i-1}, a_i+h, \dots) - f(a)}{h},
\]
to limito imenujemo
\emph{parcialni odvod}\index{Preslikava!Parcialni odvod} funkcije
$f$ po spremenljivki $x_i$ v točki $a$ in ga označimo z
\[
\frac{\partial f}{\partial x_i}(a) = f_{x_i}(a) = (D_if)(a).
\]
\end{definicija}

\begin{definicija}
Naj bo $D \subseteq \R^n$ in $f \colon D \to \R$ preslikava, $a$ pa
notranja točka množice $D$. Pravimo, da je $f$
\emph{diferenciabilna}\index{Preslikava!Diferenciabilna} v točki
$a$, če obstaja taka linearna preslikava
$L \colon \R^n \to \R$, da je
\[
f(a+h) = f(a) + L(h) + o(h),
\]
kjer je
\[
\lim_{h \to 0}\frac{\abs{o(h)}}{\norm{h}} = 0.
\]
\end{definicija}

\begin{opomba}
Pri $n=1$ je ta definicija ekvivalentna odvedljivosti $f$ v točki
$a$.
\end{opomba}

\begin{trditev}
Če tak $L$ obstaja, je enolično določen.
\end{trditev}

\begin{proof}
Predpostavimo, da sta $L_1$ in $L_2$ linearni funkciji, za kateri
je
\[
f(a+h) = f(a) + L_1(h) + o_1(h) = f(a) + L_2(h) + o_2(h),
\]
pri čemer velja
\[
\lim_{h \to 0}\frac{\abs{o_1(h)}}{\norm{h}} =
\lim_{h \to 0}\frac{\abs{o_2(h)}}{\norm{h}} = 0.
\]
Potem velja
\[
(L_1 - L_2)(h) = (o_2(h) - o_1(h)) = o(h),
\]
kjer je
\[
\lim_{h \to 0}\frac{\abs{o(h)}}{\norm{h}} = 0.
\]
Sledi, da je
\[
\lim_{h \to 0}\frac{(L_1 - L_2)(h)}{\norm{h}} = 0.
\]
Ker pa je
\[
\frac{(L_1 - L_2)(0, \dots, h, \dots)}{\abs{h}} = \ell_i,
\]
kjer je $\ell_i$ koeficient pred $i$-to spremenljivko v
$L_1 - L_2$, sledi, da je $L_1 - L_2 = 0$.
\end{proof}

\datum{2021-10-7}

\begin{opomba}
Preslikavi $L$ pravimo \emph{diferencial} funkcije $f$ v točki $a$
in ga označimo z $L = d_af$. Funkcija $h \mapsto f(a) + d_af(h)$ je
najboljša afina aproksimacija funkcije $h \mapsto f(a+h)$ v okolici
točke $a$.
\end{opomba}

\begin{izrek}
Če je $f$ v notranji točki $a \in D$ diferenciabilna, je v $a$
zvezna in parcialno odvedljiva glede na vse spremenljivke,
diferencial $f$ v točki $a$ pa je enak
\[
(d_af)(h) = \sum_{i=1}^n \frac{\partial f}{\partial x_i}(a) h_i.
\]
\end{izrek}

\begin{proof}
Naj bo $L$ diferencial $f$ v točki $a$. Sledi, da je
\[
f(a+h) = f(a) + L(h) + o(h)
\quad \text{in} \quad
\lim_{h \to 0}\frac{\abs{o(h)}}{\norm{h}} = 0.
\]
Velja
\[
L(h) = \sum_{i=1}^n \ell_i h_i,
\]
zato je $L$ zvezna v $0$. Tako je
\[
\lim_{h \to 0}f(a+h) =
f(a) + \lim_{h \to 0} L(h) + \lim_{h \to 0}o(h) =
f(a).
\]
Naj bo sedaj $h = (0,\dots, h_i, \dots,0)$. Velja
$\norm{h} = \abs{h_i}$, zato je
\[
\frac{f(a+h) - f(a)}{h_i} =
\frac{L(h)}{h_i} + \frac{o(h)}{h_i} =
\ell_i + \frac{o(h)}{h_i}.
\]
V limiti je to enako $\ell_i$, zato ima $f$ parcialni odvod. Očitno
velja tudi navedena enakost.
\end{proof}

\begin{opomba}
Obratno ne velja. Protiprimer je funkcija
\[
f(x,y)=\begin{cases}
\frac{x^2y}{x^2+y^2},& x^2+y^2\ne 0 \\
0,& x=y=0.
\end{cases}
\]
\end{opomba}

\begin{opomba}
Diferencial $d_af$ lahko identificiramo z vektorjem
\[
\left(
	\frac{\partial f}{\partial x_1}(a),
	\dots,
	\frac{\partial f}{\partial x_n}(a)
\right),
\]
ki ga imenujemo \emph{gradient}\index{Preslikava!Gradient} in
označimo z $\grad f$. Velja $(d_af)(h) = (\grad f)(a) \cdot h$.
\end{opomba}

\begin{izrek}
Naj bo $f$ v okolici $a$ parcialno odvedljiva glede na vse
spremenljivke in naj bodo parcialni odvodi zvezni v $a$. Potem je
$f$ v $a$ diferenciabilna.
\end{izrek}

\begin{proof}
Naj bo $h \in \R^n$ vektor z dovolj majhno normo, da je točka $a+h$
v konveksni\footnote{Če okolica ni konveksna, lahko pri uporabi
Lagrangevega izreka ">pademo ven"< iz nje.} okolici točke $a$, kjer
veljajo zgornje predpostavke.\footnote{Na predavanjih je bil
predstavljen dokaz za $n=2$, to pa je njegova splošna oblika.} Po
Lagrangevem izreku je
\begin{align*}
I &= f(a+h) - f(a)
\\
&= \sum_{i=1}^n \left(
	f(a_1, \dots, a_i + h_i, \dots, a_n + h_n) - 
	f(a_1, \dots, a_i, a_{i+1} + h_{i+1}, \dots, a_n + h_n)
\right)
\\
&=\sum_{i=1}^n f_{x_i}(
	a_1, \dots, a_i', a_{i+1} + h_{i+1}, \dots, a_n + h_n
) \cdot h_i,
\end{align*}
kjer je $a_i'$ med $a_i$ in $a_i + h_i$ za vse $i$. Ker so
parcialni odvodi zvezni, za
\[
\eta_i(h) =
f_{x_i}(a_1, \dots, a_i', a_{i+1} + h_{i+1}, \dots, a_n + h_n)
- f_{x_i}(a)
\]
velja
\[
\lim_{h \to 0}\eta_i(h) = 0.
\]
Naj bo
\[
o(h) = \sum_{i=1}^n \eta_i(h) \cdot h_i.
\]
Dobimo
\[
f(a+h) = f(a) + \sum_{i=1}^n f_{x_i}(a) \cdot h_i + o(h).
\]
Za dokaz obstoja diferenciala je tako dovolj dokazati, da je
\[
\lim_{h \to 0}\frac{\abs{o(h)}}{\norm{h}} = 0.
\]
Velja pa
\[
\frac{\abs{o(h)}}{\norm{h}} \leq
\sum_{i=1}^n \frac{\abs{h_i}}{\norm{h}} \cdot \abs{\eta_i(h)} \leq
\sum_{i=1}^n \abs{\eta_i(h)},
\]
kar je v limiti enako $0$.
\end{proof}

\begin{posledica}
Vse elementarne funkcije so diferenciabilne, kjer so definirane.
\end{posledica}

\newpage

\subsection{Višji parcialni odvodi}

\begin{izrek}
Naj bosta parcialna odvoda $\frac{\partial f}{\partial x_i}$ in
$\frac{\partial f}{\partial x_j}$ v okolici $a$ zvezna in naj na
tej okolici obstajata
\[
\frac{\partial}{\partial x_i}\left(
	\frac{\partial f}{\partial x_j}
\right)
\quad \text{ter} \quad
\frac{\partial}{\partial x_j}\left(
	\frac{\partial f}{\partial x_i}
\right),
\]
ki sta zvezna v $a$. Tedaj velja
\[
\frac{\partial}{\partial x_i}\left(
	\frac{\partial f}{\partial x_j}
\right)(a) =
\frac{\partial}{\partial x_j}\left(
	\frac{\partial f}{\partial x_i}
\right)(a).
\]
\end{izrek}

\datum{2021-10-12}

\begin{proof}
Dovolj je dokazati izrek za $n=2$, saj so preostale spremenljivke
pri parcialnem odvajanju konstantne. Naj na $f$, definirani v
okolici $(a,b)$, obstajata odvoda $f_x,f_y$, ki sta na tej okolici
zvezna, in parcialna odvoda $(f_x)_y$ ter $(f_y)_x$, ki sta zvezna
v $(a,b)$. Naj bo $(h,k)$ po normi dovolj majhen. Označimo
\[
\varphi(x) = f(x, b+k) - f(x, b).
\]
Potem je
\[
J =
f(a+h, b+k) - f(a, b+k) - f(a+h, b) + f(a, b) =
\varphi(a+h) - \varphi(a),
\]
kar je po Lagrangeu enako
\[
\varphi'(a') \cdot h = \left(
	f_x(a', b+h) - f_x(a', b)
\right) \cdot h
\]
za nek $a'$ med $a$ in $h$. S ponovno uporabo Lagrangevega izreka
dobimo, da je
\[
J = (f_x)_y(a', b')\cdot hk
\]
za nek $b'$ med $b$ in $b+h$. Simetrično dobimo, da je
\[
J = (f_y)_x(a'', b'')\cdot hk
\]
za $a''$ med $a$ in $a+h$ ter $b''$ med $b$ in $b+k$. Sledi, da je
\[
(f_x)_y(a', b') = (f_y)_x(a'', b'').
\]
Sedaj preprosto vzamemo limito $(h, k) \to (0, 0)$ in upoštevamo
zveznost.
\end{proof}

\begin{opomba}
Pravimo, da parcialni odvodi komutirajo in pišemo
$\frac{\partial^2 f}{\partial x_i\partial x_j}$.
\end{opomba}

\begin{definicija}
Naj bo $D$ odprta podmnožica $\R^n$. Vektorski prostor vseh
$k$-krat zvezno parcialno odvedljivih funkcij na $D$ označimo s
$\mathcal{C}^k(D)$. Prostor
\emph{gladkih funkcij}\index{Preslikava!Gladka} na $D$ je
\[
\mathcal{C}^\infty(D) = \bigcap_{k=1}^\infty \mathcal{C}^k(D).
\]
\end{definicija}

\newpage

\subsection{Diferenciabilnost preslikav iz $\R^n$ v $\R^m$}

\begin{okvir}
\begin{definicija}
Naj bo $D \subseteq \R^n$ z notranjo točko $a$ in
$F \colon D \to \R^m$ preslikava, definirana v okolici točke $a$.
Pravimo, da je $F$
\emph{diferenciabilna}\index{Preslikava!Diferenciabilna} v točki
$a$, če obstaja taka linearna preslikava $A \colon \R^n \to \R^m$,
da je
\[
F(a+h) = F(a) + A(h) + o(h),
\]
kjer je
\[
\lim_{h \to 0} \frac{\norm{o(h)}}{\norm{h}} = 0.
\]
\end{definicija}
\end{okvir}

\begin{opomba}
Podobno kot pri funkcijah je tak $A$, če obstaja, enolično določen
in mu pravimo \emph{diferencial} $F$ v točki $a$, kar označimo z
$d_aF$ ali $(DF)(a)$.
\end{opomba}

\begin{izrek}
Preslikava $F$ je diferenciabilna v $a$ natanko tedaj, ko so njene
koordinatne funkcije diferenciabilne v $a$. Tedaj velja\footnote{
Tej matriki pravimo \emph{Jacobijeva matrika}.}
\[
(DF)(a) =
\begin{bmatrix}
\dfrac{\partial f_1}{\partial x_1} &
\dots                             &
\dfrac{\partial f_1}{\partial x_n} \\ 
\vdots                            &
\ddots                            &
\vdots                            \\ 
\dfrac{\partial f_m}{\partial x_1} &
\dots                             &
\dfrac{\partial f_m}{\partial x_n}
\end{bmatrix}
\]
\end{izrek}

\begin{proof}
Naj bo $F$ diferenciabilna v $a$. Obstaja matrika $A$, za katero je
\[
F(a+h) = F(a) + A(h) + o(h)
\quad \text{in} \quad
\lim_{h \to 0} \frac{\norm{o(h)}}{\norm{h}} = 0.
\]
Sledi
\[
f_j(a+h) = f_j(a) + \left(\sum_{i=1}^n A_{j,i} h_i\right) + o_j(h).
\]
Ker je drugi člen na desni strani linearna funkcija v $h$ in
\[
\lim_{h \to 0} \frac{\abs{o_j(h)}}{\norm{h}} = 0,
\]
je $f_j$ diferenciabilna v $a$.

Predpostavimo sedaj, da so $f_1, \dots, f_m$ diferenciabilne.
Sledi, da obstajajo taki $A_{i,j}$, da je
\[
f_j(a+h) = f_j(a) + \left(\sum_{i=1}^n A_{j,i} h_i\right) + o_j(h),
\]
kjer je
\[
\lim_{h \to 0} \frac{\abs{o_j(h)}}{\norm{h}} = 0.
\]
Sedaj ni težko videti, da za $A = [A_{j,i}]$ in
$o = (o_1, \dots, o_m)$ velja
\[
F(a+h) = F(a) + A(h) + o(h)
\quad \text{in} \quad
\lim_{h \to 0} \frac{\norm{o(h)}}{\norm{h}} = 0.
\]
Sledi, da je $F$ res diferenciabilna v $a$, parcialni odvodi pa so
elementi matrike $A$.
\end{proof}

\begin{posledica}
Če so vsi parcialni odvodi funkcij $f_1, \dots, f_m$ zvezni v $a$,
je $F$ diferenciabilna v $a$.
\end{posledica}

\datum{2021-10-19}

\begin{izrek}
Naj bosta $D \subseteq \R^n$ in $\Omega \subseteq \R^m$ ter
$a \in D$ in $b \in \Omega$ notranji. Naj bo
$F \colon D \to \Omega$ diferenciabilna v $a$ z $F(a)=b$ in naj bo
$G \colon \Omega \to \R^k$ diferenciabilna v $b$. Potem je za
$\Phi = G \circ F$ diferenciabilna v $a$ in velja
\[
(D \Phi)(a) = (DG)(b)(DF)(a) =
\begin{bmatrix}
\dfrac{\partial g_1}{\partial x_1} &
\dots                             &
\dfrac{\partial g_1}{\partial x_m} \\ 
\vdots                            &
\ddots                            &
\vdots                            \\ 
\dfrac{\partial g_k}{\partial x_1} &
\dots                             &
\dfrac{\partial g_k}{\partial x_m}
\end{bmatrix}
\cdot
\begin{bmatrix}
\dfrac{\partial f_1}{\partial x_1} &
\dots                             &
\dfrac{\partial f_1}{\partial x_n} \\ 
\vdots                            &
\ddots                            &
\vdots                            \\ 
\dfrac{\partial f_m}{\partial x_1} &
\dots                             &
\dfrac{\partial f_m}{\partial x_n}
\end{bmatrix}.
\]
\end{izrek}

\begin{proof}
Velja
\[
F(a+h) = F(a) + (DF)(a) \cdot h + o_F(h),
\quad \text{kjer je} \quad
\lim_{h \to 0}\frac{\norm{o_F(h)}}{\norm{h}} = 0,
\]
in
\[
G(b+k) = G(b) + (DG)(b) \cdot h + o_G(h),
\quad \text{kjer je} \quad
\lim_{h \to 0}\frac{\norm{o_G(h)}}{\norm{h}} = 0.
\]
Sledi
\begin{align*}
\Phi(a+h) &= G(F(a+h))
\\
&= G(b + (DF)(a) \cdot h + o_F(h))
\\
&= \Phi(a) + (DG)(b) \cdot ((DF)(a) \cdot h +
o_F(h)) + o_G((DF)(a) \cdot h + o_F(h))
\\
&= \Phi(a) + (DG)(b)(DF)(a) \cdot h +
(DG)(b) \cdot o_F(h) + o_G((DF)(a) \cdot h + o_F(h))
\end{align*}
Dovolj je tako dokazati, da je
\[
\lim_{h \to 0}\frac{
\norm{(DG)(b) \cdot o_F(h) + o_G((DF)(a) \cdot h + o_F(h))}
}{\norm{h}} = 0.
\]
Velja
\[
\lim_{h \to 0}\frac{\norm{(DG)(b) \cdot o_F(h)}}{\norm{h}} = 0,
\]
saj obstaja tak $M$, da je
$\norm{(DG)(b) v} \leq M \cdot \norm{v}$ in
$\norm{(DF)(a) v} \leq M \cdot \norm{v}$.
Naj bo $\varepsilon > 0$. Potem obstaja tak $\delta > 0$, da za vse
$k$, za katere je $\norm{k} < \delta$, velja
\[
\norm{o_G(k)} \leq \varepsilon \cdot \norm{h}.
\]
Ker je
\[
\lim_{h \to 0} \norm{(DF)(a) \cdot h + o_F(h)} = 0,
\]
je za $\norm{h} < \delta_1$ zgornji izraz manjši od $\delta$.
Sledi, da je za dovolj majhne $h$
\[
\norm{o_G((DF)(a) \cdot h + o_F(h))} \leq
\varepsilon \cdot \norm{(DF)(a) \cdot h + o_F(h)}.
\]
Dobimo, da je
\[
\frac{\norm{o_G((DF)(a) \cdot h + o_F(a)}}{\norm{h}} \leq
\varepsilon \cdot
\frac{\norm{(DF)(a) \cdot h + o_F(h)}}{\norm{h}} \leq
\varepsilon \cdot \left(M + \frac{\norm{o_F(h)}}{\norm{h}}\right),
\]
torej je limita res enaka $0$.
\end{proof}

\newpage

\subsection{Izrek o implicitni funkciji}

\begin{izrek}[O implicitni funkciji]
\index{Izrek!O implicitni funkciji}
Naj bo $D \subseteq \R^2$ odprta množica in
$f \in \mathcal{C}^1(D)$. Naj bo $(a,b) \in D$ taka točka, da
velja:

\begin{enumerate}[i)]
\item $f(a,b) = 0$
\item $f_y(a,b) \ne 0$
\end{enumerate}

Tedaj obstajata okolica točke $a$ $I=(a-\delta, a+\delta)$ in
okolica točke $b$ $J=(b-\varepsilon, b+\varepsilon)$, kjer je
$I \times J \subseteq D$, za kateri za vse $x \in I$ obstaja
enoličen $y\in J$, za katera je $f(x,y) = 0$. Obstaja enolična
funkcija $\varphi \colon I \to J$, za katero velja

\begin{enumerate}[i)]
\item $\varphi(a) = b$
\item $\forall x \in I \colon f(x,\varphi(x)) = 0$
\item $\varphi \in \mathcal{C}^1(I)$ in
\[
\varphi'(x) = - \frac{f_x(x,\varphi(x))}{f_y(x,\varphi(x))}.
\]
\end{enumerate}
\end{izrek}

\begin{proof}
Brez škode za splošnost naj bo $f_y(a,b) > 0$. Sledi, da obstajata
taka $\delta_1 > 0$ in $\varepsilon > 0$, da je
\[
\overline{(a-\delta_1,a+\delta_1)} \times
\overline{(b-\varepsilon,b+\varepsilon)} \subseteq D,
\]
in je $f_y(x,y) > 0$ na tej okolici. Sledi, da je
$y \mapsto f(a,y)$ na tem intervalu strogo naraščajoča. Sledi, da
obstaja tak $0 < \delta \leq \delta_1$, za katerega za vse
$x \in (a-\delta, a+\delta)$ velja
\[
f(x, b+\varepsilon) > 0
\quad \text{in} \quad
f(x, b-\varepsilon) < 0.
\]
Ker je $y \mapsto f(x,y)$ strogo naraščajoča in zvezna, ima na
$(b-\varepsilon,b+\varepsilon)$ natanko eno ničlo.

Preostane nam dokaz, da je $\varphi$ odvedljiva. Naj bosta
$x, x+\Delta x \in I$ in označimo
$y = \varphi(x)$, $y + \Delta y = \varphi(x + \Delta x)$. Z uporabo
Lagrangevega izreka dobimo
\begin{align*}
0 &=
f(x + \Delta x, y + \Delta y) - f(x, y + \Delta y) +
f(x, y + \Delta y) - f(x, y)
\\
&= f_x(x', y + \Delta y) \Delta x + f_y(x, y') \Delta y,
\end{align*}
kjer je $x'$ med $x$ in $x + \Delta x$ ter $y'$ med $y$ in
$y + \Delta y$. Dobimo
\[
\Delta y = - \frac{f_x(x', y + \Delta y)}{f_y(x, y')} \Delta x.
\]
Obstajata taka $M$ in $m$, da je $\abs{f_x(x,y)} \leq M$ za vse
$(x,y) \in I \times J$ in
$f_y(\widetilde{x},\widetilde{y}) \geq m > 0$, saj sta parcialna
odvoda zvezna, $I \times J$ pa kompakt. Tako dobimo
\[
\Delta y \leq \frac{M}{m} \cdot \Delta x,
\]
zato je $\varphi$ zvezna. Velja pa
\[
\lim_{\Delta x \to 0}\frac{\Delta y}{\Delta x} =
\lim_{\Delta x \to 0} - \frac{f_x(x', y + \Delta y)}{f_y(x, y')} =
- \frac{f_x(x,\varphi(x))}{f_y(x,\varphi(x))}. \qedhere
\]
\end{proof}

\datum{2021-10-20}

\begin{opomba}
Če je $f \in \mathcal{C}^k$, je tudi $\varphi \in \mathcal{C}^k$.
\end{opomba}

\begin{definicija}
Naj bosta $D, \Omega \subseteq \R^n$ odprti. Preslikava
$F \colon D \to \Omega$ je
\emph{difeomorfizem}\index{Preslikava!Difeomorfizem}, če je

\begin{enumerate}[i)]
\item $F$ bijekcija,
\item $F \in \mathcal{C}^1(D)$,
\item $F^{-1} \in \mathcal{C}^1(\Omega)$.
\end{enumerate}
\end{definicija}

\begin{trditev}
Če je $F \colon D \to \Omega$ difeomorfizem, je $(DF)(x)$ obrnljiva
za vse $x$ in velja
\[
(DF^{-1})(F(x)) = (DF)^{-1}(x).
\]
\end{trditev}

\begin{proof}
Velja $F^{-1} \circ F = \id$, kar nam z odvajanjem da
\[
(DF^{-1})(F(x)) \cdot (DF)(x) = I. \qedhere
\]
\end{proof}

\begin{posledica}
Naj bo $F \colon D \to \Omega$. Če je $F$ difeomorfizem, je
$\det(DF)(x) \ne 0$ na $D$.
\end{posledica}

\begin{definicija}
Naj bo $F \colon \R^{n+m} \to \R^m$ in $F \in \mathcal{C}^1$.
Če je preslikava $x \mapsto F(x,y)$ za fiksen $y \in \R^m$
odvedljiva, označimo
\[
\frac{\partial F}{\partial x}(x,y) = (D_x F)(x,y) =
\begin{bmatrix}
\dfrac{\partial f_1}{\partial x_1} &
\dots                             &
\dfrac{\partial f_1}{\partial x_n} \\ 
\vdots                            &
\ddots                            &
\vdots                            \\ 
\dfrac{\partial f_m}{\partial x_1} &
\dots                             &
\dfrac{\partial f_m}{\partial x_n}
\end{bmatrix}.
\]
Podobno označimo odvod po $y$.
\end{definicija}

\begin{izrek}[O implicitni preslikavi]
\index{Izrek!O implicitni preslikavi}
\label{iz:imp}
Naj bo $D \subseteq \R^n \times \R^m$ odprta in $(a,b) \in D$.
Naj bo $F \colon D\to \R^m$ in $D \in \mathcal{C}^1(D)$. Če velja

\begin{enumerate}[i)]
\item $F(a,b) = 0$ in
\item $\det \frac{\partial F}{\partial y}(a,b) \ne 0$,
\end{enumerate}

potem obstajata taki okolici $U$ točke $a$ in $V$ točke $b$, pri
čemer je $U \times V \subseteq D$, in obstaja enolično določena
preslikava $\varphi \colon U \to V$, za katero velja
$\varphi \in \mathcal{C}^1(U)$ in
\begin{enumerate}[i)]
\item $\varphi(a) = b$
\item $\forall (x,y) \in U \times V \colon F(x,y) = 0 \iff
y = \varphi(x)$
\item $(D\varphi)(x) = -(D_yF)^{-1}(x,y) (D_xF)(x,y)$
\end{enumerate}
\end{izrek}

\begin{proof}
Oglejmo si preslikavo $\Phi \colon D \to \R^n \times \R^m$, za
katero je
\[
\Phi(x,y) = (x, F(x,y)).
\]
Velja $\Phi(a,b) = (a,0)$ in
\[
(D\Phi)(a,b) =
\begin{bmatrix}
I_{\R^n}    & 0           \\ 
(D_xF)(a,b) & (D_yF)(a,b)
\end{bmatrix},
\]
zato je $\det(D\Phi)(a,b) = \det (D_yF)(a,b) \ne 0$. Po izreku
\ref{iz:inv} sledi, da obstaja inverzna preslikava na okolici
$(a,0)$, ki slika po predpisu
$\Phi^{-1} \colon (x,w) \mapsto (x, G(x,w))$. Sedaj lahko
preprosto vzamemo $\varphi \equiv G(x,0)$ na dovolj majhni
okolici $a$.
\end{proof}

\begin{opomba}
Če je $F \in \mathcal{C}^k(D)$, je $\varphi \in \mathcal{C}^k(U)$.
\end{opomba}

\datum{2021-10-26}

\begin{lema}
Naj bo $f \colon D \to \R$, $f \in \mathcal{C}^1$. Naj bosta
$a,b \in D$ taki točki, da celotna daljica $(1-t)a + tb$ leži v
$D$.
Potem obstaja tak $\xi$ s te daljice, da velja
\[
f(b) - f(a) = (Df)(\xi) \cdot (b-a).
\]
\end{lema}

\begin{proof}
$\varphi \colon t \mapsto f((1-t)a + tb)$ zadošča Lagrangevem
izreku. Sledi, da obstaja tak $\tau$, da je
\[
f(b) - f(a) =
\varphi(1) - \varphi(0) =
\varphi'(\tau) =
(Df)((1-\tau)a + \tau b) \cdot (b - a). \qedhere
\]
\end{proof}

\begin{posledica}\label{pos:1}
Če obstaja tak $M$, da za vse $x$ in $j$ velja
\[
\abs{\frac{\partial f}{\partial x_j}(x)} \leq M,
\]
velja
\[
\abs{f(b) - f(a)} \leq M \sqrt{n} \cdot \abs{b - a}.
\]
\end{posledica}

\begin{proof}
Po Cauchyju je
\[
\abs{f(b) - f(a)} =
\abs{(DF)(\xi) \cdot (b - a)} \leq
\abs{(Df)(\xi)} \cdot \abs{b - a} \leq
M \sqrt{n} \cdot \abs{b - a}. \qedhere
\]
\end{proof}

\begin{posledica}\label{pos:2}
Naj bo $F \colon D \to \R^m$, $F \in \mathcal{C}^1$ in
$F = (f_1, \dots, f_m)$. Denimo, da obstaja tak $M$, da za vse $x$
in $j$ velja
\[
\abs{\frac{\partial f}{\partial x_j}(x)} \leq M.
\]
Tedaj je
\[
\norm{f(b) - f(a)} \leq M \sqrt{m \cdot n} \cdot \norm{b - a}.
\]
\end{posledica}

\begin{proof}
Po posledici \ref{pos:1} velja
\[
\norm{f(b) - f(a)} =
\sqrt{\sum_{i=1}^m \left(f_i(b) - f_i(a)\right)^2} \leq
M \sqrt{m \cdot n} \cdot \norm{b - a}. \qedhere
\]
\end{proof}

\begin{lema}\label{lm:1}
Naj bo $H \colon D \to \R^n$ in $H \in \mathcal{C}^1(D)$, kjer je
$D \subseteq \R^n$ odprta. Če je $H(0)=0$ in $(DH)(0)=0$, obstaja
tak $r>0$, da za vse $x_1, x_2 \in \overline{\mathcal{K}(0,r)}$
velja
\[
\norm{H(x_1) - H(x_2)} \leq \frac{1}{2} \norm{x_1 - x_2}.
\]
\end{lema}

\begin{proof}
Ker je $H \in \mathcal{C}^1(D)$, obstaja tak $r > 0$, da na
$\overline{\mathcal{K}(0,r)}$ velja
\[
\abs{\frac{\partial h_i}{\partial x_j}(x)} \leq \frac{1}{2n}.
\]
Po posledici \ref{pos:2} dobimo, da je
\[
\norm{H(x_1) - H(x_2)} \leq
\frac{1}{2n} \cdot n \cdot \norm{x_1 - x_2} =
\frac{1}{2} \norm{x_1 - x_2}. \qedhere
\]
\end{proof}

\begin{izrek}[O inverzni preslikavi]
\index{Izrek!O inverzni preslikavi}
\label{iz:inv}
Naj bo $D \subseteq \R^n$ odprta, $F \colon D \to \R^n$,
$F \in \mathcal{C}^1(D)$ in $\det(DF)(a) \ne 0$. Tedaj obstajata
okolica $U \subseteq D$ točke $a$ in okolica $V \subseteq \R^n$
točke $b=F(a)$, za kateri je $F \colon U \to V$ difeomorfizem.
\end{izrek}

\begin{proof}
Brez škode za splošnost naj bo $0 = a = F(a)$ in $(DF)(0) = I$ in
označimo $F(x) = x + H(x)$. Sledi, da je
\[
\lim_{x \to 0}\frac{\norm{H(x)}}{\norm{x}} = 0.
\]

Po lemi \ref{lm:1} obstaja tak $r>0$, da na
$\overline{\mathcal{K}(0,r)}$ velja
\[
\norm{F(x_1) - F(x_2)} =
\norm{x_1 - x_2 + H(x_1) - H(x_2)} \geq
\frac{1}{2} \norm{x_1 - x_2},
\]
zato je $F$ na tej okolici injektivna, njen inverz pa je zvezen.

Trdimo, da je
$\mathcal{V} = \overline{\mathcal{K}\left(0,\frac{r}{2}\right)}
\subseteq F(\overline{\mathcal{K}(0,r)})$. Za
$y \in \mathcal{V}$ tako iščemo
$x \in \overline{\mathcal{K}(0,r)}$, za katerega je
$x = -H(x) + y = T_y(x)$. Iščemo torej fiksno točko preslikave
$T_y$, ki slika iz $\overline{\mathcal{K}(0,r)}$ v
$\overline{\mathcal{K}(0,r)}$, saj je
\[
\norm{T_y(x)} = \norm{-H(x) + y} \leq r.
\]
Sledi, da je $T_y$ skrčitev za $q=\frac{1}{2}$, saj je $H$ skrčitev
z istim koeficientom. Tak $x$ torej obstaja po Banachovem
skrčitvenem načelu.\footnote{Izrek 7.4.2 v zapiskih predmeta
Analiza 1 prvega letnika.}

Naj bo $\mathcal{U} = F^{-1}(\mathcal{V}) \cap \mathcal{K}(0,r)$.
Vidimo, da je $F \colon \mathcal{U} \to \mathcal{V}$ bijekcija z
zveznim inverzom $G$. Preostane le še dokaz, da je $G$
diferenciabilna z diferencialom
\[
(DG)(y) = (DF)^{-1}(G(y)).
\]
Naj bo $y \in \mathcal{V}$ in $k \in \R^n$ dovolj majhen. Označimo
$G(y) = x$ in $G(y + k) = x + h$. Sledi, da je $y = F(x)$ in
\[
y + k = F(x + h) = F(x) + (DF)(x)\cdot h + o_F(h).
\]
Iz leme \ref{lm:1} sledi, da je
$\norm{k} \geq \frac{1}{2} \norm{h}$. Sedaj si oglejmo
\begin{align*}
\lim_{k \to 0}
\frac{\norm{G(y+k) - G(y) - (DF)^{-1}(x) \cdot h}}{\norm{k}}
&= \lim_{k \to 0}
\frac{\norm{h - (DF)^{-1}(x)((DF)(x)\cdot h + o_F(h))}}{\norm{k}}
\\
&= \lim_{k \to 0}
\frac{\norm{h}}{\norm{k}} \cdot
\frac{\norm{(DF)^{-1}(x)o_F(h)}}{\norm{h}}
\\
&\leq \lim_{k \to 0} 2M \cdot \frac{\norm{o_F(h)}}{\norm{h}},
\end{align*}
kjer je $M$ supremum iz trditve \ref{td:1} za $(DF)^{-1}(x)$. Ko
$k$ limitira proti $0$, tudi $h$ limitira proti $0$, s tem pa je
izrek dokazan.
\end{proof}

\begin{opomba}
Če je $F \in \mathcal{C}^k(D)$, je $F^{-1} \in \mathcal{C}^k(V)$.
\end{opomba}

\begin{opomba}
Naj bo $D \subseteq \R^n$ odprta, $F \colon D \to \R^n$,
$F \in \mathcal{C}^1(D)$ in $\det(DF)(x) \ne 0$ za vse $x \in D$.
Potem je $F$ \emph{lokalni difeomorfizem}.
\end{opomba}

\datum{2021-10-27}

\begin{definicija}
Naj bo $D \subseteq \R^n$ odprta in $F \colon D \to \R^m$,
$F \in \mathcal{C}^1$. \emph{Rang}\index{Preslikava!Rang}
preslikave $F$ v točki $a \in D$ je $r = \rang DF(a)$. če je rang
konstanten na $D$, pravimo, da je preslikava $F$ ranga $r$ na $D$.
\end{definicija}

\begin{opomba}
Pravimo, da je $F$ \emph{maksimalnega ranga}, če je
$r = \min\set{m,n}$. 
\end{opomba}

\begin{posledica}
Naj bo $D \subseteq \R^n$ odprta, $F \colon D \to \R^m$,
$F \in \mathcal{C}^1$ in $m < n$. Naj bo $F$ v $a \in D$
maksimalnega ranga in $F(a)=0$. Tedaj obstajajo indeksi
\[
i_1,\dots,i_{n-m}, \quad j_1,\dots,j_m \quad \text{in} \quad
\forall k,l \colon i_k \ne j_l
\]
in take $\mathcal{C}^1$ funkcije $\varphi_1,\dots,\varphi_m$,
definirane v okolici $(a_{i_1},\dots,a_{i_{n-m}})$, da je v neki
okolici $\mathcal{U}$ točke $a$ enačba $F(X)=0$ ekvivalentna
sistemu
\[
\forall k \colon x_{j_k} = \varphi_k(x_{i_1},\dots,x_{i_{n-m}}).
\]
\end{posledica}

\begin{proof}
Z ustrezno permutacijo koordinat se trditev reducira na izrek
o implicitni preslikavi.
\end{proof}

\begin{posledica}
Naj bo $D \subseteq \R^n$ odprta, $F \colon D \to \R^m$,
$F \in \mathcal{C}^1$ in $m \leq n$. Naj bo $F$ v $a \in D$
maksimalnega ranga. Potem obstaja taka okolica $\mathcal{V}$
točke $b = F(a)$ v $\R^m$ in okolica $\mathcal{U}$ točke $a$ v $D$,
da je $F \colon \mathcal{U} \to \mathcal{V}$ surjektivna.
\end{posledica}

\begin{proof}
Za $n=m$ je to posledica izreka o inverzni preslikavi. Za $m < n$
si oglejmo preslikavo $\Phi(x,y) = F(x) - y$. Velja $\Phi(a,b)=0$
in
\[
(D\Phi)(x,y) =
\begin{bmatrix}
(DF)(x) & -I
\end{bmatrix}.
\]
Brez škode za splošnost naj bo zadnjih $m$ stolpcev $(DF)(a)$
linearno neodvisnih. Po izreku o implicitni preslikavi lahko enačbo
\[
F(x) - y = 0
\]
razrešimo na $x_{n-m+1},\dots,x_n$ kot funkcije $x_1,\dots,x_{n-m}$
in $y_1,\dots,y_m$ v okolici $(a,b)$.
\end{proof}

\newpage

\subsection{Taylorjeva formula}

\begin{izrek}
Naj bo $f \in \mathcal{C}^{k+1}(D)$, kjer je $D \subseteq \R^n$
odprta. Naj bosta $a \in D$ in $h \in \R^n$ taki točki, da celotna
daljica $a+th$ za $t \in [0,1]$ leži v $D$. Potem obstaja tak
$\theta \in (0,1)$, da je
\[
f(a+h) = \sum_{i=0}^k \frac{(D_h^if)(a)}{i!} + R_k,
\]
kjer je
\[
R_n = \frac{(D_h^{k+1}f)(a + \theta h)}{(k+1)!}
\quad \text{in} \quad
D_h = \sum_{i=1}^n \frac{\partial}{\partial x_i} h_i.
\]
\end{izrek}

\begin{proof}
Naj bo $\varphi(t) = f(a + th)$. Po Taylorju za $\varphi$ velja
\[
\varphi(1) = \sum_{i=0}^k \frac{\varphi^{(i)}(0)}{i!} \cdot(1-0)^i
+ \frac{\varphi^{(k+1)}(\theta)}{(k+1)!} \cdot (1-0)^{k+1}.
\]
Ker velja
\[
\varphi^{(i)}(t) = (D_h^if)(a+th),
\]
je izrek dokazan.
\end{proof}

\begin{opomba}
Če je $f \in \mathcal{C}^\infty(D)$, lahko tvorimo Taylorjevo vrsto
\[
f(a+h) = \sum_{i=0}^\infty \frac{(D_h^if)(a)}{i!}.
\]
Če ta vrsta konvergira k $f(a+h)$ za nek $a$ in vse dovolj majhne
$h$, pravimo, da je $f$
\emph{realno analitična}\index{Preslikava!Analitična} funkcija.
\end{opomba}

\begin{posledica}
Naj bo $f \in \mathcal{C}^{k+1}(D)$, kjer je $D \subseteq \R^n$
odprta. Naj bosta $a \in D$ in $h \in \R^n$ taki točki, da celotna
daljica $a+th$ za $t \in [0,1]$ leži v $D$. Tedaj velja\footnote{
Velja $O(\norm{h}^{k+1}) \leq M \cdot \norm{h}^{k+1}$ za vse
$\norm{h} < \delta$.}
\begin{align*}
f(a+h) &= \sum_{i=0}^k \frac{(D_h^if)(a)}{i!} + o(\norm{h}^k)
\\
&= \sum_{i=0}^k \frac{(D_h^if)(a)}{i!} + O(\norm{h}^{k+1})
\end{align*}
\end{posledica}

\begin{proof}
Velja
\[
R_n = \frac{(D_h^{k+1}f)(a + \theta h)}{(k+1)!}.
\]
Ker je $f \in \mathcal{C}^{k+1}(D)$, so odvodi $f$ zvezni na $D$,
zato so na $\overline{\mathcal{K}(0,r)} \subseteq D$ omejeni. Sedaj
lahko preprosto razpišemo $(D_h^{k+1}f)$ in uporabimo
$\abs{h_i} \leq \norm{h}$.
\end{proof}

\newpage

\subsection{Ekstremi}

\begin{definicija}
Naj bo $D \subseteq \R^n$ in $f \colon D \to \R$ funkcija.

\begin{enumerate}[i)]
\item $f$ ima v $a \in D$ \emph{lokalni maksimum}, če obstaja tak
$r > 0$,
da je
\[
\forall x \in D \cap \mathcal{K}(0,r) \colon f(a) \geq f(x).
\]
\item $f$ ima v $a \in D$ \emph{globalni maksimum}, če velja
\[
\forall x \in D \colon f(a) \geq f(x).
\]
\end{enumerate}

Simetrično definiramo \emph{lokalni} in \emph{globalni minimum} ter
\emph{lokalni} in
\emph{globalni ekstrem}\index{Preslikava!Ekstrem}.
\end{definicija}

\begin{opomba}
Če je $D$ kompakt in $f$ zvezna, ima $f$ globalna ekstrema.
\end{opomba}

\begin{definicija}
Naj bo $f \colon D \to \R$ funkcija, $a \in D $ notranja točka in
naj bo $f$ v $a$ diferenciabilna. Če je $(Df)(a) = 0$, je $a$
\emph{stacionarna točka}\index{Preslikava!Stacionarna točka} za
$f$.
\end{definicija}

\datum{2021-10-28}

\begin{trditev}
Naj bo $D \subseteq \R^n$, $f \colon D \to \R$, $a \in D$
notranja in $f$ diferenciabilna v $a$. Če ima $f$ lokalni ekstrem
v $a$, je $a$ stacionarna točka za $f$.
\end{trditev}

\begin{proof}
Funkcije $\varphi_i(t) = f(a_1,\dots,t,\dots,a_n)$ imajo lokalni
ekstrem v $a_i$, zato so vsi parcialni odvodi pri $a$ enaki $0$.
\end{proof}

\begin{definicija}
Naj bo $f \in \mathcal{C}^2(D)$.
\emph{Hessejeva matrika}\index{Preslikava!Hessejeva matrika}
funkcije $f$ je matrika
\[
H_f(x) = \begin{bmatrix}
\dfrac{\partial^2 f}{\partial x_i \partial x_j}(x)
\end{bmatrix}.
\]
\end{definicija}

\begin{opomba}
V stacionarni točki $a$ je Taylorjev razvoj funkcije $f$ enak
\[
f(a+h) = f(a) + \frac{1}{2} \skl{H_f(a + \theta h)h,h}.
\]
\end{opomba}

\begin{trditev}\label{td:2}
Naj bo $f \in \mathcal{C}^2(D)$, kjer je $D \subseteq \R^n$ odprta,
in $a \in D$.

\begin{enumerate}[i)]
\item Če ima $f$ v $a$ lokalni minimum, je
$H_f(a) \geq 0$.\footnote{$H_f(a)$ je \emph{pozitivno
semidefinitna}, glej definicijo 7.8.1 v zapiskih predmeta Algebra 1
prvega letnika.}
\item Če ima $f$ v $a$ lokalni maksimum, je $H_f(a) \leq 0$.
\end{enumerate}
\end{trditev}

\begin{proof}
Naj ima $f$ v $a$ lokalni minimum. Sledi, da je $(DF)(a) = 0$. Za
$h \in \R^n$ opazujemo preslikavo $\varphi(t)=f(a + th)$. Dobimo,
da je $\varphi'(t) = 0$ in $\varphi''(t) \geq 0$, zato je
\[
0 \leq \varphi''(t) = \skl{H_f(a + th)h, h}. \qedhere
\]
\end{proof}

\begin{izrek}
Naj bo $D \subseteq \R^n$ odprta, $f \in \mathcal{C}^2(D)$ in
$a \in D$ stacionarna točka $f$.

\begin{enumerate}[i)]
\item Če je $H_f(a) > 0$, ima $f$ v $a$ strogi lokalni minimum.
\item Če je $H_f(a) < 0$, ima $f$ v $a$ strogi lokalni maksimum.
\item Če ima $H_f(a)$ pozitivne in negativne lastne vrednosti, $f$
v $a$ nima lokalnega ekstrema.
\end{enumerate}
\end{izrek}

\begin{proof}
Tretja točka sledi direktno iz trditve \ref{td:2}.

Naj bo $H_f(a) > 0$ in $h$ dovolj majhen. Potem je
\begin{align*}
f(a+h) - f(a) &= \frac{1}{2} \skl{H_f(a + \theta h)h,h}
\\
&= \frac{1}{2} \norm{h}^2
\left(\skl{H_f(a)v,v} + \skl{E(h)v,v}\right)
\end{align*}
za $v = \frac{h}{\norm{h}}$ in $E(h) = H_f(a + \theta h) - H_f(a)$.
Ker je $\setb{v \in \R^n}{\norm{v} = 1}$ kompaktna, obstaja tak
$m > 0$, da je $\skl{H_f(a)v,v} \geq m$. Ker je
$f \in \mathcal{C}^2(D)$, je $H_f$ zvezna, zato za dovolj majhne
$h$ velja $\abs{\skl{E(h)v,v}} \leq \frac{m}{2}$ in
\[
f(a+h) - f(a) \geq
\frac{1}{2} \norm{h}^2 \cdot \frac{m}{2}. \qedhere
\]
\end{proof}

\begin{posledica}
Naj bo $D \subseteq \R^2$ odprta, $f \in \mathcal{C}^2(D)$ in
$a \in D$ stacionarna točka $f$.

\begin{enumerate}[i)]
\item Če je $\det H_f(a) > 0$, ima $f$ v $a$ lokalni ekstrem.

\begin{enumerate}[a)]
\item Če je $f_{xx}(a) > 0$, ima $f$ v $a$ lokalni minimum.
\item Če je $f_{xx}(a) < 0$, ima $f$ v $a$ lokalni maksimum.
\end{enumerate}

\item Če je $\det H_f(a) < 0$, $f$ v $a$ nima lokalnega ekstrema.
\end{enumerate}
\end{posledica}

\obvs

\newpage

\subsection{Vezani ekstremi}

\begin{izrek}
Naj bo $m < n$ in $D \subseteq \R^n$ odprta. Naj bo
$G \colon D \to \R^m$ $\mathcal{C}^1$ preslikava na $D$ ranga $m$
in $G = (g_1,\dots,g_m)$. Naj bo
$M = G^{-1}(\set{0}) \ne \emptyset$ in $f \in \mathcal{C}^1(D)$. Če
ima $f$ v $p \in M$ lokalni ekstrem kot funkcija
$f \colon M \to \R$, obstajajo take konstante
$\lambda_1,\dots,\lambda_m \in \R$, da je
\[
(Df)(p) = \sum_{i=1}^m \lambda_i (Dg_i)(p).
\]
\end{izrek}

\begin{proof}
Naj bo $\Phi(x) = (f(x), G(x))$. Velja $\Phi(p) = (f(p),0)$. Če je
$\rang(D\Phi)(p) = m+1$, je $(D\Phi)(p)$ maksimalnega ranga, zato
je $\Phi$ iz okolice $p$ v okolico $(f(p),0)$ surjektivna, zato so
v zalogi vrednosti tako točke oblike $(f(p)+\varepsilon,0)$ kot
$(f(p)-\varepsilon,0)$, zato $f$ v $p$ nima lokalnega ekstrema.
Sledi, da je
\[
\rang(D\Phi)(p) = \rang(DG)(p),
\]
zato je $(Df)(p)$ linearna kombinacija $(Dg_i)(p)$.
\end{proof}

\begin{opomba}[Lagrangeva metoda]\index{Lagrangeva metoda}
Številom $\lambda_i$ pravimo \emph{Lagrangevi multiplikatorji}.
Za vsak ekstrem $f$ na $M$ obstaja tak $\lambda$, da za
\[
F(x, \lambda) = f(x) - \sum_{i = 1}^m \lambda_i g_i(x)
\]
velja $(DF)(x,\lambda) = 0$. Dobimo sistem
\[
\frac{\partial f}{\partial x_j} -
\sum_{i=1}^m \lambda_i \frac{\partial g_i(x)}{\partial x_j} = 0
\quad \text{in} \quad
g_i(x) = 0.
\]
\end{opomba}
