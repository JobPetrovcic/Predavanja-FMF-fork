\section{Preseki krivulj}

\subsection{Večkratnosti presečišč}

\datum{2022-3-17}

\begin{definicija}
\emph{Večkratnost}\index{Presečišče!Večkratnost} presečišča
projektivne premice in krivulje je stopnja pripadajočega linearnega
faktorja v faktorizaciji polinoma
\[
F(t(x,y)),
\]
kjer je $t$ parametrizacija premice.
\end{definicija}

\begin{opomba}
Vsota večkratnosti vseh presečišč je enaka $\deg F$.
\end{opomba}

\begin{trditev}
Naj bosta $f$ in $g$ polinoma v dveh spremenljivkah, monična v
spremenljivki $y$. Tedaj imata za vsako ničlo $a$ polinoma
$\Res(f,g)$ polinoma $f(a, y)$ in $g(a, y)$ skupno ničlo.
\end{trditev}

\begin{proof}
Velja
\[
\Res(f(a, y), g(a, y)) = 0,
\]
zato imata skupni faktor.
\end{proof}

\begin{trditev}
Naj bosta
\[
F(x, y, z) = \sum_{i=0}^m a_i(x,y) z^{m-i}
\quad \text{in} \quad
G(x, y, z) = \sum_{i=0}^n b_i(x,y) z^{n-i}
\]
homogena polinoma. Tedaj je $\Res(f,g)$ homogen polinom stopnje
\[
\deg F \cdot \deg G - \deg a_0 \cdot \deg b_0.
\]
\end{trditev}

\begin{proof}
Naj bo $d = \deg a_0$ in $e = \deg b_0$. Sledi, da je
\[
\deg a_i = d + i
\quad \text{in} \quad
\deg b_i = e + i.
\]
Naj bo $R = \Res(f, g)$. Dovolj je pokazati, da za vse $\lambda$
velja
\[
R(\lambda x, \lambda y) = \lambda^{mn + dn + em} R(x, y).
\]
Z uporabo $a_i(\lambda x, \lambda y) = \lambda^{d + i} a_i(x, y)$
lahko determinanto rezultante poenostavimo. Prvo vrstico razširimo
z $\lambda^e$, drugo z $\lambda^{e+1}$ in tako dalje v prvih $n$
vrsticah, nato pa podobno naredimo v zadnjih $m$ vrsticah. S tem
smo zagotovili, da so v vsakem stolpcu eksponenti $\lambda$ enaki
in jih lahko izpostavimo. Sledi, da je
\[
\prod_{i=e}^{e+n-1} \lambda^i \cdot
\prod_{i=d}^{d+m-1} \lambda^i \cdot
R(\lambda x, \lambda y) =
\prod_{i=e+d}^{d+e+m+n-1} \lambda^i \cdot R(x, y),
\]
oziroma
\[
\lambda^{\frac{n (2e + n - 1)}{2} + \frac{m(2d + m - 1)}{2}}
R(\lambda x, \lambda y) =
\lambda^{\frac{(2d + 2e + m + n - 1)(m + n)}{2}} R(x, y).
\]
Sledi, da je
\[
R(\lambda x, \lambda y) = \lambda^{mn + dn + em} R(x, y). \qedhere
\]
\end{proof}

\begin{trditev}
Naj bosta $F$ in $G$ nekonstantna polinoma brez skupnega
nekonstantnega faktorja. Tedaj je $V_h(F) \cap V_h(G)$ končna
množica. % Število je omejeno s produktom stopenj, vsota
% večkratnosti je enaka temu
\end{trditev}

\begin{proof}
Naj bo $S$ točka, ki ne leži na $V_h(FG)$. Naj bo $\ell$ premica,
ki ne gre skozi $S$. S pomočjo leme o štirih točkah lahko
privzamemo, da je $S = (0 : 0 : 1)$ in ima premica $\ell$ enačbo
$z = 0$.

Naj bo $(a : b : 0)$ projekcija presečišča prek $S$ na premico
$\ell$. Sledi, da ima presečišče koordinate $(a : b : c)$. Iščemo
torej taka števila $a$ in $b$, da imata $F(a, b, z)$ in
$G(a, b, z)$ skupno ničlo, to pa so ravno ničle polinoma
$\Res_{f,g}(a,b)$, to pa je neničeln homogen polinom v dveh
spremenljivkah. Sledi, da ga lahko faktoriziramo na linearne
faktorje, zato ima končno mnogo ničel.

Opazimo še, da na vsaki premici skozi $S$ leži končno mnogo
presečišč. V nasprotnem primeru je namreč premica vsebovana v eni
izmed krivulj, kar je v protislovju s tem, da $S$ ne leži na
krivuljah. 
\end{proof}
