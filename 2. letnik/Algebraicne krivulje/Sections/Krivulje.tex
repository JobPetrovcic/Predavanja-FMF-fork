\section{Algebraične krivulje in projektivno zaprtje}

\subsection{Definicija}

\datum{2022-2-17}

\begin{definicija}
Polinom $P \in K[x_1, \dots, x_n]$ je
\emph{nerazcepen}\index{Polinom!Nerazcepen}, če se ga ne da zapisati
kot produkt dveh nekonstantnih polinomov iz $K[x_1, \dots, x_n]$.
\end{definicija}

\begin{definicija}
Za polinom $F \in K[x,y]$ označimo njegovo množico ničel
\[
V(F) = \setb{(a,b) \in K^2}{F(a,b) = 0}.
\]
\end{definicija}

\begin{opomba}
Množicam oblike $V(f)$ pravimo
\emph{(afine) algebraične množice}\index{Algebraična množica}.
\end{opomba}

\begin{okvir}
\begin{definicija}
Množica $\mathcal{C} \subseteq K^2$ je
\emph{algebraična krivulja}\index{Algebraična krivulja}, če obstaja
tak nekonstanten polinom $F \in K[x,y]$, da je
\[
\mathcal{C} = V(F).
\]
Pravimo, da je krivulja
\emph{nerazcepna}\index{Algebraična krivulja!Nerazcepna}, če je v
zgornji definiciji $F$ nerazcepen polinom.
\end{definicija}
\end{okvir}

\begin{definicija}
\emph{Afina preslikava}\index{Afina preslikava} je kompozitum
linearne preslikave in translacije. Če je ta linearna preslikava
obrnljiva, je tudi afina preslikava obrnljiva in ji pravimo
\emph{afina transformacija}.
\end{definicija}

\begin{trditev}
Kompozitum afinih transformacij je afina transformacija.
\end{trditev}

\begin{proof}
Afine transformacije so natanko preslikave
\[
(x,y) \mapsto (ax + by + \alpha, cx + dy + \beta),
\]
kjer je $ad \ne bc$.
\end{proof}

\begin{definicija}
Krivulji $\mathcal{C}$ in $\mathcal{D}$ sta
\emph{afino ekvivalentni}\index{Algebraična krivulja!Afino ekvivalentna},
če obstaja afina transformacija $\Phi$, za katero je
$\Phi(\mathcal{C}) = \mathcal{D}$.
\end{definicija}

\begin{opomba}
Afina ekvivalenca je ekvivalenčna relacija.
\end{opomba}

\newpage

\subsection{Studyjeva lema}

\datum{2022-2-24}

\begin{definicija}
\emph{Minimalni polinom}\index{Algebraična množica!Minimalni polinom}
algebraične množice $V(f)$ je produkt nerazcepnih faktorjev $f$.
\end{definicija}

\begin{definicija}
\emph{Stopnja}\index{Algebraična množica!Stopnja} algebraične
množice je stopnja njenega minimalnega polinoma.
\end{definicija}

\begin{definicija}
Naj bo $A$ komutativen kolobar in $f, g \in A[x]$. Označimo
\[
f = \sum_{i=0}^m a_i x^{m-i}
\quad \text{in} \quad
g = \sum_{i=0}^n b_i x^{n-i}.
\]
\emph{Rezultanto}\index{Polinom!Rezultanta} polinomov $f$ in $g$
definiramo kot
\[
\Res(f,g) =
\det \begin{bmatrix}
& \tikzmark{l1} a_0 & a_1 & \dots & a_m & & & \\
& & \ddots & \ddots & \ddots & \ddots & & \\
& & & a_0 & a_1 & \dots & a_m \tikzmark{r1} & \vspace{12pt} \\
&\tikzmark{l2} b_0 & b_1 & \dots & b_n & & & \\
& & \ddots & \ddots & \ddots & \ddots & & \\
& & & b_0 & b_1 & \dots & b_n \tikzmark{r2} &
\end{bmatrix}
\DrawBox[thick, blue,fill=yellow!20, fill opacity=0.2]{l1}{r1}{\textcolor{black}{\scriptsize $(n+m) \times n$}}
\DrawBox[thick, blue,fill=yellow!20, fill opacity=0.2]{l2}{r2}{\textcolor{black}{\scriptsize $(n+m) \times m$}}
\]
\end{definicija}

\begin{izrek}
Naj bo $A$ komutativen kolobar brez deliteljev niča z enolično
faktorizacijo. Za nekonstantna polinoma $f, g \in A[x]$ sta
naslednji trditvi ekvivalentni:

\begin{enumerate}[i)]
\item $\Res(f,g) = 0$
\item $f$ in $g$ imata skupen nekonstanten faktor.
\end{enumerate}
\end{izrek}

\begin{proof}
Dokazali bomo, da sta obe trditvi ekvivalentni temu, da obstajata
$\varphi, \psi \in A[x]$, ne oba enaka $0$, za katera velja
\[
\varphi f + \psi g = 0,
\quad
\deg \varphi < \deg g
\quad \text{in} \quad
\deg \psi < \deg f.
\]
Rezultanta je enaka nič natanko tedaj, ko so vrstice linearno
odvisne, od koder dobimo polinoma $\varphi$ in $\psi$. Zaradi
pogoja s stopnjami dobimo, da imata $f$ in $g$ skupen faktor.

Za obratno smer preprosto izberemo
\[
\varphi = \frac{g}{\gcd(f,g)}
\quad \text{in} \quad
\psi = -\frac{f}{\gcd(f,g)}. \qedhere
\]
\end{proof}

\begin{lema}[Study]\index{Lema!Study}
Naj bo $f \in \C[x,y]$ nerazcepen nekonstanten polinom. Tedaj za
vsak polinom $g \in \C[x,y]$ velja
\[
f \mid g \iff V(f) \subseteq V(g).
\]
\end{lema}

\begin{proof}
Naj bo
\[
f = \sum_{i=0}^m a_i x^{m-i}
\quad \text{in} \quad
g = \sum_{i=0}^n b_i x^{n-i},
\]
kjer so $a_i, b_i \in \C[y]$.\footnote{Ker je $\C$ komutativen,
velja $\C[x,y] = \C[y][x]$.} Brez škode za splošnost naj bo
$m \geq 1$. Ker je $a_0 \ne 0$, obstaja tak $y_0$, da je
$a_0(y_0) \ne 0$.

Oglejmo si polinom $f_{y_0}(x) = f(x,y_0)$. Ker je $\C$ algebraično
zaprto polje, ima ta polinom ničlo $x_0$. Sledi, da je
$f(x_0, y_0) = 0$, zato $(x_0, y_0) \in V(g)$, zato je tudi
\[
g_{y_0}(x_0) = 0.
\]
Sledi, da imata polinoma $f_{y_0}$ in $g_{y_0}$ skupni faktor
$x - x_0$ in je njuna rezultanta enaka $0$. Sledi, da je $y_0$
ničla rezultante $\Res(f,g)$. Ker to velja za skoraj vse $y_0$, je
$\Res(f,g) = 0$, oziroma, da imata $f$ in $g$ skupni faktor, to je
$f$.
\end{proof}

\begin{opomba}
Zgornja lema je znana tudi pod imenom \emph{Nullstellensatz}.
\end{opomba}

\begin{posledica}
Za vsak nekonstanten polinom $f \in \C[x,y]$ velja
$V(f) \ne \emptyset$.
\end{posledica}

\begin{proof}
Naj bo $h$ nerazcepen faktor $f$. Tedaj za vsak $g \in \C[x,y]$
velja $\emptyset = V(h) \subseteq V(g)$, zato $h \mid g$, kar je
protislovje.
\end{proof}

\begin{posledica}
Vsaka algebraična množica enolično določa nerazcepne faktorje
pripadajočega polinoma. Vsako algebraično množico lahko na enoličen
način zapišemo kot unijo nerazcepnih.
\end{posledica}

\begin{proof}
Naj bo
\[
f = c \cdot \prod_{i=1}^k f_i^{n_i}.
\]
Sledi, da je
\[
V(f) = \bigcup_{i=1}^k V(f_i).
\]
Če je $V(f) = V(g)$, od tod sledi, da $f_i \mid g$ za vse $i$.
Simetrično dobimo $g_i \mid f$.
\end{proof}

\newpage

\subsection{Projektivna ravnina}

\datum{2022-3-3}

\begin{definicija}
Naj bo $K$ polje. \emph{Afina ravnina}\index{Afina ravnina} je
množica $A_2(K) = K^2$.
\end{definicija}

\begin{definicija}
Naj bo $K$ polje.
\emph{Projektivna ravnina}\index{Projektivna ravnina} je množica
vseh premic v $K^3$, ki potekajo skozi izhodišče. Označimo jo s
$P_2(K)$.
\end{definicija}

\begin{definicija}
\emph{Projektivne koordinate}\index{Projektivna ravnina!Koordinate}
projektivne točke je razmerje
\[
(x : y : z).
\]
\end{definicija}

\begin{opomba}
Vsakim projektivnim koordinatam, različnim od $(0 : 0 : 0)$,
ustreza natanko ena projektivna točka.
\end{opomba}

\begin{opomba}
Projektivno ravnino lahko identificiramo z afino ravnino, ki ji
dodamo \emph{točke v neskončnosti}. Točkam v projektivni ravnini,
ki so oblike $(x : y : 1)$, identificiramo s točko $(x, y)$ v afini
ravnini in jim pravimo \emph{končne točke}.

Točke $(x : y : 0)$ ustrezajo \emph{točkam v neskončnosti}, ki jih
identificiramo s snopi vzporednic.
\end{opomba}

\begin{opomba}
Projektivno ravnino $P_2(\R)$ lahko identificiramo tudi s sfero
$S^2$.
\end{opomba}

\begin{definicija}
\emph{Projektivna premica}\index{Projektivna ravnina!Premica} je
vsaka ravnina, ki gre skozi izhodišče. Identificiramo jo z afino
premico, ki ji dodamo pripadajočo točko v neskončnosti, oziroma
premico v neskončnosti.
\end{definicija}

\begin{opomba}
V sferičnem modelu so premice glavni krogi.
\end{opomba}

\begin{opomba}
Vsaki dve različni projektivni premici se sekata v natanko eni
projektivni točki. Skozi vsaki dve različni projektivni premici
poteka natanko ena projektivna premica.
\end{opomba}

\newpage

\subsection{Projektivne algebraične krivulje}

\begin{definicija}
Polinom $F \in \C[x,y,z]$ je \emph{homogen}\index{Polinom!Homogen},
če so vsi njegovi monomi iste stopnje.
\end{definicija}

\begin{opomba}
$F$ je homogen polinom stopnje $n$ natanko tedaj, ko za vse $x$,
$y$, $z$ in $\lambda$ velja
\[
F(\lambda x, \lambda y, \lambda z) = \lambda^n F(x, y, z).
\]
\end{opomba}

\begin{definicija}
Množica projektivnih ničel homogenega polinoma $F \in \C[x, y, z]$
je
\[
V_h(F) = \setb{(a : b : c) \in P_2(\C)}{F(a, b, c) = 0}.
\]
\end{definicija}

\begin{definicija}
Podmnožica $\mathcal{C} \subseteq P_2(\C)$ je
\emph{projektivna algebraična krivulja}\index{Algebraična krivulja!Projektivna},
če obstaja tak nekonstanten homogen polinom $F \in \C[x, y, z]$,
da velja
\[
\mathcal{C} = V_h(F).
\]
\end{definicija}

\begin{definicija}
\emph{Homogenizacija}\index{Polinom!Homogenizacija} polinoma
$f \in \C[x,y]$ je polinom
\[
F(x, y, z) =
z^{\deg f} \cdot f\left(\frac{x}{z}, \frac{y}{z} \right).
\]
\end{definicija}

\datum{2022-3-10}

\begin{opomba}
Naj bo $F$ homogenizacija polinoma $f$. Tedaj je
\[
V_h(F) \cap \setb{(x : y : z)}{z \ne 0} =
\setb{(x : y : 1)}{f(x, y) = 0}.
\]
Množico $V_h(F) \cap \setb{(x : y : z)}{z \ne 0}$ lahko
identificiramo z $V(f)$, preostale ničle pa s točkami v
neskončnosti.
\end{opomba}

\newpage

\subsection{Projektivne transformacije}

\begin{definicija}
\emph{Projektivna transformacija}\index{Projektivna transformacija}
je bijektivna preslikava $\Phi \colon P_2(\C) \to P_2(\C)$, ki
deluje po predpisu
\[
(x : y : z) \mapsto (ax + by + cz : dx + ey + fz : gx + hy + iz).
\]
\end{definicija}

\begin{opomba}
Preslikava je dobro definirana, saj sta obe strani homogeni.
\end{opomba}

\begin{trditev}
Preslikava z zgornjim predpisom je bijektivna natanko tedaj, ko je
\[
\begin{vmatrix}
a & b & c \\
d & e & f \\
g & h & i
\end{vmatrix}
\ne 0.
\]
\end{trditev}

\obvs

\begin{opomba}
Kompozitum projektivnih transformacij je projektivna
transformacija. Prav tako je tudi inverz projektivne transformacije
projektivna transformacija.
\end{opomba}

\begin{trditev}
Slika algebraične krivulje s projektivno transformacijo je
algebraična krivulja.
\end{trditev}

\begin{proof}
Velja
\[
\Phi(V_h(F)) =
\setb{\Phi(x : y : z)}{F(x, y, z) = 0} =
\setb{(x : y : z)}{(F \circ \Phi^{-1})(x, y, z) = 0}. \qedhere
\]
\end{proof}

\begin{definicija}
Pravimo, da so štiri točke v
\emph{splošni legi}\index{Projektivna ravnina!Splošna lega}, če
nobene tri niso kolinearne.
\end{definicija}

\begin{opomba}
Tri točke so kolinearne natanko tedaj, ko so linearno odvisne,
oziroma ko je determinanta njihovih koordinat enaka $0$.
\end{opomba}

\begin{lema}[O štirih točkah]\index{Lema!O štirih točkah}
Če so točke $p_1$, $p_2$, $p_3$ in $p_4$ ter $q_1$, $q_2$, $q_3$ in
$q_4$ v splošni legi, obstaja natanko ena projektivna
transformacija $\Phi$, za katero je $\Phi(p_i) = q_i$ za vse $i$.
\end{lema}

\begin{proof}
Dovolj je pokazati, da lahko točke $p_i$ preslikamo v točke
\[
t_1 = (1 : 0 : 0), \quad
t_2 = (0 : 1 : 0), \quad
t_3 = (0 : 0 : 0)
\quad \text{in} \quad
t_4 = (1 : 1 : 1).
\]
Naj bo $p_i = (x_i : y_i : z_i)$. Naj bo
\[
A =
\begin{bmatrix}
x_1 & x_2 & x_3 \\
y_1 & y_2 & y_3 \\
z_1 & z_2 & z_3
\end{bmatrix}^{-1}.
\]
Naj bo $\Phi$ projektivna transformacija, ki ustreza matriki $A$.
Sledi, da so točke $\Phi(p_i)$ v splošni legi, zato za
$(a : b : c) = \Phi(p_4)$ velja, da so $a, b, c \ne 0$. Sedaj
zgornjo projektivno transformacijo preprosto komponiramo s
transformacijo
\[
(x : y : z) \mapsto
\left(\frac{x}{a}, \frac{y}{b}, \frac{z}{c}\right). \qedhere
\]
\end{proof}

\newpage

\subsection{Presečišča in njihove večkratnosti}

\datum{2022-3-17}

\begin{definicija}
\emph{Večkratnost}\index{Presečišče!Večkratnost} presečišča
projektivne premice in krivulje je stopnja pripadajočega linearnega
faktorja v faktorizaciji polinoma
\[
F(t(x,y)),
\]
kjer je $t$ parametrizacija premice. Označimo jo z
$\mult_p(F \cap L)$.
\end{definicija}

\begin{opomba}
Vsota večkratnosti vseh presečišč je enaka $\deg F$.
\end{opomba}

\begin{trditev}
Naj bosta $f$ in $g$ polinoma v dveh spremenljivkah, monična v
spremenljivki $y$. Tedaj imata za vsako ničlo $a$ polinoma
$\Res(f,g)$ polinoma $f(a, y)$ in $g(a, y)$ skupno ničlo.
\end{trditev}

\begin{proof}
Velja
\[
\Res(f(a, y), g(a, y)) = 0,
\]
zato imata skupni faktor.
\end{proof}

\begin{trditev}
Naj bosta
\[
F(x, y, z) = \sum_{i=0}^m a_i(x,y) z^{m-i}
\quad \text{in} \quad
G(x, y, z) = \sum_{i=0}^n b_i(x,y) z^{n-i}
\]
homogena polinoma. Tedaj je $\Res(f,g)$ homogen polinom stopnje
\[
\deg F \cdot \deg G - \deg a_0 \cdot \deg b_0.
\]
\end{trditev}

\begin{proof}
Naj bo $d = \deg a_0$ in $e = \deg b_0$. Sledi, da je
\[
\deg a_i = d + i
\quad \text{in} \quad
\deg b_i = e + i.
\]
Naj bo $R = \Res(f, g)$. Dovolj je pokazati, da za vse $\lambda$
velja
\[
R(\lambda x, \lambda y) = \lambda^{mn + dn + em} R(x, y).
\]
Z uporabo $a_i(\lambda x, \lambda y) = \lambda^{d + i} a_i(x, y)$
lahko determinanto rezultante poenostavimo. Prvo vrstico razširimo
z $\lambda^e$, drugo z $\lambda^{e+1}$ in tako dalje v prvih $n$
vrsticah, nato pa podobno naredimo v zadnjih $m$ vrsticah. S tem
smo zagotovili, da so v vsakem stolpcu eksponenti $\lambda$ enaki
in jih lahko izpostavimo. Sledi, da je
\[
\prod_{i=e}^{e+n-1} \lambda^i \cdot
\prod_{i=d}^{d+m-1} \lambda^i \cdot
R(\lambda x, \lambda y) =
\prod_{i=e+d}^{d+e+m+n-1} \lambda^i \cdot R(x, y),
\]
oziroma
\[
\lambda^{\frac{n (2e + n - 1)}{2} + \frac{m(2d + m - 1)}{2}}
R(\lambda x, \lambda y) =
\lambda^{\frac{(2d + 2e + m + n - 1)(m + n)}{2}} R(x, y).
\]
Sledi, da je
\[
R(\lambda x, \lambda y) = \lambda^{mn + dn + em} R(x, y). \qedhere
\]
\end{proof}

\begin{trditev}
Naj bosta $F$ in $G$ nekonstantna polinoma brez skupnega
nekonstantnega faktorja. Tedaj je $V_h(F) \cap V_h(G)$ končna
množica.
\end{trditev}

\begin{proof}
Naj bo $S$ točka, ki ne leži na $V_h(FG)$. Naj bo $\ell$ premica,
ki ne gre skozi $S$. S pomočjo leme o štirih točkah lahko
privzamemo, da je $S = (0 : 0 : 1)$ in ima premica $\ell$ enačbo
$z = 0$.

Naj bo $(a : b : 0)$ projekcija presečišča prek $S$ na premico
$\ell$. Sledi, da ima presečišče koordinate $(a : b : c)$. Iščemo
torej taka števila $a$ in $b$, da imata $F(a, b, z)$ in
$G(a, b, z)$ skupno ničlo, to pa so ravno ničle polinoma
$\Res_{f,g}(a,b)$, to pa je neničeln homogen polinom v dveh
spremenljivkah. Sledi, da ga lahko faktoriziramo na linearne
faktorje, zato ima končno mnogo ničel.

Opazimo še, da na vsaki premici skozi $S$ leži končno mnogo
presečišč. V nasprotnem primeru je namreč premica vsebovana v eni
izmed krivulj, kar je v protislovju s tem, da $S$ ne leži na
krivuljah. 
\end{proof}

\datum{2022-3-24}

\begin{izrek}[Bezout]\index{Izrek!Bezout}
Naj bosta $F$ in $G$ tuja homogena polinoma v treh spremenljivkah.
Število presečišč $F$ in $G$ je navzgor omejeno z
$\deg F \cdot \deg G$.
\end{izrek}

\begin{proof}
Naj bo $S$ projektivna točka, ki ne leži na $F = 0$ in $G = 0$ ter
niti na nobeni premici, ki gre skozi dve presečišči teh krivulj,
$\ell$ pa premica, ki ne gre skozi $S$. Presečišča $F$ in $G$ prek
$S$ projeciramo na $\ell$. Po izbiri točke $S$ projekciji nobenih
dveh presečišč ne sovpadata. Sedaj s projektivno transformacijo
preslikamo $S$ v $(0 : 0 : 1)$ in $\ell$ v premico v neskončnosti.

Točka $(a : b : 0)$ je projekcija nekega presečišča natanko tedaj,
ko obstaja tak $c$, da je $F(a, b, c) = G(a, b, c) = 0$, kar je
ekvivalentno temu, da je
\[
\Res_{F,G}(a, b) = 0,
\]
zato je število presečišč manjše od stopnje rezultante. 
\end{proof}

\begin{opomba}
Ker je
\[
0 \ne F(0, 0, 1),
\]
velja $\deg a_0 = \deg b_0 = 0$. Zgornje meje na ta način zato ne
moremo izboljšati.
\end{opomba}

\begin{definicija}
\emph{Večkratnost}\index{Presečišče!Večkratnost} v točki
$(a : b : c)$ krivulj $F = 0$ in $G = 0$ je stopnja ničle
$(a : b : c)$ v $\Res_{F,G}$. Označimo jo z $\mult_p(F \cap G)$.
\end{definicija}

\begin{posledica}
Vsota večkratnosti presečišč je enaka produktu stopenj.
\end{posledica}

\begin{opomba}
Projektivne transformacije ohranjajo večkratnosti (Gibson).
\end{opomba}

\begin{opomba}
Ta definicija se ujema z definicijo večkratnosti premice in
krivulje.
\end{opomba}
