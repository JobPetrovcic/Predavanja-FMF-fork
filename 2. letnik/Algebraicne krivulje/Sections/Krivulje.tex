\section{Algebraične krivulje}

\subsection{Definicija}

\datum{2022-2-17}

\begin{definicija}
Polinom $P \in K[x_1, \dots, x_n]$ je
\emph{nerazcepen}\index{Polinom!Nerazcepen}, če se ga ne da zapisati
kot produkt dveh nekonstantnih polinomov iz $K[x_1, \dots, x_n]$.
\end{definicija}

\begin{definicija}
Za polinom $F \in K[x,y]$ označimo njegovo množico ničel
\[
V(F) = \setb{(a,b) \in K^2}{F(a,b) = 0}.
\]
\end{definicija}

\begin{okvir}
\begin{definicija}
Množica $\mathcal{C} \subseteq K^2$ je
\emph{algebraična krivulja}\index{Algebraična krivulja}, če obstaja
tak nekonstanten polinom $F \in K[x,y]$, da je
\[
\mathcal{C} = V(F).
\]
Pravimo, da je krivulja
\emph{nerazcepna}\index{Algebraična krivulja!Nerazcepna}, če je v
zgornji definiciji $F$ nerazcepen polinom.
\end{definicija}
\end{okvir}

\newpage

\subsection{Afina ekvivalentnost krivulj}

\begin{definicija}
\emph{Afina preslikava}\index{Afina preslikava} je kompozitum
linearne preslikave in translacije. Če je ta linearna preslikava
obrnljiva, je tudi afina preslikava obrnljiva in ji pravimo
\emph{afina transformacija}.
\end{definicija}

\begin{trditev}
Kompozitum afinih transformacij je afina transformacija.
\end{trditev}

\begin{proof}
Afine transformacije so natanko preslikave
\[
(x,y) \mapsto (ax + by + \alpha, cx + dy + \beta),
\]
kjer je $ad \ne bc$.
\end{proof}

\begin{definicija}
Krivulji $\mathcal{C}$ in $\mathcal{D}$ sta
\emph{afino ekvivalentni}\index{Algebraična krivulja!Afino ekvivalentna},
če obstaja afina transformacija $\Phi$, za katero je
$\Phi(\mathcal{C}) = \mathcal{D}$.
\end{definicija}

\begin{opomba}
Afina ekvivalenca je ekvivalenčna relacija.
\end{opomba}

\begin{lema}[Study]
Naj bo $G \in K[x,y]$ nerazcepen nekonstanten polinom. Tedaj za
vsak polinom $F \in K[x,y]$ velja
\[
G \mid F \iff V(G) \subseteq V(F).
\]
\end{lema}
