\section{Projektivno zaprtje}

\subsection{Projektivna ravnina}

\datum{2022-3-3}

\begin{definicija}
Naj bo $K$ polje. \emph{Afina ravnina}\index{Afina ravnina} je
množica $A_2(K) = K^2$.
\end{definicija}

\begin{definicija}
Naj bo $K$ polje.
\emph{Projektivna ravnina}\index{Projektivna ravnina} je množica
vseh premic v $K^3$, ki potekajo skozi izhodišče. Označimo jo z
$P_2(K)$.
\end{definicija}

\begin{definicija}
\emph{Projektivne koordinate}\index{Projektivna ravnina!Koordinate}
projektivne točke je razmerje
\[
(x : y : z).
\]
\end{definicija}

\begin{opomba}
Vsakim projektivnim koordinatam, različnim od $(0 : 0 : 0)$,
ustreza natanko ena projektivna točka.
\end{opomba}

\begin{opomba}
Projektivno ravnino lahko identificiramo z afino ravnino, ki ji
dodamo \emph{točke v neskončnosti}. Točkam v projektivni ravnini,
ki so oblike $(x : y : 1)$, identificiramo s točko $(x, y)$ v afini
ravnini in jim pravimo \emph{končne točke}.

Točke $(x : y : 0)$ ustrezajo \emph{točkam v neskončnosti}, ki jih
identificiramo s snopi vzporednic.
\end{opomba}

\begin{opomba}
Projektivno ravnino $P_2(\R)$ lahko identificiramo tudi s sfero
$S^2$.
\end{opomba}

\begin{definicija}
\emph{Projektivna premica}\index{Projektivna ravnina!Premica} je
vsaka ravnina, ki gre skozi izhodišče. Identificiramo jo z afino
premico, ki ji dodamo pripadajočo točko v neskončnosti, oziroma
premico v neskončnosti.
\end{definicija}

\begin{opomba}
V sferičnem modelu so premice glavni krogi.
\end{opomba}

\begin{opomba}
Vsaki dve različni projektivni premici se sekata v natanko eni
projektivni točki. Skozi vsaki dve različni projektivni premici
poteka natanko ena projektivna premica.
\end{opomba}

\newpage

\subsection{Projektivne algebraične krivulje}

\begin{definicija}
Polinom $F \in \C[x,y,z]$ je \emph{homogen}\index{Polinom!Homogen},
če so vsi njegovi monomi iste stopnje.
\end{definicija}

\begin{opomba}
$F$ je homogen polinom stopnje $n$ natanko tedaj, ko za vse $x$,
$y$, $z$ in $\lambda$ velja
\[
F(\lambda x, \lambda y, \lambda z) = \lambda^n F(x, y, z).
\]
\end{opomba}

\begin{definicija}
Množica projektivnih ničel homogenega polinoma $F \in \C[x, y, z]$
je
\[
V_h(F) = \setb{(a : b : c) \in P_2(\C)}{F(a, b, c) = 0}.
\]
\end{definicija}

\begin{definicija}
Podmnožica $\mathcal{C} \subseteq P_2(\C)$ je
\emph{projektivna algebraična krivulja}\index{Algebraična krivulja!Projektivna},
če obstaja tak nekonstanten homogen polinom $F \in \C[x, y, z]$,
da velja
\[
\mathcal{C} = V_h(F).
\]
\end{definicija}

\begin{definicija}
\emph{Homogenizacija}\index{Polinom!Homogenizacija} polinoma
$f \in \C[x,y]$ je polinom
\[
F(x, y, z) =
z^{\deg f} \cdot f\left(\frac{x}{z}, \frac{y}{z} \right).
\]
\end{definicija}

\datum{2022-3-10}

\begin{opomba}
Naj bo $F$ homogenizacija polinoma $f$. Tedaj je
\[
V_h(F) \cap \setb{(x : y : z)}{z \ne 0} =
\setb{(x : y : 1)}{f(x, y) = 0}.
\]
Množico $V_h(F) \cap \setb{(x : y : z)}{z \ne 0}$ lahko
identificiramo z $V(f)$, preostale ničle pa s točkami v
neskončnosti.
\end{opomba}

\newpage

\subsection{Projektivne transformacije}

\begin{definicija}
\emph{Projektivna transformacija} je bijektivna preslikava
$\Phi \colon P_2(\C) \to P_2(\C)$, ki deluje po predpisu
\[
(x : y : z) \mapsto (ax + by + cz : dx + ey + fz : gx + hy + iz).
\]
\end{definicija}

\begin{opomba}
Preslikava je dobro definirana, saj sta obe strani homogeni.
\end{opomba}

\begin{trditev}
Preslikava z zgornjim predpisom je bijektivna natanko tedaj, ko je
\[
\begin{vmatrix}
a & b & c \\
d & e & f \\
g & h & i
\end{vmatrix}
\ne 0.
\]
\end{trditev}

\obvs

\begin{opomba}
Kompozitum projektivnih transformacij je projektivna
transformacija. Prav tako je tudi inverz projektivne transformacije
projektivna transformacija.
\end{opomba}

\begin{trditev}
Slika algebraične krivulje s projektivno transformacijo je
algebraična krivulja.
\end{trditev}

\begin{proof}
Velja
\[
\Phi(V_h(F)) =
\setb{\Phi(x : y : z)}{F(x, y, z) = 0} =
\setb{(x : y : z)}{(F \circ \Phi^{-1})(x, y, z) = 0}. \qedhere
\]
\end{proof}

\begin{definicija}
Pravimo, da so štiri točke v
\emph{splošni legi}\index{Projektivna ravnina!Splošna lega}, če
nobene tri niso kolinearne.
\end{definicija}

\begin{opomba}
Tri točke so kolinearne natanko tedaj, ko so linearno odvisne,
oziroma ko je determinanta njihovih koordinat enaka $0$.
\end{opomba}

\begin{lema}[O štirih točkah]\index{Lema!O štirih točkah}
Če so točke $p_1$, $p_2$, $p_3$ in $p_4$ ter $q_1$, $q_2$, $q_3$ in
$q_4$ v splošni legi, obstaja natanko ena projektivna
transformacija $\Phi$, za katero je $\Phi(p_i) = q_i$ za vse $i$.
\end{lema}

\begin{proof}
Dovolj je pokazati, da lahko točke $p_i$ preslikamo v točke
\[
t_1 = (1 : 0 : 0), \quad
t_2 = (0 : 1 : 0), \quad
t_3 = (0 : 0 : 0)
\quad \text{in} \quad
t_4 = (1 : 1 : 1).
\]
Naj bo $p_i = (x_i : y_i : z_i)$. Naj bo
\[
A =
\begin{bmatrix}
x_1 & x_2 & x_3 \\
y_1 & y_2 & y_3 \\
z_1 & z_2 & z_3
\end{bmatrix}^{-1}.
\]
Naj bo $\Phi$ projektivna transformacija, ki ustreza matriki $A$.
Sledi, da so točke $\Phi(p_i)$ v splošni legi, zato za
$(a : b : c) = \Phi(p_4)$ velja, da so $a, b, c \ne 0$. Sedaj
zgornjo projektivno transformacijo preprosto komponiramo s
transformacijo
\[
(x : y : z) \mapsto
\left(\frac{x}{a}, \frac{y}{b}, \frac{z}{c}\right). \qedhere
\]
\end{proof}
