\section{Topološke lastnosti}

\subsection{Ločljivost}

\datum{2021-11-16}

\begin{definicija}
Pravimo, da topologija
\emph{loči}\index{Topološka lastnost!Ločljivost} disjunktni množici
$A, B \subseteq X$, če obstajata taki množici $U, V \in \tau$, da
velja
\[
A \subseteq U,\quad B \subseteq V,\quad B \cap U = \emptyset
\quad \text{in} \quad
A \cap V = \emptyset.
\]
Pravimo, da topologija \emph{ostro loči} množici, če sta v zgornji
definiciji $U$ in $V$ disjunktni.
\end{definicija}

\begin{definicija}
Topološki prostor $(X, \tau)$ je
\emph{Hausdorffov}\index{Topološka lastnost!Hausdorffova}, če
$\tau$ ostro loči vse točke.
\end{definicija}

\begin{trditev}
Naslednje izjave so ekvivalentne:

\begin{enumerate}[i)]
\item $X$ je Hausdorffov
\item Za vse $x \ne y$ obstaja tak $U \in \tau$, da je $x \in U$ in
$y \not \in \overline{U}$
\item Diagonala\footnote{\emph{Diagonala} je množica
$\Delta_x = \setb{(x,x)}{x \in X}$.} je zaprt podprostor v
$X \times X$.
\end{enumerate}
\end{trditev}

\begin{proof}
Če je prostor Hausdorffov, lahko preprosto vzamemo $U$ iz
definicije. Sledi, da $y \not \in \overline{U}$. Če velja druga
točka, pa lahko preprosto vzamemo $V = \overline{U}^{\mathsf{c}}$.

Naj bo $x \ne y$, $U$ in $V$ pa taki odprti množici, da je
$x \in U$ in $y \in V$. $U$ in $V$ sta disjunktni natanko tedaj, ko
$U \times V$ ne seka diagonale. Če je prostor Hausdorffov, lahko
preprosto vzamemo $U$ iz definicije. Obratno, če je diagonala
zaprta, obstaja škatlasta okolica $(x,y)$, ki ne seka diagonale,
iz te pa dobimo želene okolice.
\end{proof}

\begin{izrek}
Naj bo $Y$ Hausdorffov.

\begin{enumerate}[i)]
\item Vsaka končna podmnožica $Y$ je zaprta
\item Zaporedja v $Y$ imajo največ eno limito
\item Če sta $f, g \colon X \to Y$ zvezni, je
$\setb{x \in X}{f(x) = g(x)}$ zaprta v $X$
\item Če se zvezni preslikavi $f, g \colon X \to Y$ ujemata na
gosti podmnožici $X$, sta enaki
\item Graf zvezne preslikave $f \colon X \to Y$ je zaprt v
$X \times Y$.
\end{enumerate}
\end{izrek}

\begin{proof}
Prvi dve točki sta očitni.

Vidimo, da je $(f,g)$ zvezna, velja pa
$\setb{x \in X}{f(x) = g(x)} = (f,g)^{-1}(\Delta_x)$.

4.\ točka sledi iz 3.\ -- če se preslikavi ujemata na neki množici,
se ujemata tudi na zaprtju.

Za dokaz 5.\ točke definirajmo preslikavi
$u, v \colon X \times Y \to Y$ z $u(x,y) = f(x)$ in $v(x,y) = y$.
Množica točk ujemanja $u$ in $v$ je ravno graf
$\Gamma_f = \setb{(x,y) \in X \times Y}{y = f(x)} \subseteq
X \times Y$ funkcije $f$, ki je po 3.\ točki zaprt.
\end{proof}

\begin{izrek}
Naj bo $X$ 1-števen in $Y$ Hausdorffov. Potem je preslikava
$f \colon X \to Y$ zvezna natanko tedaj, ko za vsako konvergentno
zaporedje v $X$ velja
\[
f\left(\lim_{n\to\infty} x_n\right) = \lim_{n\to\infty} f(x_n).
\]
\end{izrek}

\begin{proof}
Privzemimo, da je $f$ zvezna in opazujmo zaporedje $(x_n)$ v $X$,
ki konvergira proti $x$. Za poljubno okolico $U$ točke $f(x)$ je
zaradi zveznosti $f^{-1}(U)$ okolica točke $x$, kar pomeni, da so v
$f^{-1}(U)$ skoraj vsi elementi zaporedja $(x_n)$. Od tod pa sledi,
da so v $U$ skoraj vsi elementi zaporedja $(f(x_n))$, torej je
$\lim_{n \to \infty} f(x_n) = f(x)$.

Predpostavimo sedaj, da $f$ ohranja limite. Naj bo $x$ točka, ki je
v zaprtju množice $A \subseteq X$, in naj bo $U_1, U_2, \dots$ neka
baza okolic za $x$. Tedaj za vsak $n$ lahko izberemo točko
$x_n \in A \cap U_1 \cap \dots \cap U_n$ (ker je ta množica
neprazna, saj je $U_1 \cap \dots \cap U_n$ odprta okolica $x$, ki
je iz zaprtja $A$). Tako smo konstruirali zaporedje $(x_n)$, ki
konvergira proti točki $x$ in so obenem vsi $f(x_n) \in f(A)$. Po
predpostavki je
\[
f(x) =
f\left(\lim_{n \to \infty} x_n\right) =
\lim_{n \to \infty} f(x_n)
\in \overline{f(A)}.
\]
To pa pomeni, da za poljuben $A \subseteq X$ velja
$f(\overline{A}) \subseteq \overline{f(A)}$, torej je $f$ zvezna po
izreku \ref{iz:1}.
\end{proof}

\begin{definicija}
Prostor $(X, \tau)$ je
\emph{Fréchetov}\index{Topološka lastnost!Fréchetova}, če $\tau$
loči točke.
\end{definicija}

\begin{trditev}
$X$ je Fréchetov natanko tedaj, ko $\tau$ finejša od topologije
končnih komplementov.
\end{trditev}

\begin{proof}
Če je prostor Fréchetov, za vse različne $x$ in $y$ obstaja tak
$V \in \tau$, da je $y \in V$ in $x \ne \in V$, zato je $y$
notranja v $X \setminus \set{x}$. Če pa so enojčki zaprti, pa
$X \setminus \set{x}$ in $X \setminus \set{y}$ ločita $x$ in $y$.
\end{proof}

\begin{trditev}
Hausdorffova in Fréchetova lastnost sta dedni in
multiplikativni.\footnote{Lastnost je \emph{multiplikativna}, če za
vse prostore $X$ in $Y$ s to lastnostjo velja, da jo ima tudi
$X \times Y$.}
\end{trditev}

\obvs

\begin{definicija}
Prostor $(X, \tau)$ je regularen, če je Hausdorffov in $\tau$ ostro
loči točke od zaprtih množic.
\end{definicija}

\begin{definicija}
Prostor $(X, \tau)$ je normalen, če je Hausdorffov in $\tau$ ostro
loči disjunktne zaprte množice.
\end{definicija}

\datum{2021-11-23}

\begin{izrek}
Če je $X$ metričen, je normalen.
\end{izrek}

\begin{proof}
Očitno je $X$ Hausdorffov, zato je dovolj dokazati, da $\tau$ ostro
loči disjunktne zaprte množice. Naj bosta $A$ in $B$ taki množici.
Sedaj preprosto vzamemo
\[
U = \setb{x \in X}{d(x,A) < d(x,B)}
\quad \text{in} \quad
V = \setb{x \in X}{d(x,B) < d(x,A)}.
\]
Ni težko videti, da sta ti množici res odprti in disjunktni.
\end{proof}

\begin{trditev}
Regularnost je dedna, normalnost pa se deduje na zaprte
podprostore.
\end{trditev}

\obvs

\begin{izrek}
Prostor, ki je regularen in 2-števen, je normalen.
\end{izrek}

\begin{proof}
Naj bosta $A$ in $B$ disjunktni zaprti množici v $X$ s števno bazo
$\mathcal{B}$.

Za vsak $a \in A$ obstajata taki disjunktni odprti množici $U_a$ in
$U_a'$, da je $a \in U_a$ in $B \subseteq U_a'$. Za $U_a$ lahko
vzamemo kar bazično okolico. Unija
\[
\bigcup_{a \in A} U_a
\]
je tako števno pokritje $A$, ki je disjunktno $B$. Podobno lahko
najdemo števno pokritje
\[
\bigcup_{b \in B} V_b
\]
množice $B$, ki je disjunktno $A$. Te množice lahko zapišemo v
zaporedjih $(U_n)$ in $(V_n)$. Sedaj preprosto vzamemo
\[
U_i' = U_i \setminus \bigcup_{j \leq i} \overline{V_j}
\quad \text{in} \quad
V_i' = V_i \setminus \bigcup_{j \leq i} \overline{U_j}.
\]
Uniji teh množic sta disjunktni in odprti.
\end{proof}

\begin{definicija}
Različne stopnje ločljivosti označimo na naslednji način:\footnote{
Oznaka ni enotna.}

\begin{enumerate}[label=$T_{\arabic*}$:, start=0]
\item Za vse $x \ne y$ ima vsaj ena izmed njiju okolico, ki ne
vsebuje druge.
\item Prostor je Fréchetov.
\item Prostor je Hausdorffov.
\item Za vsako zaprto množico $A$ in $x \not \in A$ obstajata
okolici, ki ju ostro ločita.
\item Topologija ostro loči disjunktne zaprte množice.
\end{enumerate}
\end{definicija}

\begin{trditev}
Prostor je $T_3$ natanko tedaj, ko za vsako odprto množico $U$ in
$x \in U$ obstaja taka odprta množica $V$, da je $x \in V$ in
$\overline{V} \subseteq U$.
\end{trditev}

\obvs

\begin{trditev}
Regularnost je multiplikativna.
\end{trditev}

\begin{proof}
Naj bo $U$ odprta množica in $x \in U$. Brez škode za splošnost
naj bo $U$ bazična. Sedaj uporabimo regularnost na komponentah in
vzamemo njun kartezični produkt.
\end{proof}

\begin{trditev}
Prostor je $T_4$ natanko tedaj, ko za vsako zaprto množico $A$,
vsebovano v odprti množici $U$, obstaja taka odprta množica $V$,
za katero je $A \subseteq V$ in $\overline{V} \subseteq U$.
\end{trditev}

\obvs

\newpage

\subsection{Povezanost}

\datum{2021-11-30}

\begin{okvir}
\begin{definicija}
Prostor $(X, \tau)$ je \emph{nepovezan}, če ga lahko razcepimo kot
disjunktno unijo dveh nepraznih, odprtih množic. Prostor je
\emph{povezan}\index{Topološka lastnost!Povezanost}, če ni
nepovezan.
\end{definicija}
\end{okvir}

\begin{trditev}
Naslednje izjave so ekvivalentne:

\begin{enumerate}[i)]
\item $X$ je nepovezan
\item $X$ lahko predstavimo kot disjunktno unijo dveh nepraznih,
zaprtih množic
\item V $X$ obstaja netrivialna podmnožica, ki je odprta in zaprta
\item Obstaja zvezna surjekcija $f \colon X \to \set{0,1}$.
\end{enumerate}
\end{trditev}

\obvs

\begin{izrek}
$A \subseteq \R$ je povezana natanko tedaj, ko je $A$ interval.
\end{izrek}

\begin{proof}
Če $A$ ni interval, vzamemo
\[
\setb{x \in \R}{x<t} \quad \text{in} \quad \setb{x \in \R}{x>t},
\]
kjer je $t \ne \in A$ in sta zgornji množici neprazni.

Naj bo $A$ interval, ki ga lahko zapišemo kot $U+V$. Naj bo
$a \in U$, $c \in V$ in $a < c$. Naj bo
\[
b = \sup \setb{x \in \R}{[a,x) \subseteq U}.
\]
Sledi, da je $b \in A$, saj je $b < c$. Tako je $b \in \Cl_A U$,
zato je tudi v $U$, saj je $U$ zaprt v $A$, to pa je v protislovju
z odprtostjo $U$ in definicijo $b$.
\end{proof}

\begin{izrek}
Veljajo naslednje trditve:

\begin{enumerate}[i)]
\item Naj bo $f \colon X \to Y$ zvezna. Če je $X$ povezan, je tudi
$f(X)$ povezan.
\item Če so $\set{A_\lambda}$ povezane podmnožice $X$ z nepraznim
presekom, je njihova unija povezana.
\item Povezanost je multiplikativna.
\item Če za vsaka $a, b \in A$ obstaja pot\footnote{\emph{Pot} med
$a$ in $b$ je zvezna preslikava $\gamma \colon [0,1] \to A$, za
katero je $\gamma(0) = a$ in $\gamma(1) = b$.} med njima, je $A$
povezan
\item Če je $A \subseteq X$ povezan in
$A \subseteq B \subseteq \overline{A}$, je tudi $B$ povezan.
\end{enumerate}
\end{izrek}

\begin{proof}
Uporabimo karakterizacijo s surjekcijo $f \colon X \to \set{0,1}$.
\end{proof}

\begin{izrek}[O vmesni vrednosti]\index{Izrek!O vmesni vrednosti}
Naj bo $X$ povezana in $f \colon X \to \R$. Potem je $f(X)$
interval.
\end{izrek}

\obvs

\begin{definicija}
Prostor $X$ je \emph{povezan s potmi}, če med vsakima točkama
prostora obstaja pot.
\end{definicija}

\begin{definicija}
\emph{Komponente}\index{Topološka lastnost!Povezanost!Komponenta}
prostora $X$ so maksimalne povezane podmnožice $X$. Komponento, ki
vsebuje $x$, označimo z $C(x)$. Če so vse komponente $X$ enojčki,
pravimo, da je $X$ \emph{povsem nepovezan}.
\end{definicija}

\begin{izrek}
Veljajo naslednje trditve:

\begin{enumerate}[i)]
\item Komponente tvorijo particijo prostora.
\item Komponente $X$ so zaprti prostori.
\item Slika komponente z zvezno preslikavo je vsebovana v
eni komponenti kodomene.
\end{enumerate}
\end{izrek}

\obvs

\datum{2021-12-7}

\begin{definicija}
Prostor je \emph{lokalno
povezan}\index{Topološka lastnost!Povezanost!Lokalna}, če ima bazo
iz povezanih množic.
\end{definicija}

\begin{trditev}
Prostor $X$ je lokalno povezan natanko tedaj, ko so komponente
vsake odprte podmnožice $X$ odprte.
\end{trditev}

\begin{proof}
Če je $X$ lokalno povezan, preprosto vzamemo bazno okolico vsake
točke, če pa so komponente odprte, pa preprosto vzamemo komponente
vseh odprtih množic.
\end{proof}

\begin{definicija}
\emph{Potne komponente} prostora $X$ so maksimalne podmnožice $X$,
ki so povezane s potmi. Komponento, ki vsebuje $x$, označimo z
$C'(x)$.
\end{definicija}

\begin{trditev}
Veljata naslednji trditvi:

\begin{enumerate}[i)]
\item Potne komponente tvorijo particijo prostora.
\item Slika potne komponente z zvezno preslikavo je vsebovana v
eni potni komponenti kodomene.
\end{enumerate}
\end{trditev}

\begin{definicija}
Prostor je \emph{lokalno povezan s potmi}, če ima bazo iz odprtih
množic, ki so povezane s potmi.
\end{definicija}

\begin{izrek}
V prostoru, ki je lokalno povezan s potmi, so komponente enake
komponentam s potmi.
\end{izrek}

\begin{proof}
Za vse $x$ velja $C'(x) \subseteq C(x)$. Ker je $C'(x)$ odprta v
$X$, je odprta tudi v $C(x)$. Če je $C'(x) \ne C(x)$, velja, da
lahko $C(x)$ zapišemo kot unijo $C'(x)$ in njenega komplementa, ki
je prav tako odprt, sam je unija potnih komponent.
\end{proof}

\begin{posledica}
Če je $x$ povezan in lokalno povezan s potmi, je povezan s potmi.
\end{posledica}

\newpage

\subsection{Kompaktnost}

\begin{okvir}
\begin{definicija}
Prostor je \emph{kompakten}\index{Topološka lastnost!Kompaktnost},
če ima vsako njegovo odprto pokritje končno podpokritje.
\end{definicija}
\end{okvir}

\begin{trditev}
$X$ je kompakten natanko tedaj, ko ima vsako bazično pokritje
končno podpokritje.
\end{trditev}

\obvs

\begin{izrek}
Vsak kompakten metrični prostor je omejen.
\end{izrek}

\obvs

\begin{izrek}
Zaprti intervali so kompaktni.
\end{izrek}

\begin{proof}
Glej trditev 7.5.3.\ v zapiskih Analize 1 prvega letnika.
\end{proof}

\begin{izrek}
Zvezna slika kompakta je kompakt.
\end{izrek}

\begin{proof}
Praslika pokritja slike ima končno podpokritje.
\end{proof}

\begin{trditev}
Zaprt podprostor kompaktnega prostora je kompakten.
\end{trditev}

\begin{proof}
Pokritju dodamo komplement podprostora.
\end{proof}

\datum{2021-12-14}

\begin{izrek}
Kompaktnost je multiplikativna.
\end{izrek}

\begin{proof}
Za vsako pokritje obstaja končno podpokritje množice
$\set{x} \times Y$. Tega lahko projeciramo na $X$ in dobimo odprto
okolico $x$. Ko to naredimo za vse $x \in X$, dobimo pokritje $X$,
ki ima končno podpokritje.
\end{proof}

\begin{posledica}
Vsaka omejena, zaprta podmnožica $\R^n$ je kompaktna.
\end{posledica}

\begin{trditev}
Naj bo $K$ kompakten podprostor Hausdorffovega prostora $X$. Tedaj
je $K$ zaprt v $X$.
\end{trditev}

\begin{proof}
Prostor ostro loči vsako točko $K$ od poljubne točke
$x \not \in K$. S tem dobimo pokritje $K$, ki ima končno
podpokritje. Sledi, da njihova unija in presek pripadajočih okolic
$x$ ostro ločita $x$ od $K$. Sledi, da za vse $X \not \in K$
obstaja okolica, ki ne seka $K$.
\end{proof}

\begin{izrek}[Heine-Borel-Lebesque]
Množica $A \subseteq \R^n$ je kompakten natanko tedaj, ko je $A$
zaprta in omejena.
\end{izrek}

\begin{posledica}
Vsaka preslikava $f \colon X \to \R$, kjer je $X$ kompakten, je
omejena in zavzame $\min$ in $\max$.
\end{posledica}

\begin{proof}
$f(X)$ je kompaktna, zato je zaprta in omejena.
\end{proof}

\begin{trditev}
V kompaktu ima vsaka neskončna množica stekališče.
\end{trditev}

\begin{proof}
Naj bo $A$ neskončna podmnožica brez stekališč. Za vsak $x \in X$
obstaja neka njegova okolica, ki vsebuje kvečjemu končno točk $A$.
Ker je $X$ kompakten, obstaja končno podpokritje, kar je
protislovje z neskončnostjo $A$.
\end{proof}

\begin{posledica}[Bolzano-Weierstrass]
Vsako omejeno zaporedje v $\R^n$ ima konvergentno podzaporedje.
\end{posledica}

\obvs

\begin{izrek}
Kompakten Hausdorffov prostor je normalen.
\end{izrek}

\begin{proof}
Podoben kot 2.3.8.
\end{proof}

\begin{izrek}
Kompakten metrični prostor je 2-števen in separabilen.
\end{izrek}

\begin{proof}
Za vsak $n$ vzamemo končno podpokritje pokritja s kroglami z
radijem $\frac{1}{n}$. Njihova unija je števna baza, središča
krogel pa gosta podmnožica.
\end{proof}

\begin{trditev}
$X$ je kompakten natanko tedaj, ko v vsaki družini zaprtih množic
s praznim presekom obstaja končna poddružina, ki ima prazen presek.
\end{trditev}

\obvs

\begin{izrek}[Cantor]
Padajoča veriga zaprtih množic v kompaktnem prostoru ima neprazen
presek.
\end{izrek}

\obvs

\begin{lema}[Lebesque]
Za vsako odprto pokritje $U$ metričnega prostora $X$ obstaja
\emph{Lebesqueovo število} $\lambda>0$ z lastnostjo, da je vsaka
krogla z radijem manjšim od $\lambda$ vsebovana v nekem elementu
$U$.
\end{lema}

\begin{proof}
Vzamemo končno podpokritje in za vsako točko vzamemo največjo
radij, za katerega je krogla s tem radijem še vsebovana v katerem
izmed elementov $U$. Opazimo, da je to zvezna funkcija, zato
zavzame minimum, ki pa je pozitiven.
\end{proof}

\begin{posledica}
Naj bo $f \colon X \to Y$ zvezna, kjer sta $X$ in $Y$ metrična. Če
je $X$ kompakten, je $f$ enakomerno zvezna.
\end{posledica}

\begin{proof}
Vzamemo Lebesqueovo število praslik majhnih krogel.
\end{proof}

\begin{definicija}
%Dodaj index
$X$ je lokalno kompakten, če ima vsaka točka kakšno kompaktno
okolico.
\end{definicija}

\begin{definicija}
Odprta množica $U \subseteq X$ je \emph{relativno kompaktna}, če je
$\overline{U}$ kompakt.
\end{definicija}

\begin{trditev}
Hausdorffov prostor je lokalno kompakten natanko tedaj, ko ima bazo
iz relativno kompaktnih množic.
\end{trditev}

%Add proof ig

\begin{izrek}
Vsak lokalno kompakten Hausdorffov prostor je regularen.
\end{izrek}

%Add proof later

\begin{definicija}
Prostor $X$ je \emph{Borelov}, če ga ni mogoče predstaviti kot
števno unijo zaprtih množic s prazno notranjostjo.
\end{definicija}

\begin{izrek}
Naj bodo $F_i$ zaprte množice s prazno notranjostjo v lokalno
kompaktnem zaprtem prostoru $X$. Potem ima tudi njihova unija
prazno notranjost.
\end{izrek}

\begin{proof}
Naj bo $U_1 \subseteq \bigcup F_i$ odprta. Tedaj obstaja
$x \in U_1 \setminus F_1$ in njegova relativno kompaktna okolica,
ki ne seka $F_1$. Če ta postopek nadaljujemo, dobimo zaporedje
$\overline{U_i}$ zaprtih množic, ki so vsebovane ena v drugi, zato
imajo neprazen presek, kar je protislovje.
\end{proof}
