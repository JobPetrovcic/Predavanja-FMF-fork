\section{Topološke lastnosti}

\subsection{Ločljivost}

\datum{2021-11-16}

\begin{definicija}
Pravimo, da topologija
\emph{loči}\index{Topološka lastnost!Ločljivost} disjunktni množici
$A, B \subseteq X$, če obstajata taki množici $U, V \in \tau$, da
velja
\[
A \subseteq U,\quad B \subseteq V,\quad B \cap U = \emptyset
\quad \text{in} \quad
A \cap V = \emptyset.
\]
Pravimo, da topologija \emph{ostro loči} množici, če sta v zgornji
definiciji $U$ in $V$ disjunktni.
\end{definicija}

\begin{definicija}
Topološki prostor $(X, \tau)$ je
\emph{Hausdorffov}\index{Topološka lastnost!Hausdorffova}, če
$\tau$ ostro loči vse točke.
\end{definicija}

\begin{trditev}
Naslednje izjave so ekvivalentne:

\begin{enumerate}[i)]
\item $X$ je Hausdorffov
\item Za vse $x \ne y$ obstaja tak $U \in \tau$, da je $x \in U$ in
$y \not \in \overline{U}$
\item Diagonala\footnote{\emph{Diagonala} je množica
$\Delta_x = \setb{(x,x)}{x \in X}$.} je zaprt podprostor v
$X \times X$.
\end{enumerate}
\end{trditev}

\begin{proof}
Če je prostor Hausdorffov, lahko preprosto vzamemo $U$ iz
definicije. Sledi, da $y \not \in \overline{U}$. Če velja druga
točka, pa lahko preprosto vzamemo $V = \overline{U}^{\mathsf{c}}$.

Naj bo $x \ne y$, $U$ in $V$ pa taki odprti množici, da je
$x \in U$ in $y \in V$. $U$ in $V$ sta disjunktni natanko tedaj, ko
$U \times V$ ne seka diagonale. Če je prostor Hausdorffov, lahko
preprosto vzamemo $U$ iz definicije. Obratno, če je diagonala
zaprta, obstaja škatlasta okolica $(x,y)$, ki ne seka diagonale,
iz te pa dobimo želene okolice.
\end{proof}

\begin{izrek}
Naj bo $Y$ Hausdorffov.

\begin{enumerate}[i)]
\item Vsaka končna podmnožica $Y$ je zaprta
\item Zaporedja v $Y$ imajo največ eno limito
\item Če sta $f, g \colon X \to Y$ zvezni, je
$\setb{x \in X}{f(x) = g(x)}$ zaprta v $X$
\item Če se zvezni preslikavi $f, g \colon X \to Y$ ujemata na
gosti podmnožici $X$, sta enaki
\item Graf zvezne preslikave $f \colon X \to Y$ je zaprt v
$X \times Y$.
\end{enumerate}
\end{izrek}

\begin{proof}
Prvi dve točki sta očitni.

Vidimo, da je $(f,g)$ zvezna, velja pa
$\setb{x \in X}{f(x) = g(x)} = (f,g)^{-1}(\Delta_x)$.

4.\ točka sledi iz 3.\ -- če se preslikavi ujemata na neki množici,
se ujemata tudi na zaprtju.
%
% 5. je za DN bila, dodaj pol ig
\end{proof}

\begin{izrek}
Naj bo $X$ 1-števen in $Y$ Hausdorffov. Potem je preslikava
$f \colon X \to Y$ zvezna natanko tedaj, ko za vsako konvergentno
zaporedje v $X$ velja
\[
f\left(\lim_{n\to\infty} x_n\right) = \lim_{n\to\infty} f(x_n).
\]
\end{izrek}

% ADD PROOF

\begin{definicija}
Prostor $(X, \tau)$ je
\emph{Fréchetov}\index{Topološka lastnost!Fréchetova}, če $\tau$
loči točke.
\end{definicija}

\begin{trditev}
$X$ je Fréchetov natanko tedaj, ko $\tau$ finejša od topologije
končnih komplementov.
\end{trditev}

\begin{proof}
Če je prostor Fréchetov, za vse različne $x$ in $y$ obstaja tak
$V \in \tau$, da je $y \in V$ in $x \ne \in V$, zato je $y$
notranja v $X \setminus \set{x}$. Če pa so enojčki zaprti, pa
$X \setminus \set{x}$ in $X \setminus \set{y}$ ločita $x$ in $y$.
\end{proof}

\begin{trditev}
Hausdorffova in Fréchetova lastnost sta dedni in
multiplikativni.\footnote{Lastnost je \emph{multiplikativna}, če za
vse prostore $X$ in $Y$ s to lastnostjo velja, da jo ima tudi
$X \times Y$.}
\end{trditev}

\obvs

\begin{definicija}
Prostor $(X, \tau)$ je regularen, če je Hausdorffov in $\tau$ ostro
loči točke od zaprtih množic.
\end{definicija}

\begin{definicija}
Prostor $(X, \tau)$ je normalen, če je Hausdorffov in $\tau$ ostro
loči disjunktne zaprte množice.
\end{definicija}

\datum{2021-11-23}

\begin{izrek}
Če je $X$ metričen, je normalen.
\end{izrek}

\begin{proof}
Očitno je $X$ Hausdorffov, zato je dovolj dokazati, da $\tau$ ostro
loči disjunktne zaprte množice. Naj bosta $A$ in $B$ taki množici.
Sedaj preprosto vzamemo
\[
U = \setb{x \in X}{d(x,A) < d(x,B)}
\quad \text{in} \quad
V = \setb{x \in X}{d(x,B) < d(x,A)}.
\]
Ni težko videti, da sta ti množici res odprti in disjunktni.
\end{proof}

\begin{trditev}
Regularnost je dedna, normalnost pa se deduje na zaprte
podprostore.
\end{trditev}

\obvs

\begin{izrek}[Tikonov]\index{Izrek!Tikonov}
Prostor, ki je regularen in 2-števen, je normalen.
\end{izrek}

\begin{proof}
Naj bosta $A$ in $B$ disjunktni zaprti množici v $X$ s števno bazo
$\mathcal{B}$.

Za vsak $a \in A$ obstajata taki disjunktni odprti množici $U_a$ in
$U_a'$, da je $a \in U_a$ in $B \subseteq U_a'$. Za $U_a$ lahko
vzamemo kar bazično okolico. Unija
\[
\bigcup_{a \in A} U_a
\]
je tako števno pokritje $A$, ki je disjunktno $B$. Podobno lahko
najdemo števno pokritje
\[
\bigcup_{b \in B} V_b
\]
množice $B$, ki je disjunktno $A$. Te množice lahko zapišemo v
zaporedjih $(U_n)$ in $(V_n)$. Sedaj preprosto vzamemo
\[
U_i' = U_i \setminus \bigcup_{j \leq i} \overline{V_j}
\quad \text{in} \quad
V_i' = V_i \setminus \bigcup_{j \leq i} \overline{U_j}.
\]
Uniji teh množic sta disjunktni in odprti.
\end{proof}

\begin{definicija}
Različne stopnje ločljivosti označimo na naslednji način:\footnote{
Oznaka ni enotna.}

\begin{enumerate}[label=$T_{\arabic*}$:, start=0]
\item Za vse $x \ne y$ ima vsaj ena izmed njiju okolico, ki ne
vsebuje druge.
\item Prostor je Fréchetov.
\item Prostor je Hausdorffov.
\item Prostor je regularen.
\item Prostor je normalen.
\end{enumerate}
\end{definicija}

\begin{trditev}
Prostor je $T_3$ natanko tedaj, ko za vsako odprto množico $U$ in
$x \in U$ obstaja taka odprta množica $V$, da je $x \in V$ in
$\overline{V} \subseteq U$.
\end{trditev}

\obvs

\begin{trditev}
Regularnost je multiplikativna.
\end{trditev}

\begin{proof}
Naj bo $U$ odprta množica in $x \in U$. Brez škode za splošnost
naj bo $U$ bazična. Sedaj uporabimo regularnost na komponentah in
vzamemo njun kartezični produkt.
\end{proof}

\begin{trditev}
Prostor je $T_4$ natanko tedaj, ko za vsako zaprto množico $A$,
vsebovano v odprti množici $U$, obstaja taka odprta množica $V$,
za katero je $A \subseteq V$ in $\overline{V} \subseteq U$.
\end{trditev}

\obvs

%\newpage

%\subsection{Povezanost}
